\documentclass[12pt]{article}
\usepackage{pmmeta}
\pmcanonicalname{SouslinScheme}
\pmcreated{2013-03-22 18:48:30}
\pmmodified{2013-03-22 18:48:30}
\pmowner{gel}{22282}
\pmmodifier{gel}{22282}
\pmtitle{Souslin scheme}
\pmrecord{5}{41610}
\pmprivacy{1}
\pmauthor{gel}{22282}
\pmtype{Definition}
\pmcomment{trigger rebuild}
\pmclassification{msc}{28A05}
\pmsynonym{Suslin scheme}{SouslinScheme}
%\pmkeywords{paved space}
%\pmkeywords{analytic set}
\pmdefines{regular scheme}
\pmdefines{result of a scheme}

\endmetadata

% almost certainly you want these
\usepackage{amssymb}
\usepackage{amsmath}
\usepackage{amsfonts}

% used for TeXing text within eps files
%\usepackage{psfrag}
% need this for including graphics (\includegraphics)
%\usepackage{graphicx}
% for neatly defining theorems and propositions
\usepackage{amsthm}
% making logically defined graphics
%%%\usepackage{xypic}

% there are many more packages, add them here as you need them

% define commands here
\newtheorem*{theorem*}{Theorem}
\newtheorem*{lemma*}{Lemma}
\newtheorem*{corollary*}{Corollary}
\newtheorem*{definition*}{Definition}
\newtheorem{theorem}{Theorem}
\newtheorem{lemma}{Lemma}
\newtheorem{corollary}{Corollary}
\newtheorem{definition}{Definition}

\begin{document}
\PMlinkescapeword{collection}
\PMlinkescapeword{scheme}
\PMlinkescapeword{regular}
\PMlinkescapeword{restriction}

A \emph{Souslin scheme} is a method of representing and defining analytic sets on a paved space $(X,\mathcal{F})$.
Let $\mathcal{S}$ be the collection of finite sequences of positive integers. That is $\mathcal{S}$ is the disjoint union of $\mathbb{N}^n$ for $n=1,2,\ldots$.

A Souslin scheme on $\mathcal{F}$ is a collection $(A_s)_{s\in\mathcal{S}}$ of sets in $\mathcal{F}$.
If $\mathcal{N}=\mathbb{N}^\mathbb{N}$ is Baire space then, for any $s\in\mathcal{N}$ and $n\in\mathbb{N}$, we write $s|_n\equiv (s_1,\ldots,s_n)$ for the restriction of $s$ to $\{1,\ldots,n\}$. So, $s|_n\in\mathbb{N}^n$.

The \emph{result} of the Souslin scheme $(A_s)$ is defined to be
\begin{equation*}
A=\bigcup_{s\in\mathcal{N}}\bigcap_{n=1}^\infty A_{s|_n}.
\end{equation*}
The set $\mathcal{S}$ can be partially ordered as follows. Say that $s\le t$ if $s\in\mathbb{N}^r$ and $t\in\mathbb{N}^s$ for $r\le s$, and $s_k=t_k$ for $k=1,\ldots,r$.
The scheme $(A_s)$ is said to be \emph{regular} if $A_s\supseteq A_t$ for all $s\le t$.

It can be shown that the result of a Souslin scheme is $\mathcal{F}$-analytic and, conversely, any analytic set is the result of some scheme (see equivalent definitions of analytic sets).

\begin{thebibliography}{9}
\bibitem{bourgain}
Jean Bourgain, \emph{A stabilization property and its applications in the theory of sections}. S{\'e}minaire Choquet. Initiation {\`a} l'analyse, 17 no. 1 (1977).
\end{thebibliography}

%%%%%
%%%%%
\end{document}
