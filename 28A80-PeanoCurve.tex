\documentclass[12pt]{article}
\usepackage{pmmeta}
\pmcanonicalname{PeanoCurve}
\pmcreated{2013-03-22 16:32:29}
\pmmodified{2013-03-22 16:32:29}
\pmowner{stevecheng}{10074}
\pmmodifier{stevecheng}{10074}
\pmtitle{Peano curve}
\pmrecord{13}{38723}
\pmprivacy{1}
\pmauthor{stevecheng}{10074}
\pmtype{Definition}
\pmcomment{trigger rebuild}
\pmclassification{msc}{28A80}
\pmsynonym{space-filling curve}{PeanoCurve}
\pmsynonym{space filling curve}{PeanoCurve}

\endmetadata

% this is the default PlanetMath preamble.  as your knowledge
% of TeX increases, you will probably want to edit this, but
% it should be fine as is for beginners.

% almost certainly you want these
\usepackage{amssymb}
\usepackage{amsmath}
\usepackage{amsfonts}

% used for TeXing text within eps files
%\usepackage{psfrag}
% need this for including graphics (\includegraphics)
\usepackage{graphicx}
% for neatly defining theorems and propositions
%\usepackage{amsthm}
% making logically defined graphics
%%%\usepackage{xypic}

% there are many more packages, add them here as you need them

% define commands here

\begin{document}
\PMlinkescapeword{similar}

A \emph{Peano curve} or \emph{space-filling curve} is a continuous mapping of a closed interval onto a square.  

Such mappings, introduced by Peano in 1890, played an
important role in the development of topology as a counterexample 
to the naive ideas of dimension --- while it is
true that one cannot map a space onto a space of higher dimension using a 
smooth map, this is no longer true if one only requires continuity as opposed to
smoothness.  The Peano curve and similar examples led to a rethinking of the foundations
of topology and analysis, and underscored the importance of formulating 
topological notions in a rigorous fashion.

However, still, a space-filling curve cannot ever be one-to-one;
otherwise invariance of dimension would be violated.

Many space-filling curves may be obtained as the limit of a sequence, $\langle \, h_n \mid n \in \mathbb{N} \, \rangle$, of continuous functions $h_n \colon \lbrack 0,1\rbrack \to \lbrack 0,1\rbrack \times \lbrack 0,1\rbrack$. Figure \ref{fig:graph} (\PMlinktofile{source code}{hilbert.cc}), showing the ranges of the first few approximations to Hilbert's space-filling curve, illustrates a common case in which each successive approximation is obtained by applying a recursive procedure to its predecessor.

\begin{figure}[h]
 \centering
 \includegraphics[scale=0.80]{hilbert.eps}
 \caption{The ranges of the first six approximations to Hilbert's space-filling curve}
 \label{fig:graph}
\end{figure}


%%%%%
%%%%%
\end{document}
