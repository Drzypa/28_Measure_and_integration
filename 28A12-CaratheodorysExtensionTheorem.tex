\documentclass[12pt]{article}
\usepackage{pmmeta}
\pmcanonicalname{CaratheodorysExtensionTheorem}
\pmcreated{2013-03-22 18:33:00}
\pmmodified{2013-03-22 18:33:00}
\pmowner{gel}{22282}
\pmmodifier{gel}{22282}
\pmtitle{Carath\'eodory's extension theorem}
\pmrecord{18}{41270}
\pmprivacy{1}
\pmauthor{gel}{22282}
\pmtype{Theorem}
\pmcomment{trigger rebuild}
\pmclassification{msc}{28A12}
%\pmkeywords{measure}
%\pmkeywords{algebra of sets}
%\pmkeywords{$\sigma$-algebra}
\pmrelated{Measure}
\pmrelated{OuterMeasure2}
\pmrelated{LebesgueMeasure}
\pmrelated{CaratheodorysLemma}
\pmrelated{ExistenceOfTheLebesgueMeasure}

\endmetadata

% this is the default PlanetMath preamble.  as your knowledge
% of TeX increases, you will probably want to edit this, but
% it should be fine as is for beginners.

% almost certainly you want these
\usepackage{amssymb}
\usepackage{amsmath}
\usepackage{amsfonts}

% used for TeXing text within eps files
%\usepackage{psfrag}
% need this for including graphics (\includegraphics)
%\usepackage{graphicx}
% for neatly defining theorems and propositions
\usepackage{amsthm}
% making logically defined graphics
%%%\usepackage{xypic}

% there are many more packages, add them here as you need them

% define commands here

\newtheorem*{theorem}{Theorem}

\begin{document}
\PMlinkescapeword{countably additive}
In measure theory, Carath\'eodory's extension theorem is an important result used in the construction of measures, such as the Lebesgue measure on the real number line. The result states that a \PMlinkname{countably additive}{Additive} set function on an algebra of sets can be extended to a measure on the \PMlinkname{$\sigma$-algebra}{SigmaAlgebra} generated by that algebra.

\begin{theorem}[Carath\'eodory]
Let $X$ be a set, $A$ be an algebra on $X$, and $\mathcal{A}\equiv\sigma(A)$ be the $\sigma$-algebra generated by $A$. If $\mu_0\colon A\rightarrow\mathbb{R}_+\cup\{\infty\}$ is a countably additive map then there exists a measure $\mu$ on $(X,\mathcal{A})$ such that $\mu=\mu_0$ on $A$.
\end{theorem}

\begin{thebibliography}{9}
\bibitem{williams}
David Williams, \emph{Probability with martingales},
Cambridge Mathematical Textbooks, Cambridge University Press, 1991.
\bibitem{kallenberg}
Olav Kallenberg, \emph{Foundations of modern probability}, Second edition. Probability and its Applications. Springer-Verlag, 2002.
\end{thebibliography}


\end{document}
%%%%%
%%%%%
\end{document}
