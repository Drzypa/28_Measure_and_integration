\documentclass[12pt]{article}
\usepackage{pmmeta}
\pmcanonicalname{FundamentalLemmaOfCalculusOfVariations}
\pmcreated{2013-03-22 15:02:04}
\pmmodified{2013-03-22 15:02:04}
\pmowner{matte}{1858}
\pmmodifier{matte}{1858}
\pmtitle{fundamental lemma of calculus of variations}
\pmrecord{8}{36745}
\pmprivacy{1}
\pmauthor{matte}{1858}
\pmtype{Theorem}
\pmcomment{trigger rebuild}
\pmclassification{msc}{28B15}
\pmsynonym{fundamental theorem of the calculus of variations}{FundamentalLemmaOfCalculusOfVariations}
\pmrelated{CalculusOfVariations}

\endmetadata

% this is the default PlanetMath preamble.  as your knowledge
% of TeX increases, you will probably want to edit this, but
% it should be fine as is for beginners.

% almost certainly you want these
\usepackage{amssymb}
\usepackage{amsmath}
\usepackage{amsfonts}
\usepackage{amsthm}

\usepackage{mathrsfs}

% used for TeXing text within eps files
%\usepackage{psfrag}
% need this for including graphics (\includegraphics)
%\usepackage{graphicx}
% for neatly defining theorems and propositions
%
% making logically defined graphics
%%%\usepackage{xypic}

% there are many more packages, add them here as you need them

% define commands here

\newcommand{\sR}[0]{\mathbb{R}}
\newcommand{\sC}[0]{\mathbb{C}}
\newcommand{\sN}[0]{\mathbb{N}}
\newcommand{\sZ}[0]{\mathbb{Z}}

 \usepackage{bbm}
 \newcommand{\Z}{\mathbbmss{Z}}
 \newcommand{\C}{\mathbbmss{C}}
 \newcommand{\R}{\mathbbmss{R}}
 \newcommand{\Q}{\mathbbmss{Q}}



\newcommand*{\norm}[1]{\lVert #1 \rVert}
\newcommand*{\abs}[1]{| #1 |}



\newtheorem{thm}{Theorem}
\newtheorem{defn}{Definition}
\newtheorem{prop}{Proposition}
\newtheorem{lemma}{Lemma}
\newtheorem{cor}{Corollary}
\begin{document}
The idea in the calculus of variations is to study 
stationary points of functionals. 
To derive a differential equation for such stationary
points, the following theorem is needed, and hence
named thereafter. It is also used in distribution theory
to recover traditional calculus from distributional calculus. 

\begin{thm}
Suppose $f\colon U\to \C$ is a locally integrable function on an
open subset $U\subset \sR^n$,
and suppose that
$$ 
  \int_{U} f \phi dx =0
$$
for all smooth functions with compact support $\phi\in C_0^\infty(U)$.
Then $f=0$ almost everywhere.
\end{thm}

By linearity of the integral, it is easy to see that one only needs to
prove the claim for real $f$. If $f$ is continuous, this can be seen 
by purely geometrical arguments. A full proof
based on the Lebesgue differentiation theorem is given
in \cite{hormander}. Another proof is given in \cite{lang}.

\begin{thebibliography}{9}
\bibitem{hormander}
L. H\"ormander, \emph{The Analysis of Linear Partial Differential Operators I,
(Distribution theory and Fourier Analysis)}, 2nd ed, Springer-Verlag, 1990.
\bibitem{lang}
S. Lang, \emph{Analysis II},
Addison-Wesley Publishing Company Inc., 1969.
\end{thebibliography}
%%%%%
%%%%%
\end{document}
