\documentclass[12pt]{article}
\usepackage{pmmeta}
\pmcanonicalname{CentreOfMass}
\pmcreated{2013-03-22 15:28:09}
\pmmodified{2013-03-22 15:28:09}
\pmowner{stevecheng}{10074}
\pmmodifier{stevecheng}{10074}
\pmtitle{centre of mass}
\pmrecord{12}{37322}
\pmprivacy{1}
\pmauthor{stevecheng}{10074}
\pmtype{Definition}
\pmcomment{trigger rebuild}
\pmclassification{msc}{28A75}
\pmclassification{msc}{26B15}
\pmsynonym{center of mass}{CentreOfMass}
\pmrelated{PappussTheoremForSurfacesOfRevolution}
\pmdefines{centre of gravity}
\pmdefines{centroid}
\pmdefines{center of gravity}

\endmetadata

\usepackage{amssymb}
\usepackage{amsmath}
\usepackage{amsfonts}
%\usepackage{amsthm}
\usepackage{enumerate}

% used for TeXing text within eps files
%\usepackage{psfrag}
% need this for including graphics (\includegraphics)
%\usepackage{graphicx}
% making logically defined graphics
%%%\usepackage{xypic}

% define commands here
\newcommand{\complex}{\mathbb{C}}
\newcommand{\real}{\mathbb{R}}
\newcommand{\rat}{\mathbb{Q}}
\newcommand{\nat}{\mathbb{N}}

\providecommand{\abs}[1]{\lvert#1\rvert}
\providecommand{\absW}[1]{\left\lvert#1\right\rvert}
\providecommand{\absB}[1]{\Bigl\lvert#1\Bigr\rvert}
\providecommand{\norm}[1]{\lVert#1\rVert}
\providecommand{\normW}[1]{\left\lVert#1\right\rVert}
\providecommand{\normB}[1]{\Bigl\lVert#1\Bigr\rVert}
\providecommand{\defnterm}[1]{\emph{#1}}

\newcommand{\vr}{\mathbf{r}}
\newcommand{\vmu}{\boldsymbol{\mu}}

\DeclareMathOperator{\D}{D}
\DeclareMathOperator{\linspan}{span}
\begin{document}
Let $\nu$ be a Borel signed measure for some set $E \subseteq \real^n$.
The \defnterm{centre of mass} of $E$
(with respect to $\nu$) is a vector in $\real^n$ defined by the following vector-valued integral:
\[
\vmu(E; \nu) = \frac{1}{\nu(E)} \int_{\vr \in E} \vr \, d\nu\,, \quad \nu(E) \neq 0, \pm \infty\,,
\]
provided the integral exists. Intuitively, the integral is a weighted average
of all the points of $E$.

Our abstract definition encompasses many situations.
If $E$ is a $k$-dimensional manifold, then we may take $\nu$ to be its 
absolute $k$-dimensional volume element ($d\nu = dV$).
This would give the set $E$ a unit mass density,
and in this case the centre of mass of $E$
is also called the \defnterm{centroid} of $E$.  

More generally, if we are given a measurable mass density $\rho\colon E \to \real$,
then taking $d\nu = \rho dV$, the definition defines the centre of mass for $E$
with a mass density $\rho$.
We do not restrict the density $\rho$ (or the measure $\nu$) to be non-negative;
for this allows our definition to apply even to, for example, electrical charge densities.

If $E$ is not a differentiable manifold, but is a rectifiable set, we can replace 
$dV$ by the appropriate Hausdorff measure.

The measure $\nu$ could also be discrete.  For example, $E$ could be a finite set of points $\{ x_i \} \subset \real^n$ with masses $m_i = \nu(x_i)$.  In this case, the integral $\vmu(E; \nu)$ 
reduces to a finite summation.

It is possible to economize the definition so that the set $E$ does not have to be mentioned,
although the result is somewhat unintuitive:
the \emph{centre of mass of a finite signed Borel measure} $\nu$ on $\real^n$
is defined by
\[
\vmu(\nu) = \vmu(\real^n; \nu) = \frac{1}{\nu(\real^n)} \int_{\vr \in \real^n} \vr \, d\nu
\]
(provided this exists).
If $\nu_E$ is defined by $\nu_E(S) = \nu(S \cap E)$ for all measurable $S$ and some fixed measurable $E$, then $\vmu(\nu_E) = \vmu(E ; \nu)$.

\medskip

The term \defnterm{centre of gravity} is sometimes
used loosely as a synonym for the centre of mass, but these are distinct concepts;
the centre of gravity is supposed to be a single point on which the force of gravity can
be considered to act upon. If $E$ is three-dimensional and the force of gravity is uniform throughout $E$, then the centre of mass and the centre of gravity coincide.

As shown by the above examples, the centre of mass and its related concepts 
have applications in mechanics and other areas of physics.

\section*{Symmetry principle}
If $T \colon \real^n \to \real^n$ is an invertible linear map, then by a straightforward change of variable
in the integrand, 
\[
\vmu(TE; \nu \circ T^{-1}) = T(\vmu (E; \nu) )\,.
\]
In particular, take $T$ to be an isometry, and $\nu$ to be a $k$-dimensional volume measure for $E$.
The corresponding $k$-dimensional volume measure for $TE$
must be the same as $\nu \circ T^{-1}$, because $T$ is an isometry.
Then the above equation says that 
the centroid transforms as expected under isometries. 

For example, it is intuitively obvious that the centre of mass of a disk $D \subset \real^2$ should be the centre of $D$.  (Without loss of generality, assume that $D$ is centered at the origin.)
Using the property just mentioned, this is easy to prove rigorously too:
if $T$ is the isometry that reflects across the y-axis, and $\lambda$ is the Lebesgue measure on $\real^2$,
then 
\begin{align*}
T(\vmu (D; \lambda)) &= \vmu(TD; \lambda \circ T^{-1}) \\
&= \vmu(TD; \lambda) &\text{($\lambda$ is invariant under isometries)} \\
&= \vmu(D; \lambda)\,, & \text{($TD = D$ by symmetry of the disk)}
\end{align*}
which means the x component of $\vmu(D; \lambda)$
must be zero.  Similarly, by considering reflections across the x-axis, we conclude
that the y component of $\vmu(D; \lambda)$ must be zero.
%%%%%
%%%%%
\end{document}
