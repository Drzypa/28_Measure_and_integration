\documentclass[12pt]{article}
\usepackage{pmmeta}
\pmcanonicalname{AreaFormula}
\pmcreated{2013-03-22 14:27:36}
\pmmodified{2013-03-22 14:27:36}
\pmowner{paolini}{1187}
\pmmodifier{paolini}{1187}
\pmtitle{area formula}
\pmrecord{12}{35979}
\pmprivacy{1}
\pmauthor{paolini}{1187}
\pmtype{Theorem}
\pmcomment{trigger rebuild}
\pmclassification{msc}{28A78}
\pmrelated{ChangeOfVariablesInIntegralOnMathbbRn}

% this is the default PlanetMath preamble.  as your knowledge
% of TeX increases, you will probably want to edit this, but
% it should be fine as is for beginners.

% almost certainly you want these
\usepackage{amssymb}
\usepackage{amsmath}
\usepackage{amsfonts}

% used for TeXing text within eps files
%\usepackage{psfrag}
% need this for including graphics (\includegraphics)
%\usepackage{graphicx}
% for neatly defining theorems and propositions
\usepackage{amsthm}
% making logically defined graphics
%%%\usepackage{xypic}

% there are many more packages, add them here as you need them

% define commands here
\newcommand{\R}{\mathbb R}
\newcommand{\HH}{\mathcal H}
\newtheorem{theorem}{Theorem}
\newtheorem{definition}{Definition}
\theoremstyle{remark}
\newtheorem{example}{Example}
\begin{document}
Let $\HH^m$ denote the Hausdorff measure. Let $m\le n$ and consider a Lipschitz function $f\colon \R^m \to \R^n$.
If $A\subset \R^m$ is a Lebesgue measurable set, the equality
\[
  \int_A J_f(x) \,dx = \int_{\R^n} \HH^0(f^{-1}(\{y\})\cap A) \, d\HH^m y
\]
holds, where 
\[
J_f(x) = \sqrt{\det(Df(x)\cdot Df(x)^*)}
\]
is the Jacobian determinant of $f$ in the point $x$ and represent the $m$-volume of the image of the unit cube under the linear map $Df(x)$.

If $u\in L^1(\R^m)$ then one has
\[
  \int_{\R^m} u(x) J_f(x)\, dx = \int_{\R^n}\sum_{x\in f^{-1}(\{y\})} u(x)\, d\HH^m y.
\]

Notice that this formula is a generalization of the change of variables in integrals on $\R^n$.
%%%%%
%%%%%
\end{document}
