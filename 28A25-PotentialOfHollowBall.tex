\documentclass[12pt]{article}
\usepackage{pmmeta}
\pmcanonicalname{PotentialOfHollowBall}
\pmcreated{2013-03-22 17:16:46}
\pmmodified{2013-03-22 17:16:46}
\pmowner{pahio}{2872}
\pmmodifier{pahio}{2872}
\pmtitle{potential of hollow ball}
\pmrecord{8}{39619}
\pmprivacy{1}
\pmauthor{pahio}{2872}
\pmtype{Example}
\pmcomment{trigger rebuild}
\pmclassification{msc}{28A25}
\pmclassification{msc}{26B10}
\pmclassification{msc}{26B15}
\pmrelated{JacobiDeterminant}
\pmrelated{ChangeOfVariablesInIntegralOnMathbbRn}
\pmrelated{SubstitutionNotation}
\pmrelated{ModulusOfComplexNumber}

% this is the default PlanetMath preamble.  as your knowledge
% of TeX increases, you will probably want to edit this, but
% it should be fine as is for beginners.

% almost certainly you want these
\usepackage{amssymb}
\usepackage{amsmath}
\usepackage{amsfonts}

% used for TeXing text within eps files
%\usepackage{psfrag}
% need this for including graphics (\includegraphics)
%\usepackage{graphicx}
% for neatly defining theorems and propositions
%\usepackage{amsthm}
% making logically defined graphics
%%%\usepackage{xypic}

% there are many more packages, add them here as you need them

% define commands here
\newcommand{\sijoitus}[2]%
{\operatornamewithlimits{\Big/}_{\!\!\!#1}^{\,#2}}
\begin{document}
\PMlinkescapeword{mass}

Let\, $(\xi,\,\eta,\,\zeta)$\, be a point bearing a mass\, $m$\, and\, $(x,\,y,\,z)$\, a \PMlinkescapetext{variable} point.  If the distance of these points is $r$, we can define the {\em potential} of\, $(\xi,\,\eta,\,\zeta)$\, in\, $(x,\,y,\,z)$\, as
$$\frac{m}{r} = \frac{m}{\sqrt{(x-\xi)^2+(y-\eta)^2+(z-\zeta)^2}}.$$
The relevance of this concept appears from the fact that its partial derivatives
$$\frac{\partial}{\partial x}\!\left(\frac{m}{r}\right) = -\frac{m(x-\xi)}{r^3},\quad
\frac{\partial}{\partial y}\!\left(\frac{m}{r}\right) = -\frac{m(y-\eta)}{r^3},\quad
\frac{\partial}{\partial z}\!\left(\frac{m}{r}\right) = -\frac{m(z-\zeta)}{r^3}$$
are the components of the gravitational \PMlinkescapetext{force} with which the material point\, $(\xi,\,\eta,\,\zeta)$\, acts on one mass unit in the point\, $(x,\,y,\,z)$\, (provided that the \PMlinkescapetext{measure units} are chosen suitably).

The potential of a set of points\, $(\xi,\,\eta,\,\zeta)$\, is the sum of the potentials of individual points, i.e. it may lead to an integral.\\

We determine the potential of all points\, $(\xi,\,\eta,\,\zeta)$\, of a hollow ball, where the matter is located between two concentric spheres with radii $R_0$ and $R\, (>R_0)$.  Here the \PMlinkescapetext{density} of mass is assumed to be presented by a continuous function \, $\varrho = \varrho(r)$\, at the distance $r$ from the centre $O$.  Let $a$ be the distance from $O$ of the point $A$, where the potential is to be determined.  We chose $O$ the origin and the ray $OA$ the positive $z$-axis.

For obtaining the potential in $A$ we must integrate over the ball shell where $R_0 \le r \le R$.  We use the spherical coordinates $r$, $\varphi$ and $\psi$ which are tied to the Cartesian coordinates via
$$x = r\cos\varphi\cos\psi,\quad y = r\cos\varphi\sin\psi,\quad z = r\sin\varphi;$$
for attaining all points we set
$$R_0 \le r \le R,\quad -\frac{\pi}{2} \le \varphi \le \frac{\pi}{2},\quad
0 \le \psi < 2\pi.$$
The cosines law implies that\, $PA = \sqrt{r^2-2ar\sin\varphi+a^2}$.  Thus the potential is the triple integral
\begin{align}
V(a) = 
\int_{R_0}^R \int_{-\frac{\pi}{2}}^\frac{\pi}{2} \int_0^{2\pi}\!
\!\frac{\varrho(r)\,r^2\cos\varphi}{\sqrt{r^2-2ar\sin\varphi+a^2}}\,dr\,d\varphi\,d\psi
= 2\pi\int_{R_0}^R \varrho(r)\,r\,dr\int_{-\frac{\pi}{2}}^\frac{\pi}{2}
\frac{r\cos\varphi\,d\varphi}{\sqrt{r^2-2ar\sin\varphi+a^2}},
\end{align}
where the factor\, $r^2\cos\varphi$\, is the coefficient for the coordinate changing
$$\left|\frac{\partial(x,\,y,\,z)}{\partial(r,\,\varphi,\,\psi)}\right| =
\!\mod\!\left|\begin{matrix}
\cos\varphi\cos\psi & \cos\varphi\sin\psi & \sin\varphi \\
-r\sin\varphi\cos\psi & -r\sin\varphi\sin\psi & r\cos\varphi \\
-r\cos\varphi\sin\psi & r\cos\varphi\cos\psi & 0
\end{matrix}\right|.$$

We get from the latter integral
\begin{align}
\int_{-\frac{\pi}{2}}^\frac{\pi}{2}
\frac{r\cos\varphi\,d\varphi}{\sqrt{r^2-2ar\sin\varphi+a^2}}
= -\frac{1}{a}\sijoitus{\varphi=-\frac{\pi}{2}}{\quad\frac{\pi}{2}}\sqrt{r^2-2ar\sin\varphi+a^2}
= \frac{1}{a}[(r+a)-|r-a|].
\end{align}
Accordingly we have the two cases:

$1^\circ$.\, The point $A$ is outwards the hollow ball, i.e. $a > R$.\, Then we have\, $|r-a| = a-r$\, for all\, 
$r\in[R_0,\,R]$.\, The value of the integral (2) is $\frac{2r}{a}$, and (1) gets the form
$$V(a) = \frac{4\pi}{a}\int_{R_0}^R \varrho(r)\,r^2\,dr = \frac{M}{a},$$
where $M$ is the mass of the hollow ball.  Thus {\em the potential outwards the hollow ball is exactly the same as in the case that all mass were concentrated to the centre}.  A correspondent statement concerns the attractive \PMlinkescapetext{force}
                 $$V'(a) = -\frac{M}{a^2}.$$


$2^\circ$.\, The point $A$ is in the cavity of the hollow ball, i.e. $a < R_0$ .\, Then\, $|r-a| = r-a$\, on the interval of integration of (2).  The value of (2) is equal to 2, and (1) yields
           $$V(a) = 4\pi\int_{R_0}^R \varrho(r)\,r\,dr,$$
which is \PMlinkescapetext{independent} on $a$.  That is, {\em the potential of the hollow ball, when the \PMlinkescapetext{density} of mass depends only on the distance from the centre, has in the cavity a constant value, and the hollow ball influences in no way on a mass inside it}.

\begin{thebibliography}{8}
\bibitem{lindelof}{\sc Ernst Lindel\"of}: {\em Differentiali- ja integralilasku
ja sen sovellutukset II}.\, Mercatorin Kirjapaino Osakeyhti\"o, Helsinki (1932).
\end{thebibliography} 


%%%%%
%%%%%
\end{document}
