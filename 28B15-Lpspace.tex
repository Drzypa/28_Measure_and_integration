\documentclass[12pt]{article}
\usepackage{pmmeta}
\pmcanonicalname{Lpspace}
\pmcreated{2013-03-22 12:21:32}
\pmmodified{2013-03-22 12:21:32}
\pmowner{Mathprof}{13753}
\pmmodifier{Mathprof}{13753}
\pmtitle{$L^p$-space}
\pmrecord{28}{32047}
\pmprivacy{1}
\pmauthor{Mathprof}{13753}
\pmtype{Definition}
\pmcomment{trigger rebuild}
\pmclassification{msc}{28B15}
\pmsynonym{$L^p$ space}{Lpspace}
\pmsynonym{essentially bounded function}{Lpspace}
\pmrelated{MeasureSpace}
\pmrelated{Norm}
\pmrelated{EssentialSupremum}
\pmrelated{Measure}
\pmrelated{FeynmannPathIntegral}
\pmrelated{AmenableGroup}
\pmrelated{VectorPnorm}
\pmrelated{VectorNorm}
\pmrelated{SobolevInequality}
\pmrelated{L2SpacesAreHilbertSpaces}
\pmrelated{RieszFischerTheorem}
\pmrelated{BoundedLinearFunctionalsOnLpmu}
\pmrelated{ConvolutionsOfComplexFunctionsOnLocallyCompactG}
\pmdefines{$p$-integrable function}
\pmdefines{$L^\infty$}
\pmdefines{essentially bounded}
\pmdefines{$L^p$-norm}

\usepackage{amssymb}
\usepackage{amsmath}
\usepackage{amsfonts}

% used for TeXing text within eps files
%\usepackage{psfrag}
% need this for including graphics (\includegraphics)
%\usepackage{graphicx}
% for neatly defining theorems and propositions
%\usepackage{amsthm}
% making logically defined graphics
%%%\usepackage{xypic} 

% there are many more packages, add them here as you need them

% define commands here
\newcommand{\md}{d}
\newcommand{\mv}[1]{\mathbf{#1}}	% matrix or vector
\newcommand{\mvt}[1]{\mv{#1}^{\mathrm{T}}}
\newcommand{\mvi}[1]{\mv{#1}^{-1}}
\newcommand{\mderiv}[1]{\frac{\md}{\md {#1}}} %d/dx
\newcommand{\mnthderiv}[2]{\frac{\md^{#2}}{\md {#1}^{#2}}} %d^n/dx
\newcommand{\mpderiv}[1]{\frac{\partial}{\partial {#1}}} %partial d^n/dx
\newcommand{\mnthpderiv}[2]{\frac{\partial^{#2}}{\partial {#1}^{#2}}} %partial d^n/dx
\newcommand{\borel}{\mathfrak{B}}
\newcommand{\integers}{\mathbb{Z}}
\newcommand{\rationals}{\mathbb{Q}}
\newcommand{\reals}{\mathbb{R}}
\newcommand{\complexes}{\mathbb{C}}
\newcommand{\naturals}{\mathbb{N}}
\newcommand{\defined}{:=}
\newcommand{\var}{\mathrm{var}}
\newcommand{\cov}{\mathrm{cov}}
\newcommand{\corr}{\mathrm{corr}}
\newcommand{\set}[1]{\left\{#1\right\}}
\newcommand{\powerset}[1]{\mathcal{P}(#1)}
\newcommand{\bra}[1]{\langle#1 \vert}
\newcommand{\ket}[1]{\vert \hspace{1pt}#1\rangle}
\newcommand{\braket}[2]{\langle #1 \ket{#2}}
\newcommand{\abs}[1]{\left|#1\right|}
\newcommand{\norm}[1]{\left|\left|#1\right|\right|}
\newcommand{\esssup}{\mathrm{ess\ sup}}
\newcommand{\Lspace}[1]{L^{#1}}
\newcommand{\Lone}{\Lspace{1}}
\newcommand{\Ltwo}{\Lspace{2}}
\newcommand{\Lp}{\Lspace{p}}
\newcommand{\Lq}{\Lspace{q}}
\newcommand{\Linf}{\Lspace{\infty}} 

%hope this works.......................
\begin{document}
\paragraph{Definition}
Let $(X, \borel, \mu)$ be a measure space. Let $0<p < \infty$.  The \emph{$\Lp$-norm} of a function $f:X\rightarrow \complexes$ is defined as
\begin{equation}
\norm{f}_{p} \defined \left(\int_{X}  \abs{f}^p d\mu \right)^{\frac{1}{p}}
\end{equation}
when the integral exists.  The set of functions with finite $\Lp$-norm forms a vector space $V$ with the usual pointwise addition and scalar 
multiplication of functions.  In particular, the set of functions with zero $\Lp$-norm form a linear subspace of $V$, which for this article 
will be called $K$.  We are then interested in the quotient space $V/K$, which consists of complex functions on $X$ with finite $\Lp$-norm, 
identified up to equivalence almost everywhere.  This quotient space is the complex $\Lp$-space on $X$.

\paragraph{Theorem} If $1 \leq p < \infty$, the vector space $V/K$ is complete with respect to the $\Lp$ norm.

\paragraph{The space $\Linf$.}
The space $\Linf$ is somewhat special, and may be defined without explicit reference to an integral.  First, the $\Linf$-norm of $f$ is 
defined to be the essential supremum of $\abs{f}$:
\begin{equation}
\norm{f}_{\infty} \defined \esssup \abs{f} = \inf \set{a \in \reals: \mu(\set{x: \abs{f(x)} > a}) = 0}
\end{equation}
However, if $\mu$ is the trivial measure, then essential supremum of every measurable function
is defined to be 0.

The definitions of $V$, $K$, and $\Linf$ then proceed as above, and again we have that $L^\infty$ is complete.  Functions in $\Linf$ are also called {\em essentially bounded}.
%\paragraph{Theorem 3} The $\Linf$-norm is the limit of $\Lp$ norms.
%proof to follow.

\paragraph{Example}
Let $X = [0,1]$ and $f(x) = \frac{1}{\sqrt{x}}$.  Then $f \in \Lone(X)$ but $f \notin \Ltwo(X)$.

%Something deletes final chars in these files on submit so I'm adding a line of useless information...........................
%%%%%
%%%%%
\end{document}
