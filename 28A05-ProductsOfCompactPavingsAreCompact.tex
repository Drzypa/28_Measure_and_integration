\documentclass[12pt]{article}
\usepackage{pmmeta}
\pmcanonicalname{ProductsOfCompactPavingsAreCompact}
\pmcreated{2013-03-22 18:45:09}
\pmmodified{2013-03-22 18:45:09}
\pmowner{gel}{22282}
\pmmodifier{gel}{22282}
\pmtitle{products of compact pavings are compact}
\pmrecord{5}{41530}
\pmprivacy{1}
\pmauthor{gel}{22282}
\pmtype{Theorem}
\pmcomment{trigger rebuild}
\pmclassification{msc}{28A05}
%\pmkeywords{compact paving}
%\pmkeywords{generalized Cartesian product}
\pmrelated{SumsOfCompactPavingsAreCompact}

% almost certainly you want these
\usepackage{amssymb}
\usepackage{amsmath}
\usepackage{amsfonts}

% used for TeXing text within eps files
%\usepackage{psfrag}
% need this for including graphics (\includegraphics)
%\usepackage{graphicx}
% for neatly defining theorems and propositions
\usepackage{amsthm}
% making logically defined graphics
%%%\usepackage{xypic}

% there are many more packages, add them here as you need them

% define commands here
\newtheorem*{theorem*}{Theorem}
\newtheorem*{lemma*}{Lemma}
\newtheorem*{corollary*}{Corollary}
\newtheorem*{definition*}{Definition}
\newtheorem{theorem}{Theorem}
\newtheorem{lemma}{Lemma}
\newtheorem{corollary}{Corollary}
\newtheorem{definition}{Definition}

\begin{document}
\PMlinkescapeword{product}
\PMlinkescapeword{functions}
\PMlinkescapeword{index set}
\PMlinkescapeword{subsets}
\PMlinkescapeword{product paving}
\PMlinkescapeword{compact}
\PMlinkescapeword{compact paving}
\PMlinkescapeword{theorem}
\PMlinkescapeword{equivalent}
\PMlinkescapeword{subset}
\PMlinkescapeword{finite}
\PMlinkescapeword{intersection}
\PMlinkescapeword{compactness}
\PMlinkescapeword{equation}
\PMlinkescapeword{satisfies}
\PMlinkescapeword{simple}

Suppose that $(K_i,\mathcal{K}_i)$ is a paved space for each $i$ in index set $I$. The \PMlinkname{product}{GeneralizedCartesianProduct} $\prod_{i\in I} K_i$ is the set of all functions $x\colon I\rightarrow\bigcup_iK_i$ such that $x_i\in K_i$ for each $i$, and the product of subsets $S_i\subseteq K_i$ is
\begin{equation*}
\prod_{i\in I}S_i=\left\{x\in\prod_{i\in I}K_i\colon x_i\in S_i\text{ for each }i\in I\right\}.
\end{equation*}
Then, the product paving is defined by
\begin{equation*}
\prod_{i\in I} \mathcal{K}_i=\left\{\prod_{i\in I}S_i\colon S_i\in\mathcal{K}_i\text{ for each }i\in I\right\}.
\end{equation*} 

\begin{theorem}\label{thm:1}
Let $(K_i,\mathcal{K}_i)$ be compact paved spaces for $i\in I$. Then, $\prod_i\mathcal{K}_i$ is a compact paving on $\prod_iK_i$.
\end{theorem}

Note that this result is a version of Tychonoff's theorem applying to paved spaces and, together with the fact that all compact pavings are closed subsets of a compact space, is easily seen to be equivalent to Tychonoff's theorem.

Theorem \ref{thm:1} is simple to prove directly. Suppose that $\{A_j\colon j\in J\}$ is a subset of $\prod_i \mathcal{K}_i$ satisfying the finite intersection property. Writing $A_j=\prod_{i\in I}S_{ij}$ for $S_{ij}\in\mathcal{K}_i$ gives
\begin{equation}\label{eq:1}
\bigcap_{j\in J^\prime}A_j=\prod_{i\in I}\left(\bigcap_{j\in J^\prime}S_{ij}\right)
\end{equation}
for any $J^\prime\subseteq J$. By the finite intersection property, this is nonempty whenever $J^\prime$ is finite, so $\bigcap_{j\in J^\prime}S_{ij}\not=\emptyset$. Consequently, $\{S_{ij}\colon j\in J\}\subseteq\mathcal{K}_i$ satisfies the finite intersection property and, by compactness of $\mathcal{K}_i$, the intersection $\bigcap_{j\in J}S_{ij}$ is nonempty. So equation (\ref{eq:1}) shows that $\bigcap_{j\in J}A_j$ is nonempty.

%%%%%
%%%%%
\end{document}
