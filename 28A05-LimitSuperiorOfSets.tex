\documentclass[12pt]{article}
\usepackage{pmmeta}
\pmcanonicalname{LimitSuperiorOfSets}
\pmcreated{2013-03-22 13:13:22}
\pmmodified{2013-03-22 13:13:22}
\pmowner{Koro}{127}
\pmmodifier{Koro}{127}
\pmtitle{limit superior of sets}
\pmrecord{8}{33689}
\pmprivacy{1}
\pmauthor{Koro}{127}
\pmtype{Definition}
\pmcomment{trigger rebuild}
\pmclassification{msc}{28A05}
\pmclassification{msc}{60A99}
\pmdefines{limit inferior of sets}
\pmdefines{infinitely often}
\pmdefines{i.o.}

\endmetadata

% this is the default PlanetMath preamble.  as your knowledge
% of TeX increases, you will probably want to edit this, but
% it should be fine as is for beginners.

% almost certainly you want these
\usepackage{amssymb}
\usepackage{amsmath}
\usepackage{amsfonts}

% used for TeXing text within eps files
%\usepackage{psfrag}
% need this for including graphics (\includegraphics)
%\usepackage{graphicx}
% for neatly defining theorems and propositions
%\usepackage{amsthm}
% making logically defined graphics
%%%\usepackage{xypic}

% there are many more packages, add them here as you need them

% define commands here
\begin{document}
Let $A_1,A_2,\dots$ be a sequence of sets. 
The limit superior of sets is defined by
\[\limsup A_n = \bigcap_{n=1}^\infty \bigcup_{k=n}^\infty A_k.\]

It is easy to see that $x\in \limsup A_n$ if and only if $x\in A_n$ for infinitely many values of $n$.
Because of this, in probability theory the notation $[A_n \operatorname{i.o.}]$ is often used to refer to $\limsup A_n$, where i.o. stands for \textit{infinitely often}.

The limit inferior of sets is defined by

\[\liminf A_n = \bigcup_{n=1}^\infty \bigcap_{k=n}^\infty A_k,\]

and it can be shown that $x\in \liminf A_n$ if and only if $x$ belongs to $A_n$ for all but finitely many values of $n$.
%%%%%
%%%%%
\end{document}
