\documentclass[12pt]{article}
\usepackage{pmmeta}
\pmcanonicalname{UniversallyMeasurable}
\pmcreated{2013-03-22 18:36:58}
\pmmodified{2013-03-22 18:36:58}
\pmowner{gel}{22282}
\pmmodifier{gel}{22282}
\pmtitle{universally measurable}
\pmrecord{8}{41350}
\pmprivacy{1}
\pmauthor{gel}{22282}
\pmtype{Definition}
\pmcomment{trigger rebuild}
\pmclassification{msc}{28A20}
\pmclassification{msc}{28A05}
%\pmkeywords{$\sigma$-algebra}
%\pmkeywords{measurable}
\pmrelated{CompleteMeasure}
\pmrelated{AnalyticSet2}
\pmdefines{universal completion}
\pmdefines{universally complete}

% almost certainly you want these
\usepackage{amssymb}
\usepackage{amsmath}
\usepackage{amsfonts}

% used for TeXing text within eps files
%\usepackage{psfrag}
% need this for including graphics (\includegraphics)
%\usepackage{graphicx}
% for neatly defining theorems and propositions
\usepackage{amsthm}
% making logically defined graphics
%%%\usepackage{xypic}

% there are many more packages, add them here as you need them

% define commands here
\newtheorem*{theorem*}{Theorem}
\newtheorem*{lemma*}{Lemma}
\newtheorem*{corollary*}{Corollary}
\newtheorem*{definition*}{Definition}
\newtheorem{theorem}{Theorem}
\newtheorem{lemma}{Lemma}
\newtheorem{corollary}{Corollary}
\newtheorem{definition}{Definition}

\begin{document}
\PMlinkescapeword{finite}
\PMlinkescapeword{theorem}
\PMlinkescapeword{theory}
\PMlinkescapeword{equivalent}
\PMlinkescapeword{term}
\PMlinkescapeword{definitions}
\PMlinkescapeword{ranges}
\PMlinkescapeword{contains}
\PMlinkescapeword{even}
\PMlinkescapeword{weaker}
\PMlinkescapeword{axiom}
\PMlinkescapeword{measurable set}
Given a measurable space $(\Omega,\mathcal{F})$, a subset $S\subseteq\Omega$ is said to be \emph{universally measurable} if it lies in the \PMlinkname{completion}{CompleteMeasure} of the $\sigma$-algebra $\mathcal{F}$ with respect to every finite measure $\mu$ on $(\Omega,\mathcal{F})$.
That is, for every such $\mu$, there exist $A,B\in\mathcal{F}$ such that $A\subseteq S\subseteq B$ and $\mu(B\setminus A)=0$.

If, for any finite measure $\mu$ we denote the completion of $\mathcal{F}$ by $\mathcal{F}^*_\mu$ then, the collection of universally measurable sets is
\begin{equation*}
\mathcal{F}^*=\bigcap_\mu\mathcal{F}^*_\mu
\end{equation*}
where $\mu$ ranges over the finite measures on $(\Omega,\mathcal{F})$.
Being an intersection of $\sigma$-algebras, the collection $\mathcal{F}^*$ of universally measurable sets is itself a $\sigma$-algebra, and is called the \emph{universal completion} of $\mathcal{F}$.
The $\sigma$-algebra $\mathcal{F}$ is called \emph{universally complete} if it contains every set which is universally measurable with respect to $\mathcal{F}$. That is, if $\mathcal{F}^*=\mathcal{F}$. The universal completion of a $\sigma$-algebra is itself universally complete.

A function $f\colon A\rightarrow B$ between measurable spaces $(A,\mathcal{A})$ and $(B,\mathcal{B})$ is said to be universally measurable if it is $\mathcal{A}^*/\mathcal{B}^*$-measurable, so that $f^{-1}(S)\in\mathcal{A}^*$ for all $S\in\mathcal{B}^*$. In fact, this is equivalent to requiring that $f$ be $\mathcal{A}^*/\mathcal{B}$-measurable so, in particular, every measurable function is universally measurable.

As \PMlinkname{any $\sigma$-finite measure is equivalent to a probability measure}{AnySigmaFiniteMeasureIsEquivalentToAProbabilityMeasure}, the term ``finite measure'' may be replaced by ``$\sigma$-finite measure'' in the definitions above.
In particular, every $\sigma$-finite measure $\mu$ on $(\Omega,\mathcal{F})$ can be uniquely extended to the universal completion of $\mathcal{F}$.

A subset of a topological space $X$ is said to be universally measurable if it is universally measurable with respect to the Borel $\sigma$-algebra on $X$. It can be shown, for example, that the image of any continuous map between Polish spaces is universally measurable.

An important example of the use of universally measurable functions comes from the theory of continuous-time stochastic processes. The first time that a continuous process hits a given value is a universally measurable time, as stated by the d\'ebut theorem. However, it is not always measurable.

It is well known that the axiom of choice is required to construct non-Lebesgue measurable subsets of $\mathbb{R}^n$ and, it can be shown that this statement extends to all universally measurable subsets of $\mathbb{R}^n$. That is, the Zermelo-Fraenkel axioms of set  theory, without the axiom of choice, are consistent with the statement that all subsets of $\mathbb{R}^n$ are universally measurable. This remains true even if we allow a weaker version of the AOC, in the form of the axiom of dependent choice.

As every universally measurable subset of the real numbers is Lebesgue measurable, the proof of Vitali's Theorem gives an example of the use of the axiom of choice to construct a non-universally measurable set. It is also possible to construct \PMlinkname{universally but non-Borel measurable sets}{ALebesgueMeasurableButNonBorelSet} without using the axiom of choice.
%%%%%
%%%%%
\end{document}
