\documentclass[12pt]{article}
\usepackage{pmmeta}
\pmcanonicalname{ExtendingACapacityToACartesianProduct}
\pmcreated{2013-03-22 18:47:38}
\pmmodified{2013-03-22 18:47:38}
\pmowner{gel}{22282}
\pmmodifier{gel}{22282}
\pmtitle{extending a capacity to a Cartesian product}
\pmrecord{6}{41592}
\pmprivacy{1}
\pmauthor{gel}{22282}
\pmtype{Theorem}
\pmcomment{trigger rebuild}
\pmclassification{msc}{28A12}
\pmclassification{msc}{28A05}
%\pmkeywords{capacity}
%\pmkeywords{compact paved space}

\endmetadata

% almost certainly you want these
\usepackage{amssymb}
\usepackage{amsmath}
\usepackage{amsfonts}

% used for TeXing text within eps files
%\usepackage{psfrag}
% need this for including graphics (\includegraphics)
%\usepackage{graphicx}
% for neatly defining theorems and propositions
\usepackage{amsthm}
% making logically defined graphics
%%%\usepackage{xypic}

% there are many more packages, add them here as you need them

% define commands here
\newtheorem*{theorem*}{Theorem}
\newtheorem*{lemma*}{Lemma}
\newtheorem*{corollary*}{Corollary}
\newtheorem*{definition*}{Definition}
\newtheorem{theorem}{Theorem}
\newtheorem{lemma}{Lemma}
\newtheorem{corollary}{Corollary}
\newtheorem{definition}{Definition}

\begin{document}
\PMlinkescapeword{set function}
\PMlinkescapeword{subset}
\PMlinkescapeword{onto}
\PMlinkescapeword{finite unions}
\PMlinkescapeword{compact}
\PMlinkescapeword{paving}
\PMlinkescapeword{projections}
\PMlinkescapeword{property}
\PMlinkescapeword{set functions}
\PMlinkescapeword{closure}
\PMlinkescapeword{finite}
\PMlinkescapeword{consequence}

A capacity on a set $X$ can be extended to a set function on a Cartesian product $X\times K$ simply by projecting any subset onto $X$, and then applying the original capacity.

\begin{theorem*}
Suppose that $(X,\mathcal{F})$ is a paved space such that $\mathcal{F}$ is closed under finite unions and finite intersections, and that $(K,\mathcal{K})$ is a compact paved space.
Define $\mathcal{G}$ to be the closure under finite unions and finite intersections of the paving $\mathcal{F}\times\mathcal{K}$ on $X\times K$.

If $I$ is an $\mathcal{F}$-capacity and $\pi_X\colon X\times K\to X$ is the projection map, we can form the composition
\begin{align*}
&I\circ\pi_X\colon\mathcal{P}(X\times K)\to\mathbb{R},\\
&I\circ\pi_X(S) = I(\pi_X(S)).
\end{align*}
Then $\pi_X(S)\in \mathcal{F}_\delta$ for any $S\in\mathcal{G}_\delta$, and $I\circ\pi_X$ is a $\mathcal{G}$-capacity.
\end{theorem*}

This result justifies looking at capacities when considering projections from the Cartesian product $X\times K$ onto $X$. We see that the property of being a capacity is preserved under composing with such projections. However, additivity of set functions is not preserved, so the corresponding result would not be true if  ``capacity'' was replaced by ``measure'' or ``outer measure''.

Recall that if $S\subseteq X\times K$ is $(\mathcal{G},I\circ\pi_X)$-capacitable then, for any $\epsilon>0$, there is an $A\in\mathcal{G}_\delta$ such that $A\subseteq S$ and $I\circ\pi_X(A)>I\circ\pi_X(S)-\epsilon$. However, $\pi_X(A)\subseteq\pi_X(S)$ and, by the above theorem, $\pi_X(A)\in\mathcal{F}_\delta$. This has the following consequence.
\begin{lemma*}
Let $S\subseteq X\times K$ be $(\mathcal{G},I\circ\pi_X)$-capacitable. Then, $\pi_X(S)$ is $(\mathcal{F},I)$-capacitable.
\end{lemma*}

%%%%%
%%%%%
\end{document}
