\documentclass[12pt]{article}
\usepackage{pmmeta}
\pmcanonicalname{ProofOfFunctionalMonotoneClassTheorem}
\pmcreated{2013-03-22 18:38:44}
\pmmodified{2013-03-22 18:38:44}
\pmowner{gel}{22282}
\pmmodifier{gel}{22282}
\pmtitle{proof of functional monotone class theorem}
\pmrecord{4}{41387}
\pmprivacy{1}
\pmauthor{gel}{22282}
\pmtype{Proof}
\pmcomment{trigger rebuild}
\pmclassification{msc}{28A20}
%\pmkeywords{measurable function}
%\pmkeywords{$\pi$-system}
%\pmkeywords{Dynkin system}

\endmetadata

% almost certainly you want these
\usepackage{amssymb}
\usepackage{amsmath}
\usepackage{amsfonts}

% used for TeXing text within eps files
%\usepackage{psfrag}
% need this for including graphics (\includegraphics)
%\usepackage{graphicx}
% for neatly defining theorems and propositions
\usepackage{amsthm}
% making logically defined graphics
%%%\usepackage{xypic}

% there are many more packages, add them here as you need them

% define commands here
\newtheorem*{theorem*}{Theorem}
\newtheorem*{lemma*}{Lemma}
\newtheorem*{corollary*}{Corollary}
\newtheorem*{definition*}{Definition}
\newtheorem{theorem}{Theorem}
\newtheorem{lemma}{Lemma}
\newtheorem{corollary}{Corollary}
\newtheorem{definition}{Definition}

\begin{document}
\PMlinkescapeword{monotone class theorem}
\PMlinkescapeword{theory}
\PMlinkescapeword{simple}
\PMlinkescapeword{generating}
\PMlinkescapeword{generates}
\PMlinkescapeword{generate}
\PMlinkescapeword{application}
\PMlinkescapeword{states}
\PMlinkescapeword{closed}
\PMlinkescapeword{generated by}
\PMlinkescapeword{theorem}
\PMlinkescapeword{finite}
\PMlinkescapeword{products}

We start by proving the following version of the monotone class theorem.

\begin{theorem}\label{thm:1}
Let $(X,\mathcal{A})$ be a measurable space and $\mathcal{S}$ be a \PMlinkname{$\pi$-system}{PiSystem} generating the \PMlinkname{$\sigma$-algebra}{SigmaAlgebra} $\mathcal{A}$.
Suppose that $\mathcal{H}$ be a vector space of real-valued functions on $X$ containing the constant functions and satisfying the following,
\begin{itemize}
\item if $f\colon X\rightarrow\mathbb{R}_+$ is bounded and there is a sequence of nonnegative functions $f_n\in \mathcal{H}$ increasing pointwise to $f$, then $f\in \mathcal{H}$.
\item for every set $A\in\mathcal{S}$ the characteristic function $1_A$ is in $\mathcal{H}$.
\end{itemize}
Then, $\mathcal{H}$ contains every bounded and measurable function from $X$ to $\mathbb{R}$.
\end{theorem}


Let $\mathcal{D}$ consist of the collection of subsets $B$ of $X$ such that the characteristic function $1_B$ is in $\mathcal{H}$. Then, by the conditions of the theorem, the constant function $1_X$ is in $V$ so that $X\in\mathcal{D}$, and $\mathcal{S}\subseteq\mathcal{D}$. For any $A\subseteq B$ in $\mathcal{D}$ then $1_{B\setminus A}=1_B-1_A\in \mathcal{H}$, as $\mathcal{H}$ is closed under linear combinations, and therefore $B\setminus A$ is in $\mathcal{D}$.
If $A_n\in\mathcal{D}$ is an increasing sequence, then $1_{A_n}\in \mathcal{H}$ increases pointwise to $1_{\bigcup_nA_n}$, which is therefore in $\mathcal{H}$, and $\bigcup_nA_n\in\mathcal{D}$. It follows that $\mathcal{D}$ is a Dynkin system, and Dynkin's lemma shows that it contains the $\sigma$-algebra $\mathcal{A}$.

We have shown that $1_A\in \mathcal{H}$ for every $A\in\mathcal{A}$. Now consider any bounded and measurable function $f\colon X\rightarrow\mathbb{R}$ taking values in a finite set $S\subseteq\mathbb{R}$. Then,
\begin{equation*}
f=\sum_{s\in S} s1_{f^{-1}(\{s\})}
\end{equation*}
is in $\mathcal{H}$.

We denote the floor function by $\lfloor\cdot\rfloor$. That is, $\lfloor a\rfloor$ is defined to be the largest integer less than or equal to the real number $a$.
Then, for any nonnegative bounded and measurable $f\colon X\rightarrow\mathbb{R}$, the sequence of functions $f_n(x)=2^{-n}\lfloor 2^n f(x)\rfloor$ each take values in a finite set, so are in $\mathcal{H}$, and increase pointwise to $f$. So, $f\in \mathcal{H}$.

Finally, as every measurable and bounded function $f\colon X\rightarrow\mathbb{R}$ can be written as the difference of its positive and negative parts $f=f_+-f_-$, then $f\in \mathcal{H}$.


We now extend this result to prove the following more general form of the theorem.

\begin{theorem}\label{thm:2}
Let $X$ be a set and $\mathcal{K}$ be a collection of bounded and real valued functions on $X$ which is closed under multiplication, so that $fg\in\mathcal{K}$ for all $f,g\in\mathcal{K}$. Let $\mathcal{A}$ be the $\sigma$-algebra on $X$ generated by $\mathcal{K}$.

Suppose that $\mathcal{H}$ is a vector space of bounded real valued functions on $X$ containing $\mathcal{K}$ and the constant functions, and satisfying the following
\begin{itemize}
\item if $f\colon X\rightarrow\mathbb{R}$ is bounded and there is a sequence of nonnegative functions $f_n\in \mathcal{H}$ increasing pointwise to $f$, then $f\in \mathcal{H}$.
\end{itemize}
Then, $\mathcal{H}$ contains every bounded and real valued $\mathcal{A}$-measurable function on $X$.
\end{theorem}

Let us start by showing that $\mathcal{H}$ is closed under uniform convergence. That is, if $f_n$ is a sequence in $\mathcal{H}$ and $\Vert f_n-f\Vert \equiv \sup_x|f_n(x)-f(x)|$ converges to zero, then $f\in\mathcal{H}$. By passing to a subsequence if necessary, we may assume that $\Vert f_n-f_m\Vert\le 2^{-n}$ for all $m\ge n$. Define $g_n\equiv f_n-2^{1-n}+2+\Vert f\Vert$. Then $g_n\in\mathcal{H}$ since $\mathcal{H}$ is a vector space containing the constant functions. Also, $g_n$ are nonnegative functions increasing pointwise to $f+2+\Vert f\Vert$ which must therefore be in $\mathcal{H}$, showing that $f\in\mathcal{H}$ as required.

Now let $\mathcal{H}_0$ consist of linear combinations of constant functions and functions in $\mathcal{K}$ and $\mathcal{\bar H}_0$ be its \PMlinkname{closure}{Closure} under uniform convergence. Then $\mathcal{\bar H}_0\subseteq\mathcal{H}$ since we have just shown that $\mathcal{H}$ is closed under uniform convergence.
As $\mathcal{K}$ is already closed under products, $\mathcal{H}_0$ and $\mathcal{\bar H}_0$ will also be closed under products, so are \PMlinkname{algebras}{Algebra}. In particular, $p(f)\in\mathcal{\bar H}_0$ for every $f\in\mathcal{\bar H}_0$ and polynomial $p\in\mathbb{R}[X]$.
Then, for any continuous function $p\colon\mathbb{R}\rightarrow\mathbb{R}$, the Weierstrass approximation theorem says that there is a sequence of polynomials $p_n$ converging uniformly to $p$ on bounded intervals, so $p_n(f)\rightarrow p(f)$ uniformly. It follows that $p(f)\in\mathcal{\bar H}_0$. In particular, the minimum of any two functions $f,g\in\mathcal{\bar H}_0$, $f\wedge g = f-|f-g|$ and the maximum $f\vee g=f+|g-f|$ will be in $\mathcal{\bar H}_0$.

We let $\mathcal{S}$ consist of the sets $A\subseteq X$ such that there is a sequence of nonnegative $f_n\in\mathcal{\bar H}_0$ increasing pointwise to $1_A$. Once it is shown that this is a $\pi$-system generating the $\sigma$-algebra $\mathcal{A}$, then the result will follow from theorem \ref{thm:1}.

If $f_n,g_n\in\mathcal{\bar H}_0$ are nonnegative functions increasing pointwise to $1_A,1_B$ then $f_ng_n$ increases pointwise to $1_{A\cap B}$, so $A\cap B\in\mathcal{S}$ and $\mathcal{S}$ is a $\pi$-system.

Finally, choose any $f\in\mathcal{K}$ and $a\in\mathbb{R}$. Then, $f_n=((n(f-a))\vee 0)\wedge 1$ is a sequence of functions in $\mathcal{\bar H}_0$ increasing pointwise to $1_{f^{-1}((a,\infty))}$. So, $f^{-1}((a,\infty))\in\mathcal{S}$. As intervals of the form $(a,\infty)$ generate the Borel $\sigma$-algebra on $\mathbb{R}$, it follows that $\mathcal{S}$ generates the $\sigma$-algebra $\mathcal{A}$, as required.

%%%%%
%%%%%
\end{document}
