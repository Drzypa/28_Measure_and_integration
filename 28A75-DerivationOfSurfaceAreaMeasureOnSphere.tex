\documentclass[12pt]{article}
\usepackage{pmmeta}
\pmcanonicalname{DerivationOfSurfaceAreaMeasureOnSphere}
\pmcreated{2013-03-22 14:57:55}
\pmmodified{2013-03-22 14:57:55}
\pmowner{rspuzio}{6075}
\pmmodifier{rspuzio}{6075}
\pmtitle{derivation of  surface area measure on sphere}
\pmrecord{6}{36664}
\pmprivacy{1}
\pmauthor{rspuzio}{6075}
\pmtype{Derivation}
\pmcomment{trigger rebuild}
\pmclassification{msc}{28A75}

% this is the default PlanetMath preamble.  as your knowledge
% of TeX increases, you will probably want to edit this, but
% it should be fine as is for beginners.

% almost certainly you want these
\usepackage{amssymb}
\usepackage{amsmath}
\usepackage{amsfonts}

% used for TeXing text within eps files
%\usepackage{psfrag}
% need this for including graphics (\includegraphics)
%\usepackage{graphicx}
% for neatly defining theorems and propositions
%\usepackage{amsthm}
% making logically defined graphics
%%%\usepackage{xypic}

% there are many more packages, add them here as you need them

% define commands here
\begin{document}
The sphere of radius $r$ can be described parametrically by spherical coordinates (what else ;) ) as follows:
 $$ x = r \sin u \sin v $$
 $$ y = r \sin u \cos v $$
 $$ z = r \cos u $$
Then, using trigonometric identities to simplify expressions we find that
 $$\frac{\partial (x, y)}{\partial (u,v)} =
\left| \begin{matrix}
r \cos u \sin v & r \sin u \cos v \\
r \cos u \cos v & -r \sin u \sin v
\end{matrix} \right| =
- r^2 \cos u \sin u$$
$$\frac{\partial (y, z)}{\partial (u,v)} =
\left| \begin{matrix}
r \cos u \cos v & -r \sin u \sin v \\
- r \sin u & 0
\end{matrix} \right| =
- r^2 \sin^2 u \sin v$$
$$\frac{\partial (z, x)}{\partial (u,v)} =
\left| \begin{matrix}
- r \sin u & 0 \\
r \cos u \sin v & r \sin u \cos v
\end{matrix} \right| =
r^2 \sin^2 u \cos v$$
and hence, using more trigonometric identities, we find that
 $$\sqrt{ \left(  \frac{\partial (x,y)}{\partial (u,v)} \right)^2 +  \left( \frac{\partial (y,z)}{\partial (u,v)} \right)^2 + \left( \frac{\partial (z,x)}{\partial (u,v)} \right)^2 } =$$
 $$\sqrt{ r^4 \cos^2 u \sin^2 u + r^4 \sin^4 u \sin^2 v + r^4 \sin^4 u \cos^2 v } = r^2 \sin u.$$
This means that, on a sphere
 $$d^2 A = r^2 \sin u \> du \, dv.$$
Note that in the case of a unit sphere, ($r = 1$) this agrees with the formula presented in the second paragraph of subsection 2 of the main entry.

To return to the main entry \PMlinkid{click here}{6660}
%%%%%
%%%%%
\end{document}
