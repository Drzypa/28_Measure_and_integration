\documentclass[12pt]{article}
\usepackage{pmmeta}
\pmcanonicalname{sigmaalgebra}
\pmcreated{2013-03-22 14:00:04}
\pmmodified{2013-03-22 14:00:04}
\pmowner{drini}{3}
\pmmodifier{drini}{3}
\pmtitle{$\sigma$-algebra}
\pmrecord{7}{34835}
\pmprivacy{1}
\pmauthor{drini}{3}
\pmtype{Definition}
\pmcomment{trigger rebuild}
\pmclassification{msc}{28A60}

\usepackage{graphicx}
%%%\usepackage{xypic} 
\usepackage{bbm}
\newcommand{\Z}{\mathbbmss{Z}}
\newcommand{\C}{\mathbbmss{C}}
\newcommand{\R}{\mathbbmss{R}}
\newcommand{\Q}{\mathbbmss{Q}}
\newcommand{\mathbb}[1]{\mathbbmss{#1}}
\newcommand{\figura}[1]{\begin{center}\includegraphics{#1}\end{center}}
\newcommand{\figuraex}[2]{\begin{center}\includegraphics[#2]{#1}\end{center}}
\begin{document}
Let $X$ be a set. A $\sigma$-algebra is a collection $M$ of subsets of $X$ such that
\begin{itemize}
\item $X\in M$
\item If $A\in M$ then $X-A\in M$.
\item If $A_1,A_2,A_3,\ldots$ is a countable subcollection of $M$, that is, $A_j\in M$ for $j=1,2,3,\ldots$ (the subcollection can be finite) then the union of all of them is also in $M$:
\[\bigcup_{j=1}^\infty A_i\in M.\]
\end{itemize}
%%%%%
%%%%%
\end{document}
