\documentclass[12pt]{article}
\usepackage{pmmeta}
\pmcanonicalname{CentreOfMassOfHalfdisc}
\pmcreated{2013-03-22 17:20:57}
\pmmodified{2013-03-22 17:20:57}
\pmowner{pahio}{2872}
\pmmodifier{pahio}{2872}
\pmtitle{centre of mass of half-disc}
\pmrecord{10}{39706}
\pmprivacy{1}
\pmauthor{pahio}{2872}
\pmtype{Example}
\pmcomment{trigger rebuild}
\pmclassification{msc}{28A75}
\pmclassification{msc}{26B15}
\pmsynonym{center of mass of half-disc}{CentreOfMassOfHalfdisc}
\pmsynonym{centroid of half-disc}{CentreOfMassOfHalfdisc}
%\pmkeywords{double integral}
\pmrelated{SubstitutionNotation}
\pmrelated{CentreOfMassOfPolygon}
\pmrelated{CenterOfGravityOfCircularSector}
\pmrelated{AreaOfSphericalZone}

\endmetadata

% this is the default PlanetMath preamble.  as your knowledge
% of TeX increases, you will probably want to edit this, but
% it should be fine as is for beginners.

% almost certainly you want these
\usepackage{amssymb}
\usepackage{amsmath}
\usepackage{amsfonts}

% used for TeXing text within eps files
%\usepackage{psfrag}
% need this for including graphics (\includegraphics)
%\usepackage{graphicx}
% for neatly defining theorems and propositions
%\usepackage{amsthm}
% making logically defined graphics
%%%\usepackage{xypic}

% there are many more packages, add them here as you need them

% define commands here
\newcommand{\sijoitus}[2]%
{\operatornamewithlimits{\Big/}_{\!\!\!#1}^{\,#2}}
\begin{document}
Let $E$ be the upper half-disc of the disc\, $x^2+y^2 \leqq R$\, in $\mathbb{R}^2$ with a \PMlinkescapetext{constant} surface-density 1.  By the symmetry, its centre of mass lies on its medium radius, and therefore we only have to calculate the ordinate $Y$ of the centre of mass.  For doing that, one can use the double integral
$$Y \;=\; \frac{1}{\nu(E)}\iint_E y\,dx\,dy,$$
where\, $\nu(E) = \frac{\pi R^2}{2}$\, is the area of the half-disc.  The region of integration is defined by
$$E \;=\; \{(x,\,y)\in\mathbb{R}^2\,\vdots\;\; -R\leqq x \leqq R,\; 0 \leqq y \leqq \sqrt{R^2-x^2}\}.$$
Accordingly we may write
$$Y \;=\; \frac{2}{\pi R^2}\!\int_{-R}^R\!dx\int_0^{\sqrt{R^2-x^2}}\!y\,dy \;=\; 
\frac{2}{\pi R^2}\!\int_{-R}^R\frac{R^2\!-\!x^2}{2}\,dx \;=\; 
\frac{2}{\pi R^2}\sijoitus{x=-R}{\quad R}\left(\frac{R^2x}{2}-\frac{x^3}{6}\right) \;=\; \frac{4R}{3\pi}.$$
Thus the centre of mass is the point\, $(0,\,\frac{4R}{3\pi})$.

%%%%%
%%%%%
\end{document}
