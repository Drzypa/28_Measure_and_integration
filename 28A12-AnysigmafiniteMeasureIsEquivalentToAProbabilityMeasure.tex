\documentclass[12pt]{article}
\usepackage{pmmeta}
\pmcanonicalname{AnysigmafiniteMeasureIsEquivalentToAProbabilityMeasure}
\pmcreated{2013-03-22 18:33:44}
\pmmodified{2013-03-22 18:33:44}
\pmowner{gel}{22282}
\pmmodifier{gel}{22282}
\pmtitle{any $\sigma$-finite measure is equivalent to a probability measure}
\pmrecord{6}{41285}
\pmprivacy{1}
\pmauthor{gel}{22282}
\pmtype{Theorem}
\pmcomment{trigger rebuild}
\pmclassification{msc}{28A12}
\pmclassification{msc}{28A10}
%\pmkeywords{sigma-finite}
%\pmkeywords{measure}
\pmrelated{SigmaFinite}
\pmrelated{Measure}

\endmetadata

% this is the default PlanetMath preamble.  as your knowledge
% of TeX increases, you will probably want to edit this, but
% it should be fine as is for beginners.

% almost certainly you want these
\usepackage{amssymb}
\usepackage{amsmath}
\usepackage{amsfonts}

% used for TeXing text within eps files
%\usepackage{psfrag}
% need this for including graphics (\includegraphics)
%\usepackage{graphicx}
% for neatly defining theorems and propositions
\usepackage{amsthm}
% making logically defined graphics
%%%\usepackage{xypic}

% there are many more packages, add them here as you need them

% define commands here
\newtheorem*{theorem*}{Theorem}

\begin{document}
\PMlinkescapeword{equivalent}
The following theorem states that for any \PMlinkname{$\sigma$-finite}{SigmaFinite} measure $\mu$, there is an equivalent probability measure $\mathbb{P}$ --- that is, the sets $A$ satisfying $\mu(A)=0$ are the same as those satisfying $\mathbb{P}(A)=0$.
This result allows statements about probability measures to be generalized to arbitrary $\sigma$-finite measures.

\begin{theorem*}
Any nonzero $\sigma$-finite measure $\mu$ on a measurable space $(X,\mathcal{A})$ is equivalent to a probability measure $\mathbb{P}$ on $(X,\mathcal{A})$. In particular, there is a positive measurable function $f\colon X\rightarrow(0,\infty)$ satisfying $\int f\,d\mu=1$, and $\mathbb{P}(A)=\int_Af\,d\mu$ for all $A\in\mathcal{A}$.
\end{theorem*}
\begin{proof}
Let $A_1,A_2,\ldots$ be a sequence in $\mathcal{A}$ such that $\mu(A_k)<\infty$ and $\bigcup_kA_k=X$. Then it is easily verified that
\begin{equation*}
g\equiv\sum_{k=1}^\infty 2^{-k}\frac{1_{A_k}}{1+\mu(A_k)}
\end{equation*}
satisfies $1\ge g>0$ and $\int g\,d\mu<\infty$. So, setting $f=g/\int g\,d\mu$, we have $\int f\,d\mu=1$ and therefore $\mathbb{P}(A)\equiv\int_Af\,d\mu$ is a probability measure equivalent to $\mu$.
\end{proof}
%%%%%
%%%%%
\end{document}
