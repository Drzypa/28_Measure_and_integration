\documentclass[12pt]{article}
\usepackage{pmmeta}
\pmcanonicalname{BoundedLinearFunctionalsOnLinftymu}
\pmcreated{2013-03-22 18:38:08}
\pmmodified{2013-03-22 18:38:08}
\pmowner{gel}{22282}
\pmmodifier{gel}{22282}
\pmtitle{bounded linear functionals on $L^\infty(\mu)$}
\pmrecord{5}{41375}
\pmprivacy{1}
\pmauthor{gel}{22282}
\pmtype{Theorem}
\pmcomment{trigger rebuild}
\pmclassification{msc}{28A25}
\pmrelated{BoundedLinearFunctionalsOnLpmu}
\pmrelated{RadonNikodymTheorem}
\pmrelated{LpNormIsDualToLq}

% almost certainly you want these
\usepackage{amssymb}
\usepackage{amsmath}
\usepackage{amsfonts}

% used for TeXing text within eps files
%\usepackage{psfrag}
% need this for including graphics (\includegraphics)
%\usepackage{graphicx}
% for neatly defining theorems and propositions
\usepackage{amsthm}
% making logically defined graphics
%%%\usepackage{xypic}

% there are many more packages, add them here as you need them

% define commands here
\newtheorem*{theorem*}{Theorem}
\newtheorem*{lemma*}{Lemma}
\newtheorem*{corollary*}{Corollary}
\newtheorem*{definition*}{Definition}
\newtheorem{theorem}{Theorem}
\newtheorem{lemma}{Lemma}
\newtheorem{corollary}{Corollary}
\newtheorem{definition}{Definition}

\begin{document}
\PMlinkescapeword{bounded}
\PMlinkescapeword{property}
\PMlinkescapeword{satisfies}
For any measure space $(X,\mathfrak{M},\mu)$ and $g\in L^1(\mu)$, the following linear map can be defined
\begin{align*}
&\Phi_g\colon L^\infty (\mu)\rightarrow\mathbb{R},\\
&f\mapsto\Phi_g(f)\equiv\int fg\,d\mu.
\end{align*}
It is easily shown that $\Phi_g$ is \PMlinkname{bounded}{OperatorNorm}, so is a member of the dual space of $\L^\infty(\mu)$. However, unless the measure space consists of a finite set of atoms, not every element of the dual of $L^\infty(\mu)$ can be written like this. Instead, it is necessary to restrict to linear maps satisfying a bounded convergence property.

\begin{theorem*}
Let $(X,\mathfrak{M},\mu)$ be a \PMlinkname{$\sigma$-finite}{SigmaFinite} measure space and $V$ be the space of bounded linear maps $\Phi\colon L^\infty(\mu)\rightarrow\mathbb{R}$ satisfying bounded convergence. That is, if $|f_n|\le 1$ are in $L^\infty(\mu)$ and $f_n(x)\rightarrow 0$ for almost every $x\in X$, then $\Phi(f_n)\rightarrow 0$.

Then $g\mapsto\Phi_g$ gives an isometric isomorphism from $L^1(\mu)$ to $V$.
\end{theorem*}
\begin{proof}
First, the operator norm $\Vert\Phi_g\Vert$ is equal to the $L^1$-norm of $g$ (see \PMlinkname{$L^p$-norm is dual to $L^q$}{LpNormIsDualToLq}), so the map $g\mapsto\Phi_g$ gives an isometric embedding from $L^1$ into the dual of $L^\infty$. Furthermore, dominated convergence implies that $\Phi_g$ satisfies bounded convergence so $\Phi_g\in V$. We just need to show that $g\mapsto\Phi_g$ maps onto $V$.

So, suppose that $\Phi\in V$. It needs to be shows that $\Phi=\Phi_g$ for some $g\in L^1$.
Defining an \PMlinkname{additive set function}{Additive} $\nu\colon\mathfrak{M}\rightarrow\mathbb{R}$ by
\begin{equation*}
\nu(A)=\Phi(1_A)
\end{equation*}
for every set $A\in\mathfrak{M}$, the bounded convergence property for $\Phi$ implies that $\nu$ is countably additive and is therefore a finite signed measure. So, the Radon-Nikodym theorem gives a $g\in L^1$ such that $\nu(A)=\int_A g\,d\mu$ for every $A\in\mathfrak{M}$.
Then, the equality
\begin{equation*}
\Phi(fh)= \int fg\,d\mu
\end{equation*}
is satisfied for $f=1_A$ with any $A\in\mathfrak{M}$ and the functional monotone class theorem extends this to any bounded and measurable $f\colon X\rightarrow\mathbb{C}$, giving $\Phi_g=\Phi$.
\end{proof}

%%%%%
%%%%%
\end{document}
