\documentclass[12pt]{article}
\usepackage{pmmeta}
\pmcanonicalname{KochCurve}
\pmcreated{2013-03-22 12:05:34}
\pmmodified{2013-03-22 12:05:34}
\pmowner{akrowne}{2}
\pmmodifier{akrowne}{2}
\pmtitle{Koch curve}
\pmrecord{8}{31186}
\pmprivacy{1}
\pmauthor{akrowne}{2}
\pmtype{Definition}
\pmcomment{trigger rebuild}
\pmclassification{msc}{28A33}
\pmclassification{msc}{28A80}
\pmsynonym{Koch snowflake}{KochCurve}
%\pmkeywords{fractals}

\usepackage{amssymb}
\usepackage{amsmath}
\usepackage{amsfonts}
\usepackage{graphicx}
%%%\usepackage{xypic}
\begin{document}
A Koch curve is a fractal generated by a replacement rule.  This rule is, at each step, to replace the middle $1/3$ of each line segment with two sides of a right triangle having sides of length equal to the replaced segment.  Two applications of this rule on a single line segment gives us:

\begin{center}
\includegraphics[scale=.6]{koch1}
\end{center}

To generate the Koch curve, the rule is applied indefinitely, with a starting line segment.  Note that, if the length of the initial line segment is $l$, the length $L_K$ of the Koch curve at the $n$th step will be

$$ L_K = \left( \frac{4}{3} \right)^n l $$

This quantity increases without bound; hence the Koch curve has infinite length.  However, the curve still bounds a finite area.  We can prove this by noting that in each step, we add an amount of area equal to the area of all the equilateral triangles we have just created.  We can bound the area of each triangle of side length $s$ by $s^2$ (the square containing the triangle.)  Hence, at step $n$, the area $A_K$ ``under'' the Koch curve (assuming $l=1$) is 

\begin{eqnarray*} A_K & < & \left(\frac{1}{3}\right)^2 + 3 \left(\frac{1}{9}\right)^2 + 9 \left(\frac{1}{27}\right)^2 +  \cdots \\ 
 & = & \sum_{i=1}^n \frac{1}{3^{i-1}} 
\end{eqnarray*}

but this is a geometric series of ratio less than one, so it converges.  Hence a Koch curve has infinite length and bounds a finite area.

A Koch snowflake is the figure generated by applying the Koch replacement rule to an equilateral triangle indefinitely.
%%%%%
%%%%%
%%%%%
\end{document}
