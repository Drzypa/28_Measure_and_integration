\documentclass[12pt]{article}
\usepackage{pmmeta}
\pmcanonicalname{VitaliConvergenceTheorem}
\pmcreated{2013-03-22 16:17:10}
\pmmodified{2013-03-22 16:17:10}
\pmowner{stevecheng}{10074}
\pmmodifier{stevecheng}{10074}
\pmtitle{Vitali convergence theorem}
\pmrecord{9}{38401}
\pmprivacy{1}
\pmauthor{stevecheng}{10074}
\pmtype{Theorem}
\pmcomment{trigger rebuild}
\pmclassification{msc}{28A20}
\pmsynonym{uniform-integrability convergence theorem}{VitaliConvergenceTheorem}
\pmrelated{ModesOfConvergenceOfSequencesOfMeasurableFunctions}
\pmrelated{UniformlyIntegrable}
\pmrelated{DominatedConvergenceTheorem}

\usepackage{amssymb}
\usepackage{amsmath}
\usepackage{amsfonts}
\usepackage{amsthm}
\usepackage{enumerate}
%\usepackage{graphicx}
%\usepackage{psfrag}
%%%\usepackage{xypic}

% define commands here
\newcommand{\complex}{\mathbb{C}}
\newcommand{\real}{\mathbb{R}}
\newcommand{\rat}{\mathbb{Q}}
\newcommand{\nat}{\mathbb{N}}
\newcommand{\Le}{\mathbf{L}}

\providecommand{\abs}[1]{\lvert#1\rvert}
\providecommand{\absW}[1]{\left\lvert#1\right\rvert}
\providecommand{\absB}[1]{\Bigl\lvert#1\Bigr\rvert}
\providecommand{\norm}[1]{\lVert#1\rVert}
\providecommand{\normW}[1]{\left\lVert#1\right\rVert}
\providecommand{\normB}[1]{\Bigl\lVert#1\Bigr\rVert}


\begin{document}
Let $f_1, f_2, \dotsc$ be $\Le^p$-integrable functions on some measure space, for $1 \leq p < \infty$.

The sequence $\{ f_n \}$ converges in $\Le^p$ to a measurable function $f$ 
if and and only if

\begin{enumerate}[i]
\item
the sequence $\{ f_n \}$ converges to $f$ in measure;
\item
the functions $\{ \abs{f_n}^p \}$ are uniformly integrable; and
\item
for every $\epsilon > 0$, there exists a set $E$
of finite measure, such that $\int_{E^\mathrm{c}} \abs{f_n}^p < \epsilon$
for all $n$.
\end{enumerate}

\subsection*{Remarks}

This theorem can be used as a replacement for the more
well-known dominated convergence theorem, when a
dominating \PMlinkescapetext{factor} cannot be found for the functions
$f_n$ to be integrated.
(If this theorem is known, the dominated convergence theorem
can be derived as a special case.)

In a finite measure space, condition (iii) is trivial.
In fact, condition (iii) is the tool used to reduce considerations
in the general case to the case of a finite measure space.

In probability \PMlinkescapetext{theory}, the definition of ``uniform integrability''
is slightly different from its definition in general measure theory;
either definition may be used in the statement of this theorem.

\begin{thebibliography}{3}
\bibitem{Folland}
Gerald B. Folland. {\it Real Analysis: Modern Techniques and Their Applications}, second ed. Wiley-Interscience, 1999.

\bibitem{Rosenthal}
Jeffrey S. Rosenthal. {\it A First Look at Rigorous Probability Theory}.
World Scientific, 2003.
\end{thebibliography}

%%%%%
%%%%%
\end{document}
