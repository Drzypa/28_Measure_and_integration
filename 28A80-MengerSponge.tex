\documentclass[12pt]{article}
\usepackage{pmmeta}
\pmcanonicalname{MengerSponge}
\pmcreated{2013-03-22 14:27:05}
\pmmodified{2013-03-22 14:27:05}
\pmowner{mathwizard}{128}
\pmmodifier{mathwizard}{128}
\pmtitle{Menger sponge}
\pmrecord{23}{35968}
\pmprivacy{1}
\pmauthor{mathwizard}{128}
\pmtype{Definition}
\pmcomment{trigger rebuild}
\pmclassification{msc}{28A80}
%\pmkeywords{fractal}
\pmrelated{Fractal}
\pmrelated{CantorSet}
\pmrelated{SierpinskiGasket}
\pmdefines{Sierpinski carpet}

\endmetadata

% this is the default PlanetMath preamble.  as your knowledge
% of TeX increases, you will probably want to edit this, but
% it should be fine as is for beginners.

% almost certainly you want these
\usepackage{amssymb}
\usepackage{amsmath}
\usepackage{amsfonts}

% used for TeXing text within eps files
%\usepackage{psfrag}
% need this for including graphics (\includegraphics)
\usepackage{graphicx}
% for neatly defining theorems and propositions
%\usepackage{amsthm}
% making logically defined graphics
%%%\usepackage{xypic}

% there are many more packages, add them here as you need them

% define commands here
\begin{document}
\\PMlinkescapeword{divisions}
\PMlinkescapeword{equivalent}
\PMlinkescapeword{unit}

A \emph{Sierpinski carpet} is the set of all points $(x, y)$ such that $x$ or $y$ is in
the Cantor set.  An equivalent and perhaps simpler definition is:

Let $S_0$ be a unit square.  Let $S_{n+1}$ be $S_n$, with each square divided
into ninths, by being divided into thirds horizontally and vertically, and the central resulting square removed, and the other resulting squares treated separately in further divisions.  The limit as $n \to \infty$ of $S_n$ is a Sierpinski carpet. An approximation is shown in figure \ref{fig:sierpinski}.

\begin{figure}[h]
\begin{centering}
\includegraphics[scale=0.5]{sierpinski_carpet.eps}
\caption{An approximation of the Sierpinski carpet}\label{fig:sierpinski}
\end{centering}
\end{figure}

The \emph{Menger sponge} is a fractal embedded in 3-dimensional space.  It can be seen as a 3-d generalization of the Sierpinski carpet, which is itself a 2-dimensional generalization of the Cantor set.  The Menger sponge is almost always represented as being constructed from Cantor sets using the ``middle third'' rule.

The Menger sponge consists of all points $(x, y, z)$ such that $(x, y)$, $(y,z)$, and $(x,z)$ are all in Sierpinski carpets.  Each ``face'' is a Sierpinski carpet.

Similarily to the Sierpinski carpet the Menger sponge can be constructed in the following way:

Start with a unit cube and split it into 27 smaller cubes of equal size. Remove the central cube and the ones joining a face with it. Then start over with the remaining smaller cubes.

\begin{figure}[h]
\begin{centering}
\includegraphics[scale=0.4]{menger.ps}
\caption{An iteration of the Menger sponge -- created in Blender 2.35.
(The \PMlinktofile{Blender file}{menger.blend} for this picture.)}\label{fig:menger}
\end{centering}
\end{figure}
%%%%%
%%%%%
\end{document}
