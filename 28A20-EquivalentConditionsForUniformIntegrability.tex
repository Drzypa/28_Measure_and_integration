\documentclass[12pt]{article}
\usepackage{pmmeta}
\pmcanonicalname{EquivalentConditionsForUniformIntegrability}
\pmcreated{2013-03-22 18:40:17}
\pmmodified{2013-03-22 18:40:17}
\pmowner{gel}{22282}
\pmmodifier{gel}{22282}
\pmtitle{equivalent conditions for uniform integrability}
\pmrecord{7}{41418}
\pmprivacy{1}
\pmauthor{gel}{22282}
\pmtype{Theorem}
\pmcomment{trigger rebuild}
\pmclassification{msc}{28A20}
%\pmkeywords{measure space}
%\pmkeywords{Lebesgue integral}

% almost certainly you want these
\usepackage{amssymb}
\usepackage{amsmath}
\usepackage{amsfonts}

% used for TeXing text within eps files
%\usepackage{psfrag}
% need this for including graphics (\includegraphics)
%\usepackage{graphicx}
% for neatly defining theorems and propositions
\usepackage{amsthm}
% making logically defined graphics
%%%\usepackage{xypic}

% there are many more packages, add them here as you need them

% define commands here
\newtheorem*{theorem*}{Theorem}
\newtheorem*{lemma*}{Lemma}
\newtheorem*{corollary*}{Corollary}
\newtheorem*{definition*}{Definition}
\newtheorem{theorem}{Theorem}
\newtheorem{lemma}{Lemma}
\newtheorem{corollary}{Corollary}
\newtheorem{definition}{Definition}

\begin{document}
\PMlinkescapeword{properties}
\PMlinkescapeword{equivalence}
\PMlinkescapeword{finite}
\PMlinkescapeword{symmetric}

Let $(\Omega,\mathcal{F},\mu)$ be a measure space and $S$ be a bounded subset of $L^1(\Omega,\mathcal{F},\mu)$. That is, $\int|f|\,d\mu$ is bounded over $f\in S$. Then, the following are equivalent.

\begin{enumerate}
\item\label{eps delta} For every $\epsilon>0$ there is a $\delta>0$ so that
\begin{equation*}
\int_A |f|\,d\mu<\epsilon
\end{equation*}
for all $f\in S$ and $\mu(A)<\delta$.
\item\label{eps K} For every $\epsilon>0$ there is a $K>0$ satisfying
\begin{equation*}
\int_{|f|>K} |f|\,d\mu<\epsilon
\end{equation*}
for all $f\in S$.
\item\label{Phi} There is a measurable function $\Phi\colon\mathbb{R}\rightarrow [0,\infty)$ such that $\Phi(x)/|x|\rightarrow\infty$ as $|x|\rightarrow\infty$ and
\begin{equation*}
\int \Phi(f)\,d\mu
\end{equation*}
is bounded over all $f\in S$. Moreover, the function $\Phi$ can always be chosen to be symmetric and convex.
\end{enumerate}

So, for bounded subsets of $L^1$, either of the above properties can be used to define uniform integrability. Conversely, when the measure space is finite, then conditions (\ref{eps K}) and (\ref{Phi}) are easily shown to imply that $S$ is bounded in $L^1$.


To show the equivalence of these statements, let us suppose that $\int|f|\,d\mu< L$ for $f\in S$.

\vspace{\baselineskip}\noindent{\bf (\ref{eps delta}) implies (\ref{eps K})}

For $\epsilon>0$, property (\ref{eps delta}) gives a $\delta>0$ so that $\int_A|f|\,d\mu<\epsilon$ whenever $f\in S$ and $\mu(A)<\delta$. Choosing $K>L/\delta$, Markov's inequality gives
\begin{equation*}
\mu(|f|>K)\le K^{-1}\int|f|\,d\mu\le L/K<\delta
\end{equation*}
and, therefore, $\int_{|f|>K}|f|\,d\mu<\epsilon$.

\noindent{\bf (\ref{eps K}) implies (\ref{Phi})}

For each $n=1,2,\ldots$, property (\ref{eps K}) gives a $K_n$ satisfying
\begin{equation*}
\int (|f|-K_n)_+\,d\mu\le\int_{|f|>K_n}|f|\,d\mu\le 2^{-n}.
\end{equation*}
Without loss of generality, the $K_n$ can be chosen to be increasing to infinity, so we can define $\Phi(x)=\sum_n(|x|-K_n)_+$. Then,
\begin{equation*}
\int\Phi(f)\,d\mu=\sum_n\int(|f|-K_n)_+\,d\mu\le\sum_n 2^{-n}=1.
\end{equation*}

\noindent{\bf (\ref{Phi}) implies (\ref{eps delta})}

First, suppose that $\int\Phi(f)\,d\mu< M$ for $f\in S$.
For $\epsilon>0$, the condition that $\Phi(x)/|x|\rightarrow\infty$ as $|x|\rightarrow\infty$ gives a $K>0$ such that $\Phi(x)/|x|\ge 2M/\epsilon$ whenever $|x|>K$.
Setting $\delta=\epsilon/2K$,
\begin{equation*}\begin{split}
\int_A|f|\,d\mu
&\le \int_{|f|>K}|f|\,d\mu +K\mu(A)\\
&< (\epsilon/2M)\int_{|f|>K}\Phi(f)\,d\mu +K\delta\\
&< \epsilon/2 +\epsilon/2 =\epsilon.
\end{split}\end{equation*}
whenever $\mu(A)<\delta$ and $f\in S$.

%%%%%
%%%%%
\end{document}
