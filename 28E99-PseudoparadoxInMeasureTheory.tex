\documentclass[12pt]{article}
\usepackage{pmmeta}
\pmcanonicalname{PseudoparadoxInMeasureTheory}
\pmcreated{2013-03-22 14:38:40}
\pmmodified{2013-03-22 14:38:40}
\pmowner{rspuzio}{6075}
\pmmodifier{rspuzio}{6075}
\pmtitle{pseudoparadox in measure theory}
\pmrecord{11}{36232}
\pmprivacy{1}
\pmauthor{rspuzio}{6075}
\pmtype{Theorem}
\pmcomment{trigger rebuild}
\pmclassification{msc}{28E99}
\pmsynonym{one-dimensional Banach-Tarski paradox}{PseudoparadoxInMeasureTheory}
\pmrelated{VitalisTheorem}
\pmrelated{BanachTarskiParadox}

% this is the default PlanetMath preamble.  as your knowledge
% of TeX increases, you will probably want to edit this, but
% it should be fine as is for beginners.

% almost certainly you want these
\usepackage{amssymb}
\usepackage{amsmath}
\usepackage{amsfonts}

% used for TeXing text within eps files
%\usepackage{psfrag}
% need this for including graphics (\includegraphics)
%\usepackage{graphicx}
% for neatly defining theorems and propositions
%\usepackage{amsthm}
% making logically defined graphics
%%%\usepackage{xypic}

% there are many more packages, add them here as you need them

% define commands here
\begin{document}
The interval $[0,1)$ can be subdivided into an countably infinite collection of disjoint subsets $A_i, i = 0,1,2,\ldots$ such that, by translating each set one obtains a collection of disjoint sets $B_i$ such that $\bigcup_{i=1}^\infty B_i = [0,2)$

This paradox challenges the naive ``\PMlinkescapetext{cut and paste}'' notion that if one \PMlinkescapetext{cuts} a set into a countable number of pieces and reassembles them, the result will have the same measure as the original set.

The resolution to this and similar paradoxes lies in the fact that the sets $A_i$ were not defined constructively.  To show that they exist, one needs to appeal to the non-constructive axiom of choice.  What the paradox shows is that one can't have one's cake and eat it too --- either one can cling to the naive ``\PMlinkescapetext{cut and paste}'' picture and forego non-constructive techniques as the intuitionist school of mathematics does, or else if, like the majority of mathematicians, one wants to keep the powerful tools provided by non-constructive techniques in set theory, one must give up the naive notion that every set is measurable and limit ``\PMlinkescapetext{cut and paste operations}'' to operations involving measurable sets.

It might be worth mentioning that it is essential that there be an infinite number of sets $A_i$.  As an elegant argument posted by jihema shows, it is not possible to find a finite collection of disjoint subsets of $[0,1)$ such that a union of translations of these subsets equals $[0,2)$.  In higher dimensions, the situation is worse because, as Banach and Tarski showed, it is possible to derive analogous paradoxes involving only a finite number of subsets.
%%%%%
%%%%%
\end{document}
