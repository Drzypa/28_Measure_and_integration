\documentclass[12pt]{article}
\usepackage{pmmeta}
\pmcanonicalname{mathbbLpVsmathbbLq}
\pmcreated{2013-03-22 15:22:05}
\pmmodified{2013-03-22 15:22:05}
\pmowner{yark}{2760}
\pmmodifier{yark}{2760}
\pmtitle{$\mathbb{L}^p$ vs $\mathbb{L}^q$}
\pmrecord{11}{37194}
\pmprivacy{1}
\pmauthor{yark}{2760}
\pmtype{Topic}
\pmcomment{trigger rebuild}
\pmclassification{msc}{28A25}
%\pmkeywords{norm}
%\pmkeywords{L^p space}

\usepackage{amsmath}
\usepackage{amsfonts}

\def\L{\mathbb{L}}
\def\R{\mathbb{R}}

\begin{document}
\PMlinkescapeword{connection}
\PMlinkescapeword{right}
\PMlinkescapeword{side}

Let $(X,\mathcal{A},\mu)$ be a measure space and $1\leq p,q \leq\infty$.
Generally there is no connection between
$\L^p(\mu)$ and $\L^q(\mu)$ as sets.
However, for some special measures,
there is an interesting relationship between them.
A few examples:
\begin{enumerate}
\item If $\lambda^n$ is the Lebesgue measure on $\R^n$ and $p\neq q$,
then $\L^p(\lambda^n)\not\subseteq \L^q(\lambda^n)$
for all $n\in\mathbb{N}^+$.
Here is an example for $n=1$ and $1\leq p<q<\infty$.
Let
\[
  f(x):=\begin{cases}
  x^{\frac{-1}{p}},& x>1 \\
  0,& x \leq 1
  \end{cases}
\]
and
\[
  g(x):=\begin{cases}
  x^{\frac{-1}{q}},& x\in(0,1) \\
  0,& x\not\in (0,1).
  \end{cases}
\]
This gives $\|f\|_q^q=\frac{p}{q-p}$, $\|g\|_p^p=\frac{q}{q-p}$
and $\|f\|_p=\|g\|_q=\infty$.
So $f\in\L^q\setminus\L^p$ and $g\in\L^p\setminus\L^q$.
For the $\infty$-norm,
$\chi_\R\in\L^\infty\setminus\L^q$,
where $\chi$ is the characteristic function,
and also $f\not\in\L^\infty$. 
\item If $p<q$ then $l^p\subseteq l^q$.
This is trivial if $q = \infty$.
Now let $x=(x_0,x_1,\dots)\in l^p$ and $q<\infty$.
Then
\[
  \|x\|_q^q = \sum_{n=0}^\infty |x_n|^q = \sum_{n=0}^\infty |x_n|^p |x_n|^{q-p} \leq \sum_{n=0}^\infty |x_n|^p \|x\|_\infty^{q-p} = \|x\|_\infty^{q-p} \|x\|_p^p <\infty,
\]
so $x\in l^q$.
\item If $\mu(X)$ is finite and $p<q$,
then $\L^q \subseteq \L^p$.
This is easy if $q = \infty$,
because $|f|\leq\|f\|_\infty$ almost everywhere,
so $\|f\|_p^p= \int |f|^p d\mu \leq \int \|f\|_\infty^p d\mu
= \|f\|_\infty^p \mu(X)<\infty$.
Now let $q<\infty$,
thus 
\begin{align*}
  \|f\|_p^p&=\int |f|^p d\mu \\
  &=\int_{|f|>1}|f|^p d\mu + \int_{|f|\leq 1}|f|^p d\mu \\
  &\leq \int_{|f|>1}|f|^q d\mu + \int_{|f|\leq 1}d\mu \\
  &\leq \|f\|_q^q+\mu(X) \\
  &<\infty.
\end{align*}
\end{enumerate}

Finally, we prove an interesting property for $p$-norms:
if $X$ is a finite measure space,
then for any measurable function $f$ on $X$
the equality $\lim\limits_{p\to\infty}\|f\|_p=\|f\|_\infty$ holds.
We have already seen that $\|f\|_p\leq\|f\|_\infty\mu(X)^{\frac{1}{p}}$.
Now for any $\varepsilon\in(0,\|f\|_\infty)$
define $A_\varepsilon := \{ x \in X : |f(x)|\geq\|f\|_\infty-\varepsilon \}$,
$\delta_\varepsilon:=\mu(A_\varepsilon)>0$
and $g:=(\|f\|_\infty-\varepsilon)\chi_{A_\varepsilon}$.
Since $|g|\leq|f|$,
we have $\|g\|_p=(\|f\|_\infty-\varepsilon)\delta_\varepsilon^{\frac{1}{p}}\leq\|f\|_p\leq\|f\|_\infty\mu(X)^{\frac{1}{p}}$.
Now we take $\liminf$ on the left and $\limsup$ on the right side:
$\|f\|_\infty-\varepsilon \leq \liminf\limits_{p\to\infty}\|f\|_p \leq \limsup\limits_{p\to\infty}\|f\|_p \leq \|f\|_\infty$.
Taking $\varepsilon\downarrow 0$
gives $\lim\limits_{p\to\infty}\|f\|_p=\|f\|_\infty$.

%%%%%
%%%%%
\end{document}
