\documentclass[12pt]{article}
\usepackage{pmmeta}
\pmcanonicalname{ClarksonInequality}
\pmcreated{2013-03-22 16:04:59}
\pmmodified{2013-03-22 16:04:59}
\pmowner{georgiosl}{7242}
\pmmodifier{georgiosl}{7242}
\pmtitle{Clarkson inequality}
\pmrecord{13}{38142}
\pmprivacy{1}
\pmauthor{georgiosl}{7242}
\pmtype{Theorem}
\pmcomment{trigger rebuild}
\pmclassification{msc}{28A25}
%\pmkeywords{L^p space}
%\pmkeywords{uniformly convex space}

% this is the default PlanetMath preamble.  as your knowledge
% of TeX increases, you will probably want to edit this, but
% it should be fine as is for beginners.

% almost certainly you want these
\usepackage{amssymb}
\usepackage{amsmath}
\usepackage{amsfonts}

% used for TeXing text within eps files
%\usepackage{psfrag}
% need this for including graphics (\includegraphics)
%\usepackage{graphicx}
% for neatly defining theorems and propositions
%\usepackage{amsthm}
% making logically defined graphics
%%%\usepackage{xypic}

% there are many more packages, add them here as you need them

% define commands here

\begin{document}
The \emph{Clarkson inequality} says that for all $f,g \in L^p$, for $2\leq p<\infty$  we have:
\[
\left\| \frac{f+g}{2} \right\|^p_p +
\left\| \frac{f-g}{2} \right\|^p_p
                                     \le
\frac{1}{2}\left( \|f\|^p_p + \|g\|^p_p\right).
\]
\\The inequality can be used to prove that $L^p$ space is uniformly convex for $2\leq p<\infty$.

\textbf{Remark}.
If $1 < p < 2$, then the Clarkson inequality becomes:
\[
\left\| \frac{f+g}{2} \right\|^q_p +
\left\| \frac{f-g}{2} \right\|^q_p
                                     \le
\left(\frac{1}{2} \|f\|^p_p +\frac{1}{2} \|g\|^p_p\right)^\frac{1}{p-1}
\].




for functions $f,\,g \in L^p$, where $q=\frac{p}{p-1}$.


%%%%%
%%%%%
\end{document}
