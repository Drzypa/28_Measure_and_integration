\documentclass[12pt]{article}
\usepackage{pmmeta}
\pmcanonicalname{ExampleOfFunctionNotLebesgueMeasurableWithMeasurableLevelSets}
\pmcreated{2013-03-22 15:51:22}
\pmmodified{2013-03-22 15:51:22}
\pmowner{cvalente}{11260}
\pmmodifier{cvalente}{11260}
\pmtitle{example of function not Lebesgue Measurable with measurable level sets}
\pmrecord{7}{37841}
\pmprivacy{1}
\pmauthor{cvalente}{11260}
\pmtype{Example}
\pmcomment{trigger rebuild}
\pmclassification{msc}{28B15}
\pmrelated{measurableFunctions}
\pmrelated{VitalisTheorem}
\pmrelated{MeasurableFunctions}

\endmetadata

% this is the default PlanetMath preamble.  as your knowledge
% of TeX increases, you will probably want to edit this, but
% it should be fine as is for beginners.

% almost certainly you want these
\usepackage{amssymb}
\usepackage{amsmath}
\usepackage{amsfonts}

% used for TeXing text within eps files
%\usepackage{psfrag}
% need this for including graphics (\includegraphics)
%\usepackage{graphicx}
% for neatly defining theorems and propositions
%\usepackage{amsthm}
% making logically defined graphics
%%%\usepackage{xypic}

% there are many more packages, add them here as you need them

% define commands here
\begin{document}
Consider $V$ as in Vitali's theorem. Define the function $f: [0,1] \to [0,+\infty[$ by:

\begin{displaymath}
f(x) = 
\begin{cases}
x & \text{if}\: x \notin V\\
2+x & \text{if}\: x \in V
\end{cases}
\end{displaymath}

The level sets of $f$ will either be the empty set, or a singleton  and thus measurable.

\begin{displaymath}
f^{-1} \left( \left\{ x \right\} \right) =
\begin{cases}
\{x\}   & \text{if}\: 0 \le x \le 1 \wedge x \notin V\\
\{2-x\} & \text{if}\: 2 \le x \le 3 \wedge x-2 \in V\\
\{\}    & \text{otherwise}
\end{cases}
\end{displaymath}

$f$ is not a measurable function since $f^{-1}([2,+\infty[) = V$ and $V$ is not a measurable set.
%%%%%
%%%%%
\end{document}
