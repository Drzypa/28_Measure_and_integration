\documentclass[12pt]{article}
\usepackage{pmmeta}
\pmcanonicalname{Partition1}
\pmcreated{2013-03-22 15:57:50}
\pmmodified{2013-03-22 15:57:50}
\pmowner{Wkbj79}{1863}
\pmmodifier{Wkbj79}{1863}
\pmtitle{partition}
\pmrecord{8}{37978}
\pmprivacy{1}
\pmauthor{Wkbj79}{1863}
\pmtype{Definition}
\pmcomment{trigger rebuild}
\pmclassification{msc}{28-00}
\pmclassification{msc}{26A42}
\pmsynonym{subinterval partition}{Partition1}
\pmrelated{Subinterval}

\endmetadata

% this is the default PlanetMath preamble.  as your knowledge
% of TeX increases, you will probably want to edit this, but
% it should be fine as is for beginners.

% almost certainly you want these
\usepackage{amssymb}
\usepackage{amsmath}
\usepackage{amsfonts}

% used for TeXing text within eps files
%\usepackage{psfrag}
% need this for including graphics (\includegraphics)
%\usepackage{graphicx}
% for neatly defining theorems and propositions
%\usepackage{amsthm}
% making logically defined graphics
%%%\usepackage{xypic}

% there are many more packages, add them here as you need them

% define commands here

\begin{document}
Let $a,b \in \mathbb{R}$ with $a<b$.  A {\sl partition\/} of an interval $[a,b]$ is a set of nonempty subintervals $\{ [a,x_1), [x_1,x_2), \dots , [x_{n-1}, b] \}$ for some positive integer $n$.  That is, $a<x_1<x_2<\dots<x_{n-1}<b$.  Note that $n$ is the number of subintervals in the partition.

Subinterval partitions are useful for defining Riemann integrals.

Note that subinterval partition is a specific case of a \PMlinkname{partition}{Partition} of a set since the subintervals are defined so that they are pairwise disjoint.
%%%%%
%%%%%
\end{document}
