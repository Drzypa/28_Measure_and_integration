\documentclass[12pt]{article}
\usepackage{pmmeta}
\pmcanonicalname{AbsolutelyContinuous}
\pmcreated{2013-03-22 13:26:12}
\pmmodified{2013-03-22 13:26:12}
\pmowner{Koro}{127}
\pmmodifier{Koro}{127}
\pmtitle{absolutely continuous}
\pmrecord{10}{33997}
\pmprivacy{1}
\pmauthor{Koro}{127}
\pmtype{Definition}
\pmcomment{trigger rebuild}
\pmclassification{msc}{28A12}
\pmrelated{RadonNikodymTheorem}
\pmrelated{AbsolutelyContinuousFunction2}
\pmdefines{absolute continuity}

% this is the default PlanetMath preamble.  as your knowledge
% of TeX increases, you will probably want to edit this, but
% it should be fine as is for beginners.

% almost certainly you want these
\usepackage{amssymb}
\usepackage{amsmath}
\usepackage{amsfonts}

% used for TeXing text within eps files
%\usepackage{psfrag}
% need this for including graphics (\includegraphics)
%\usepackage{graphicx}
% for neatly defining theorems and propositions
%\usepackage{amsthm}
% making logically defined graphics
%%%\usepackage{xypic}
% there are many more packages, add them here as you need them
\usepackage{mathrsfs}

% define commands here
\begin{document}
Let $\mu$ and $\nu$ be signed measures or complex measures on the same measurable space
$(\Omega, \mathscr{S})$. We say that $\nu$ is \emph{absolutely continuous}
with respect to $\mu$ if, for each $A\in \mathscr{S}$ such that $|\mu|(A)=0$,
it holds that $\nu(A)=0$. This is usually denoted by $\nu \ll \mu$.

\textbf{Remarks.}

If $\mu$ and $\nu$ are signed measures and $(\nu^+, \nu^-)$ is the Jordan decomposition of $\nu$, the following \PMlinkescapetext{propositions} are equivalent:
\begin{enumerate}
\item $\nu\ll\mu$;
\item $\nu^+\ll\mu$ and $\nu^-\ll\mu$;
\item $|\nu|\ll|\mu|$.
\end{enumerate}

If $\nu$ is a finite signed or complex measure and $\nu\ll\mu$, the following useful property holds: for each $\varepsilon>0$, there is a $\delta>0$ such that 
$|\nu|(E)<\varepsilon$ whenever $|\mu|(E)<\delta$.
%%%%%
%%%%%
\end{document}
