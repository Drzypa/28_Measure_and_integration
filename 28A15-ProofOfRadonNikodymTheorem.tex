\documentclass[12pt]{article}
\usepackage{pmmeta}
\pmcanonicalname{ProofOfRadonNikodymTheorem}
\pmcreated{2013-03-22 18:58:03}
\pmmodified{2013-03-22 18:58:03}
\pmowner{Ziosilvio}{18733}
\pmmodifier{Ziosilvio}{18733}
\pmtitle{proof of Radon-Nikodym theorem}
\pmrecord{5}{41827}
\pmprivacy{1}
\pmauthor{Ziosilvio}{18733}
\pmtype{Proof}
\pmcomment{trigger rebuild}
\pmclassification{msc}{28A15}
\pmsynonym{Hilbert spaces proof of Radon-Nikodym's theorem}{ProofOfRadonNikodymTheorem}
\pmsynonym{measure- theoretic proof of Radon-Nikodym theorem}{ProofOfRadonNikodymTheorem}

% this is the default PlanetMath preamble.  as your knowledge
% of TeX increases, you will probably want to edit this, but
% it should be fine as is for beginners.

% almost certainly you want these
\usepackage{amssymb}
\usepackage{amsmath}
\usepackage{amsfonts}

% used for TeXing text within eps files
%\usepackage{psfrag}
% need this for including graphics (\includegraphics)
%\usepackage{graphicx}
% for neatly defining theorems and propositions
%\usepackage{amsthm}
% making logically defined graphics
%%%\usepackage{xypic}

% there are many more packages, add them here as you need them

% define commands here

\begin{document}
The following proof of Radon-Nikodym theorem
is based on the original argument by John von Neumann.
We suppose that $\mu$ and $\nu$ are real, nonnegative, and finite.
The extension to the $\sigma$-finite case is a standard exercise,
as is $\mu$-a.e. uniqueness of Radon-Nikodym derivative.
Having done this, the thesis also holds for signed and complex-valued measures.

Let $(X,\mathcal{F})$ be a measurable space
and let $\mu,\nu:\mathcal{F}\to[0,R]$ two finite measures on $X$
such that $\nu(A)=0$ for every $A\in\mathcal{F}$ such that $\mu(A)=0$.
Then $\sigma=\mu+\nu$ is a finite measure on $X$
such that $\sigma(A)=0$ if and only if $\mu(A)=0$.

Consider the linear functional
\begin{math}
T:L^2(X,\mathcal{F},\sigma)\to\mathbb{R}
\end{math}
defined by
\begin{equation} \label{eq:T}
Tu=\int_Xu\;d\mu\;\;\forall u\in L^2(X,\mathcal{F},\sigma)\;.
\end{equation}
$T$ is well-defined
because $\mu$ is finite and dominated by $\sigma$, so that
\begin{math}
L^2(X,\mathcal{F},\sigma)
\subseteq L^2(X,\mathcal{F},\mu)
\subseteq L^1(X,\mathcal{F},\mu);
\end{math}
it is also linear and bounded because
\begin{math}
|Tu|\leq\|u\|_{L^2(X,\mathcal{F},\sigma)}\cdot\sqrt{\sigma(X)}.
\end{math}
By Riesz representation theorem, there exists $g\in L^2(X,\mathcal{F},\sigma)$ such that
\begin{equation} \label{eq:g}
Tu=\int_Xu\;d\mu=\int_Xu\cdot g\,d\sigma
\end{equation}
for every $u\in L^2(X,\mathcal{F},\sigma)$.
Then
\begin{math}
\mu(A)=\int_Ag\,d\sigma
\end{math}
for every $A\in\mathcal{F}$,
so that $0<g\leq 1$ $\mu$- and $\sigma$-a.e.
(Consider the former with $A=\{x\mid g(x)\leq 0\}$ or $A=\{x\mid g(x)>1\}$.)
Moreover, the second equality in (\ref{eq:q})
holds when $u=\chi_A$ for $A\in\mathcal{F}$,
thus also when $u$ is a simple measurable function
by linearity of integral,
and finally when $u$ is a ($\mu$- and $\sigma$-a.e.)
nonnegative $\mathcal{F}$-measurable function
because of the monotone convergence theorem.

Now, $1/g$ is $\mathcal{F}$-measurable
and nonnegative $\mu$- and $\sigma$-a.e.;
moreover, $\dfrac{1}{g}\cdot g=1$ $\sigma$- and $\mu$-a.e.
Thus, for every $A\in\mathcal{F}$,
\begin{equation} \label{eq:1/g}
\int_A\frac{1}{g}\,d\mu
=\int_Ad\sigma
=\sigma(A)
\end{equation}
Since $\sigma$ is finite, $1/g\in L^1(X,\mathcal{F},\mu)$,
and so is $f=\dfrac{1}{g}-1$.
Then for every $A\in\mathcal{F}$
\begin{displaymath}
\nu(A)=\sigma(A)-\mu(A)
=\int_A\left(\frac{1}{g}-1\right)d\mu
=\int_Af\,d\mu\,.
\end{displaymath}

%%%%%
%%%%%
\end{document}
