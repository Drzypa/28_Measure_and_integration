\documentclass[12pt]{article}
\usepackage{pmmeta}
\pmcanonicalname{UniformlyIntegrable}
\pmcreated{2013-03-22 15:22:55}
\pmmodified{2013-03-22 15:22:55}
\pmowner{stevecheng}{10074}
\pmmodifier{stevecheng}{10074}
\pmtitle{uniformly integrable}
\pmrecord{23}{37211}
\pmprivacy{1}
\pmauthor{stevecheng}{10074}
\pmtype{Definition}
\pmcomment{trigger rebuild}
\pmclassification{msc}{28A20}
\pmsynonym{uniform integrability}{UniformlyIntegrable}
\pmsynonym{uniform absolute continuity}{UniformlyIntegrable}
\pmrelated{VitaliConvergenceTheorem}
\pmrelated{ConditionalExpectationsAreUniformlyIntegrable}

% this is the default PlanetMath preamble.  as your knowledge
% of TeX increases, you will probably want to edit this, but
% it should be fine as is for beginners.

% almost certainly you want these
\usepackage{amssymb}
\usepackage{amsmath}
\usepackage{amsfonts}

% used for TeXing text within eps files
%\usepackage{psfrag}
% need this for including graphics (\includegraphics)
%\usepackage{graphicx}
% for neatly defining theorems and propositions
%\usepackage{amsthm}
% making logically defined graphics
%%%\usepackage{xypic}

% there are many more packages, add them here as you need them

% define commands here
\newcommand{\Le}{\mathbf{L}}
\newcommand{\real}{\mathbb{R}}
\newcommand{\E}{\mathbb{E}}
\providecommand{\abs}[1]{\lvert#1\rvert}

\providecommand{\absB}[1]{\Bigl\lvert#1\Bigr\rvert}
\providecommand{\defnterm}[1]{\emph{#1}}
\begin{document}
Let $\mu$ be a positive measure on a measurable space.
A collection of functions $\{ f_\alpha \} \subset \Le^1(\mu)$
is \defnterm{uniformly integrable}, if for every $\epsilon > 0$, there
exists $\delta > 0$ such that
\begin{align*}
\absB{ \int_E f_\alpha \, d\mu } < \epsilon \quad \textrm{whenever $\mu(E) < \delta$, for any $\alpha$.}
\end{align*}

(The absolute value sign outside of the integral above may appear under the integral sign instead without affecting the definition.)


The usefulness of this definition comes from the Vitali convergence theorem,
which uses it to characterize the convergence of functions
in $\Le^1(\mu)$.

\subsection*{Definition in probability theory}

In probability \PMlinkescapetext{theory}, a different, and slightly stronger, definition of ``uniform integrability'', is more commonly used:

A collection of functions $\{ f_\alpha \} \subset \Le^1(\mu)$
is \defnterm{uniformly integrable}, if for every $\epsilon > 0$, there
exists $t \geq 0$ such that
%\mathbb{E} \bigl[ \abs{f_\alpha} \, 1_{[\abs{f_\alpha} \geq t]} \bigr] =
\begin{align*}
\int_{[\abs{f_\alpha} \geq t]} \abs{f_\alpha}  \, d\mu < \epsilon
\quad \textrm{for every $\alpha$.}
\end{align*}

Assuming $\mu$ is a probability measure, this definition is equivalent
to the previous one together with the condition that $\int \abs{f_\alpha} \, d\mu$ is uniformly bounded for all $\alpha$.

\subsection*{Properties}

\begin{enumerate}
\item
If a finite number of collections are uniformly integrable,
then so is their finite union.
\item
A single $f \in \Le^1(\mu)$ is always uniformly integrable.

To see this, observe that $f$ must be almost everywhere non-infinite.
Thus $f \cdot 1_{[ \abs{f} > k]}$ goes to zero a.e. as $k \to \infty$, and it is bounded
by $\abs{f}$.
Then $\int_{[\abs{f} > k]} \abs{f}d\mu \to 0$ by the dominated convergence theorem.
Choosing $k$ big enough so that $\int_{[\abs{f} > k]} \abs{f}d\mu < \epsilon$, and letting $\delta = \epsilon/k$, we have, when $\mu(E) < \delta$,
\begin{align*}
\int_E \abs{f}d\mu= \int_{E \cap [\abs{f} \leq k]} \abs{f}d\mu+ \int_{E \cap [\abs{f} > k]} \abs{f} d\mu
\leq k \mu(E) + \epsilon = 2\epsilon\,.
\end{align*}
\end{enumerate}

\subsection*{Examples}
\begin{enumerate}
\item
If $g$ is an integrable function, then 
the collection consisting of
all measurable functions $f$ dominated by $g$ --- that is, $\abs{f} \leq g$ ---
is uniformly integrable.

\item
If $X$ is a $\Le^1$ random variable on a probability space $\Omega$,
then the set of all of its conditional expectations, \[
\{ \E[X \mid \mathcal{G}] \colon \mathcal{G}\text{ is a $\sigma$-algebra of $\Omega$} \}\,,
\]
 is always uniformly integrable.

\item
If there is an unbounded increasing function $\phi\colon [0, \infty) \to [0, \infty)$ such that 
\[
\int \abs{f_\alpha} \phi(\abs{f_\alpha}) \, d\mu
\]
is uniformly bounded for all $\alpha$,
then the collection $\{ f_\alpha \}$ is uniformly integrable.

\end{enumerate}

\begin{thebibliography}{3}
\bibitem{Chung}
Kai Lai Chung. {\it A Course in Probability Theory}, third ed. Academic Press, 2001.

\bibitem{Folland}
Gerald B. Folland. {\it Real Analysis: Modern Techniques and Their Applications}, second ed. Wiley-Interscience, 1999.

\bibitem{Rosenthal}
Jeffrey S. Rosenthal. {\it A First Look at Rigorous Probability Theory}.
World Scientific, 2003.

\end{thebibliography}

%%%%%
%%%%%
\end{document}
