\documentclass[12pt]{article}
\usepackage{pmmeta}
\pmcanonicalname{CounterexampleOfFubinisTheoremForTheLebesgueIntegral}
\pmcreated{2013-03-22 18:18:15}
\pmmodified{2013-03-22 18:18:15}
\pmowner{rmilson}{146}
\pmmodifier{rmilson}{146}
\pmtitle{counter-example of Fubini's theorem for the Lebesgue integral}
\pmrecord{6}{40924}
\pmprivacy{1}
\pmauthor{rmilson}{146}
\pmtype{Example}
\pmcomment{trigger rebuild}
\pmclassification{msc}{28A35}

\usepackage{amsmath}
\usepackage{amsfonts}
\usepackage{amssymb}
\newcommand{\reals}{\mathbb{R}}
\newcommand{\natnums}{\mathbb{N}}
\newcommand{\cnums}{\mathbb{C}}
\newcommand{\znums}{\mathbb{Z}}
\newcommand{\lp}{\left(}
\newcommand{\rp}{\right)}
\newcommand{\lb}{\left[}
\newcommand{\rb}{\right]}
\newcommand{\supth}{^{\text{th}}}
\newtheorem{proposition}{Proposition}
\newtheorem{definition}[proposition]{Definition}

\newtheorem{theorem}[proposition]{Theorem}
\begin{document}
The following observation demonstrates the necessity of the
integrability assumption in Fubini's theorem.  Let \[Q= \{ (x,y)\in
\mathbb{R}^2: x\geq0, y\geq 0\}\] denote the upper, right quadrant.
Let $R\subset Q$ be the region in the quadrant bounded by the lines
$y=x, y=x-1$, and let let $S\subset Q$ be a similar region, but this
time bounded by the lines $y=x-1,\; y=x-2$.  Let \[f = \chi_{S}-
\chi_R,\] where $\chi$ denotes a characteristic function.

Observe that the Lebesgue measure of $R$ and of $S$ is infinite.
Hence, $f$ is not a Lebesgue-integrable function.  However for every
$x\geq 0$ the function
\[ g(x) = \int_0^\infty f(x,y)\, dy \]
is integrable.
Indeed,
\[ g(x) = \left\{
  \begin{array}{cl}
    -x & \mbox{ for } 0 \leq x \leq 1,\\
    x-2 & \mbox{ for } 1\leq x \leq 2,\\
    0 & \mbox{ for } x\geq 2.
  \end{array} \right.
\]
Similarly, for $y\geq 0$, the function
\[ h(y) = \int_0^\infty f(x,y)\, dx\]
is integrable.  Indeed, 
\[ h(y) = 0,\quad y\geq 0.\]
Hence, the values of the iterated integrals  
$$ \int_0^\infty g(x) \, dx =   -1,$$
$$ \int_0^\infty h(y)\, dy =  0,$$
are finite, but do not agree.  This does not contradict Fubini's
theorem because the value of the planar Lebesgue integral 
\[ \int_Q f(x,y)\, d\mu(x,y), \]
where $\mu(x,y)$ is the planar Lebesgue measure, is not defined.
%%%%%
%%%%%
\end{document}
