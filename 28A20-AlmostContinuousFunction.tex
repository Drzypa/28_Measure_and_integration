\documentclass[12pt]{article}
\usepackage{pmmeta}
\pmcanonicalname{AlmostContinuousFunction}
\pmcreated{2013-03-22 16:13:45}
\pmmodified{2013-03-22 16:13:45}
\pmowner{Wkbj79}{1863}
\pmmodifier{Wkbj79}{1863}
\pmtitle{almost continuous function}
\pmrecord{4}{38329}
\pmprivacy{1}
\pmauthor{Wkbj79}{1863}
\pmtype{Definition}
\pmcomment{trigger rebuild}
\pmclassification{msc}{28A20}
\pmsynonym{almost continuous}{AlmostContinuousFunction}
\pmrelated{LusinsTheorem2}

\endmetadata

\usepackage{amssymb}
\usepackage{amsmath}
\usepackage{amsfonts}

\usepackage{psfrag}
\usepackage{graphicx}
\usepackage{amsthm}
%%\usepackage{xypic}

\begin{document}
Let $m$ denote Lebesgue measure, $A$ be a Lebesgue measurable subset of $\mathbb{R}$, and $f:A \to \mathbb{C}$ (or $f:A \to \mathbb{R}$).  Then $f$ is \emph{almost continuous} if, for every $\varepsilon>0$, there exists a closed subset $F$ of $\mathbb{R}$ such that $F \subseteq A$, $m(A-F)<\varepsilon$, and $f|_F$ is continuous.
%%%%%
%%%%%
\end{document}
