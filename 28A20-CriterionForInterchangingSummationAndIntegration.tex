\documentclass[12pt]{article}
\usepackage{pmmeta}
\pmcanonicalname{CriterionForInterchangingSummationAndIntegration}
\pmcreated{2013-03-22 16:20:05}
\pmmodified{2013-03-22 16:20:05}
\pmowner{rspuzio}{6075}
\pmmodifier{rspuzio}{6075}
\pmtitle{criterion for interchanging summation and integration}
\pmrecord{9}{38465}
\pmprivacy{1}
\pmauthor{rspuzio}{6075}
\pmtype{Result}
\pmcomment{trigger rebuild}
\pmclassification{msc}{28A20}

\endmetadata

% this is the default PlanetMath preamble.  as your knowledge
% of TeX increases, you will probably want to edit this, but
% it should be fine as is for beginners.

% almost certainly you want these
\usepackage{amssymb}
\usepackage{amsmath}
\usepackage{amsfonts}

% used for TeXing text within eps files
%\usepackage{psfrag}
% need this for including graphics (\includegraphics)
%\usepackage{graphicx}
% for neatly defining theorems and propositions
%\usepackage{amsthm}
% making logically defined graphics
%%%\usepackage{xypic}

% there are many more packages, add them here as you need them

% define commands here

\begin{document}
The following criterion for interchanging integration and summation
is often useful in practise:  Suppose one has a sequence of measurable
functions $f_k \colon M \to \mathbb{R}$ (The index $k$ runs over 
non-negative integers.) on some measure space $M$ 
and can find another sequence of measurable
functions $g_k \colon M \to \mathbb{R}$ such that $|f_k (x)| \le
g_k (x)$ for all $k$ and almost all $x$ and $\sum_{k=0}^\infty g_k(x)$ 
converges for almost all $x \in M$ and $\sum_{k=0}^\infty \int
g_k(x) \, dx < \infty$.  Then
 \[\int_M \sum_{k=0}^\infty f_k(x) \, dx = \sum_{k=0}^\infty 
 \int_M f_k(x) \, dx\]

This criterion is a corollary of the monotone and dominated
convergence theorems.  Since the $g_k$'s are nonnegative, the
sequence of partial sums is increasing, hence, by the monotone
convergence theorem, $\int_M \sum_{k=0}^\infty g_k(x) \, dx < \infty$. 
Since 
$\sum_{k=0}^\infty g_k(x)$ converges for almost all $x$, 
 \[ \left| \sum_{k=0}^n f_k(x) \right |\le  \sum_{k=0}^n |f_k(x)| \le
 \sum_{k=0}^n g_k(x) \le \sum_{k=0}^\infty g_k(x),\]
the dominated convergence theorem implies that we may integrate
the sequence of partial sums term-by-term, which is tantamount to
saying that we may switch integration and summation.

As an example of this method, consider the following:
 \[\int_{-\infty}^{+\infty} \sum_{k=1}^\infty {\cos (x/k) \over 
x^2 + k^4} \, dx\]
The idea behind the method is to pick our $g$'s as simple as possible
so that it is easy to integrate them and apply the criterion.  A good
choice here is $g_k (x) = 1 / (x^2 + k^4)$.  We then have $\int_{-\infty}^{+\infty} g_k (x) \, dx = \pi / k^2$ and, as
$\sum_{k=1}^\infty k^{-2} < \infty$, we can interchange summation
and integration:
 \[\sum_{k=1}^\infty \int_{-\infty}^{+\infty} {\cos (x/k) \over 
x^2 + k^4} \, dx.\]
Doing the integrals, we obtain the answer
 \[\pi \sum_{k=1}^\infty {e^{-k} \over k^2}\]
%%%%%
%%%%%
\end{document}
