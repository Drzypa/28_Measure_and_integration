\documentclass[12pt]{article}
\usepackage{pmmeta}
\pmcanonicalname{ProofOfExtendingACapacityToACartesianProduct}
\pmcreated{2013-03-22 18:47:41}
\pmmodified{2013-03-22 18:47:41}
\pmowner{gel}{22282}
\pmmodifier{gel}{22282}
\pmtitle{proof of extending a capacity to a Cartesian product}
\pmrecord{5}{41593}
\pmprivacy{1}
\pmauthor{gel}{22282}
\pmtype{Proof}
\pmcomment{trigger rebuild}
\pmclassification{msc}{28A12}
\pmclassification{msc}{28A05}
%\pmkeywords{capacity}
%\pmkeywords{compact paved space}

% almost certainly you want these
\usepackage{amssymb}
\usepackage{amsmath}
\usepackage{amsfonts}

% used for TeXing text within eps files
%\usepackage{psfrag}
% need this for including graphics (\includegraphics)
%\usepackage{graphicx}
% for neatly defining theorems and propositions
\usepackage{amsthm}
% making logically defined graphics
%%%\usepackage{xypic}

% there are many more packages, add them here as you need them

% define commands here
\newtheorem*{theorem*}{Theorem}
\newtheorem*{lemma*}{Lemma}
\newtheorem*{corollary*}{Corollary}
\newtheorem*{definition*}{Definition}
\newtheorem{theorem}{Theorem}
\newtheorem{lemma}{Lemma}
\newtheorem{corollary}{Corollary}
\newtheorem{definition}{Definition}

\begin{document}
\PMlinkescapeword{finite}
\PMlinkescapeword{compact}
\PMlinkescapeword{closure}
\PMlinkescapeword{paving}
\PMlinkescapeword{property}
\PMlinkescapeword{satisfies}
\PMlinkescapeword{increasing}
\PMlinkescapeword{sequence}
\PMlinkescapeword{decreasing}
\PMlinkescapeword{projection}
\PMlinkescapeword{compact paving}
\PMlinkescapeword{inequality}
\PMlinkescapeword{compactness}
\PMlinkescapeword{set function}

Let $(X,\mathcal{F})$ be a paved space such that $\mathcal{F}$ is closed under finite unions and finite intersections, and $(K,\mathcal{K})$ be a compact paved space.
Define $\mathcal{G}$ to be the closure under finite unions and finite intersections of the paving $\mathcal{F}\times\mathcal{K}$ on $X\times K$.
For an $\mathcal{F}$-capacity $I$, define
\begin{align*}
&\tilde I\colon\mathcal{P}(X\times K)\to\mathbb{R},\\
&\tilde I(S) = I(\pi_X(S)),
\end{align*}
where $\pi_X$ is the projection map onto $X$. We show that $\tilde I$ is a $\mathcal{G}$-capacity and that $\pi_X(S)\in\mathcal{F}_\delta$ whenever $S\in\mathcal{G}_\delta$.

Clearly, the property that $\tilde I$ is an increasing set function follows from the fact that $I$ satisfies this property. Furthermore, if $S_n\subseteq X\times K$ is an increasing sequence of sets with $S=\bigcup_nS_n$ then $\pi_X(S_n)$ is an increasing sequence and
\begin{equation*}
\tilde I(S)=I(\pi_X(S))=I\left(\bigcup_n\pi_X(S_n)\right)=\lim_{n\rightarrow\infty}I(\pi_X(S_n))=\lim_{n\rightarrow\infty}\tilde I(S_n).
\end{equation*}

To prove that $\tilde I$ is a $\mathcal{G}$-capacity, it only remains to show that if $S_n$ is a sequence in $\mathcal{G}$ decreasing to $S\subseteq X\times K$ then $\tilde I(S_n)\rightarrow \tilde I(S)$.
Note that any $S$ in $\mathcal{G}$ can be written as $S=\bigcap_{j=1}^m\bigcup_{k=1}^{n_j}A_{j,k}\times K_{j,k}$ for sets $A_{j,k}\in\mathcal{F}$ and $K_{j,k}\in\mathcal{K}$. The projection onto $X$ is then
\begin{equation*}
\pi_X(S)=\bigcup\left\{\bigcap_{j=1}^m A_{j,k_j}\colon k_j\le n_j,\ \bigcap_{j=1}^mK_{j,k_j}\not=\emptyset\right\}
\end{equation*}
which, as $\mathcal{F}$ is closed under finite unions and finite intersections, must be in $\mathcal{F}$.
Furthermore, for any $x\in X$,
\begin{equation*}
S_x\equiv\left\{y\in K\colon (x,y)\in S\right\}
=\bigcap_{j=1}^m\bigcup\left\{K_{j,k}\colon k\le n_j, x\in A_{j,k} \right\}.
\end{equation*}
This shows that $S_x$ is in the closure $\mathcal{K}^*$ of $\mathcal{K}$ under finite unions and finite intersections. Furthermore, since \PMlinkname{compact pavings are closed subsets of a compact topological space}{CompactPavingsAreClosedSubsetsOfACompactSpace}, $\mathcal{K}^*$ is itself a compact paving.

Now let $S_n$ be a decreasing sequence of sets in $\mathcal{G}$ and set $S=\bigcap_nS_n$. Then $\pi_X(S)\subseteq\pi_X(S_n)$ for each $n$, giving $\pi_X(S)\subseteq\bigcap_n\pi_X(S_n)$. To prove the reverse inequality, consider $x\in\bigcap_n\pi_X(S_n)$. Then, $(S_n)_x$ is a nonempty set in $\mathcal{K}^*$ for all $n$. By compactness, $S_x=\bigcap_n(S_n)_x$ must also be nonempty and therefore $x\in \pi_X(S)$. This shows that
\begin{equation*}
\bigcap_n\pi_X(S_n)=\pi_X(S).
\end{equation*}
Furthermore, as we have shown that $\pi_X(S_n)\in\mathcal{F}$ and, as $I$ is an $\mathcal{F}$-capacity,
\begin{equation*}
\tilde I(S_n)=I(\pi_X(S_n))\rightarrow I(\pi_X(S))=\tilde I(S).
\end{equation*}
So $\tilde I$ is a $\mathcal{G}$-capacity.

We finally show that if $S\in\mathcal{G}_\delta$ then $\pi_X(S)\in \mathcal{F}_\delta$. By definition, there is a sequence $S_n\in\mathcal{G}$ such that $S=\bigcap_nS_n$. Setting $S^\prime_n=\bigcap_{m\le n}S_m$ then, since $\mathcal{G}$ is closed under finite unions and finite intersections, $S^\prime_n\in\mathcal{G}$. Furthermore, $S^\prime_n$ decreases to $S$ so, as shown above, $\pi_X(S^\prime_n)\in\mathcal{F}$ and
\begin{equation*}
\pi_X(S)=\bigcap_n\pi_X(S^\prime_n)\in\mathcal{F}_\delta
\end{equation*}
as required.

%%%%%
%%%%%
\end{document}
