\documentclass[12pt]{article}
\usepackage{pmmeta}
\pmcanonicalname{ProofOfCapacityGeneratedByAMeasure}
\pmcreated{2013-03-22 18:47:55}
\pmmodified{2013-03-22 18:47:55}
\pmowner{gel}{22282}
\pmmodifier{gel}{22282}
\pmtitle{proof of capacity generated by a measure}
\pmrecord{5}{41598}
\pmprivacy{1}
\pmauthor{gel}{22282}
\pmtype{Proof}
\pmcomment{trigger rebuild}
\pmclassification{msc}{28A12}
\pmclassification{msc}{28A05}
%\pmkeywords{finite measure}
%\pmkeywords{capacity}

\endmetadata

% almost certainly you want these
\usepackage{amssymb}
\usepackage{amsmath}
\usepackage{amsfonts}

% used for TeXing text within eps files
%\usepackage{psfrag}
% need this for including graphics (\includegraphics)
%\usepackage{graphicx}
% for neatly defining theorems and propositions
\usepackage{amsthm}
% making logically defined graphics
%%%\usepackage{xypic}

% there are many more packages, add them here as you need them

% define commands here
\newtheorem*{theorem*}{Theorem}
\newtheorem*{lemma*}{Lemma}
\newtheorem*{corollary*}{Corollary}
\newtheorem*{definition*}{Definition}
\newtheorem{theorem}{Theorem}
\newtheorem{lemma}{Lemma}
\newtheorem{corollary}{Corollary}
\newtheorem{definition}{Definition}

\begin{document}
\PMlinkescapeword{sequence}
\PMlinkescapeword{increasing}
\PMlinkescapeword{decreasing}
\PMlinkescapeword{measures}
\PMlinkescapeword{subsets}
\PMlinkescapeword{inequality}
\PMlinkescapeword{subset}
\PMlinkescapeword{completion}

For a finite measure space $(X,\mathcal{F},\mu)$, define
\begin{align*}
&\mu^*\colon\mathcal{P}(X)\to\mathbb{R}_+,\\
&\mu^*(S)=\inf\left\{\mu(A)\colon A\in\mathcal{F},\ A\supseteq S\right\}.
\end{align*}
We show that $\mu^*$ is an $\mathcal{F}$-capacity and that a subset $S\subseteq X$ is $(\mathcal{F},\mu^*)$-\PMlinkname{capacitable}{ChoquetCapacity} if and only if it is in the \PMlinkname{completion}{CompleteMeasure} of $\mathcal{F}$ with respect to $\mu$.

Note, first of all, that $\mu^*(S)=\mu(S)$ for any $S\in\mathcal{F}$.
That $\mu^*$ is increasing follows directly from the definition. If $A_n\in\mathcal{F}$ is a decreasing sequence of sets then $A=\bigcap_nA_n$ is also in $\mathcal{F}$ and, by \PMlinkname{continuity from above}{PropertiesForMeasure} for measures,
\begin{equation*}
\mu^*(A_n)=\mu(A_n)\rightarrow\mu(A)=\mu^*(A)
\end{equation*}
as $n\rightarrow\infty$.

Now suppose that $S_n$ is an increasing sequence of subsets of $X$ and set $S=\bigcup_nS_n$. Then, $\mu^*(S_n)\le\mu^*(S)$ for each $n$ and, hence, $\lim_{n\rightarrow\infty}\mu^*(S_n)\le\mu^*(S)$.

To prove the reverse inequality, choose any $\epsilon>0$ and sequence $A_n\in\mathcal{F}$ with $S_n\subseteq A_n$ and $\mu(A_n)\le\mu^*(S_n)+2^{-n}\epsilon$.
Then, $A_m\cap A_n\supseteq S_n$ whenever $m\ge n$ and, therefore,
\begin{equation*}
\mu(A_n\setminus A_m) = \mu(A_n)-\mu(A_m\cap A_n)\le\mu(A_n)-\mu^*(S_n)\le 2^{-n}\epsilon.
\end{equation*}
Additivity of $\mu$ then gives
\begin{equation*}
\mu\left(\bigcup_{m\le n}A_m\right)
\le\mu(A_n)+ \sum_{m=1}^{n-1}\mu(A_m\setminus A_n)\le\mu^*(S_n)+\epsilon.
\end{equation*}
So, by continuity from below for measures,
\begin{equation*}
\mu^*(S)\le\mu\left(\bigcup_nA_n\right)
=\lim_{n\rightarrow\infty}\mu\left(\bigcup_{m\le n}A_m\right)\le\lim_{n\rightarrow\infty}\mu^*(S_n)+\epsilon.
\end{equation*}
Choosing $\epsilon$ arbitrarily small shows that $\mu^*(S_n)\rightarrow\mu^*(S)$ and, therefore, $\mu^*$ is indeed an $\mathcal{F}$-capacity.

Now suppose that $S$ is in the completion of $\mathcal{F}$ with respect to $\mu$, so that there exists $A,B\in\mathcal{F}$ with $A\subseteq S\subseteq B$ and $\mu(B\setminus A)=0$. Then,
\begin{equation*}
\mu^*(A)=\mu(A)=\mu(B)\ge\mu^*(S)
\end{equation*}
and $S$ is indeed $(\mathcal{F},\mu^*)$-capacitable.
Conversely, let $S$ be $(\mathcal{F},\mu^*)$-capacitable. Then, there exists $A_n,B_n\in\mathcal{F}$ such that $A_n\subseteq S\subseteq B_n$ and
\begin{equation*}
\mu(A_n)\ge\mu^*(S)-1/n,\ \mu(B_n)\le\mu^*(S)+1/n.
\end{equation*}
Setting $A=\bigcup_nA_n$ and $B=\bigcap_nB_n$ gives $A\subseteq S\subseteq B$ and
\begin{equation*}
\mu(A)\ge\mu^*(S)\ge\mu(B).
\end{equation*}
So $\mu(B\setminus A)=\mu(B)-\mu(A)=0$, as required.

%%%%%
%%%%%
\end{document}
