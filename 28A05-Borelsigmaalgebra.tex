\documentclass[12pt]{article}
\usepackage{pmmeta}
\pmcanonicalname{Borelsigmaalgebra}
\pmcreated{2013-03-22 12:00:31}
\pmmodified{2013-03-22 12:00:31}
\pmowner{djao}{24}
\pmmodifier{djao}{24}
\pmtitle{Borel $\sigma$-algebra}
\pmrecord{10}{30951}
\pmprivacy{1}
\pmauthor{djao}{24}
\pmtype{Definition}
\pmcomment{trigger rebuild}
\pmclassification{msc}{28A05}
\pmsynonym{Borel $\sigma$ algebra}{Borelsigmaalgebra}
\pmsynonym{Borel sigma algebra}{Borelsigmaalgebra}
\pmrelated{SigmaAlgebra}
\pmrelated{OuterRegular}
\pmrelated{LebesgueMeasure}
\pmdefines{Borel subset}
\pmdefines{Borel set}

\usepackage{amssymb}
\usepackage{amsmath}
\usepackage{amsfonts}
\usepackage{graphicx}
%%%\usepackage{xypic}
\begin{document}
For any topological space $X$, the \emph{Borel sigma algebra} of $X$ is the $\sigma$--algebra $\mathcal{B}$ generated by the open sets of $X$. In other words, the Borel sigma algebra is equal to the intersection of all sigma algebras $\mathcal{A}$ of $X$ having the property that every open set of $X$ is an element of $\mathcal{A}$.

An element of $\mathcal{B}$ is called a \emph{Borel subset} of $X$, or a \emph{Borel set}.


%%%%%
%%%%%
%%%%%
\end{document}
