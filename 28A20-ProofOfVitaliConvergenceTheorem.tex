\documentclass[12pt]{article}
\usepackage{pmmeta}
\pmcanonicalname{ProofOfVitaliConvergenceTheorem}
\pmcreated{2013-03-22 17:31:04}
\pmmodified{2013-03-22 17:31:04}
\pmowner{stevecheng}{10074}
\pmmodifier{stevecheng}{10074}
\pmtitle{proof of Vitali convergence theorem}
\pmrecord{5}{39909}
\pmprivacy{1}
\pmauthor{stevecheng}{10074}
\pmtype{Proof}
\pmcomment{trigger rebuild}
\pmclassification{msc}{28A20}

\endmetadata

% The standard font packages
\usepackage{amssymb}
\usepackage{amsmath}
\usepackage{amsfonts}

% For neatly defining theorems and definitions
\usepackage{amsthm}

% Including EPS/PDF graphics (\includegraphics)
%\usepackage{graphicx}

% Making matrix-based graphics
%%%\usepackage{xypic}

% Enumeration lists with different styles
\usepackage{enumerate}

% Set up the theorem environments
%\newtheorem{thm}{Theorem}
\newtheorem*{thm*}{Theorem}

\newcommand{\defnterm}[1]{\emph{#1}}

% The standard number systems
\newcommand{\complex}{\mathbb{C}}
\newcommand{\real}{\mathbb{R}}
\newcommand{\rat}{\mathbb{Q}}
\newcommand{\nat}{\mathbb{N}}
\newcommand{\intset}{\mathbb{Z}}

% Absolute values and norms
% Normal, wide, and big versions of the delimeters
\newcommand{\abs}[1]{\lvert#1\rvert}
\newcommand{\absW}[1]{\left\lvert#1\right\rvert}
\newcommand{\absB}[1]{\Bigl\lvert#1\Bigr\rvert}
\newcommand{\norm}[1]{\lVert#1\rVert}
\newcommand{\normW}[1]{\left\lVert#1\right\rVert}
\newcommand{\normB}[1]{\Bigl\lVert#1\Bigr\rVert}

\newcommand{\Le}{\mathbf{L}}
\newcommand{\indc}{\mathbf{1}}
\begin{document}
\PMlinkescapeword{entire}
\PMlinkescapeword{simple}
\PMlinkescapeword{argument}
\PMlinkescapeword{variation}
\PMlinkescapeword{words}


\begin{thm*}
Let $f_1, f_2, \dotsc$ be $\Le^p$-integrable functions on 
a measure space $(X, \mu)$, for $1 \leq p < \infty$.
The following conditions are necessary and sufficient for
$f_n$ to be a Cauchy sequence in the $\Le^p(X,\mu)$ norm:

\begin{enumerate}[(i)]
\item
the sequence $f_n$ is Cauchy in measure;
\item
the functions $\{ \abs{f_n}^p \}$ are uniformly integrable; and
\item
for each $\epsilon > 0$, there is a set $A$
of finite measure, with $\norm{f_n \indc(X \setminus A)} < \epsilon$
for all $n$.
\end{enumerate}
\end{thm*}

\begin{proof}
We abbreviate $\abs{f_n - f_m}$ by $f_{mn}$.

\begin{description}
\item[Necessity of (i).]
Fix $t > 0$, and let
$E_{mn} = \{ f_{mn} \geq t \}$.
Then
\[
\mu(E_{mn})^{1/p} = \frac{1}{t} \norm{ t \, \indc(E_{mn}) }
\leq \frac{1}{t} \norm{ f_{mn} } \to 0\,,
\quad
\text{as $m, n \to \infty$.}
\]

\item[Necessity of (ii).]
Select $N$ such that $\norm{f_n - f_N} < \epsilon$ when $n \geq N$.
The family $\{ \abs{f_1}^p, \dotsc, \abs{f_{N-1}}^p, \abs{f_N}^p \}$
is uniformly integrable
because it consists of only \emph{finitely} many integrable functions.

So for every $\epsilon > 0$, 
there is $\delta > 0$ such that $\mu(E) < \delta$
implies $\norm{f_n \indc(E)} < \epsilon$ for $n \leq N$.
On the other hand, for $n > N$,
\[
\norm{f_n \indc(E)} \leq \norm{ (f_n - f_N) \indc(E)} + \norm{f_N \indc(E)} < 2\epsilon
\]
for the same sets $E$,
and thus the entire infinite sequence $\{ \abs{f_n}^p \}$ is
uniformly integrable too.


\item[Necessity of (iii).]
Select $N$ such that $\norm{f_n - f_N} < \epsilon$
for all $n \geq N$.
Let $\varphi$ be a simple function approximating $f_N$ in $\Le^p$ norm up to $\epsilon$.
Then $\norm{f_n - \varphi} < 2\epsilon$ for all $n \geq N$.
Let $A_N = \{ \varphi \neq 0 \}$ be the support of $\varphi$, which must
have finite measure.
It follows that
\begin{align*}
\norm{f_n \indc(X \setminus A_N)} = \norm{f_n - f_n \indc(A_N)} 
&\leq \norm{f_n - \varphi} + \norm{\varphi - f_n \indc(A_N)} \\
&= \norm{f_n - \varphi} + \norm{(\varphi - f_n) \indc(A_N)} \\
&< 2\epsilon + 2\epsilon\,.
\end{align*}

For each $n < N$, we can similarly construct sets $A_n$
of finite measure, 
such that $\norm{f_n \indc(X \setminus A_n)} < 4\epsilon$.
If we set $A = A_1 \cup \dotsb \cup A_{N-1} \cup A_N$, a finite union,
then $A$ has finite measure, and clearly 
$\norm{f_n \indc(X \setminus A)} < 4\epsilon$ for any $n$.

\item[Sufficiency.]
We show $f_{mn}$ 
to be small for large $m,n$ by a multi-step estimate:
\begin{align*}
\norm{f_{mn}} 
&\leq 
\norm{f_{mn} \indc(A \setminus E_{mn})} + 
\norm{f_{mn} \indc(E_{mn})} +
\norm{f_{mn} \indc(X \setminus A)}\,.
\end{align*}

Use condition (iii) to choose $A$ of finite measure
such that $\norm{f_n \indc(X \setminus A)} < \epsilon$
for every $n$.
Then $\norm{f_{mn} \indc(X \setminus A)} < 2 \epsilon$.

Let $t = \epsilon/\mu(A)^{1/p} > 0$,
and $E_{mn} = \{ f_{mn} \geq t \}$.
By condition (ii) choose $\delta > 0$ so that 
$\norm{f_n \indc(E)} < \epsilon$ whenever
$\mu(E) < \delta$.
By condition (i), take $N$ such that if $m, n \geq N$,
then
$\mu(E_{mn}) < \delta$;
it follows immediately that $\norm{ f_{mn} \indc(E_{mn}) } < 2\epsilon$.

Finally, $\norm{f_{mn} \indc(A \setminus E_{mn})} \leq t \mu(A)^{1/p} = \epsilon$, since $f_{mn} < t$ on the complement of $E_{mn}$.
Hence $\norm{f_{mn}} < 5\epsilon$ for $m, n \geq N$.
\qedhere
\end{description}
\end{proof}

\noindent
\emph{Remark.}  In the statement of the theorem, instead 
of dealing with Cauchy sequences,
we can directly speak of convergence of $f_n$ to $f$
in $\Le^p$ and in measure.
This variation of the theorem
is easily proved,
for:
\begin{itemize}
\item
a sequence converges in $\Le^p$ if and only if it is
Cauchy in $\Le^p$;
\item
a sequence that converges in measure is automatically Cauchy in measure;
\item
a simple adaptation of the argument
shows that $f_n \to f$ in $\Le^p$ implies $f_n \to f$ in measure; and
\item
the limit in measure is unique.
\end{itemize}


%%%%%
%%%%%
\end{document}
