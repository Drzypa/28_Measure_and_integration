\documentclass[12pt]{article}
\usepackage{pmmeta}
\pmcanonicalname{InfiniteProductMeasure}
\pmcreated{2013-03-22 16:23:14}
\pmmodified{2013-03-22 16:23:14}
\pmowner{CWoo}{3771}
\pmmodifier{CWoo}{3771}
\pmtitle{infinite product measure}
\pmrecord{13}{38532}
\pmprivacy{1}
\pmauthor{CWoo}{3771}
\pmtype{Definition}
\pmcomment{trigger rebuild}
\pmclassification{msc}{28A35}
\pmclassification{msc}{60A10}
\pmrelated{ProductSigmaAlgebra}
\pmdefines{totally finite measure}

\usepackage{amssymb,amscd}
\usepackage{amsmath}
\usepackage{amsfonts}

% used for TeXing text within eps files
%\usepackage{psfrag}
% need this for including graphics (\includegraphics)
%\usepackage{graphicx}
% for neatly defining theorems and propositions
%\usepackage{amsthm}
% making logically defined graphics
%%\usepackage{xypic}
\usepackage{pst-plot}
\usepackage{psfrag}

% define commands here

\begin{document}
Let $(E_i,\mathcal{B}_i,\mu_i)$ be measure spaces, where $i\in I$ an index set, possibly infinite.  We define the \emph{product} of $(E_i,\mathcal{B}_i,\mu_i)$ as follows:
\begin{enumerate}
\item let $E=\prod E_i$, the Cartesian product of $E_i$,
\item let $\mathcal{B}=\sigma((\mathcal{B}_i)_{i\in I})$, the smallest sigma algebra containing subsets of $E$ of the form $\prod B_i$ where $B_i=E_i$ for all but a finite number of $i\in I$.
\end{enumerate}

Then $(E,\mathcal{B})$ is a measurable space.  The next task is to define a measure $\mu$ on $(E,\mathcal{B})$ so that $(E,\mathcal{B},\mu)$ becomes in addition a measure space.  Before proceeding to define $\mu$, we make the assumption that \begin{quote} each $\mu_i$ is a \emph{totally finite measure}, that is, $\mu_i(E_i)< \infty$.\end{quote}  In fact, we can now turn each $(E_i,\mathcal{B}_i,\mu_i)$ into a probability space by introducing for each $i\in I$ a new measure: $$\overline{\mu}_i=\frac{\mu_i}{\mu_i(E_i)}.$$

With the assumption that each $(E_i,\mathcal{B}_i,\mu_i)$ is a probability space, it can be shown that there is a \emph{unique} measure $\mu$ defined on $\mathcal{B}$ such that, for any $B\in \mathcal{B}$ expressible as a product of $B_i\in \mathcal{B}_i$ with $B_i=E_i$ for all $i\in I$ except on a finite subset $J$ of $I$:
$$\mu(B)=\prod_{j\in J} \mu_j(B_j).$$

Then $(E,\mathcal{B},\mu)$ becomes a measure space, and in particular, a probability space.  $\mu$ is sometimes written $\prod \mu_i$.

\textbf{Remarks}.
\begin{itemize}
\item If $I$ is infinite, one sees that the total finiteness of $\mu_i$ can not be dropped.  For example, if $I$ is the set of positive integers, assume $\mu_1(E_1)<\infty$ and $\mu_2(E_2)=\infty$.  Then $\mu(B)$ for $$B:=B_1\times \prod_{i>1}E_i = B_1\times E_2 \times \prod_{i>2}E_i \mbox{, where }B_1\in \mathcal{B}_1$$ would not be well-defined (on the one hand, it is $\mu_1(B_1)<\infty$, but on the other it is $\mu_1(B_1)\mu_2(E_2)=\infty$).
\item The above construction agrees with the result when $I$ is finite (see \PMlinkname{finite product measure}{ProductMeasure}).
\end{itemize}
%%%%%
%%%%%
\end{document}
