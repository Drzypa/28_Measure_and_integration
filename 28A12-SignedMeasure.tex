\documentclass[12pt]{article}
\usepackage{pmmeta}
\pmcanonicalname{SignedMeasure}
\pmcreated{2013-03-22 13:26:55}
\pmmodified{2013-03-22 13:26:55}
\pmowner{Koro}{127}
\pmmodifier{Koro}{127}
\pmtitle{signed measure}
\pmrecord{8}{34013}
\pmprivacy{1}
\pmauthor{Koro}{127}
\pmtype{Definition}
\pmcomment{trigger rebuild}
\pmclassification{msc}{28A12}

% this is the default PlanetMath preamble.  as your knowledge
% of TeX increases, you will probably want to edit this, but
% it should be fine as is for beginners.
\usepackage{mathrsfs}
% almost certainly you want these
\usepackage{amssymb}
\usepackage{amsmath}
\usepackage{amsfonts}

% used for TeXing text within eps files
%\usepackage{psfrag}
% need this for including graphics (\includegraphics)
%\usepackage{graphicx}
% for neatly defining theorems and propositions
%\usepackage{amsthm}
% making logically defined graphics
%%%\usepackage{xypic}

% there are many more packages, add them here as you need them

% define commands here
\begin{document}
A \emph{signed measure} on a measurable space $(\Omega,\mathscr{S})$ is a function $\mu:\mathscr{S}\rightarrow \mathbb{R}\cup\{+\infty\}$ which is \PMlinkname{$\sigma$-additive}{Additive} and such that $\mu(\emptyset)=0$.

\textbf{Remarks.} 
\begin{enumerate}
\item The usual (positive) measure is a particular case of signed measure, in which $|\mu| = \mu$ (see Jordan decomposition.)

\item Notice that the value $-\infty$ is not allowed. For some authors, a signed measure can only take finite values (so that $+\infty$ is not allowed either). This is sometimes useful because it turns the space of all signed measures into a normed vector space, with the natural operations, and the norm given by $\|\mu\| = |\mu|(\Omega)$.

\item An important example of signed measures arises from the usual measures in the following way: Let $(\Omega,\mathscr{S},\mu)$ be a measure space, and let $f$ be a (real valued) measurable function such that 
\[\int_{\{x\in \Omega:f(x)<0\}} |f| d\mu <\infty.\]
Then a signed measure is defined by
\[A\mapsto \int_A fd\mu.\]
\end{enumerate}
%%%%%
%%%%%
\end{document}
