\documentclass[12pt]{article}
\usepackage{pmmeta}
\pmcanonicalname{MeasurableFunction}
\pmcreated{2013-03-22 12:50:50}
\pmmodified{2013-03-22 12:50:50}
\pmowner{CWoo}{3771}
\pmmodifier{CWoo}{3771}
\pmtitle{measurable function}
\pmrecord{18}{33176}
\pmprivacy{1}
\pmauthor{CWoo}{3771}
\pmtype{Definition}
\pmcomment{trigger rebuild}
\pmclassification{msc}{28A20}
\pmsynonym{Borel measurable}{MeasurableFunction}
%\pmkeywords{measurable}
\pmrelated{ExampleOfFunctionNotLebesgueMeasurableWithMeasurableLevelSets}
\pmrelated{LusinsTheorem2}
\pmrelated{BorelGroupoid}
\pmrelated{BorelMorphism}
\pmdefines{Borel measurable function}

% this is the default PlanetMath preamble.  as your knowledge
% of TeX increases, you will probably want to edit this, but
% it should be fine as is for beginners.

% almost certainly you want these
\usepackage{amssymb}
\usepackage{amsmath}
\usepackage{amsfonts}

% used for TeXing text within eps files
%\usepackage{psfrag}
% need this for including graphics (\includegraphics)
%\usepackage{graphicx}
% for neatly defining theorems and propositions
%\usepackage{amsthm}
% making logically defined graphics
%%%\usepackage{xypic}

% there are many more packages, add them here as you need them

% define commands here
\begin{document}
Let $\big(X,\mathcal{B}(X)\big)$ and $\big(Y, \mathcal{B}(Y)\big)$ be two measurable spaces.  Then a function $f\colon X\to Y$ is called a \emph{measurable function} if:
$$f^{-1}\big(\mathcal{B}(Y)\big) \subseteq \mathcal{B}(X)$$
where $f^{-1}\big(\mathcal{B}(Y)\big) = \{f^{-1}(E)\mid E\in\mathcal{B}(Y)\}$.\\  

In other words,  the inverse image of every $\mathcal{B}(Y)$-measurable set is $\mathcal{B}(X)$-measurable.  The space of all measurable functions $f\colon X\to Y$ is denoted as $$\mathcal{M}\big(\big(X,\mathcal{B}(X)\big),\big(Y, \mathcal{B}(Y)\big)\big).$$  Any measurable function into $(\mathbb{R},\mathcal{B}(\mathbb{R}))$, where $\mathcal{B}(\mathbb{R})$ is the Borel sigma algebra of the real numbers $\mathbb{R}$, is called a \emph{Borel measurable function}.{\footnote {More generally, a measurable function is called \emph{Borel measurable} if the range space $Y$ is a topological space with $\mathcal{B}(Y)$ the sigma algebra generated by all open sets of $Y$.}}   The space of all Borel measurable functions from a measurable space $(X,\mathcal{B}(X))$ is denoted by  $\displaystyle{\mathcal{L}^0\big(X,\mathcal{B}(X)\big)}$.  

Similarly, we write $\displaystyle{\bar{\mathcal{L}}^0\big(X,\mathcal{B}(X)\big)}$ for  $\displaystyle{\mathcal{M}\big(\big(X,\mathcal{B}(X)), (\bar{\mathbb{R}},\mathcal{B}(\bar{\mathbb{R}})\big)\big)}$, where $\mathcal{B}(\bar{\mathbb{R}})$ is the Borel sigma algebra of $\bar{\mathbb{R}}$, the set of extended real numbers. \\

\textbf{Remark}.  If $f:X\to Y$ and $g:Y\to Z$ are measurable functions, then so is $g\circ f:X\to Z$, for if $E$ is $\mathcal{B}(Z)$-measurable, then $g^{-1}(E)$ is $\mathcal{B}(Y)$-measurable, and $f^{-1}\big(g^{-1}(E)\big)$ is $\mathcal{B}(X)$-measurable.  But $f^{-1}\big(g^{-1}(E)\big)=(g\circ f)^{-1}(E)$, which implies that $g\circ f$ is a measurable function.

\textbf{Example:}
\begin{itemize}
\item Let $E$ be a subset of a measurable space $X$.  Then the characteristic function $\chi_E$ is a measurable function if and only if $E$ is measurable.
\end{itemize}
%%%%%
%%%%%
\end{document}
