\documentclass[12pt]{article}
\usepackage{pmmeta}
\pmcanonicalname{SequenceOfSetsConvergence}
\pmcreated{2013-03-22 16:15:12}
\pmmodified{2013-03-22 16:15:12}
\pmowner{Andrea Ambrosio}{7332}
\pmmodifier{Andrea Ambrosio}{7332}
\pmtitle{sequence of sets convergence}
\pmrecord{8}{38358}
\pmprivacy{1}
\pmauthor{Andrea Ambrosio}{7332}
\pmtype{Definition}
\pmcomment{trigger rebuild}
\pmclassification{msc}{28A05}

\endmetadata

% this is the default PlanetMath preamble.  as your knowledge
% of TeX increases, you will probably want to edit this, but
% it should be fine as is for beginners.

% almost certainly you want these
\usepackage{amssymb}
\usepackage{amsmath}
\usepackage{amsfonts}

% used for TeXing text within eps files
%\usepackage{psfrag}
% need this for including graphics (\includegraphics)
%\usepackage{graphicx}
% for neatly defining theorems and propositions
%\usepackage{amsthm}
% making logically defined graphics
%%%\usepackage{xypic}

% there are many more packages, add them here as you need them

% define commands here

\begin{document}
Let $\left\{ A_{n}\right\} _{n=1}^{\infty }$ be a sequence of sets,
and $A$ a set.

The sequence $\left\{ A_{n}\right\} _{n=1}^{\infty }$ is said to \emph{\PMlinkescapetext{converge} from below} to $A$, (shortly, $A_{n}\uparrow A$ or $A_n \nearrow A$), iff \\
1) $A_{n}\subseteq A_{n+1}$ \ $\forall n\geq 1$ \\
2) $\displaystyle A=\bigcup_{n=1}^{\infty }A_{n}$

The sequence $\left\{ A_{n}\right\} _{n=1}^{\infty }$ is said to \emph{\PMlinkescapetext{converge} from above} to $A$, (shortly, $A_{n}\downarrow A$ or $A_n \searrow A$), iff \\
1) $A_{n+1}\subseteq A_{n}$ \ $\forall n\geq 1$ \\
2) $\displaystyle A=\bigcap_{n=1}^{\infty }A_{n}$


In both cases the less accurate notation%
\[
A=\lim_{n\longrightarrow \infty }A_{n}
\]

is also used.
%%%%%
%%%%%
\end{document}
