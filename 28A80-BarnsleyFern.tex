\documentclass[12pt]{article}
\usepackage{pmmeta}
\pmcanonicalname{BarnsleyFern}
\pmcreated{2013-03-22 16:05:33}
\pmmodified{2013-03-22 16:05:33}
\pmowner{paolini}{1187}
\pmmodifier{paolini}{1187}
\pmtitle{Barnsley fern}
\pmrecord{7}{38154}
\pmprivacy{1}
\pmauthor{paolini}{1187}
\pmtype{Example}
\pmcomment{trigger rebuild}
\pmclassification{msc}{28A80}

% this is the default PlanetMath preamble.  as your knowledge
% of TeX increases, you will probably want to edit this, but
% it should be fine as is for beginners.

% almost certainly you want these
\usepackage{amssymb}
\usepackage{amsmath}
\usepackage{amsfonts}

% used for TeXing text within eps files
%\usepackage{psfrag}
% need this for including graphics (\includegraphics)
\usepackage{graphicx}
% for neatly defining theorems and propositions
\usepackage{amsthm}
% making logically defined graphics
%%%\usepackage{xypic}

% there are many more packages, add them here as you need them

% define commands here
\newcommand{\R}{\mathbb R}
\newtheorem{theorem}{Theorem}
\newtheorem{definition}{Definition}
\theoremstyle{remark}
\newtheorem{example}{Example}
\begin{document}
The \emph{Barnsley fern} $F$ is the only non-empty compact subset of $\R^2$ satisfying
the relation
\[
  F = \bigcup_{i=1}^4 T_i(F)
\]
where $T_i\colon \R^2\to \R^2$ are the following linear mappings:
\begin{align*}
T_1(x,y)&=(0.85 x + 0.04 y, -0.04 x + 0.85 y+1.6),\\
T_2(x,y)&=(0.2 x - 0.26 y, 0.23 x + 0.22 y + 1.6),\\
T_3(x,y)&=(-0.15 x +0.28 y, 0.26 x+0.24 y +0.44),\\
T_4(x,y)&=(0,0.16 y).
\end{align*}
\begin{figure}
\includegraphics{felcex}
\end{figure}
%%%%%
%%%%%
\end{document}
