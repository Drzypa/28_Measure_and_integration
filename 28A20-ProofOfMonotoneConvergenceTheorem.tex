\documentclass[12pt]{article}
\usepackage{pmmeta}
\pmcanonicalname{ProofOfMonotoneConvergenceTheorem}
\pmcreated{2013-03-22 13:29:56}
\pmmodified{2013-03-22 13:29:56}
\pmowner{paolini}{1187}
\pmmodifier{paolini}{1187}
\pmtitle{proof of monotone convergence theorem}
\pmrecord{7}{34075}
\pmprivacy{1}
\pmauthor{paolini}{1187}
\pmtype{Proof}
\pmcomment{trigger rebuild}
\pmclassification{msc}{28A20}
\pmclassification{msc}{26A42}

% this is the default PlanetMath preamble.  as your knowledge
% of TeX increases, you will probably want to edit this, but
% it should be fine as is for beginners.

% almost certainly you want these
\usepackage{amssymb}
\usepackage{amsmath}
\usepackage{amsfonts}

% used for TeXing text within eps files
%\usepackage{psfrag}
% need this for including graphics (\includegraphics)
%\usepackage{graphicx}
% for neatly defining theorems and propositions
%\usepackage{amsthm}
% making logically defined graphics
%%%\usepackage{xypic}

% there are many more packages, add them here as you need them

% define commands here
\newtheorem{theorem}{Theorem}
\begin{document}
It is enough to prove the following

\begin{theorem}
Let $(X,\mu)$ be a measurable space and let $f_k\colon X\to {\mathbb R}\cup\{+\infty\}$ 
be a monotone increasing sequence of positive measurable functions (i.e. $0\le f_1 \le f_2 \le \ldots$). Then $f(x)=\lim_{k\to \infty} f_k(x)$ is measurable and
\[
  \lim_{n\to\infty} \int_X f_k\, d\mu = \int_{X} f(x) \, d\mu.
\]
\end{theorem}

First of all by the monotonicity of the sequence we have
\[
  f(x)= \sup_k f_k(x)
\]
hence we know that $f$ is measurable. Moreover being $f_k\le f$ for all $k$, by the monotonicity of the integral, we immediately get
\[
  \sup_k \int_X f_k\, d\mu \le \int_X f(x)\, d\mu.
\]

So take any simple measurable function $s$ such that $0\le s \le f$. Given also $\alpha<1$ define
\[
  E_k = \{ x \in X \colon f_k(x) \ge \alpha s(x)\}.
\]
The sequence $E_k$ is an increasing sequence of measurable sets. Moreover the union of all $E_k$ is the whole space $X$ since 
$\lim_{k\to\infty} f_k(x)=f(x) \ge s(x) > \alpha s(x)$. Moreover it holds
\[
  \int_X f_k\, d\mu \ge \int_{E_k} f_k\, d\mu \ge \alpha\int_{E_k} s\, d\mu.
\]
Since $s$ is a simple measurable function it is easy to check that 
$E\mapsto \int_E s\, d\mu$ is a measure and hence
\[
  \sup_k \int_X f_k\, d\mu \ge \alpha \int_X s\, d\mu.
\]
But this last inequality holds for every $\alpha<1$ and for all simple measurable functions $s$ with $s\le f$. Hence by the definition of Lebesgue integral
\[
  \sup_k \int_X f_k \, d\mu \ge \int_X f\, d\mu
\]
which completes the proof.
%%%%%
%%%%%
\end{document}
