\documentclass[12pt]{article}
\usepackage{pmmeta}
\pmcanonicalname{MeasureZeroInmathbbRn}
\pmcreated{2013-03-22 17:57:11}
\pmmodified{2013-03-22 17:57:11}
\pmowner{asteroid}{17536}
\pmmodifier{asteroid}{17536}
\pmtitle{measure zero in $\mathbb{R}^n$}
\pmrecord{6}{40452}
\pmprivacy{1}
\pmauthor{asteroid}{17536}
\pmtype{Theorem}
\pmcomment{trigger rebuild}
\pmclassification{msc}{28A05}

\endmetadata

% this is the default PlanetMath preamble.  as your knowledge
% of TeX increases, you will probably want to edit this, but
% it should be fine as is for beginners.

% almost certainly you want these
\usepackage{amssymb}
\usepackage{amsmath}
\usepackage{amsfonts}

% used for TeXing text within eps files
%\usepackage{psfrag}
% need this for including graphics (\includegraphics)
%\usepackage{graphicx}
% for neatly defining theorems and propositions
%\usepackage{amsthm}
% making logically defined graphics
%%%\usepackage{xypic}

% there are many more packages, add them here as you need them

% define commands here

\begin{document}
\subsection{Measure Zero in $\mathbb{R}^n$}
In $\mathbb{R}^n$ with the Lebesgue measure $m$ there is a \PMlinkescapetext{simple} characterization of the sets that have measure zero.

$\,$

{\bf Theorem -} A subset $X \subseteq \mathbb{R}^n$ has zero Lebesgue measure if and only if for every $\epsilon >0$ there is a sequence of compact rectangles $\{R_i\}_{i \in \mathbb{N}}$ that cover $X$ and such that $\sum_{i} m(R_i) < \epsilon$.

$\,$

\subsection{Measure Zero Avoiding Measure Theory}

In some circumstances one may want to avoid the whole \PMlinkescapetext{theory} of Lebesgue measure and integration and still be interested in having a notion of measure zero, like for example when studying Riemann integrals or constructing the Lebesgue measure from an historical \PMlinkescapetext{point} of view. Another interesting example where sets of measure zero arise and there is no reason to introduce measures or integrals is when studying the \PMlinkescapetext{type }of sets where a function of bounded variation is not differentiable (this sets have always measure zero).

Nevertheless, the notion of measure zero is not lost in this situation. Since the Lebesgue measure of compact rectangles can be easily calculated and defined from the start (see \PMlinkname{Jordan content of an $n$-cell}{JordanContentOfAnNCell} for example), the condition stated in the previous theorem can be taken as the definition of measure zero.

$\,$

{\bf Definition -} A set $X$ in $\mathbb{R}^n$ is said to have \emph{measure zero} if for every $\epsilon >0$ there is a sequence of compact rectangles $\{R_i\}_{i \in \mathbb{N}}$ that cover $X$ and such that $\sum_{i} m(R_i) < \epsilon$, where $m$ is the \PMlinkname{Jordan content}{JordanContentOfAnNCell}.

$\,$

Similarly, the notion of \emph{almost everywhere} remains essentially the same. One just has to work with the previous definition.
%%%%%
%%%%%
\end{document}
