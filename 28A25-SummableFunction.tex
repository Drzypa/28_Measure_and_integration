\documentclass[12pt]{article}
\usepackage{pmmeta}
\pmcanonicalname{SummableFunction}
\pmcreated{2013-03-22 18:12:14}
\pmmodified{2013-03-22 18:12:14}
\pmowner{ehremo}{15714}
\pmmodifier{ehremo}{15714}
\pmtitle{summable function}
\pmrecord{8}{40783}
\pmprivacy{1}
\pmauthor{ehremo}{15714}
\pmtype{Definition}
\pmcomment{trigger rebuild}
\pmclassification{msc}{28A25}
%\pmkeywords{summable Lebesgue integrable}
\pmrelated{LebesgueIntegrable}

% this is the default PlanetMath preamble.  as your knowledge
% of TeX increases, you will probably want to edit this, but
% it should be fine as is for beginners.

% almost certainly you want these
\usepackage{amssymb}
\usepackage{amsmath}
\usepackage{amsfonts}

% used for TeXing text within eps files
%\usepackage{psfrag}
% need this for including graphics (\includegraphics)
%\usepackage{graphicx}
% for neatly defining theorems and propositions
%\usepackage{amsthm}
% making logically defined graphics
%%%\usepackage{xypic}

% there are many more packages, add them here as you need them

% define commands here
\def\v#1{\mathbf{#1}}
\def\reals{\mathbb{R}}
\begin{document}
A measurable function $f : \Omega \to \reals$ where $(\Omega, \mathcal{A}, \mu)$ is a measure space is said to be {\bf summable} if the Lebesgue integral of the absolute value of $f$ exists and is finite,
\begin{equation*}
\int_{\Omega} |f| d\mu < +\infty
\end{equation*}
An alternative way of expressing this condition is to assert that $f \in L^1(\Omega)$.

Note that some authors distinguish between integrable and summable: an integrable function is one for which the above integral exists; a summable function is one for which the integral exists and is finite.
%%%%%
%%%%%
\end{document}
