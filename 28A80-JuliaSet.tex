\documentclass[12pt]{article}
\usepackage{pmmeta}
\pmcanonicalname{JuliaSet}
\pmcreated{2013-03-22 17:15:26}
\pmmodified{2013-03-22 17:15:26}
\pmowner{rspuzio}{6075}
\pmmodifier{rspuzio}{6075}
\pmtitle{Julia set}
\pmrecord{7}{39593}
\pmprivacy{1}
\pmauthor{rspuzio}{6075}
\pmtype{Definition}
\pmcomment{trigger rebuild}
\pmclassification{msc}{28A80}

% this is the default PlanetMath preamble.  as your knowledge
% of TeX increases, you will probably want to edit this, but
% it should be fine as is for beginners.

% almost certainly you want these
\usepackage{amssymb}
\usepackage{amsmath}
\usepackage{amsfonts}

% used for TeXing text within eps files
%\usepackage{psfrag}
% need this for including graphics (\includegraphics)
%\usepackage{graphicx}
% for neatly defining theorems and propositions
%\usepackage{amsthm}
% making logically defined graphics
%%%\usepackage{xypic}

% there are many more packages, add them here as you need them

% define commands here

\begin{document}
Let $U$ be an open subset of the complex plane and let $f \colon U \to U$
be analytic.  Denote the $n$-th iterate of $f$ by $f^n$, i.e. $f^1 = f$ 
and $f^{n+1} = f \circ f^n$.  Then the \emph{Julia set} of $f$ is the
subset $J$ of $U$ characterized by the following property: if $z \in J$
then the restriction of $\{f^n \mid n \in \mathbb{N}\}$ to any neighborhood
of $z$ is not a normal family.

It can also be shown that the Julia set of $f$ is the closure of the set of
repelling periodic points of $f$.  (Repelling periodic point means that, for
some $n$, we have $f^n (z) = z$ and $|f'(z)| > 1$.)

A simple example is afforded by the map $f(z) = z^2$; in this case, the Julia
set is the unit circle.  In general, however, things are much more complicated
and the Julia set is a fractal.

From the definition, it follows that the Julia set is closed under $f$ and
its inverse --- $f(J) = J$ and $f^{-1} (J) = J$.  Topologically, Julia sets
are perfect and have empty interior.


%%%%%
%%%%%
\end{document}
