\documentclass[12pt]{article}
\usepackage{pmmeta}
\pmcanonicalname{RadonMeasure}
\pmcreated{2013-03-22 15:49:41}
\pmmodified{2013-03-22 15:49:41}
\pmowner{ptr}{5636}
\pmmodifier{ptr}{5636}
\pmtitle{Radon measure}
\pmrecord{16}{37797}
\pmprivacy{1}
\pmauthor{ptr}{5636}
\pmtype{Definition}
\pmcomment{trigger rebuild}
\pmclassification{msc}{28C05}
\pmclassification{msc}{28C15}
%\pmkeywords{topological measure theory}
\pmrelated{BorelMeasure}
\pmrelated{SigmaFiniteBorelMeasureAndRelatedBorelConcepts}
\pmdefines{Radon space}

% this is the default PlanetMath preamble.  as your knowledge
% of TeX increases, you will probably want to edit this, but
% it should be fine as is for beginners.

% almost certainly you want these
\usepackage{amssymb}
\usepackage{amsmath}
\usepackage{amsfonts}

% used for TeXing text within eps files
%\usepackage{psfrag}
% need this for including graphics (\includegraphics)
%\usepackage{graphicx}
% for neatly defining theorems and propositions
\usepackage{amsthm}
% making logically defined graphics
%%%\usepackage{xypic}

% there are many more packages, add them here as you need them

% define commands here
\begin{document}
Let $X$ be a Hausdorff space. A Borel measure $\mu$ on $X$ is said to be a \emph{Radon measure} if it is:
\begin{enumerate}
\item finite on compact sets, 
\item inner regular (tight), $\mu(A) = \sup \{\mu (V) \mid \text{compact}\ V \subset A\}$.
\end{enumerate}
A finite Radon measure satisfies $\mu(A) = \inf \{\mu(G) \mid \text{open}\ G \supset A\}$.

Radon measures are not necessarily locally finite, although this is the case for locally compact and metric spaces. (Counterexample: spaces where only finite subsets are compact.) 

A \emph{Radon space} is a topological space on which every finite Borel measure is a Radon measure, this is the case, e.g. for Polish spaces or Hausdorff spaces that are continuous images of Polish spaces.


Radon measures are the ``most important class of measures on arbitrary Hausdorff topological spaces'' (K\"onig~\cite{koe97}, p.xiv) and formed the base of the development of integration theory by Bourbaki and Schwartz. In particular for locally compact spaces one often \emph{defines} Radon measures as linear functionals $\mu$ on the space $C^c(X)$ of continuous functions with compact support (`Riesz representation definition'). Berg \emph{et al.} give the following summary \cite{ber84}, p. 62\emph{f.}:

Given the finctional $\mu$ one defines \emph{set functions}, in fact Borel measures, the \emph{outer measure} $\mu^*$  and the \emph{essential outer measure} $\mu^\bullet$ given by \begin{alignat}
\mu^* (G) &= \sup \{\mu(f) | f\in C^c(X), 0 \le f \le 1_G\} \text{ for open }G\subset X\text{ and}\\
\mu^* (A) &= \inf\{\mu^* (A \cap K) | K \in \mathfrak{K} (X)\} \text{ for }A\subset X,\\
\mu^\bullet (A) &= \sup \{\mu^*(A \cap K) | K \in \mathfrak{K}(X)\} \text{ for }A\subset X.
\end{alignat}

$\mu^\bullet$ is a Radon measure in our sense, while $\mu^*$ is not always Radon. For locally compact and $\sigma$-compact spaces, however, both coincide (on the Borel algebra) and are equivalent to our Radon measure. For general Hausdorff spaces, Bourbaki introduces $W^* (A) = \sup \{(W_K)(A\cap K) | K\in \mathfrak{K}(X)\}$, where 
W, called a \emph{Radon premeasure}, associates a Radon measure $W_K$ to each compact $K \subset X$, with $W_K|L = W_L, L\in\mathfrak{K}$ . This is a Radon measure (on Borel sets), Bourbaki, however, calls it only so if it is in addition locally finite.

Consider now Borel measures $\nu : \mathfrak{B}\mapsto [0, \infty]$ which are
\begin{itemize}
\item finite on compact sets, $\nu|\mathfrak{K} <\infty$,
\item inner regular on the open sets, $\nu(G) = \sup \{\nu(K)|K\subset G\}$ for $G$ open, and $K$ compact,
\item outer regular, $\nu(B) = \inf \{\nu(G) | B\subset G$ for open $G$ and Borel $B$.
\end{itemize}
then the measures $\nu$ correspond bijectively to locally finite Radon measures $\mu$ on $X$. 











\begin{thebibliography}{99}
\bibitem{ber84}    Christian Berg, Jens Peter Reus, Paul Ressel: Harmonic analysis on semigroups. -- Berlin, 1984 (Graduate Texts in Mathematics; 100) 
\bibitem{koe97}    Heinz K\"onig: Measure and integration~: an advanced course in basic procedures and applications.-- Berlin, 1997. 
%\PMlinkexternal{ZBl. 0887.28001}
%{http://www.emis.de:80/cgi-in/zmen/ZMATH/en/command.html?first=1\&maxdocs=3\&type=html\&an=0887.28001\&format=complete}
\end{thebibliography}
\PMlinkescapeword{development}
\PMlinkescapeword{base}
\PMlinkescapeword{theory}
\PMlinkescapeword{representation}
%%%%%
%%%%%
\end{document}
