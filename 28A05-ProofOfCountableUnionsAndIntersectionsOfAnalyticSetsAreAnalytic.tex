\documentclass[12pt]{article}
\usepackage{pmmeta}
\pmcanonicalname{ProofOfCountableUnionsAndIntersectionsOfAnalyticSetsAreAnalytic}
\pmcreated{2013-03-22 18:46:19}
\pmmodified{2013-03-22 18:46:19}
\pmowner{gel}{22282}
\pmmodifier{gel}{22282}
\pmtitle{proof of countable unions and intersections of analytic sets are analytic}
\pmrecord{4}{41564}
\pmprivacy{1}
\pmauthor{gel}{22282}
\pmtype{Proof}
\pmcomment{trigger rebuild}
\pmclassification{msc}{28A05}
%\pmkeywords{analytic set}
%\pmkeywords{compact paving}
%\pmkeywords{product paving}
%\pmkeywords{direct sum paving}

\endmetadata

% almost certainly you want these
\usepackage{amssymb}
\usepackage{amsmath}
\usepackage{amsfonts}

% used for TeXing text within eps files
%\usepackage{psfrag}
% need this for including graphics (\includegraphics)
%\usepackage{graphicx}
% for neatly defining theorems and propositions
\usepackage{amsthm}
% making logically defined graphics
%%%\usepackage{xypic}

% there are many more packages, add them here as you need them

% define commands here
\newtheorem*{theorem*}{Theorem}
\newtheorem*{lemma*}{Lemma}
\newtheorem*{corollary*}{Corollary}
\newtheorem*{definition*}{Definition}
\newtheorem{theorem}{Theorem}
\newtheorem{lemma}{Lemma}
\newtheorem{corollary}{Corollary}
\newtheorem{definition}{Definition}

\begin{document}
\PMlinkescapeword{sequence}
\PMlinkescapeword{analytic}
\PMlinkescapeword{compact}
\PMlinkescapeword{direct sum}
\PMlinkescapeword{subset}

Let $(X,\mathcal{F})$ be a paved space and $(A_n)_{n\in\mathbb{N}}$ be a sequence of $\mathcal{F}$-\PMlinkname{analytic sets}{AnalyticSet2}. We show that their union and intersection is also analytic.

From the definition of $\mathcal{F}$-analytic sets, there exist compact paved spaces $(K_n,\mathcal{K}_n)$ and $S_n\in(\mathcal{F}\times\mathcal{K}_n)_{\sigma\delta}$ such that
\begin{equation*}
A_n=\left\{x\in X\colon (x,y)\in S_n\text{ for some }y\in K_n\right\}.
\end{equation*}

We start by showing that $\bigcap_nA_n$ is analytic. Let $K=\prod_nK_n$ and $\mathcal{K}=\prod_n\mathcal{K}_n$ be the product paving, and $\pi_n\colon K\rightarrow K_n$ be the projection map.
Then $x\in\bigcap_nA_n$ if and only if for each $n$ there is a $y_n\in K_n$ with $(x,y_n)\in S_n$. Equivalently, setting $y=(y_1,y_2,\cdots)$, then $(x,y)\in \bigcap_n\pi^{-1}_n(S_n)$. However, this is in $(\mathcal{F}\times\mathcal{K}_n)_{\sigma\delta}$ and we can write,
\begin{equation*}
\bigcap_nA_n = \pi_X\left(\bigcap_n\pi^{-1}_n(S_n)\right),
\end{equation*}
where $\pi_X\colon X\times K\rightarrow X$ is the projection map.
As products of compact pavings are compact, $(K,\mathcal{K})$ is compact and it follows from the definition that $\bigcap_nA_n$ is $\mathcal{F}$-analytic.


We now show that $\bigcup_n A_n$ is analytic. Let $K=\sum_nK_n$ and $\mathcal{K}=\sum_n\mathcal{K}_n$ be the direct sum paving, \PMlinkname{which is compact}{SumsOfCompactPavingsAreCompact}. Also, write $S_n=\bigcap_{m=1}^\infty T_{m,n}$ for $T_{m,n}\in(\mathcal{F}\times\mathcal{K}_n)_\sigma$.
We identify $K_n$ with a subset of $K$, so that $K$ is the union of the disjoint sets $K_n$.
Then $x\in\bigcup_nA_n$ if and only if $(x,y)\in S_n$ for some $n$ and some $y\in K$,
\begin{equation*}
\bigcup_nA_n=\pi_X\left(\bigcup_nS_n\right).
\end{equation*}
However, the fact that $K_{n_1},K_{n_2}$ are disjoint for $n_1\not= n_2$ says that $T_{m,n_1},T_{m,n_2}$ are disjoint and, therefore,
\begin{equation*}
\bigcup_nS_n=\bigcup_n\bigcap_mT_{m,n}=\bigcap_m\bigcup_nT_{m,n}\in\left(\mathcal{F}\times\mathcal{K}\right)_{\sigma\delta}.
\end{equation*}
So $\bigcup_nA_n$ is $\mathcal{F}$-analytic.

%%%%%
%%%%%
\end{document}
