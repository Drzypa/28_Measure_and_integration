\documentclass[12pt]{article}
\usepackage{pmmeta}
\pmcanonicalname{LinftyXmu}
\pmcreated{2013-03-22 13:59:46}
\pmmodified{2013-03-22 13:59:46}
\pmowner{ack}{3732}
\pmmodifier{ack}{3732}
\pmtitle{$L^{\infty}(X, \mu)$}
\pmrecord{11}{34797}
\pmprivacy{1}
\pmauthor{ack}{3732}
\pmtype{Definition}
\pmcomment{trigger rebuild}
\pmclassification{msc}{28A25}

\usepackage{amssymb}
\usepackage{amsmath}
\usepackage{amsfonts}

\DeclareMathOperator{\esssup}{ess sup}
\newcommand*{\norm}[1]{\left\lVert #1\right\rVert}
\newcommand*{\abs}[1]{\left\lVert #1\right\rVert}
\begin{document}
Let $X$ be a nonempty set and $\mathcal{A}$ be a $\sigma$-algebra on $X$. Also, let $\mu$ be a non-negative measure defined on $\mathcal{A}$. 
Two complex valued functions $f$ and $g$ are said to be equal almost everywhere on $X$ 
(denoted as $f = g$ a.e. if $\mu \{x \in X : f(x) \ne g(x) \} = 0.$ The relation of being equal almost everywhere on $X$ defines an equivalence relation. 
It is a common practice in the integration theory to denote the equivalence class containing $f$ by $f$ itself. 
It is easy to see that if $f_1,f_2$ are equivalent and $g_1,g_2$ are equivalent, then $f_1 + g_1, f_2+g_2$ are equivalent, and $f_1  g_1, f_2g_2$ are equivalent. 
This naturally defines addition and multiplication among the equivalent classes of such functions.
For a measureable 
$f \colon X \to \mathbb{C}$, we define  
$$
  \norm{f}_{\text{ess}} = \operatorname{inf}\{M > 0 \colon \mu \{x : |f(x)| > M\} = 0\},
$$
called the essential supremum of $|f|$ on $X$. 
Now we define, 
$$
  L^{\infty}(X,\mu) = \{f : X \to \mathbb{C} : \norm{f}_{\text{ess}} < \infty\}.
$$ 
Here the elements of $L^{\infty}(X,\mu)$ are equivalence classes. 
\subsubsection*{Properties of $L^\infty(X,\mu)$}
\begin{enumerate}
\item The space $L^{\infty}(X,\mu)$ is a normed linear space with the 
   norm $\norm{\cdot}_{\text{ess}}$. Also, the metric defined by 
   the norm is complete, making $L^{\infty}(X,\mu)$, a Banach space. 
\item  $L^{\infty}(X,\mu)$ is the dual of $L^1(X,\mu)$ if $X$ is $\sigma$-finite. 
\item  $L^{\infty}(X,\mu)$ is closed under pointwise multiplication, and
with this multiplication it becomes an algebra. 
Further, $L^{\infty}(X,\mu)$ is also a \PMlinkname{$C^*$-algebra}{CAlgebra} with the involution defined by $f^*(x) = \overline{f(x)}$. Since this $C^*$-algebra is also a dual of some Banach space, it is called von Neumann algebra.
\end{enumerate}
%%%%%
%%%%%
\end{document}
