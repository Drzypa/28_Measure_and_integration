\documentclass[12pt]{article}
\usepackage{pmmeta}
\pmcanonicalname{CounterexampleToTonellisTheorem}
\pmcreated{2013-03-22 18:16:36}
\pmmodified{2013-03-22 18:16:36}
\pmowner{rmilson}{146}
\pmmodifier{rmilson}{146}
\pmtitle{counter-example to Tonelli's theorem}
\pmrecord{4}{40882}
\pmprivacy{1}
\pmauthor{rmilson}{146}
\pmtype{Example}
\pmcomment{trigger rebuild}
\pmclassification{msc}{28A35}

\endmetadata

\usepackage{amsmath}
\usepackage{amsfonts}
\usepackage{amssymb}
\newcommand{\reals}{\mathbb{R}}
\newcommand{\natnums}{\mathbb{N}}
\newcommand{\cnums}{\mathbb{C}}
\newcommand{\znums}{\mathbb{Z}}
\newcommand{\lp}{\left(}
\newcommand{\rp}{\right)}
\newcommand{\lb}{\left[}
\newcommand{\rb}{\right]}
\newcommand{\supth}{^{\text{th}}}
\newtheorem{proposition}{Proposition}
\newtheorem{definition}[proposition]{Definition}

\newtheorem{theorem}[proposition]{Theorem}
\begin{document}
The following observation demonstrates the necessity of the
$\sigma$-finite assumption in Tonelli's and Fubini's theorem.  Let $X$
denote the closed unit interval $[0,1]$ equipped with Lebesgue measure
and $Y$ the same set, but this time equipped with counting measure
$\nu$.  Let
\[ f(x,y) = \left\{
  \begin{array}{cl}
    1 & \mbox{ if } x=y,\\
    0 & \mbox{ otherwise}.
  \end{array}\right.
\]
Observe that 
\[ \int_Y \left( \int_X f(x,y) d\mu(x)\right) d\nu(y) = 0,\]
while
\[ \int_X \left( \int_Y f(x,y) d\nu(y)\right) d\mu(x) = 1.\]
The iterated integrals do not give the same value, this despite the
fact that the integrand is a non-negative function.  

Also observe that there does not exist a simple function on $X\times
Y$ that is dominated by $f$.  Hence,
\[ \int_{X\times Y} f(x,y) d (\mu(x)\times \nu(y) = 0.\]
Therefore,  the integrand is $L^1$ integrable
relative to the product measure. However, as we observed above, the
iterated integrals do not agree.  This observation illustrates the need for the
$\sigma$-finite assumption for Fubini's theorem.

%%%%%
%%%%%
\end{document}
