\documentclass[12pt]{article}
\usepackage{pmmeta}
\pmcanonicalname{TonellisTheorem}
\pmcreated{2013-03-22 14:15:49}
\pmmodified{2013-03-22 14:15:49}
\pmowner{jirka}{4157}
\pmmodifier{jirka}{4157}
\pmtitle{Tonelli's theorem}
\pmrecord{7}{35713}
\pmprivacy{1}
\pmauthor{jirka}{4157}
\pmtype{Theorem}
\pmcomment{trigger rebuild}
\pmclassification{msc}{28A35}
\pmrelated{FubinisTheorem}
\pmrelated{FubinisTheoremForTheLebesgueIntegral}

% this is the default PlanetMath preamble.  as your knowledge
% of TeX increases, you will probably want to edit this, but
% it should be fine as is for beginners.

% almost certainly you want these
\usepackage{amssymb}
\usepackage{amsmath}
\usepackage{amsfonts}

% used for TeXing text within eps files
%\usepackage{psfrag}
% need this for including graphics (\includegraphics)
%\usepackage{graphicx}
% for neatly defining theorems and propositions
\usepackage{amsthm}
% making logically defined graphics
%%%\usepackage{xypic}

% there are many more packages, add them here as you need them

% define commands here
\theoremstyle{theorem}
\newtheorem*{thm}{Theorem}
\newtheorem*{lemma}{Lemma}
\newtheorem*{conj}{Conjecture}
\newtheorem*{cor}{Corollary}
\newtheorem*{example}{Example}
\newtheorem*{prop}{Proposition}
\theoremstyle{definition}
\newtheorem*{defn}{Definition}
\theoremstyle{remark}
\newtheorem*{rmk}{Remark}
\begin{document}
Here denote $L^+(X)$ as the space of measurable functions $X \to [0,\infty]$.  Furthermore all integrals are Lebesgue integrals.

\begin{thm}[Tonelli]
Suppose $(X,{\mathcal{M}},\mu)$ and $(Y,{\mathcal{N}},\nu)$ are \PMlinkname{$\sigma$-finite}{SigmaFinite}
measure spaces.  If $f \in L^+(X \times Y)$, then the functions
$x \mapsto \int_Y f(x,y) d\nu(y)$ and
$y \mapsto \int_X f(x,y) d\mu(x)$ are in $L^+(X)$ and $L^+(Y)$ respectively, and furthermore if we denote by $\mu \times \nu$ the product measure, then
\begin{equation*}
\int_{X \times Y} f \, d(\mu \times \nu) =
\int_X \left[ \int_Y f(x,y) \,d\nu(y) \right] d\mu(x) = 
\int_Y \left[ \int_X f(x,y) \,d\mu(x) \right] d\nu(y) .
\end{equation*}
\end{thm} 

Basically this says that you can switch the \PMlinkescapetext{order} of integrals, or integrate over the product space as long as everything is positive and the spaces are $\sigma$-finite.  Do note that we allow the functions to take on the value of
infinity with the standard conventions used in Lebesgue integration.  That is, $0 \cdot \infty = 0$, so that if a function is infinite on a set of measure 0, then this does not contribute anything to the value of the integral.
See the entry on extended real numbers for further discussion.

If we take the counting measure on ${\mathbb{N}}$, then one can \PMlinkescapetext{state} the Tonelli theorem for sums.

\begin{thm}[Tonelli for sums]
Suppose that $f_{ij} \geq 0$ for all $i,j \in {\mathbb{N}}$, then
\begin{equation*}
\sum_{i,j \in {\mathbb{N}}} f_{ij}
=
\sum_{i=1}^\infty \sum_{j=1}^\infty f_{ij}
=
\sum_{j=1}^\infty \sum_{i=1}^\infty f_{ij} .
\end{equation*}
\end{thm}

In the above theorem we have used ${\mathbb{N}}$ as our \PMlinkescapetext{index} set for simplicity and familiarity of notation.  
If you would have an uncountable number of non-zero elements $f_{ij}$ then
all the sums would be infinite and the result would be trivial.
So the theorem for arbitrary \PMlinkescapetext{index}
sets just reduces to the above case.

\begin{thebibliography}{9}
\bibitem{folland}
Gerald B.\@ Folland. \emph{\PMlinkescapetext{Real Analysis, Modern Techniques
and Their Applications}}. John Wiley \& Sons, Inc., New York, New York, 1999
\end{thebibliography}
%%%%%
%%%%%
\end{document}
