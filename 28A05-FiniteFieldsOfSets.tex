\documentclass[12pt]{article}
\usepackage{pmmeta}
\pmcanonicalname{FiniteFieldsOfSets}
\pmcreated{2013-03-22 15:47:49}
\pmmodified{2013-03-22 15:47:49}
\pmowner{rspuzio}{6075}
\pmmodifier{rspuzio}{6075}
\pmtitle{finite fields of sets}
\pmrecord{8}{37758}
\pmprivacy{1}
\pmauthor{rspuzio}{6075}
\pmtype{Theorem}
\pmcomment{trigger rebuild}
\pmclassification{msc}{28A05}
\pmclassification{msc}{03E20}

% this is the default PlanetMath preamble.  as your knowledge
% of TeX increases, you will probably want to edit this, but
% it should be fine as is for beginners.

% almost certainly you want these
\usepackage{amssymb}
\usepackage{amsmath}
\usepackage{amsfonts}

% used for TeXing text within eps files
%\usepackage{psfrag}
% need this for including graphics (\includegraphics)
%\usepackage{graphicx}
% for neatly defining theorems and propositions
%\usepackage{amsthm}
% making logically defined graphics
%%%\usepackage{xypic}

% there are many more packages, add them here as you need them

% define commands here
\begin{document}
If $S$ is a finite set then any field of subsets of $S$ (see ``field of 
sets'' in the entry on rings of sets) can be described as the set of unions 
of subsets of a partition of $S$.

Note that, if $P$ is a partition of $S$ and $A,B \subset P$, we
have
\begin{eqnarray*}
\overline{\bigcup A} &=& \bigcup (P \setminus A) \\
(\bigcup A) \cup (\bigcup B) &=& \bigcup (A \cup B) \\
(\bigcup A) \cap (\bigcup B) &=& \bigcup (A \cap B) \\
\end{eqnarray*}
so $\{ \bigcup X \mid X \subset P \}$ is a field of sets.

Now assume that $\mathcal{F}$ is a field of subsets of a finite set
$S$.  Let us define the set of ``prime elements'' of $\mathcal{F}$ as
follows:
 \[P = \{ X \in (\mathcal{F} \setminus {\varnothing}) \mid (Y \subset X)
 \wedge (Y \in \mathcal{F}) \Rightarrow (Y = \varnothing \vee Y = X) \} \]
The choice of terminology ``prime element'' is meant to be a suggestive
mnemonic of how the only divisors of a prime number are 1 and the
number itself.

We claim that $P$ is a partition.  To justify this claim, we need to
show that elements of $P$ are pairwise disjoint and that $\bigcup P =
S$.

Suppose that $A$ and $B$ are prime elements.  Since, by definition, $A
\in \mathcal{F}$ and $B \in \mathcal{F}$ and $\mathcal{F}$ is a field
of sets, $A \cap B \in \mathcal{F}$.  Since $A \cap B \subset A$, we
must either have $A \cap B = \varnothing$ or $A \cap B = A$.  In the
former case, $A$ and $B$ are disjoint, whilst in the latter case $A =
B$.

Suppose that $x$ is any element of $S$.  Then we claim that the set
$X$ defined as 
 \[ X = \bigcup \{ Y \in \mathcal F \mid x \in Y \} \]
is a prime element of $\mathcal{F}$.  To begin, note that, since
$\mathcal{F}$ is finite, a forteriori any subset of $\mathcal{F}$ is
finite and, since fields of sets are assumed to be closed under
intersection, it follows that the intersection of a susbet of
$\mathcal{F}$ is an element of $\mathcal{F}$, in particular $X \in
\mathcal{F}$.  

Suppose that $Z \subset X$ and $Z \in \mathcal{F}$.  If $x \notin Z$,
then $x \in \overline{Z}$.  Since $\mathcal{F}$ is a field of sets,
$\overline{Z} \in \mathcal{F}$.  Hence, by the construction of $X$, it
is the case that $X \subset \overline{Z}$, hence $X \cap Z =
\varnothing$.  Together with $Z \subset X$, this implies $Z =
\varnothing$.  If $x \in Z$, then, by construction, $X \subset Z$,
which implies $X = Z$.  

Thus, we see that $X$ is a prime set.  Since $x$ was arbitrarily
chosen, this means that every element of $S$ is contained in a prime
element of $\mathcal{F}$, so the union of all prime elements is $S$
itself.  Together with the previously shown fact that prime elements
are pairwise disjoint, this shows that the prime elements for a
partition of $S$.

Let $A$ be an arbitrary element of $\mathcal{F}$.  Since $P \subset
\mathcal{F}$, it is the case that $(\forall X \in P) A \cap X \in
\mathcal{F}$.  Since $P$ is a partition of $S$,
 \[ A = \bigcup \{ A \cap X \mid X \in P \} \]
so every element of $\mathcal{F}$ can be expressed as a union of
elements of $P$.
%%%%%
%%%%%
\end{document}
