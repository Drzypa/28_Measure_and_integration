\documentclass[12pt]{article}
\usepackage{pmmeta}
\pmcanonicalname{Quasiinvariant}
\pmcreated{2013-03-22 15:56:00}
\pmmodified{2013-03-22 15:56:00}
\pmowner{Mathprof}{13753}
\pmmodifier{Mathprof}{13753}
\pmtitle{quasi-invariant}
\pmrecord{12}{37942}
\pmprivacy{1}
\pmauthor{Mathprof}{13753}
\pmtype{Definition}
\pmcomment{trigger rebuild}
\pmclassification{msc}{28A12}
\pmrelated{RepresentationsOfLocallyCompactGroupoids}

\endmetadata

% this is the default PlanetMath preamble.  as your knowledge
% of TeX increases, you will probably want to edit this, but
% it should be fine as is for beginners.

% almost certainly you want these
\usepackage{amssymb}
\usepackage{amsmath}
\usepackage{amsfonts}

% used for TeXing text within eps files
%\usepackage{psfrag}
% need this for including graphics (\includegraphics)
%\usepackage{graphicx}
% for neatly defining theorems and propositions
\usepackage{amsthm}
% making logically defined graphics
%%%\usepackage{xypic}

% there are many more packages, add them here as you need them

% define commands here
\theoremstyle{definition}
\newtheorem{definition}{Definition}
\begin{document}
\begin{definition}
Let $(E, \mathcal{B})$ be a measurable space, and $T : E \to E$ be a measurable map.  A measure $\mu$ on $(E, \mathcal{B})$ is said to be \emph{quasi-invariant} under $T$ if $\mu \circ T^{-1}$ is absolutely continuous with respect to  $\mu$.  That is, for all $A \in \mathcal{B}$ with $\mu(A)=0$, we also have $\mu(T^{-1}(A)) = 0$.  We also say that $T$ leaves $\mu$ quasi-invariant.
\end{definition}

As a \PMlinkescapetext{simple} example, let $E = \mathbb{R}$ with $\mathcal{B}$ the  \PMlinkname{Borel $\sigma$-algebra}{BorelSigmaAlgebra}, and $\mu$ be Lebesgue measure.  If $T(x) = x + 5$, then $\mu$ is quasi-invariant under $T$.  If $S(x)=0$, then $\mu$ is not quasi-invariant under $S$.  (We have $\mu(\{0\}) = 0$, but $\mu(T^{-1}(\{0\})) = \mu(\mathbb{R}) = \infty$).

To give another example, take $E$ to be the nonnegative integers and
declare  every subset of $E$ to be  a measurable set. Fix $\lambda > 0$.
Let $\mu(\{n\}) = \frac{\lambda^n}{n!}$ and extend $\mu$ to all subsets
by additivity.  Let $T$ be the shift function: $n \to n+1$. Then
$\mu$ is quasi-invariant under $T$ and not \PMlinkname{invariant}{HaarMeasure}.
%%%%%
%%%%%
\end{document}
