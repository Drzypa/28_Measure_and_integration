\documentclass[12pt]{article}
\usepackage{pmmeta}
\pmcanonicalname{QuadraticJuliaSet}
\pmcreated{2013-03-22 17:14:40}
\pmmodified{2013-03-22 17:14:40}
\pmowner{PrimeFan}{13766}
\pmmodifier{PrimeFan}{13766}
\pmtitle{quadratic Julia set}
\pmrecord{9}{39574}
\pmprivacy{1}
\pmauthor{PrimeFan}{13766}
\pmtype{Definition}
\pmcomment{trigger rebuild}
\pmclassification{msc}{28A80}

% this is the default PlanetMath preamble.  as your knowledge
% of TeX increases, you will probably want to edit this, but
% it should be fine as is for beginners.

% almost certainly you want these
\usepackage{amssymb}
\usepackage{amsmath}
\usepackage{amsfonts}

% used for TeXing text within eps files
%\usepackage{psfrag}
% need this for including graphics (\includegraphics)
%\usepackage{graphicx}
% for neatly defining theorems and propositions
%\usepackage{amsthm}
% making logically defined graphics
%%%\usepackage{xypic}

% there are many more packages, add them here as you need them

% define commands here

\begin{document}
For each complex number $c$, there is an associated quadratic map
$f_c\colon\mathbb{C}\to\mathbb{C}$ defined by $f_c(z) = z^2 + c$.
Since polynomials are analytic, it follows that $f_c$ has a Julia set
$J(f_c)$, which we call the \emph{quadratic Julia set} associated to
$c$ and denote by $J_c$.

The function can also be viewed as having $\mathbb{R}^2$ as its domain
and codomain.  If $c = a + ib$, then
\[
  f_c(x, y) = (x^2 - y^2 + a, 2xy + b).
\]
The characterization of the Julia set $J_c$ as all points $z$ for which the
collection of iterates $\{f^n\colon n\in\mathbb{N}\}$ is not a normal
family can be roughly interpreted as saying that the Julia set
includes only those points exhibiting chaotic behavior.  In
particular, points whose orbit under $f$ goes to infinity are omitted
from $J_c$, as well as points whose orbit converges to a point.

Sometimes for aesthetic purposes a Julia set is displayed with points
of the latter type included.  However, the chaoticity of the true Julia
set can be exploited to plot an approximation very quickly.  Given a
single point $z$ known to be in the quadratic Julia set $J_c$, its inverses 
under $f$, that is, the square roots of $z - c$, are also in $J_c$.  Moreover,
by the chaoticity condition the ``backwards orbit'' of $z$ (selecting just one 
square root at each step) will be distributed fairly evenly
over $J_c$, so this gives a computationally inexpensive method to plot
Julia sets.

Before the advent of computers, the French mathematician Gaston Julia
proved under what conditions a Julia set is connected or not
connected.  After computers became available, it became possible to
make pictures displaying some of the complexity of these Julia sets,
and the Mandelbrot set, a kind of index into connected quadratic Julia
sets, was discovered.

In the same way that some people see recognizable shapes in clouds,
some people see recognizable shapes in Julia sets, and some of them
have been named accordingly. To give two examples: the San Marco
dragon at $\frac{-3}{4} + 0i$ and the Douady rabbit at $\frac{-1}{8} +
\frac{745}{1000}i$ (the coordinates can be varied by small values and
still give very similar shapes).

% not sure what to do with this sentence
Julia sets can be generalized to other iterated holomorphic functions
on the complex plane or in a 3-dimensional space.

\begin{thebibliography}{1}
\bibitem{hl} H. Lauwerier, translated by Sophia Gill-Hoffst\"adt. {\it Fractals: Endlessly Repeated Geometric Figures} Princeton: Princeton University Press (1991): 124 - 151
\end{thebibliography}

\PMlinkescapeword{index}
\PMlinkescapeword{similar}
\PMlinkescapeword{type}
%%%%%
%%%%%
\end{document}
