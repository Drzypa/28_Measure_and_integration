\documentclass[12pt]{article}
\usepackage{pmmeta}
\pmcanonicalname{CompleteMeasure}
\pmcreated{2013-03-22 14:06:56}
\pmmodified{2013-03-22 14:06:56}
\pmowner{Koro}{127}
\pmmodifier{Koro}{127}
\pmtitle{complete measure}
\pmrecord{5}{35520}
\pmprivacy{1}
\pmauthor{Koro}{127}
\pmtype{Definition}
\pmcomment{trigger rebuild}
\pmclassification{msc}{28A12}
\pmrelated{UniversallyMeasurable}
\pmdefines{completion}
\pmdefines{complete}

\endmetadata

% this is the default PlanetMath preamble.  as your knowledge
% of TeX increases, you will probably want to edit this, but
% it should be fine as is for beginners.

% almost certainly you want these
\usepackage{amssymb}
\usepackage{amsmath}
\usepackage{amsfonts}
\usepackage{mathrsfs}

% used for TeXing text within eps files
%\usepackage{psfrag}
% need this for including graphics (\includegraphics)
%\usepackage{graphicx}
% for neatly defining theorems and propositions
%\usepackage{amsthm}
% making logically defined graphics
%%%\usepackage{xypic}

% there are many more packages, add them here as you need them

% define commands here
\newcommand{\C}{\mathbb{C}}
\newcommand{\R}{\mathbb{R}}
\newcommand{\N}{\mathbb{N}}
\newcommand{\Z}{\mathbb{Z}}
\newcommand{\Per}{\operatorname{Per}}
\begin{document}
A measure space $(X,\mathscr{S},\mu)$ is said to be \emph{complete} if every subset of a set of measure $0$ is measurable (and consequently, has measure $0$); i.e. if for all $E\in\mathscr{S}$ such that $\mu(E)=0$ and for all $S\subset E$ we have $\mu(S)=0$. 

If a measure space is not complete, there exists a \PMlinkname{completion}{CompletionOfAMeasureSpace} of it, which is a complete measure space $(X,\overline{\mathscr{S}},\overline{\mu})$ such that $\mathscr{S}\subset\overline{\mathscr{S}}$ and $\overline {\mu}_{|\mathscr{S}} = \mu$, where $\overline{\mathscr{S}}$ is the smallest $\sigma$-algebra containing both $\mathscr{S}$ and all subsets of elements of zero measure of $\mathscr{S}$.
%%%%%
%%%%%
\end{document}
