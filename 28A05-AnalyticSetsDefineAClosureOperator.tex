\documentclass[12pt]{article}
\usepackage{pmmeta}
\pmcanonicalname{AnalyticSetsDefineAClosureOperator}
\pmcreated{2013-03-22 18:46:30}
\pmmodified{2013-03-22 18:46:30}
\pmowner{gel}{22282}
\pmmodifier{gel}{22282}
\pmtitle{analytic sets define a closure operator}
\pmrecord{4}{41568}
\pmprivacy{1}
\pmauthor{gel}{22282}
\pmtype{Theorem}
\pmcomment{trigger rebuild}
\pmclassification{msc}{28A05}
%\pmkeywords{analytic set}
%\pmkeywords{closure operator}
%\pmkeywords{paving}

\endmetadata

% almost certainly you want these
\usepackage{amssymb}
\usepackage{amsmath}
\usepackage{amsfonts}

% used for TeXing text within eps files
%\usepackage{psfrag}
% need this for including graphics (\includegraphics)
%\usepackage{graphicx}
% for neatly defining theorems and propositions
\usepackage{amsthm}
% making logically defined graphics
%%%\usepackage{xypic}

% there are many more packages, add them here as you need them

% define commands here
\newtheorem*{theorem*}{Theorem}
\newtheorem*{lemma*}{Lemma}
\newtheorem*{corollary*}{Corollary}
\newtheorem*{definition*}{Definition}
\newtheorem{theorem}{Theorem}
\newtheorem{lemma}{Lemma}
\newtheorem{corollary}{Corollary}
\newtheorem{definition}{Definition}

\begin{document}
\PMlinkescapeword{collection}
\PMlinkescapeword{subsets}
\PMlinkescapeword{analytic}
\PMlinkescapeword{analytic sets}
\PMlinkescapeword{properties}

For a paving $\mathcal{F}$ on a set $X$, we denote the collection of all $\mathcal{F}$-\PMlinkname{analytic sets}{AnalyticSet2} by $a(\mathcal{F})$.
Then, $\mathcal{F}\mapsto a(\mathcal{F})$ is a closure operator on the subsets of $X$.
That is,
\begin{enumerate}
\item\label{1} $\mathcal{F}\subseteq a(\mathcal{F})$.
\item\label{2} If $\mathcal{F}\subseteq\mathcal{G}$ then $a(\mathcal{F})\subseteq a(\mathcal{G})$.
\item\label{3} $a(a(\mathcal{F}))=a(\mathcal{F})$.
\end{enumerate}

For example, if $\mathcal{G}$ is a collection of $\mathcal{F}$-analytic sets then $\mathcal{G}\subseteq a(\mathcal{F})$ gives $a(\mathcal{G})\subseteq a(a(\mathcal{F}))=a(\mathcal{F})$ and so all $\mathcal{G}$-analytic sets are also $\mathcal{F}$-analytic. In particular, for a metric space, the analytic sets are the same regardless of whether they are defined with respect to the collection of open, closed or Borel sets.

Properties \ref{1} and \ref{2} follow directly from the definition of analytic sets. We just need to prove \ref{3}. So, for any $A\in a(a(\mathcal{F}))$ we show that $A\in a(\mathcal{F})$. First, there is a \PMlinkname{compact paved space}{PavedSpace} $(K,\mathcal{K})$ and $S\in \left(a(\mathcal{F})\times\mathcal{K}\right)_{\sigma\delta}$ such that $A$ is equal to the projection $\pi_X(S)$.
Write
\begin{equation*}
S=\bigcap_{m=1}^\infty\bigcup_{n=1}^\infty A_{m,n}\times B_{m,n}
\end{equation*}
for $A_{m,n}\in a(\mathcal{F})$ and $B_{m,n}\in\mathcal{K}$. It is clear that $A_{m,n}\times B_{m,n}$ is $\mathcal{F}\times\mathcal{K}$-analytic and, as countable unions and intersections of analytic sets are analytic, $S$ is also $\mathcal{F}\times\mathcal{K}$-analytic. Finally, since projections of analytic sets are analytic, $A=\pi_X(S)$ must be $\mathcal{F}$-analytic as required.

%%%%%
%%%%%
\end{document}
