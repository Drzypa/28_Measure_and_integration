\documentclass[12pt]{article}
\usepackage{pmmeta}
\pmcanonicalname{SierpinskiGasket}
\pmcreated{2013-05-18 22:53:46}
\pmmodified{2013-05-18 22:53:46}
\pmowner{mathwizard}{128}
\pmmodifier{unlord}{1}
\pmtitle{Sierpi\'nski gasket}
\pmrecord{41}{33011}
\pmprivacy{1}
\pmauthor{mathwizard}{1}
\pmtype{Definition}
\pmcomment{trigger rebuild}
\pmclassification{msc}{28A80}
\pmsynonym{Sierpinski triangle}{SierpinskiGasket}
\pmsynonym{Sierpinski gasket}{SierpinskiGasket}
\pmsynonym{Sierpi\'nski triangle}{SierpinskiGasket}
\pmrelated{MengerSponge}

\usepackage{amssymb}
\usepackage{amsmath}
\usepackage{amsfonts}

% used for TeXing text within eps files
%\usepackage{psfrag}
% need this for including graphics (\includegraphics)
\usepackage{graphicx}
% for neatly defining theorems and propositions
%\usepackage{amsthm}
% making logically defined graphics
%%%\usepackage{xypic}

% there are many more packages, add them here as you need them

% define commands here
\begin{document}
Let $S_0$ be a triangular area, and define $S_{n+1}$ to be obtained from $S_n$ by replacing each triangular area in $S_n$ with three similar and similarly
oriented triangular areas each intersecting with each of the other two at exactly one vertex, each one half the linear scale of the original in size.  
The limiting set as
$n\rightarrow \infty$ (alternately the intersection of all these sets) is a {\emph Sierpi\'nski gasket}, also known as a {\emph Sierpi\'nski triangle}.
\begin{figure}[h]
\begin{centering}
\includegraphics[width=5cm]{s01.png}
\includegraphics[width=5cm]{s02.png}
\caption{Sierpi\'nski gasket stage 0, a single triangle, and at stage 1, three triangles}
\end{centering}
\end{figure}
%\begin{figure}[h]
%\begin{centering}
%\includegraphics[width=5cm]{s02.eps}
%\caption{Stage 1, three triangles}
%\end{centering}
%\end{figure}
\begin{figure}[h]
\begin{centering}
\includegraphics[width=5cm]{s03.png}
\includegraphics[width=5cm]{s04.png}
\caption{Stage 2, nine triangles, and stage $n$, $3^n$ triangles}
\end{centering}
\end{figure}
%\begin{figure}[h]
%\begin{centering}
%\includegraphics[width=5cm]{s04.eps}
%\caption{Stage $n$, $3^n$ triangles}
%\end{centering}
%\end{figure}
%%%%%
%%%%%
% rerender..................
\end{document}
