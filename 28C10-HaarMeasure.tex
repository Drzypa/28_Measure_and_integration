\documentclass[12pt]{article}
\usepackage{pmmeta}
\pmcanonicalname{HaarMeasure}
\pmcreated{2013-03-22 12:40:55}
\pmmodified{2013-03-22 12:40:55}
\pmowner{djao}{24}
\pmmodifier{djao}{24}
\pmtitle{Haar measure}
\pmrecord{9}{32959}
\pmprivacy{1}
\pmauthor{djao}{24}
\pmtype{Definition}
\pmcomment{trigger rebuild}
\pmclassification{msc}{28C10}
\pmdefines{left Haar measure}
\pmdefines{right Haar measure}
\pmdefines{bi-invariant Haar measure}

% this is the default PlanetMath preamble.  as your knowledge
% of TeX increases, you will probably want to edit this, but
% it should be fine as is for beginners.

% almost certainly you want these
\usepackage{amssymb}
\usepackage{amsmath}
\usepackage{amsfonts}

% used for TeXing text within eps files
%\usepackage{psfrag}
% need this for including graphics (\includegraphics)
%\usepackage{graphicx}
% for neatly defining theorems and propositions
%\usepackage{amsthm}
% making logically defined graphics
%%%\usepackage{xypic} 

% there are many more packages, add them here as you need them

% define commands here
\begin{document}
\section{Definition of Haar measures}

Let $G$ be a locally compact topological group, and denote by $\mathcal{B}$ the sigma algebra generated by the closed compact subsets of $G$. A \emph{left Haar measure} on $G$ is a measure $\mu$ on $\mathcal{B}$ which is:
\begin{enumerate}
\item outer regular on all sets $B \in \mathcal{B}$
\item inner regular on all open sets $U \in \mathcal{B}$
\item finite on all compact sets $K \in \mathcal{B}$
\item invariant under left translation: $\mu(gB) = \mu(B)$ for all sets $B \in \mathcal{B}$
\item nontrivial: $\mu(B) > 0$ for all non-empty open sets $B \in \mathcal{B}$.
\end{enumerate}

A \emph{right Haar measure} on $G$ is defined similarly, except with left translation invariance replaced by right translation invariance ($\mu(Bg) = \mu(B)$ for all sets $B \in \mathcal{B}$). A \emph{bi-invariant Haar measure} is a Haar measure that is both left invariant and right invariant.

\section{Existence of Haar measures}

For any discrete topological group $G$, the counting measure on $G$ is a bi-invariant Haar measure. More generally, every locally compact topological group $G$ has a left Haar measure $\mu$, which is unique up to scalar multiples. In addition, $G$ also admits a right Haar measure, and for an abelian group $G$ the left and right Haar measures are always equal. The Haar measure plays an important role in the development of Fourier analysis and representation theory on locally compact groups such as Lie groups and profinite groups.

%%%%%
%%%%%
\end{document}
