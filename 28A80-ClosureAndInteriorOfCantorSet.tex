\documentclass[12pt]{article}
\usepackage{pmmeta}
\pmcanonicalname{ClosureAndInteriorOfCantorSet}
\pmcreated{2013-03-22 15:43:21}
\pmmodified{2013-03-22 15:43:21}
\pmowner{rspuzio}{6075}
\pmmodifier{rspuzio}{6075}
\pmtitle{closure and interior of Cantor set}
\pmrecord{4}{37671}
\pmprivacy{1}
\pmauthor{rspuzio}{6075}
\pmtype{Theorem}
\pmcomment{trigger rebuild}
\pmclassification{msc}{28A80}

% this is the default PlanetMath preamble.  as your knowledge
% of TeX increases, you will probably want to edit this, but
% it should be fine as is for beginners.

% almost certainly you want these
\usepackage{amssymb}
\usepackage{amsmath}
\usepackage{amsfonts}

% used for TeXing text within eps files
%\usepackage{psfrag}
% need this for including graphics (\includegraphics)
%\usepackage{graphicx}
% for neatly defining theorems and propositions
%\usepackage{amsthm}
% making logically defined graphics
%%%\usepackage{xypic}

% there are many more packages, add them here as you need them

% define commands here
\begin{document}
The Cantor set is closed and its interior is empty.

To prove the first assertion, note that each of the sets $C_0, C_1,
C_2, \ldots$, being the union of a finite number of closed intervals
is closed.  Since the Cantor set is the intersection of all these sets
and intersections of closed sets are closed, it follows that the
Cantor set is closed.

To prove the second assertion, it suffices to show that given any open
interval $I$, no matter how small, at least one point of that interval
will not belong to the Cantor set.  To accomplish this, the ternary
characterization of the Cantor set is useful.  Because rational
numbers whose denominators are powers of 3 are dense, there exists a
rational number $n/3^m$ contained in $I$.  Expressed in base 3, this
rational number has a finite expansion.  If this expansion contains
the digit ``1'', then our number does not belong to Cantor set, and we
are done.  If not, since $I$ is open, there must exist a number $k >
m$ such that $n/3^m + 1/3^k \in I$.  Now, the last digit of the ternary 
expansion of this number is ``1'' by construction, so we also find a 
number not belonging to the interval in this case.
%%%%%
%%%%%
\end{document}
