\documentclass[12pt]{article}
\usepackage{pmmeta}
\pmcanonicalname{TailEvent}
\pmcreated{2013-03-22 17:07:18}
\pmmodified{2013-03-22 17:07:18}
\pmowner{fernsanz}{8869}
\pmmodifier{fernsanz}{8869}
\pmtitle{tail event}
\pmrecord{9}{39424}
\pmprivacy{1}
\pmauthor{fernsanz}{8869}
\pmtype{Definition}
\pmcomment{trigger rebuild}
\pmclassification{msc}{28A05}
%\pmkeywords{sigma algebra}
%\pmkeywords{zero-one law}
%\pmkeywords{sigma algebra induced by random variables}
\pmrelated{SigmaAlgebra}
\pmrelated{KolmogorovZeroOneLaw}
\pmdefines{tail sigma algebra}

\endmetadata

% this is the default PlanetMath preamble.  as your knowledge
% of TeX increases, you will probably want to edit this, but
% it should be fine as is for beginners.

% almost certainly you want these
\usepackage{amssymb}
\usepackage{amsmath}
\usepackage{amsfonts}
\usepackage{amsthm}

% define commands here
\theoremstyle{definition}
\newtheorem*{defn}{Definition}
\theoremstyle{remark}
\newtheorem{rem}{Remark}
\numberwithin{equation}{section}
\newcommand{\N}{\mathbb N}
\begin{document}
\title{Tail event}%
\author{Fernando Sanz Gamiz}%

\begin{defn}
Let $\Omega$ be a set and $\mathcal F$ a sigma algebra of subsets
of $\Omega$. Given the random variables $\{X_n, n \in \N\}$, defined
on the measurable space $(\Omega,\mathcal F)$, the \emph{tail
events} are the events of the \emph{tail $\sigma$-algebra}
$$\mathcal F_{\infty}=\bigcap^{\infty}_{n=1}\sigma
(X_n,X_{n+1},\cdots)$$ where $\sigma (X_n,X_{n+1},\cdots)$ is the
$\sigma$-algebra induced by $(X_n,X_{n+1},\cdots)$.
\end{defn}

\medskip

\begin{rem}
One can intuitively think of tail events as those events whose
ocurrence or not is not affected by altering any finite number of
random variables in the sequence. Some examples are $$\{\lim \sup
X_n <c \}, \sum X_n \mbox{converges }, \lim  X_n \mbox{ exists}$$
\end{rem}

\medskip

\begin{rem}
One of the most important theorems in probability theory due to
Kolomogorv, is the Kolmogorov zero-one law which states that, in the case of independent random variables, the
probability of any tail event is 0 or 1 (provided there is a
probability measure defined on $(\Omega,\mathcal F)$)
\end{rem}
%%%%%
%%%%%
\end{document}
