\documentclass[12pt]{article}
\usepackage{pmmeta}
\pmcanonicalname{ProofOfFatouLebesgueTheorem}
\pmcreated{2013-03-22 15:58:50}
\pmmodified{2013-03-22 15:58:50}
\pmowner{Wkbj79}{1863}
\pmmodifier{Wkbj79}{1863}
\pmtitle{proof of Fatou-Lebesgue theorem}
\pmrecord{13}{37997}
\pmprivacy{1}
\pmauthor{Wkbj79}{1863}
\pmtype{Proof}
\pmcomment{trigger rebuild}
\pmclassification{msc}{28A20}

% this is the default PlanetMath preamble.  as your knowledge
% of TeX increases, you will probably want to edit this, but
% it should be fine as is for beginners.

% almost certainly you want these
\usepackage{amssymb}
\usepackage{amsmath}
\usepackage{amsfonts}

% used for TeXing text within eps files
%\usepackage{psfrag}
% need this for including graphics (\includegraphics)
%\usepackage{graphicx}
% for neatly defining theorems and propositions
\usepackage{amsthm}
% making logically defined graphics
%%%\usepackage{xypic}

% there are many more packages, add them here as you need them

% define commands here

\begin{document}
Since $\displaystyle \left| \int g \, d\mu \right| \le \int |g| \, d\mu \le \int \Phi \, d\mu < \infty$, we have that $\displaystyle \int g \, d\mu > - \infty$.  Similarly, $\displaystyle \int h \, d\mu < \infty$.

The inequality $\displaystyle \liminf_{n \to \infty} \int f_n \, d\mu \le \limsup_{n \to \infty} \int f_n \, d\mu$ is obvious by definition of $\liminf$ and $\limsup$.

Define a sequence of functions $k_n \colon X \to \mathbb{R}$ by $k_n(x)=f_n(x)+\Phi (x)$.  Then each $k_n$ is nonnegative (since $-f_n \le |f_n| \le \Phi$) and integrable (since $k_n \le |f_n|+\Phi \le 2\Phi$), as is $\displaystyle k := \liminf_{n \to \infty} k_n$.  Fatou's lemma yields that $\displaystyle \int k \, d\mu \le \liminf_{n \to \infty} \int k_n \, d\mu$.  Thus:

\begin{center}
$\begin{array}{ll}
\displaystyle \int g \, d\mu + \int \Phi \, d\mu & \displaystyle = \int (g+\Phi) \, d\mu \\
\\
& \displaystyle = \int k \, d\mu \\
\\
& \displaystyle \le \liminf_{n \to \infty} \int k_n \, d\mu \\
\\
& \displaystyle = \liminf_{n \to \infty} \int (f_n+\Phi) \, d\mu \\
\\
& \displaystyle = \liminf_{n \to \infty} \left( \int f_n \, d\mu + \int \Phi \, d\mu \right) \\
\\
& \displaystyle = \liminf_{n \to \infty} \int f_n \, d\mu + \liminf_{n \to \infty} \int \Phi \, d\mu \\
\\
& \displaystyle = \liminf_{n \to \infty} \int f_n \, d\mu + \int \Phi \, d\mu \end{array}$
\end{center}

Since $\displaystyle \int \Phi \, d\mu < \infty$, it follows that $\displaystyle \int g \, d\mu \le \liminf_{n \to \infty} \int f_n \, d\mu$.

Note that $|-f_n|=|f_n| \le \Phi$.  Thus,

\begin{center}
\begin{tabular}{ll}
$\displaystyle -\int h \, d\mu$ & $\displaystyle = \int -h \, d\mu$\\
\\
& $\displaystyle = \int -\limsup_{n \to \infty} f_n \, d\mu$ \\
\\
& $\displaystyle = \int \liminf_{n \to \infty} \left( -f_n \right) \, d\mu$ \\
\\
& $\displaystyle \le \liminf_{n \to \infty} \int -f_n \, d\mu$ by a previous \PMlinkescapetext{argument}, \\
\\
& $\displaystyle = \liminf_{n \to \infty} \left( -\int f_n \, d\mu \right)$ \\
\\
& $\displaystyle = -\limsup_{n \to \infty} \int f_n \, d\mu.$ \end{tabular}
\end{center}

Hence, $\displaystyle \limsup_{n \to \infty} \int f_n \, d\mu \le \int h \, d\mu$.  It follows that $\displaystyle -\infty < \int g \, d\mu \le \liminf_{n \to \infty} \int f_n \, d\mu \le \limsup_{n \to \infty} \int f_n \, d\mu \le \int h \, d\mu < \infty$.  $\qedsymbol$


%%%%%
%%%%%
\end{document}
