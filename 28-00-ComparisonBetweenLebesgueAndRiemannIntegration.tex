\documentclass[12pt]{article}
\usepackage{pmmeta}
\pmcanonicalname{ComparisonBetweenLebesgueAndRiemannIntegration}
\pmcreated{2013-03-22 15:29:14}
\pmmodified{2013-03-22 15:29:14}
\pmowner{Mathprof}{13753}
\pmmodifier{Mathprof}{13753}
\pmtitle{comparison between Lebesgue and Riemann Integration}
\pmrecord{7}{37343}
\pmprivacy{1}
\pmauthor{Mathprof}{13753}
\pmtype{Example}
\pmcomment{trigger rebuild}
\pmclassification{msc}{28-00}

% this is the default PlanetMath preamble. as your knowledge
% of TeX increases, you will probably want to edit this, but
% it should be fine as is for beginners.

% almost certainly you want these
\usepackage{amssymb}
\usepackage{amsmath}
\usepackage{amsfonts}
\usepackage{url}
%\usepackage{fullpage}

% used for TeXing text within eps files
%\usepackage{psfrag}
% need this for including graphics (\includegraphics)
%\usepackage{graphicx}
% for neatly defining theorems and propositions
\usepackage{amsthm}
\newtheorem{theorem}{Theorem}[section]

% there are many more packages, add them here as you need them

% define commands here
\newcommand{\mv}[1]{\mathbf{#1}} % matrix or vector
\newcommand{\cov}{\mathrm{cov}}
\newcommand{\mvt}[1]{\mv{#1}^{\mathrm{T}}}
\newcommand{\mvi}[1]{\mv{#1}^{-1}}
\newcommand{\mpderiv}[1]{\frac{\partial}{\partial {#1}}}
\newcommand{\borel}{\mathfrak{B}} \newcommand{\defined}{:=}
\newcommand{\1}{{{\bf 1}}}
\begin{document}
The Riemann and Lebesgue integral are defined in different ways, with
the latter generally perceived as the more general.  The aim of 
this article is to clarify this claim by providing a number of,
hopefully simple and convincing, examples and arguments. We restrict
this article to a discussion of proper and improper integrals. For
extensions and even more general definitions of the integral we refer
to  \url{http://www.math.vanderbilt.edu/~schectex/ccc/gauge/}:

\section{Proper Integrals}
\label{sec:some-examples}

\subsection*{The Dirichlet Function}
\label{sec:example-1}
Our first example shows that functions exists that are Lebesgue
integrable but not Riemann integrable.

Consider the characteristic function of the rational numbers in
$[0,1]$, i.e.,
\begin{equation*}
  1_{\mathbb{Q}} (x) = 
  \begin{cases}
    1, &\quad \text{if $x$ rational}, \\
0, &\quad \text{elsewhere}.
  \end{cases}
\end{equation*}
This function, known as the Dirichlet function, is not Riemann
integrable.  To see this, take an arbitrary partition of the interval
$[0,1]$. The supremum of $1_{\mathbb{Q}}$ on any interval
(with non-empty interior) is~$1$, whereas its infimum is~$0$.  Hence,
the upper Riemann sum of $1_{\mathbb{Q}}$ is~$1$ while the
lower Riemann sum is~$0$.  Clearly, the upper and lower Riemann sums
converge to~1 and~0, respectively, in the limit of the size of largest
interval in the partition going to zero.  Obviously, these limits are
not the same. As a bounded function is Riemann integrable if and only
if the upper and lower Riemann sums converge to the same number, the
Riemann integral of $1_{\mathbb{Q}}$ cannot exist.

On the other hand, $1_{\mathbb{Q}}$ turns out to be
Lebesgue integrable, which we now show. Let us enumerate the rationals
in $[0,1]$ as $\{q_0, q_1, \ldots \}$. Now cover each $q_i$ with an
open set $O_i$ of size $\epsilon/2^i$.  Hence, the set
$\{q_0,q_1,\ldots\}$ is contained in the set $\subset G_\epsilon :=
\cup_{i=0}^\infty O_i$.  Now it is known that every countable union of
open sets forms a Lebesgue measurable set. Therefore, $G_\epsilon$ is
also a Lebesgue measurable set.  Consequently, the Lebesgue integral
of this set's characteristic function, i.e.,
\begin{equation*}
  1_{G_\epsilon}(x) =
  \begin{cases}
    1, &\quad\text{if } x \in O, \\
    0, &\quad\text{elsewhere},
  \end{cases}
\end{equation*}
exists. 

In fact, the integral of $1_{G_\epsilon}$ is less than or equal to
$\epsilon\sum_{i=0}^\infty 2^{-i} = \epsilon$ as the total length of
the union of sets $O_i$ is less than or equal to $\epsilon$.  (There
might be overlaps between the $O_i$.) Now taking the limit
$\epsilon\to 0$ it follows that for all $\epsilon$,
\begin{equation*}
  \int 1_{\mathbb{Q}}\,dx \leq  \int
  1_{G_\epsilon} \, dx \leq \epsilon
\end{equation*}
Since this holds for all $\epsilon >0$, the left hand side must be~0.

\subsection*{A 'Worse' Kind of Dirichlet Function}

The above is, arguably, somewhat simple. The only reason that the
Dirichlet function is Lebesgue, but not Riemann, integrable, is that
its spikes occur on the rationals, a set of numbers which is, in
comparison to the irrational numbers, a very small set.  By modifying
the Dirichlet function on \emph{a set of measure zero}, that is, by
removing its spikes, it becomes the zero function, which is evidently
Riemann integrable.  This reasoning might lead us to conjecture that
it is possible to turn any Lebesgue integrable function into a Riemann
integrable function by modifying it on a set of measure zero. This
conjecture, however, is false as the next example shows.

Interestingly, the function we seek can be obtained in a more or less
direct way from the previous example. First cover the rationals by
open sets $O_i$ of length $y_i$, where the sequence of numbers of
$\{y_i\}_{i=0,\ldots}$ is such that $y_i>0$ for all $i$ but such that
$\sum_i y_i = 1/2$.  Observe that $G_y:=\cup_i O_i$ is a measurable
set with measure (length) less than or equal to $1/2$. But this
implies that the complement $G^c_y$ of $G_y$, which contains only
irrational numbers, has length at least $1/2$.

Observe that the characteristic function $1_{G_y}$ is
nowhere continuous on $G^c_y$, as the rationals lie dense in the
reals. Since, also, $G_y^c$  has measure at least equal
to $1/2$ it is impossible to remove these discontinuity points by a
mere modification on a countable number of measure zero sets. 

Now there is a theorem by Lebesgue stating that a bounded function $f$
is Riemann integrable if and only if $f$ is continuous almost
everywhere.  Apparently, $1_{G_y}$ is bounded and
discontinuous on a set with measure larger than $0$. Thus, we may
conclude that $1_{G_y}$ is not Riemann integrable.  

To prove that $1_{G_y}$ is Lebesgue integrable follows
easily if we approach the subject of integration from a somewhat more
abstract point of view. This is the topic of the next section.

\subsection*{Exchanging Limits and Integrals}

Let us first present one further example: the sequence of
characteristic functions of $\cup_{i=1}^n O_i$, where $O_i, i=1\ldots
n$ \emph{and } $n<\infty$, are the open sets appearing in the
definition of the 'worse Dirichlet function'.  Clearly, any such
function has a finite number of discontinuities, hence is Riemann
integrable. However, these functions converge point-wise to
$1_{G_y}$, which is \emph{not} Riemann integrable.
Apparently, sequences of Riemann integrable functions may converge to
non-Riemann integrable functions.  Interestingly, the sequence of
integrals of $1_{\cup_{i=1}^n O_i}$, i.e. a sequence of
reals, \emph{has} a limit as it is increasing and bounded by $1/2$.
This somewhat disturbing inconsistency is satisfactory resolved
by Lebesgue's theory of integration.

As a matter of fact, the advantage of Lebesgue integration is perhaps
best appreciated by interpreting this example from a more abstract
(functional analysis) point of view.  Stated a bit differently, we
might approach the subject not from the bottom up (looking at
individual functions) but from the top down (looking at classes of
functions).  In more detail, suppose we are allowed to apply an
operator $T$ to any function that is an element of some function
space. It would be nice if this space is closed under taking
(point-wise) limits.  In other words, besides being allowed to apply
$T$ to some sequence of functions $f_n$, we are also allowed to apply
$T$ to the function $f$ obtained as the limit of $f_n$.  It would be
even nicer if  $\lim_n Tf_n$ is the same as $T \lim_n f_n = Tf$.
(This is, for instance, useful when it is simple to compute $T f_n$
for each $n$, but difficult to compute $Tf$, while the latter might be
what really interests us.)

In the present case, i.e, integration, we perceive the integral of a
function as a (continuous linear) operator.  The class of Lebesgue
integrable functions has the desired abstract properties (simple
conditions to check whether the exchange of integral and limit is
allowed), whereas the class of Riemann integrable functions does not.

Applying this to the above example, viz. the integration of
$1_{G_y}$, we use Lebesgue Dominated Convergence Theorem,
which states that when a sequence $\left\{f_n\right\}$ of Lebesgue
measurable functions is bounded by a Lebesgue integrable function, the
function $f$ obtained as the pointwise limit $f_n$ is also Lebesgue
integrable, and $\int \lim_n f_n = \int f = \lim_n \int f_n$. Since,
for all $n$, $1_{\cup_i^n O_i}$ is bounded and Lebesgue
integrable,  $1_{G_y}$ is also Lebesgue integrable,
and reversing the (pointwise) limit and the integral is allowed.


\subsection{Fubini's Theorem}
\label{sec:fubinis-theorem}

Admittedly the function $1_{G_y}$ is rather artificial. 
A really powerful example of the consequences of being allowed to
reverse integral and limit is provided by (the proof of)
Fubini's theorem applied to the rectangle $Q=[a,b] \times [c,d]$.
Compare the following two theorems.  See, for instance,
\cite{apostol69:_calcul} or \cite{lang83:_under_analy} for proofs of
the first theorem, and \cite{jones00:_lebes_integ_euclid_spaces} for
the second theorem.

\begin{theorem}
  \label{thr:3}
  Riemann Case. Assume $f$ to be Riemann integrable on $Q$. Assume
  also that the one-dimensional function $x\mapsto f(x,y)$ is Riemann
  integrable for almost all $y\in[c,d]$. Then the function $y\mapsto
  \int_a^b f(x,y) \,dx$ is Riemann integrable and $\int_Q f(x,y) \,dx
  \,dy = \int_c^d \left(\int_a^b f(x,y) \,dx\right) \,dy$.  Note: both
  conditions are satisfied if $f$ is continuous on $Q$.
\end{theorem}

\begin{theorem}
  Lebesgue Case. Assume that $f$ is Lebesgue integrable on $Q$. Then
  the function $x\mapsto f(x,y)$ is Lebesgue integrable for almost all
  $y$ on $[c,d]$. As a consequence, the function $y\mapsto \int_a^b
  f(x,y) \,dx$ is Lebesgue integrable and $\int_Q f(x,y) \,dx \,dy =
  \int_c^d \left(\int_a^b f(x,y) \,dx\right) \,dy$.
\end{theorem}

Observe that the second \emph{assumption} in the Riemann case has
turned into a \emph{consequence} in the Lebesgue case.  The main
reason behind this difference is precisely that the class of Lebesgue
measurable functions is closed under taking limits (under a bounded
condition), whereas the class of Riemann integrable functions is not.

\section{Improper Integrals}
\label{sec:improper-integrals}
From the above the reader may conclude that whenever a function is
Riemann integrable, it is Lebesgue integrable. This is true as long we
only include \emph{proper} integrals. If, on the other hand, we also
consider \emph{improper} integrals the statement is no longer valid.
There exist functions whose improper Riemann integral exists, whereas
the Lebesgue integral does not.  Concentrating on functions defined on
subsets of $\mathbb{R}^n$ the situation is as shown by the following
Venn Diagram:
\begin{figure}[h]
  \begin{center}
  \begin{picture}(6,40)
    \put(1,10){\circle{50}}
    \put(20,10){\circle{50}}
    \put(11,10){\circle{15}}
    \put(25,6){$RI$}
    \put(7,6){$R$}
    \put(-10,6){$L$}
  \end{picture}
  \end{center}
\caption{$R$, $RI$, and $L$, are  the classes of Riemann, improper
  Riemann, and Lebesgue integrable functions, respectively.}
\end{figure}

In the previous we already discussed the inclusion  $R\subset L$.

Let us now integrate the function $1/\sqrt x$ over $[0,1]$ to show
that some functions exist $RI\cup L$ but are not in $R$. As $1/\sqrt
x$ is not bounded on $[0,1]$ every upper Riemann sum is infinite.  On
the other hand, both the improper Riemann integral and the Lebesgue
integral exist, and give the same result.

Secondly, $L$ is not contained in $RI$ as follows from the fact that
the Dirichlet function is not $RI$. 

Finally, $RI$ is also not a subset of $L$. Define $f:
[0,\infty)\mapsto [-1,1]$ as $1$ on $[0,1)$, $-1/2$ on $[1,2)$, $1/3$
on $[2,3)$, $-1/4$ on $[3,4)$, etc.  The Riemann integral and the
Lebesgue integral of $f$ over $[0,n)$ are both equal to
$\sum_{k=1}^{n} (-1)^{k+1} 1/k.$ It is well known that this
alternating sum converges, implying the existence of the improper
Riemann integral. However, since a function is Lebesgue integrable if
and only if its absolute value is also Lebesgue integrable, $f$ is not
in $L$.

\begin{thebibliography}{Apo69}

\bibitem[Apo69]{apostol69:_calcul}
T.M. Apostol.
\newblock {\em Calculus}, volume~2.
\newblock John Wiley \& Sons, 1969.

\bibitem[Jon00]{jones00:_lebes_integ_euclid_spaces}
F.~Jones.
\newblock {\em Lebesgue Integration on Euclidean Spaces}.
\newblock Jones and Bartlett, revised edition edition, 2000.

\bibitem[Lan83]{lang83:_under_analy}
S.~Lang.
\newblock {\em Undergraduate Analysis}.
\newblock Springer-Verlag, 1983.

\end{thebibliography}
%%%%%
%%%%%
\end{document}
