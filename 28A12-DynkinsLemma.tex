\documentclass[12pt]{article}
\usepackage{pmmeta}
\pmcanonicalname{DynkinsLemma}
\pmcreated{2013-03-22 18:33:05}
\pmmodified{2013-03-22 18:33:05}
\pmowner{gel}{22282}
\pmmodifier{gel}{22282}
\pmtitle{Dynkin's lemma}
\pmrecord{11}{41273}
\pmprivacy{1}
\pmauthor{gel}{22282}
\pmtype{Theorem}
\pmcomment{trigger rebuild}
\pmclassification{msc}{28A12}
\pmsynonym{pi-system d-system lemma}{DynkinsLemma}
%\pmkeywords{pi-system}
%\pmkeywords{Dynkin system}
%\pmkeywords{sigma algebra}
\pmrelated{PiSystem}
\pmrelated{DynkinSystem}
\pmrelated{UniquenessOfMeasuresExtendedFromAPiSystem}

% this is the default PlanetMath preamble.  as your knowledge
% of TeX increases, you will probably want to edit this, but
% it should be fine as is for beginners.

% almost certainly you want these
\usepackage{amssymb}
\usepackage{amsmath}
\usepackage{amsfonts}

% used for TeXing text within eps files
%\usepackage{psfrag}
% need this for including graphics (\includegraphics)
%\usepackage{graphicx}
% for neatly defining theorems and propositions
\usepackage{amsthm}
% making logically defined graphics
%%%\usepackage{xypic}

% there are many more packages, add them here as you need them

% define commands here

\newtheorem*{lemma}{Lemma}
\begin{document}
Dynkin's lemma is a result in measure theory showing that the \PMlinkname{$\sigma$-algebra}{SigmaAlgebra} generated by any given \PMlinkname{$\pi$-system}{PiSystem} on a set $X$ coincides with the Dynkin system generated the $\pi$-system. The result can be used to prove that measures are uniquely determined by their values on $\pi$-systems generating the required $\sigma$-algebra. For example, the Borel $\sigma$-algebra on $\mathbb{R}$ is generated by the $\pi$-system of open intervals $(a,b)$ for $a<b$ and consequently the Lebesgue measure $\mu$ is uniquely determined by the property that $\mu( (a,b) )=b-a$.

Note that this lemma generalizes the statement that a Dynkin system which is also a $\pi$-system is a $\sigma$-algebra.


\begin{lemma}[Dynkin]
Let $A$ be a $\pi$-system on a set $X$. Then $\mathcal{D}(A)=\sigma(A)$. That is, the smallest Dynkin system containing $A$ coincides with the $\sigma$-algebra generated by $A$.
\end{lemma}
\begin{proof}
As $A$ is a $\pi$-system, the set $\mathcal{D}_1\equiv\{S\subseteq X:S\cap T\in \mathcal{D}(A)\text{ for every }T\in A\}$ contains $A$. We show that $\mathcal{D}_1$ is also a Dynkin system.

First, for every $T\in A$, $X\cap T=T\in A$ so $X$ is in $\mathcal{D}_1$. Second, if $S_1\subseteq S_2$ are in $\mathcal{D}_1$ and $T\in A$ then $(S_2\setminus S_1)\cap T = (S_2\cap T)\setminus(S_1\cap T)$ is in $\mathcal{D}(A)$ showing that $S_2\setminus S_1\in\mathcal{D}_1$.
Finally. if $S_n\in\mathcal{D}_1$ is a sequence increasing to $S\subseteq X$ and $T\in A$ then $S_n\cap T$ is a sequence in $\mathcal{D}(A)$ increasing to $S\cap T$. As Dynkin systems are closed under limits of increasing sequences this shows that $S\cap T\in\mathcal{D}(A)$ and therefore $S\in\mathcal{D}_1$. So $\mathcal{D}_1$ is indeed a Dynkin system. In particular, $\mathcal{D}(A)\subseteq\mathcal{D}_1$ and $S\cap T\in\mathcal{D}(A)$ for all $S\in\mathcal{D}(A)$ and $T\in A$.

We now set $\mathcal{D}_2\equiv\{S\subseteq X:S\cap T\in\mathcal{D}(A)\text{ for every }T\in\mathcal{D}(A)\}$ which, as shown above, contains $A$. Also, as in the argument above for $\mathcal{D}_1$, $\mathcal{D}_2$ is a Dynkin system. Therefore, $\mathcal{D}(A)$ is contained in $\mathcal{D}_2$ and it follows that $S\cap T\in\mathcal{D}(A)$ for any $S,T\in\mathcal{D}(A)$. So $\mathcal{D}(A)$ is both a $\pi$-system and a Dynkin system.

We can now show that $\mathcal{D}(A)$ is a $\sigma$-algebra. As it is a Dynkin system, $S^c=X\setminus S\in\mathcal{D}(A)$ for every $S\in\mathcal{D}(A)$ and, as it is also a $\pi$-system, this shows that $\mathcal{D}(A)$ is an algebra of sets on $X$. Finally, choose any sequence $A_n\in\mathcal{D}(A)$. Then, $\bigcup_{m=1}^nA_m$ is a sequence in $\mathcal{D}(A)$ increasing to $\bigcup_nA_n$ which, as $\mathcal{D}(A)$ is Dynkin system, must be in $\mathcal{D}(A)$. So, $\mathcal{D}(A)$ is a $\sigma$-algebra and must contain $\sigma(A)$. Conversely, as $\sigma(A)$ is a Dynkin system (as it is a $\sigma$-algebra) containing $A$, it must also contain $\mathcal{D}(A)$.
\end{proof}

\begin{thebibliography}{9}
\bibitem{williams}
David Williams, \emph{Probability with martingales},
Cambridge Mathematical Textbooks, Cambridge University Press, 1991.
\end{thebibliography}
%%%%%
%%%%%
\end{document}
