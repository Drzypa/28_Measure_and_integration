\documentclass[12pt]{article}
\usepackage{pmmeta}
\pmcanonicalname{ConstructionOfOuterMeasures}
\pmcreated{2013-03-22 18:33:17}
\pmmodified{2013-03-22 18:33:17}
\pmowner{gel}{22282}
\pmmodifier{gel}{22282}
\pmtitle{construction of outer measures}
\pmrecord{10}{41277}
\pmprivacy{1}
\pmauthor{gel}{22282}
\pmtype{Theorem}
\pmcomment{trigger rebuild}
\pmclassification{msc}{28A12}
%\pmkeywords{outer measure}
\pmrelated{OuterMeasure}
\pmrelated{LebesgueOuterMeasure}
\pmrelated{CaratheodorysLemma}

% this is the default PlanetMath preamble.  as your knowledge
% of TeX increases, you will probably want to edit this, but
% it should be fine as is for beginners.

% almost certainly you want these
\usepackage{amssymb}
\usepackage{amsmath}
\usepackage{amsfonts}

% used for TeXing text within eps files
%\usepackage{psfrag}
% need this for including graphics (\includegraphics)
%\usepackage{graphicx}
% for neatly defining theorems and propositions
\usepackage{amsthm}
% making logically defined graphics
%%%\usepackage{xypic}

% there are many more packages, add them here as you need them

% define commands here
\newtheorem*{theorem}{Theorem}

\begin{document}
The following theorem is used in measure theory to construct \PMlinkname{outer measures}{OuterMeasure2} on a set $X$, starting with a non-negative function on a collection of subsets of $X$. For example, if we take $X$ to be the real numbers, $\mathcal{C}$ to be the collection of bounded open intervals of $\mathbb{R}$ and define $p$ by $p((a,b))=b-a$ for real numbers $a<b$, then the Lebesgue outer measure is obtained.

\begin{theorem}
Let $X$ be a set, $\mathcal{C}$ be a family of subsets of $X$ containing the empty set and $p\colon\mathcal{C}\rightarrow \mathbb{R}\cup\{\infty\}$ be a function satisfying $p(\emptyset)=0$.
Then the function $\mu^*\colon\mathcal{P}(X)\rightarrow\mathbb{R}\cup\{\infty\}$ defined by
\begin{equation}\label{eq:1}
\mu^*(A)=\inf\left\{\sum_{i=1}^\infty p(A_i): A_i\in\mathcal{C},\ A\subseteq\bigcup_{i=1}^\infty A_i\right\}
\end{equation}
is an outer measure.
\end{theorem}
\begin{proof}
The definition of $\mu^*$ immediately gives $\mu^*(A)\le\mu^*(B)$ for sets $A\subseteq B$, and if $A=\emptyset$ then we can take $A_i=\emptyset$ in (\ref{eq:1}) to obtain $\mu^*(\emptyset)\le\sum_ip(\emptyset)=0$, giving $\mu^*(\emptyset)=0$. Only the countable subadditivity of $\mu^*$ remains to be shown.
That is, if $A_i$ is a sequence in $\mathcal{P}(X)$ then
\begin{equation}\label{eq:2}
\mu^*\left(\bigcup_i A_i\right)\le\sum_i\mu^*(A_i).
\end{equation}
To prove this inequality, we may restrict to the case where $\mu^*(A_i)<\infty$ for each $i$ so that, choosing any $\epsilon>0$, equation (\ref{eq:1}) says that there exists a sequence $A_{i,j}\in \mathcal{C}$ such that $A_i\subseteq\bigcup_j A_{i,j}$ and,
\begin{equation*}
\sum_{j=1}^\infty p(A_{i,j})\le\mu^*(A_i)+2^{-i}\epsilon.
\end{equation*}
As $\bigcup_iA_i\subseteq\bigcup_{i,j}A_{i,j}$, equation (\ref{eq:1}) defining $\mu^*$ gives
\begin{equation*}
\mu^*\left(\bigcup_iA_i\right)\le\sum_{i,j}p(A_{i,j})=\sum_i\sum_jp(A_{i,j})\le\sum_i(\mu^*(A_i)+2^{-i}\epsilon)=\sum_i\mu^*(A_i)+\epsilon.
\end{equation*}
As $\epsilon>0$ is arbitrary, this proves subadditivity (\ref{eq:2}).
\end{proof}

Although this result is rather general, placing few restrictions on the function $p$, there is no guarantee that the outer measure $\mu^*$ will agree with $p$ for the sets in $\mathcal{C}$ nor that $\mathcal{C}$ will consist of \PMlinkname{$\mu^*$-measurable}{CaratheodorysLemma} sets.

For example, if $X=\mathbb{R}$, $\mathcal{C}$ consists of the bounded open intervals, and $p((a,b))=(b-a)^2$ for real numbers $a<b$, then $\mu^*((a,b))=0\not=p((a,b))$.

Alternatively if $p((a,b))=\sqrt{b-a}$ for all $a<b$ then it follows that $\mu^*((a,b))=\sqrt{b-a}$ so
\begin{equation*}
\mu^*((0,1))+\mu^*([1,2))=1+1\not=\mu^*((0,2))=\sqrt{2},
\end{equation*}
and $(0,1)$ is not $\mu^*$-measurable.




%%%%%
%%%%%
\end{document}
