\documentclass[12pt]{article}
\usepackage{pmmeta}
\pmcanonicalname{sigmaalgebra}
\pmcreated{2013-03-22 12:00:28}
\pmmodified{2013-03-22 12:00:28}
\pmowner{yark}{2760}
\pmmodifier{yark}{2760}
\pmtitle{$\sigma$-algebra}
\pmrecord{16}{30950}
\pmprivacy{1}
\pmauthor{yark}{2760}
\pmtype{Definition}
\pmcomment{trigger rebuild}
\pmclassification{msc}{28A60}
\pmsynonym{sigma-algebra}{sigmaalgebra}
\pmsynonym{sigma algebra}{sigmaalgebra}
\pmsynonym{$\sigma$ algebra}{sigmaalgebra}
\pmsynonym{Borel structure}{sigmaalgebra}
\pmsynonym{$\sigma$-field}{sigmaalgebra}
\pmsynonym{sigma-field}{sigmaalgebra}
\pmsynonym{sigma field}{sigmaalgebra}
\pmsynonym{$\sigma$ field}{sigmaalgebra}
\pmrelated{Algebra2}
\pmrelated{BorelSigmaAlgebra}
\pmrelated{MathcalFMeasurableFunction}
\pmrelated{RingOfSets}
\pmdefines{generated by}

\endmetadata

\usepackage{amssymb}

\def\emptyset{\varnothing}
\def\F{\mathcal{F}}
\def\R{\mathbb{R}}
\def\powerset#1{\mathcal{P}(#1)}

\begin{document}
\PMlinkescapeword{preserve}

\section*{Introduction}

When defining a measure for a set $E$
we usually cannot hope to make every subset of $E$ measurable.
Instead we must usually restrict our attention
to a specific collection of subsets of $E$,
requiring that this collection be closed under operations
that we would expect to preserve measurability.
A $\sigma$-algebra is such a collection.

\section*{Definition}

Given a set $E$, a \emph{$\sigma$-algebra} in $E$
is a collection $\F$ of subsets of $E$ such that:
\begin{itemize}
\item $\emptyset\in\F$.
\item Any union of countably many elements of $\F$
      is an element of $\F$.
\item The complement of any element of $\F$ in $E$
      is an element of $\F$.
\end{itemize}

\section*{Notes}

It follows from the definition that any $\sigma$-algebra $\F$ in $E$
also satisfies the properties:
\begin{itemize}
\item $E\in\F$.
\item Any intersection of countably many elements of $\F$
      is an element of $\F$.
\end{itemize}

Note that a $\sigma$-algebra is a field of sets
that is closed under countable unions and countable intersections
(rather than just finite unions and finite intersections).

Given any collection $C$ of subsets of $E$,
the $\sigma$-algebra $\sigma(C)$ \emph{generated by} $C$
is defined to be the smallest $\sigma$-algebra in $E$
such that $C\subseteq \sigma(C)$.
This is well-defined,
as the intersection of any non-empty collection of $\sigma$-algebras in $E$
is also a $\sigma$-algebra in $E$.

\section*{Examples}

For any set $E$, 
the power set $\powerset{E}$ is a $\sigma$-algebra in $E$,
as is the set $\{\emptyset,E\}$.

A more interesting example is the 
\PMlinkname{Borel $\sigma$-algebra}{BorelSigmaAlgebra} in $\R$,
which is the $\sigma$-algebra generated by the open subsets of $\R$,
or, equivalently,
the $\sigma$-algebra generated by the compact subsets of $\R$.
%%%%%
%%%%%
%%%%%
\end{document}
