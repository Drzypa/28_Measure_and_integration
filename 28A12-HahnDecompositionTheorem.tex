\documentclass[12pt]{article}
\usepackage{pmmeta}
\pmcanonicalname{HahnDecompositionTheorem}
\pmcreated{2013-03-22 13:26:59}
\pmmodified{2013-03-22 13:26:59}
\pmowner{Koro}{127}
\pmmodifier{Koro}{127}
\pmtitle{Hahn decomposition theorem}
\pmrecord{10}{34014}
\pmprivacy{1}
\pmauthor{Koro}{127}
\pmtype{Theorem}
\pmcomment{trigger rebuild}
\pmclassification{msc}{28A12}
\pmdefines{Hahn decomposition}

% this is the default PlanetMath preamble.  as your knowledge
% of TeX increases, you will probably want to edit this, but
% it should be fine as is for beginners.

% almost certainly you want these
\usepackage{amssymb}
\usepackage{amsmath}
\usepackage{amsfonts}

% used for TeXing text within eps files
%\usepackage{psfrag}
% need this for including graphics (\includegraphics)
%\usepackage{graphicx}
% for neatly defining theorems and propositions
%\usepackage{amsthm}
% making logically defined graphics
%%%\usepackage{xypic}

% there are many more packages, add them here as you need them
\usepackage{mathrsfs}
% define commands here
\begin{document}
\PMlinkescapeword{decomposition}\PMlinkescapeword{decompositions}
Let $\mu$ be a signed measure in the measurable space $(\Omega,\mathscr{S})$. There are two measurable sets $A$ and $B$ such that:
\begin{enumerate}
\item $A\cup B = \Omega$ and $A\cap B = \emptyset$;
\item $\mu(E)\geq 0$ for each $E\in\mathscr{S}$ such that $E\subset A$;
\item $\mu(E)\leq 0$ for each $E\in\mathscr{S}$ such that $E\subset B$.
\end{enumerate}

The pair $(A,B)$ is called a \emph{Hahn decomposition} for $\mu$.
This decomposition is not unique, but any other such decomposition $(A',B')$ satisfies $\mu(A'\vartriangle A) = \mu(B\vartriangle B') = 0$ (where $\vartriangle$ denotes the symmetric difference), so the two decompositions differ in a set of measure 0.
%%%%%
%%%%%
\end{document}
