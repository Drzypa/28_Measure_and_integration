\documentclass[12pt]{article}
\usepackage{pmmeta}
\pmcanonicalname{ProofOfVitalisTheorem}
\pmcreated{2013-03-22 13:45:50}
\pmmodified{2013-03-22 13:45:50}
\pmowner{paolini}{1187}
\pmmodifier{paolini}{1187}
\pmtitle{proof of Vitali's Theorem}
\pmrecord{7}{34467}
\pmprivacy{1}
\pmauthor{paolini}{1187}
\pmtype{Proof}
\pmcomment{trigger rebuild}
\pmclassification{msc}{28A05}
\pmrelated{ProofOfPsuedoparadoxInMeasureTheory}

% this is the default PlanetMath preamble.  as your knowledge
% of TeX increases, you will probably want to edit this, but
% it should be fine as is for beginners.

% almost certainly you want these
\usepackage{amssymb}
\usepackage{amsmath}
\usepackage{amsfonts}

% used for TeXing text within eps files
%\usepackage{psfrag}
% need this for including graphics (\includegraphics)
%\usepackage{graphicx}
% for neatly defining theorems and propositions
\usepackage{amsthm}
% making logically defined graphics
%%%\usepackage{xypic}

% there are many more packages, add them here as you need them

% define commands here
\newcommand{\R}{\mathbb R}
\newtheorem{theorem}{Theorem}
\newtheorem{definition}{Definition}
\theoremstyle{remark}
\newtheorem{example}{Example}
\begin{document}
Consider the equivalence relation in $[0,1)$ given by
\[
  x\sim y \quad \Leftrightarrow\quad x-y\in\mathbb Q
\]
and let $\mathcal F$ be the family of all equivalence classes of $\sim$.
Let $V$ be a \PMlinkescapetext{section} of $\mathcal F$ i.e. put in $V$ an 
element for each equivalence class of $\sim$ (notice that we are using the axiom 
of choice).

Given $q\in\mathbb Q\cap [0,1)$ define 
\[
V_q=((V+q)\cap [0,1))\cup((V+q-1)\cap[0,1))
\]
that is $V_q$ is obtained translating $V$ by a quantity $q$ to the right and then cutting the piece which goes beyond the point $1$ and putting it on the left, starting from $0$.

Now notice that given $x\in[0,1)$ there exists $y\in V$ such that $x\sim y$ (because $V$ is a section of $\sim$) and hence there exists $q\in \mathbb Q\cap[0,1)$ 
such that $x\in V_q$. So 
\[
\bigcup_{q\in\mathbb Q\cap[0,1)} V_q = [0,1).
\]

Moreover all the $V_q$ are disjoint. In fact if $x\in V_q\cap V_p$ then $x-q$ (modulus $[0,1)$) and $x-p$ are both in $V$ which is not possible since they differ by a rational quantity $q-p$ (or $q-p+1$).

Now if $V$ is Lebesgue measurable, clearly also $V_q$ are measurable and $\mu(V_q)=\mu(V)$. Moreover by the countable additivity of $\mu$ we have
\[
  \mu([0,1)) = \sum_{q\in \mathbb Q\cap[0,1)} \mu(V_q) 
  = \sum_q \mu(V).
\]
So if $\mu(V)=0$ we had $\mu([0,1))=0$ and if $\mu(V)>0$ we had $\mu([0,1))=+\infty$.

So the only possibility is that $V$ is not Lebesgue measurable.
%%%%%
%%%%%
\end{document}
