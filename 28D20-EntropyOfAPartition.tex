\documentclass[12pt]{article}
\usepackage{pmmeta}
\pmcanonicalname{EntropyOfAPartition}
\pmcreated{2013-03-22 14:31:56}
\pmmodified{2013-03-22 14:31:56}
\pmowner{Koro}{127}
\pmmodifier{Koro}{127}
\pmtitle{entropy of a partition}
\pmrecord{6}{36076}
\pmprivacy{1}
\pmauthor{Koro}{127}
\pmtype{Definition}
\pmcomment{trigger rebuild}
\pmclassification{msc}{28D20}
\pmclassification{msc}{37A35}
\pmrelated{HartleyFunction}
\pmdefines{entropy of a $\sigma$-algebra}
\pmdefines{entropy of a sigma-algebra}
\pmdefines{measurable partition}

% this is the default PlanetMath preamble.  as your knowledge
% of TeX increases, you will probably want to edit this, but
% it should be fine as is for beginners.

% almost certainly you want these
\usepackage{amssymb}
\usepackage{amsmath}
\usepackage{amsfonts}
\usepackage{mathrsfs}

% used for TeXing text within eps files
%\usepackage{psfrag}
% need this for including graphics (\includegraphics)
%\usepackage{graphicx}
% for neatly defining theorems and propositions
%\usepackage{amsthm}
% making logically defined graphics
%%%\usepackage{xypic}

% there are many more packages, add them here as you need them

% define commands here
\newcommand{\C}{\mathbb{C}}
\newcommand{\R}{\mathbb{R}}
\newcommand{\N}{\mathbb{N}}
\newcommand{\Z}{\mathbb{Z}}
\newcommand{\Per}{\operatorname{Per}}
\begin{document}
Let $(X,\mathscr{B},\mu)$ be a probability space. A measurable partition of $X$ is a partition such that each of its
elements is a measurable set (i.e. an element of $\mathscr{B}$).

Given a finite measurable partition $\mathcal{P}$,
its \emph{\PMlinkescapetext{entropy}} is
\[ H_\mu(\mathcal{P})=\sum_{P\in\mathcal{P}} -\mu(P)\log\mu(P),\]
where we assume $0\log 0 = 0$ for convenience.

\textbf{Remarks.}
\begin{enumerate}
        \item   Entropy can be interpreted as a measure of the \textit{a priori} uncertainity about the
                outcome of the measurement an experiment, assuming that we are measuring it through the given partition
                (i.e., we are going to be told in which atom of the partition the result is).
                Thus, the finer a partition is, the higher the resulting entropy. In particular, the trivial
                partition $\{X\}$ has entropy $0$, since there is only one possible outcome, so there is no
                uncertainity at all. On the other hand, the measurement gives no information at all
                about the ``real'' outcome of the experiment, which reflects the complementary intepretation of entropy:
                as the information gained from the measurement.
                This is because of the intuitive fact that more
                uncertainity about the outcome of the measurement means that more information will be obtained from
                knowing it about the ``real'' outcome.


        \item   Equally intuitive is the fact that among all measurable partitions of $X$ into $n$ atoms, the maximum
                possible
                entropy is attained at those in which the atoms are equally likely (i.e., all atoms have equal
                measure $1/n$).
                This can be proved by means of standard calculus, and
                a direct computation shows that the maximum value is $\log n$.

       \item   Since the definition of entropy involves only the measure of atoms of the given partition, two
                partitions which are equal modulo measure zero have the same entropy.
       \item There is a natural correspondence between finite measurable partitions and finite sub-$\sigma$-algebras of $\mathscr{B}$. For this reason,
to each finite sub-$\sigma$-algebra $\mathscr{P}$ we can define its entropy by $H_\mu(\mathcal{P})$ where $\mathcal{P}$ is the (unique) partition which generates $\mathscr{P}$. For short, we denote this entropy by $H_\mu(\mathscr{P})$.
\end{enumerate}
%%%%%
%%%%%
\end{document}
