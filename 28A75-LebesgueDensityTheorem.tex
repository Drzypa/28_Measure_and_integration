\documentclass[12pt]{article}
\usepackage{pmmeta}
\pmcanonicalname{LebesgueDensityTheorem}
\pmcreated{2013-03-22 13:21:02}
\pmmodified{2013-03-22 13:21:02}
\pmowner{bbukh}{348}
\pmmodifier{bbukh}{348}
\pmtitle{Lebesgue density theorem}
\pmrecord{7}{33869}
\pmprivacy{1}
\pmauthor{bbukh}{348}
\pmtype{Theorem}
\pmcomment{trigger rebuild}
\pmclassification{msc}{28A75}

\usepackage{amssymb}
\usepackage{amsmath}
\usepackage{amsfonts}
\begin{document}
Let $\mu$ be the Lebesgue measure on $\mathbb{R}^n$, and for a
measurable set $A\subset \mathbb{R}^n$ define the density of $A$ in
$\epsilon$-neighborhood of $x\in\mathbb{R}^n$ by
\begin{equation*}
d_\epsilon(x)=\frac{\mu(A\cap B_\epsilon(x))}{\mu(B_\epsilon(x))}
\end{equation*}
where $B_\epsilon(x)$ denotes the ball of radius $\epsilon$
centered at $x$.

The Lebesgue density theorem asserts that for almost every point of
$A$ the density
\begin{equation*}\label{eq:densdef}
d(x)=\lim_{\epsilon\to 0} d_{\epsilon}(x)
\end{equation*}
exists and is equal to $1$.

In other words, for every measurable set $A$ the density of $A$ is
$0$ or $1$ almost everywhere. However, it is a curious fact that
if $\mu(A)>0$ and $\mu(\mathbb{R}^n\setminus A)>0$, then there are
always points of $\mathbb{R}^n$ where the density is neither $0$
nor $1$ \cite[Lemma 4]{cite:croft_latticepts}.

\begin{thebibliography}{1}

\bibitem{cite:croft_latticepts}
Hallard~T. Croft.
\newblock Three lattice-point problems of {Steinhaus}.
\newblock {\em Quart. J. Math. Oxford (2)}, 33:71--83, 1982.
\newblock \PMlinkexternal{Zbl
  0499.10035}{http://www.emis.de/cgi-bin/zmen/ZMATH/en/quick.html?type=html&an=0499.10035}.

\end{thebibliography}
%%%%%
%%%%%
\end{document}
