\documentclass[12pt]{article}
\usepackage{pmmeta}
\pmcanonicalname{KolmogorovZerooneLaw}
\pmcreated{2013-03-22 17:07:21}
\pmmodified{2013-03-22 17:07:21}
\pmowner{fernsanz}{8869}
\pmmodifier{fernsanz}{8869}
\pmtitle{Kolmogorov zero-one law}
\pmrecord{10}{39425}
\pmprivacy{1}
\pmauthor{fernsanz}{8869}
\pmtype{Definition}
\pmcomment{trigger rebuild}
\pmclassification{msc}{28A05}
%\pmkeywords{tail event}
%\pmkeywords{tail sigma algebra}
%\pmkeywords{Hewitt-Savage zero-one law}
%\pmkeywords{sigma algebra induced by random variables}
\pmrelated{TailEvent}

% this is the default PlanetMath preamble.  as your knowledge
% of TeX increases, you will probably want to edit this, but
% it should be fine as is for beginners.

% almost certainly you want these
\usepackage{amssymb}
\usepackage{amsmath}
\usepackage{amsfonts}
\usepackage{amsthm}

% define commands here
\newtheorem*{thm}{Theorem}
\theoremstyle{remark}
\newtheorem{rem}{Remark}
\numberwithin{equation}{section}
\newcommand{\N}{\mathbb N}
\newcommand{\F}{\mathcal F}
\begin{document}
\title{Kolomogorov zero-one law}%
\author{Fernando Sanz Gamiz}%

\begin{thm}[Kolmogorov]
Let $\Omega$ be a set, $\F$ a sigma-algebra of subsets of $\Omega$
and $P$ a probability measure. Given the independent random
variables $\{X_n, n \in \N\}$, defined on $(\Omega,\F, P)$, it
happens that $$P(A)=0 \mbox{ or } P(A)=1, A \in \F_{\infty},$$
i.e.,the probability of any tail event is 0 or 1.
\end{thm}

\medskip

\begin{proof}
Define $\F_n = \sigma(X_1,X_2,...,X_n)$. As any event in
$\sigma(X_{n+1},X_{n+2},...)$ is independent of any event in
$\sigma(X_1,X_2,...,X_n)$ \footnote{this assertion should be proved
actually, because independence of random variables is defined for
every finite number of them and we are dealing with events involving
an infinite number. By two successive applications of the Monotone
Class Theorem, one can readily prove this is in fact correct}, any
event in the tail $\sigma$-algebra $\F_{\infty}$ is independent of
any event in $\bigcup_{n=1}^{\infty} \F_n$; hence, any event in
$\F_{\infty}$ is independent of any event in
$\sigma(\bigcup_{n=1}^{\infty} \F_n)$ \footnote{again by application
of the Monotone Class Theorem}. But $\F_{\infty} \subset
\sigma(\bigcup_{n=1}^{\infty} \F_n)$ \footnote{because $\F_{\infty}
\subset \sigma(X_1,X_2,...)=\sigma(\bigcup_{n=1}^{\infty} \F_n)$,
this last equality being easily proved}, so any tail event is
independent of itself, i.e., $P(A)=P(A\cap A)=P(A)P(A)$ which
implies $P(A)=0$ or $P(A)=1$.
\end{proof}
%%%%%
%%%%%
\end{document}
