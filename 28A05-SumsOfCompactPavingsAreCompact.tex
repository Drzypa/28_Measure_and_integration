\documentclass[12pt]{article}
\usepackage{pmmeta}
\pmcanonicalname{SumsOfCompactPavingsAreCompact}
\pmcreated{2013-03-22 18:45:15}
\pmmodified{2013-03-22 18:45:15}
\pmowner{gel}{22282}
\pmmodifier{gel}{22282}
\pmtitle{sums of compact pavings are compact}
\pmrecord{5}{41532}
\pmprivacy{1}
\pmauthor{gel}{22282}
\pmtype{Theorem}
\pmcomment{trigger rebuild}
\pmclassification{msc}{28A05}
\pmsynonym{disjoint unions of compact pavings are compact}{SumsOfCompactPavingsAreCompact}
%\pmkeywords{compact paving}
%\pmkeywords{direct sum}
\pmrelated{ProductsOfCompactPavingsAreCompact}
\pmdefines{direct sum of pavings}
\pmdefines{disjoint union of pavings}

\endmetadata

% almost certainly you want these
\usepackage{amssymb}
\usepackage{amsmath}
\usepackage{amsfonts}

% used for TeXing text within eps files
%\usepackage{psfrag}
% need this for including graphics (\includegraphics)
%\usepackage{graphicx}
% for neatly defining theorems and propositions
\usepackage{amsthm}
% making logically defined graphics
%%%\usepackage{xypic}

% there are many more packages, add them here as you need them

% define commands here
\newtheorem*{theorem*}{Theorem}
\newtheorem*{lemma*}{Lemma}
\newtheorem*{corollary*}{Corollary}
\newtheorem*{definition*}{Definition}
\newtheorem{theorem}{Theorem}
\newtheorem{lemma}{Lemma}
\newtheorem{corollary}{Corollary}
\newtheorem{definition}{Definition}

\begin{document}
\PMlinkescapeword{direct sum}
\PMlinkescapeword{index set}
\PMlinkescapeword{compact}
\PMlinkescapeword{compact paving}
\PMlinkescapeword{finite}
\PMlinkescapeword{compactness}
\PMlinkescapeword{satisfies}
\PMlinkescapeword{union}
\PMlinkescapeword{disjoint}
\PMlinkescapeword{subsets}

Suppose that $(K_i,\mathcal{K}_i)$ is a paved space for each $i$ in an index set $I$. The direct sum, or \PMlinkname{disjoint union}{DisjointUnion}, $\sum_{i\in I}K_i$ is the union of the disjoint sets $K_i\times\{i\}$. The direct sum of the paving $\mathcal{K}_i$ is defined as
\begin{equation*}
\sum_{i\in I}\mathcal{K}_i=\left\{\sum_{i\in I}S_i\colon S_i\in\mathcal{K}_i\cup\{\emptyset\}\text{ is empty for all but finitely many }i\right\}.
\end{equation*}

\begin{theorem*}
Let $(K_i,\mathcal{K}_i)$ be compact paved spaces for $i\in I$. Then, $\sum_i\mathcal{K}_i$ is a compact paving on $\sum_iK_i$.
\end{theorem*}

The paving $\mathcal{K}^\prime$ consisting of subsets of $\sum_i\mathcal{K}_i$ of the form $\sum_iS_i$ where $S_i=\emptyset$ for all but a single $i\in I$ is easily shown to be compact.
Indeed, if $\mathcal{K}^{\prime\prime}\subseteq \mathcal{K}^\prime$ satisfies the finite intersection property then there is an $i\in I$ such that $S\subseteq K_i\times\{i\}$ for every $S\in\mathcal{K}^{\prime\prime}$. Compactness of $\mathcal{K}_i$ gives $\bigcap\mathcal{K}^{\prime\prime}\not=\emptyset$.

Then, as $\sum_i\mathcal{K}_i$ consists of finite unions of sets in $\mathcal{K}^\prime$, it is a compact paving (see compact pavings are closed subsets of a compact space).

%%%%%
%%%%%
\end{document}
