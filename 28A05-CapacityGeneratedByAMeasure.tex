\documentclass[12pt]{article}
\usepackage{pmmeta}
\pmcanonicalname{CapacityGeneratedByAMeasure}
\pmcreated{2013-03-22 18:47:35}
\pmmodified{2013-03-22 18:47:35}
\pmowner{gel}{22282}
\pmmodifier{gel}{22282}
\pmtitle{capacity generated by a measure}
\pmrecord{7}{41591}
\pmprivacy{1}
\pmauthor{gel}{22282}
\pmtype{Theorem}
\pmcomment{trigger rebuild}
\pmclassification{msc}{28A05}
\pmclassification{msc}{28A12}
\pmsynonym{outer measure generated by a measure}{CapacityGeneratedByAMeasure}
%\pmkeywords{measure}
%\pmkeywords{capacity}
%\pmkeywords{outer measure}
\pmdefines{outer measure generated by}

\endmetadata

% almost certainly you want these
\usepackage{amssymb}
\usepackage{amsmath}
\usepackage{amsfonts}

% used for TeXing text within eps files
%\usepackage{psfrag}
% need this for including graphics (\includegraphics)
%\usepackage{graphicx}
% for neatly defining theorems and propositions
\usepackage{amsthm}
% making logically defined graphics
%%%\usepackage{xypic}

% there are many more packages, add them here as you need them

% define commands here
\newtheorem*{theorem*}{Theorem}
\newtheorem*{lemma*}{Lemma}
\newtheorem*{corollary*}{Corollary}
\newtheorem*{definition*}{Definition}
\newtheorem{theorem}{Theorem}
\newtheorem{lemma}{Lemma}
\newtheorem{corollary}{Corollary}
\newtheorem{definition}{Definition}

\begin{document}
\PMlinkescapeword{set function}
\PMlinkescapeword{sequence}
\PMlinkescapeword{subset}
\PMlinkescapeword{extension}
\PMlinkescapeword{completion}
\PMlinkescapeword{equivalent}

Any \PMlinkname{finite measure}{SigmaFinite} can be extended to a set function on the power set of the underlying space. As the following result states, this will be a Choquet capacity.

\begin{theorem*}
Let $(X,\mathcal{F},\mu)$ be a finite measure space. Then,
\begin{align*}
&\mu^*\colon\mathcal{P}(X)\to\mathbb{R}_+,\\
&\mu^*(S)=\inf\left\{\mu(A)\colon A\in\mathcal{F},\ A\supseteq S\right\} 
\end{align*}
is an $\mathcal{F}$-capacity. Furthermore, a subset $S\subseteq X$ is $(\mathcal{F},\mu^*)$-capacitable if and only if it is in the \PMlinkname{completion}{CompleteMeasure} of $\mathcal{F}$ with respect to $\mu$.
\end{theorem*}

Note that, as well as being a capacity, $\mu^*$ is also an outer measure (see \PMlinkname{here}{ConstructionOfOuterMeasures}), which does not require the finiteness of $\mu$.
Clearly, $\mu^*(A)=\mu(A)$ for all $A\in\mathcal{F}$, so $\mu^*$ is an extension of $\mu$ to the power set of $X$, and is referred to as the \emph{outer measure generated by $\mu$}.

Recall that a subset $S\subseteq X$ is in the completion of $\mathcal{F}$ with respect to $\mu$ if and only if there are sets $A,B\in\mathcal{F}$ with $A\subseteq S\subseteq B$ and $\mu(B\setminus A)=0$ which, by the above theorem, is equivalent to the capacitability of $S$.

%%%%%
%%%%%
\end{document}
