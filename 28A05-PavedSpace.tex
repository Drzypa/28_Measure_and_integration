\documentclass[12pt]{article}
\usepackage{pmmeta}
\pmcanonicalname{PavedSpace}
\pmcreated{2013-03-22 18:44:50}
\pmmodified{2013-03-22 18:44:50}
\pmowner{gel}{22282}
\pmmodifier{gel}{22282}
\pmtitle{paved space}
\pmrecord{6}{41523}
\pmprivacy{1}
\pmauthor{gel}{22282}
\pmtype{Definition}
\pmcomment{trigger rebuild}
\pmclassification{msc}{28A05}
\pmsynonym{paving}{PavedSpace}
\pmsynonym{paved set}{PavedSpace}
\pmrelated{F_sigmaSet}
\pmrelated{G_deltaSet}
\pmrelated{AnalyticSet2}
\pmdefines{paving}
\pmdefines{compact paving}
\pmdefines{compactly paved by}
\pmdefines{product paving}

% almost certainly you want these
\usepackage{amssymb}
\usepackage{amsmath}
\usepackage{amsfonts}

% used for TeXing text within eps files
%\usepackage{psfrag}
% need this for including graphics (\includegraphics)
%\usepackage{graphicx}
% for neatly defining theorems and propositions
\usepackage{amsthm}
% making logically defined graphics
%%%\usepackage{xypic}

% there are many more packages, add them here as you need them

% define commands here
\newtheorem*{theorem*}{Theorem}
\newtheorem*{lemma*}{Lemma}
\newtheorem*{corollary*}{Corollary}
\newtheorem*{definition*}{Definition}
\newtheorem{theorem}{Theorem}
\newtheorem{lemma}{Lemma}
\newtheorem{corollary}{Corollary}
\newtheorem{definition}{Definition}

\begin{document}
\PMlinkescapeword{compact}
\PMlinkescapeword{finite}
\PMlinkescapeword{operations}
\PMlinkescapeword{order}
\PMlinkescapeword{contain}

A \emph{paving} on a set $X$ is any collection $\mathcal{A}$ of subsets of $X$, and $(X,\mathcal{A})$ is said to be a \emph{paved space}.
Given any two paved spaces $(X,\mathcal{A})$ and $(Y,\mathcal{B})$, the product paving $\mathcal{A}\times\mathcal{B}$ is defined as
\begin{equation*}
\mathcal{A}\times\mathcal{B} = \left\{A\times B\colon A\in\mathcal{A},B\in\mathcal{B}\right\}.
\end{equation*}

A paved space $(K,\mathcal{K})$ is said to be \emph{compact} if every subcollection of $\mathcal{K}$ satisfying the finite intersection property has nonempty intersection. Equivalently, if any $\mathcal{K}^\prime\subseteq\mathcal{K}$ has empty intersection then there is a finite $\mathcal{K}^{\prime\prime}\subseteq\mathcal{K}^\prime$ with empty intersection. Then, $\mathcal{K}$ is said to be a \emph{compact paving}, and $K$ is \emph{compactly paved} by $\mathcal{K}$.
An example of compact pavings is given by the collection of all \PMlinkname{compact subsets}{Compact} of a Hausdorff topological space.

For any paving $\mathcal{A}$, the notation $\mathcal{A}_\sigma$ is often used to denote countable unions of elements of $\mathcal{A}$,
\begin{equation*}
\mathcal{A}_\sigma\equiv\left\{\bigcup_{n=1}^\infty A_n\colon A_n\in\mathcal{A}\text{ for all }n\in\mathbb{N}\right\}.
\end{equation*}
Similarly, $\mathcal{A}_\delta$ denotes the countable intersections of elements of $\mathcal{A}$,
\begin{equation*}
\mathcal{A}_\delta\equiv\left\{\bigcap_{n=1}^\infty A_n\colon A_n\in\mathcal{A}\text{ for all }n\in\mathbb{N}\right\}.
\end{equation*}
These operations can be combined in any order so that, for example, $\mathcal{A}_{\sigma\delta}=(\mathcal{A}_\sigma)_\delta$ is the collection of countable intersections of countable unions of elements of $\mathcal{A}$.

Note: In the definition of a paved space, some authors additionally require a paving $\mathcal{K}$ to contain the empty set.

\begin{thebibliography}{9}
\bibitem{bichteler}
K. Bichteler, \emph{Stochastic integration with jumps}. Encyclopedia of Mathematics and its Applications, 89. Cambridge University Press, 2002.
\bibitem{dellacherie}
Claude Dellacherie, Paul-Andr\'e Meyer, \emph{Probabilities and potential}. North-Holland Mathematics Studies, 29. North-Holland Publishing Co., 1978.
\bibitem{he}
Sheng-we He, Jia-gang Wang, Jia-an Yan, \emph{Semimartingale theory and stochastic calculus.} Kexue Chubanshe (Science Press), CRC Press, 1992.
\bibitem{rao}
M.M. Rao, \emph{Measure theory and integration}. Second edition. Monographs and Textbooks in Pure and Applied Mathematics, 265. Marcel Dekker Inc., 2004.
\end{thebibliography}

%%%%%
%%%%%
\end{document}
