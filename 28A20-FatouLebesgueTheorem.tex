\documentclass[12pt]{article}
\usepackage{pmmeta}
\pmcanonicalname{FatouLebesgueTheorem}
\pmcreated{2013-03-22 13:12:53}
\pmmodified{2013-03-22 13:12:53}
\pmowner{Koro}{127}
\pmmodifier{Koro}{127}
\pmtitle{Fatou-Lebesgue theorem}
\pmrecord{7}{33679}
\pmprivacy{1}
\pmauthor{Koro}{127}
\pmtype{Theorem}
\pmcomment{trigger rebuild}
\pmclassification{msc}{28A20}
\pmrelated{FatousLemma}

\endmetadata

% this is the default PlanetMath preamble.  as your knowledge
% of TeX increases, you will probably want to edit this, but
% it should be fine as is for beginners.

% almost certainly you want these
\usepackage{amssymb}
\usepackage{amsmath}
\usepackage{amsfonts}

% used for TeXing text within eps files
%\usepackage{psfrag}
% need this for including graphics (\includegraphics)
%\usepackage{graphicx}
% for neatly defining theorems and propositions
%\usepackage{amsthm}
% making logically defined graphics
%%%\usepackage{xypic}

% there are many more packages, add them here as you need them

% define commands here
\begin{document}
Let $(X,\mu)$ be a measure space. If $\Phi\colon X\to \mathbb{R}$ is a nonnegative function with $\int \Phi d\mu <\infty$, and if $f_1, f_2,\dots$ is a sequence of measurable functions such that $|f_n|\leq \Phi$ for each $n$, then 
\[g=\liminf_{n\rightarrow\infty} f_n \;\;\textnormal{and}\; 
h=\limsup_{n\rightarrow\infty} f_n\]
are both integrable, and 
\[-\infty < \int g d\mu\leq \liminf_{n\rightarrow\infty}\int f_nd\mu\leq
\limsup_{k\rightarrow\infty}\int f_n d\mu\leq \int h d\mu < \infty.\]
%%%%%
%%%%%
\end{document}
