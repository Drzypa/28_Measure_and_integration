\documentclass[12pt]{article}
\usepackage{pmmeta}
\pmcanonicalname{CompletionOfAMeasureSpace}
\pmcreated{2013-03-22 14:06:59}
\pmmodified{2013-03-22 14:06:59}
\pmowner{Koro}{127}
\pmmodifier{Koro}{127}
\pmtitle{completion of a measure space}
\pmrecord{9}{35521}
\pmprivacy{1}
\pmauthor{Koro}{127}
\pmtype{Derivation}
\pmcomment{trigger rebuild}
\pmclassification{msc}{28A12}

% this is the default PlanetMath preamble.  as your knowledge
% of TeX increases, you will probably want to edit this, but
% it should be fine as is for beginners.

% almost certainly you want these
\usepackage{amssymb}
\usepackage{amsmath}
\usepackage{amsfonts}
\usepackage{mathrsfs}

% used for TeXing text within eps files
%\usepackage{psfrag}
% need this for including graphics (\includegraphics)
%\usepackage{graphicx}
% for neatly defining theorems and propositions
%\usepackage{amsthm}
% making logically defined graphics
%%%\usepackage{xypic}

% there are many more packages, add them here as you need them

% define commands here
\newcommand{\C}{\mathbb{C}}
\newcommand{\R}{\mathbb{R}}
\newcommand{\N}{\mathbb{N}}
\newcommand{\Z}{\mathbb{Z}}
\newcommand{\Per}{\operatorname{Per}}
\begin{document}
If the measure space $(X,\mathscr{S},\mu)$ is not complete, then it can be completed in the following way. Let 
$$\mathscr{Z} = \bigcup_{E\in\mathscr{S}\, ,\mu(E)=0} \mathscr{P}(E),$$
i.e. the family of all subsets of sets whose $\mu$-measure is zero.
Define
$$\overline{\mathscr{S}} = \{A \cup B : A\in \mathscr{S}, \, B\in \mathscr{Z}\}.$$
We assert that $\overline{\mathscr{S}}$ is a $\sigma$-algebra. In fact, it clearly contains the emptyset, and it is closed under countable unions because both $\mathscr{S}$ and $\mathscr{Z}$ are. We thus need to show that it is closed under complements. Let $A\in \mathscr{S}$, $B\in\mathscr{Z}$ and suppose $E\in\mathscr{S}$ is such that $B\subset E$ and $\mu(E)=0$.
Then we have $$(A\cup B)^c = A^c\cap B^c = A^c\cap (E-(E-B))^c = A^c\cap (E^c\cup (E-B)) = (A^c\cap E^c) \cup (A^c\cap(E-B)),$$
where $A^c\cap E^c\in \mathscr{S}$ and  $A^c\cap(E-B)\in\mathscr{Z}$. Hence $(A\cup B)^c \in \overline{\mathscr{S}}$. 

Now we define $\overline{\mu}$ on $\overline{\mathscr{S}}$ by $\overline{\mu}(A\cup B) = \mu(A)$, whenever $A\in \mathscr{S}$ and $B\in\mathscr{Z}$. It is easily verified that this defines in fact a measure, and that $(X,\overline{\mathscr{S}},\overline{\mu})$ is the completion of $(X,\mathscr{S},\mu)$.
%%%%%
%%%%%
\end{document}
