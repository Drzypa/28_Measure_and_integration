\documentclass[12pt]{article}
\usepackage{pmmeta}
\pmcanonicalname{RadonNikodymTheorem}
\pmcreated{2013-03-22 13:26:15}
\pmmodified{2013-03-22 13:26:15}
\pmowner{Koro}{127}
\pmmodifier{Koro}{127}
\pmtitle{Radon-Nikodym theorem}
\pmrecord{9}{33998}
\pmprivacy{1}
\pmauthor{Koro}{127}
\pmtype{Theorem}
\pmcomment{trigger rebuild}
\pmclassification{msc}{28A15}
\pmrelated{AbsolutelyContinuous}
\pmrelated{BoundedLinearFunctionalsOnLpmu}
\pmrelated{MartingaleProofOfTheRadonNikodymTheorem}
\pmrelated{BoundedLinearFunctionalsOnLinftymu}
\pmdefines{Radon-Nikodym derivative}

% this is the default PlanetMath preamble.  as your knowledge
% of TeX increases, you will probably want to edit this, but
% it should be fine as is for beginners.

% almost certainly you want these
\usepackage{amssymb}
\usepackage{amsmath}
\usepackage{amsfonts}

% used for TeXing text within eps files
%\usepackage{psfrag}
% need this for including graphics (\includegraphics)
%\usepackage{graphicx}
% for neatly defining theorems and propositions
%\usepackage{amsthm}
% making logically defined graphics
%%%\usepackage{xypic}
\usepackage{mathrsfs}
% there are many more packages, add them here as you need them

% define commands here
\begin{document}
Let $\mu$ and $\nu$ be two $\sigma$-finite measures on the same measurable space $(\Omega, \mathscr{S})$, such that $\nu\ll \mu$ 
(i.e. $\nu$ is absolutely continuous with respect to $\mu$.) 
Then there exists a measurable function $f$, which is nonnegative 
and finite, such that for each $A\in \mathscr{S}$, 
\[\nu(A)=\int_A fd\mu.\]
This function is unique (any other function satisfying these
conditions is equal to $f$ $\mu$-almost everywhere,) and it is called
the \emph{Radon-Nikodym derivative} of $\nu$ with respect to $\mu$, 
denoted by $f = \frac{d\nu}{d\mu}$.

\textbf{Remark.} The theorem also holds if $\nu$ is a signed measure. Even if $\nu$ is not $\sigma$-finite the theorem holds, with the exception that $f$ is not necessarely finite.

\textbf{Some properties of the Radon-Nikodym derivative}

Let $\nu$, $\mu$, and $\lambda$ be $\sigma$-finite measures in 
$(\Omega,\mathscr{S})$. 

\begin{enumerate}
\item If $\nu \ll \lambda$ and $\mu\ll\lambda$, then
\[\frac{d(\nu+\mu)}{d\lambda} = 
\frac{d\nu}{d\lambda}+\frac{d\mu}{d\lambda}\;\; 
\mu\mbox{-almost everywhere};\]

\item If $\nu\ll\mu\ll\lambda$, then
\[\frac{d\nu}{d\lambda}=\frac{d\nu}{d\mu}\frac{d\mu}{d\lambda}
\;\; \mu\mbox{-almost everywhere};\]

\item If $\mu\ll\lambda$ and $g$ is a $\mu$-integrable function, then 
\[\int_\Omega gd\mu = \int_\Omega g\frac{d\mu}{d\lambda}d\lambda;\]

\item If $\mu\ll\nu$ and $\nu \ll\mu$, then 
\[\frac{d\mu}{d\nu}=\left(\frac{d\nu}{d\mu}\right)^{-1}.\]

\end{enumerate}
%%%%%
%%%%%
\end{document}
