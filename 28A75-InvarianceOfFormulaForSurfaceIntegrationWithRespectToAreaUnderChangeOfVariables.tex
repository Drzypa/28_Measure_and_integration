\documentclass[12pt]{article}
\usepackage{pmmeta}
\pmcanonicalname{InvarianceOfFormulaForSurfaceIntegrationWithRespectToAreaUnderChangeOfVariables}
\pmcreated{2013-03-22 15:07:32}
\pmmodified{2013-03-22 15:07:32}
\pmowner{rspuzio}{6075}
\pmmodifier{rspuzio}{6075}
\pmtitle{invariance of formula for surface integration with respect to area under change of variables}
\pmrecord{5}{36865}
\pmprivacy{1}
\pmauthor{rspuzio}{6075}
\pmtype{Proof}
\pmcomment{trigger rebuild}
\pmclassification{msc}{28A75}

% this is the default PlanetMath preamble.  as your knowledge
% of TeX increases, you will probably want to edit this, but
% it should be fine as is for beginners.

% almost certainly you want these
\usepackage{amssymb}
\usepackage{amsmath}
\usepackage{amsfonts}

% used for TeXing text within eps files
%\usepackage{psfrag}
% need this for including graphics (\includegraphics)
%\usepackage{graphicx}
% for neatly defining theorems and propositions
%\usepackage{amsthm}
% making logically defined graphics
%%%\usepackage{xypic}

% there are many more packages, add them here as you need them

% define commands here
\begin{document}
First, we can use the chain rule for Jacobians to see how one of the terms in parentheses transforms:
 $$\frac{\partial (x, y)}{\partial (u, v)} =  \frac{\partial (x, y)}{\partial (u', v')} \frac{\partial (u', v')}{\partial (u, v)}$$
 A similar story holds for the other two factors.  Combining them, we conclude that
 $$\sqrt{ \left(  \frac{\partial (x,y)}{\partial (u,v)} \right)^2 +  \left( \frac{\partial (y,z)}{\partial (u,v)} \right)^2 + \left( \frac{\partial (z,x)}{\partial (u,v)} \right)^2 } =$$
$$\sqrt{ \left(  \frac{\partial (x,y)}{\partial (u',v')} \frac{\partial (u', v')}{\partial (u, v)} \right)^2  +  \left( \frac{\partial (y,z)}{\partial (u',v')} \frac{\partial (u', v')}{\partial (u, v)} \right)^2 + \left( \frac{\partial (z,x)}{\partial (u',v')} \frac{\partial (u', v')}{\partial (u, v)} \right)^2 } =$$
 $$\frac{\partial (u', v')}{\partial (u, v)} \sqrt{ \left(  \frac{\partial (x,y)}{\partial (u',v')} \right)^2 +  \left( \frac{\partial (y,z)}{\partial (u',v')} \right)^2 + \left( \frac{\partial (z,x)}{\partial (u',v')} \right)^2 }$$

Since the factor in parentheses in front of the square root is the Jacobi determinant, we can apply the rule change of variables in multidimensional integrals to conclude that
 $$\int f(u,v) \sqrt{ \left(  \frac{\partial (x,y)}{\partial (u,v)} \right)^2 +  \left( \frac{\partial (y,z)}{\partial (u,v)} \right)^2 + \left( \frac{\partial (z,x)}{\partial (u,v)} \right)^2 } \> du \, dv = $$
 $$\int f(u',v') \sqrt{ \left(  \frac{\partial (x,y)}{\partial (u',v')} \right)^2 +  \left( \frac{\partial (y,z)}{\partial (u',v')} \right)^2 + \left( \frac{\partial (z,x)}{\partial (u',v')} \right)^2 } \> du' \, dv',$$
which shows that our formula gives the same answer for $\int_S f(u,v) \, d^2 A$, no matter how we choose to parameterize $S$.
%%%%%
%%%%%
\end{document}
