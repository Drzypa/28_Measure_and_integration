\documentclass[12pt]{article}
\usepackage{pmmeta}
\pmcanonicalname{ProofOfFubinisTheoremForTheLebesgueIntegral}
\pmcreated{2013-03-22 15:21:52}
\pmmodified{2013-03-22 15:21:52}
\pmowner{azdbacks4234}{14155}
\pmmodifier{azdbacks4234}{14155}
\pmtitle{proof of Fubini's theorem for the Lebesgue integral}
\pmrecord{4}{37189}
\pmprivacy{1}
\pmauthor{azdbacks4234}{14155}
\pmtype{Proof}
\pmcomment{trigger rebuild}
\pmclassification{msc}{28A35}

\endmetadata

% this is the default PlanetMath preamble.  as your knowledge
% of TeX increases, you will probably want to edit this, but
% it should be fine as is for beginners.

% almost certainly you want these
\usepackage{amssymb}
\usepackage{amsmath}
\usepackage{amsfonts}

% used for TeXing text within eps files
%\usepackage{psfrag}
% need this for including graphics (\includegraphics)
%\usepackage{graphicx}
% for neatly defining theorems and propositions
%\usepackage{amsthm}
% making logically defined graphics
%%%\usepackage{xypic}

% there are many more packages, add them here as you need them

% define commands here
\begin{document}

Let $\mu_x$ and $\mu_y$ be measures on $X$ and $Y$ respectively, let $\mu$ be the product measure $\mu_x \otimes \mu_y$, and let $f(x,y)$ be $\mu$-integrable on $A\subset X\times Y$. Then 
\[ \int_A f(x,y) d\mu = \int_X\left(\int_{A_x} f(x,y) d\mu_y\right) d\mu_x = \int_Y\left(\int_{A_y} f(x,y) d\mu_x\right) d\mu_y\]
where 
\[A_x = \{y\mid (x,y)\in A\} , A_y = \{x\mid (x,y)\in A\}\]

\textbf{Proof}: Assume for now that $f(x,y)\geq 0$. Consider the set
\[U = X\times Y\times\mathbb{R}\]
equipped with the measure
\[\mu_u = \mu_x \otimes \mu_y \otimes \mu^1 = \mu \otimes \mu^1 = \mu_x \otimes \lambda\]
where $\mu^1$ is ordinary Lebesgue measure and $\lambda = \mu_y \otimes \mu^1$. Also consider the set $W\subset U$ defined by 
\[W = \{(x,y,z)\mid (x,y)\in A, 0\leq z\leq f(x,y)\}\]
Then
\[\mu_{u}\left(W\right) = \int_A f(x,y) d\mu\]
And
\[\mu_{u}\left(W\right) = \int_X \lambda\left(W_x\right) d\mu_x\]
where
\[W_x = \{(y,z)\mid (x,y,z)\in W\}\]
However, we also have that 
\[\lambda\left(W_x\right) = \int_{A_x} f(x,y) d\mu_y\]
Combining the last three equations gives us Fubini's theorem. To remove the restriction that $f(x,y)$ be nonnegative, write $f$ as 
\[f(x,y) = f^{+}(x,y) - f^{-}(x,y)\]
where
\[f^{+}(x,y) = \frac{\vert f(x,y)\vert + f(x,y)}{2}, f^{-}(x,y) = \frac{\vert f(x,y)\vert - f(x,y)}{2}\]
are both nonnegative.
%%%%%
%%%%%
\end{document}
