\documentclass[12pt]{article}
\usepackage{pmmeta}
\pmcanonicalname{FunctionalMonotoneClassTheorem}
\pmcreated{2013-03-22 18:38:36}
\pmmodified{2013-03-22 18:38:36}
\pmowner{gel}{22282}
\pmmodifier{gel}{22282}
\pmtitle{functional monotone class theorem}
\pmrecord{4}{41384}
\pmprivacy{1}
\pmauthor{gel}{22282}
\pmtype{Theorem}
\pmcomment{trigger rebuild}
\pmclassification{msc}{28A20}
%\pmkeywords{measurable space}
\pmrelated{MonotoneClassTheorem}

% almost certainly you want these
\usepackage{amssymb}
\usepackage{amsmath}
\usepackage{amsfonts}

% used for TeXing text within eps files
%\usepackage{psfrag}
% need this for including graphics (\includegraphics)
%\usepackage{graphicx}
% for neatly defining theorems and propositions
\usepackage{amsthm}
% making logically defined graphics
%%%\usepackage{xypic}

% there are many more packages, add them here as you need them

% define commands here
\newtheorem*{theorem*}{Theorem}
\newtheorem*{lemma*}{Lemma}
\newtheorem*{corollary*}{Corollary}
\newtheorem*{definition*}{Definition}
\newtheorem{theorem}{Theorem}
\newtheorem{lemma}{Lemma}
\newtheorem{corollary}{Corollary}
\newtheorem{definition}{Definition}

\begin{document}
\PMlinkescapeword{monotone class theorem}
\PMlinkescapeword{theory}
\PMlinkescapeword{simple}
\PMlinkescapeword{generating}
\PMlinkescapeword{generates}
\PMlinkescapeword{application}
\PMlinkescapeword{states}
\PMlinkescapeword{closed}
\PMlinkescapeword{generated by}
\PMlinkescapeword{theorem}
\PMlinkescapeword{finite}

The monotone class theorem is a result in measure theory which allows statements about particularly simple classes of functions to be generalized to arbitrary measurable and bounded functions.

\begin{theorem}
Let $(X,\mathcal{A})$ be a measurable space and $\mathcal{S}$ be a \PMlinkname{$\pi$-system}{PiSystem} generating the \PMlinkname{$\sigma$-algebra}{SigmaAlgebra} $\mathcal{A}$.
Suppose that $\mathcal{H}$ be a vector space of real-valued functions on $X$ containing the constant functions and satisfying the following,
\begin{itemize}
\item if $f\colon X\rightarrow\mathbb{R}$ is bounded and there is a sequence of nonnegative functions $f_n\in \mathcal{H}$ increasing pointwise to $f$, then $f\in \mathcal{H}$.
\item for every set $A\in\mathcal{S}$ the characteristic function $1_A$ is in $\mathcal{H}$.
\end{itemize}
Then, $\mathcal{H}$ contains every bounded and measurable function from $X$ to $\mathbb{R}$.
\end{theorem}

That $\mathcal{H}$ is a vector space just means that it is closed under taking linear combinations, so $\lambda f+\mu g\in \mathcal{H}$ whenever $f,g\in\mathcal{H}$ and $\lambda,\mu\in\mathbb{R}$.

As an example application, consider \PMlinkname{Fubini's theorem}{FubinisTheorem}, which states that for any two finite measure spaces $(X,\mathcal{A},\mu)$ and $(Y,\mathcal{B},\nu)$ then we may commute the order of integration
\begin{equation}\label{eq:1}
\int\int f(x,y)\,d\mu(x)\,d\nu(y)=\int\int f(x,y)\,d\nu(y)\,d\mu(x).
\end{equation}
Here, $f\colon X\times Y\rightarrow\mathbb{R}$ is a bounded and $\mathcal{A}\otimes\mathcal{B}$-measurable function. The space $\mathcal{H}$ of functions for which this identity holds is easily shown to be linearly closed and, by the monotone convergence theorem, is closed under taking monotone limits of functions. Furthermore, the $\pi$-system of sets of the form $A\times B$ for $A\in\mathcal{A}$, $B\in\mathcal{B}$ generates the $\sigma$-algebra $\mathcal{A}\otimes\mathcal{B}$ and
\begin{equation*}
\int\int 1_{A\times B}(x,y)\,d\mu(x)\,d\nu(y)=\mu(A)\nu(B)=\int\int 1_{A\times B}(x,y)\,d\nu(y)\,d\mu(x).
\end{equation*}
So, $1_{A\times B}\in \mathcal{H}$ and the monotone class theorem allows us to conclude that all real valued and bounded $\mathcal{A}\otimes\mathcal{B}$-measurable functions are in $\mathcal{H}$, and equation (\ref{eq:1}) is satisfied.

An alternative, more general form of the monotone class theorem is as follows. The version of the monotone class theorem given above follows from this by letting $\mathcal{K}$ be the characteristic functions $1_A$ for $A\in\mathcal{S}$.

\begin{theorem}\label{thm:2}
Let $X$ be a set and $\mathcal{K}$ be a collection of bounded and real valued functions on $X$ which is closed under multiplication, so that $fg\in\mathcal{K}$ for all $f,g\in\mathcal{K}$. Let $\mathcal{A}$ be the $\sigma$-algebra on $X$ generated by $\mathcal{K}$.

Suppose that $\mathcal{H}$ is a vector space of real valued functions on $X$ containing $\mathcal{K}$ and the constant functions, and satisfying the following
\begin{itemize}
\item if $f\colon X\rightarrow\mathbb{R}$ is bounded and there is a sequence of nonnegative functions $f_n\in \mathcal{H}$ increasing pointwise to $f$, then $f\in \mathcal{H}$.
\end{itemize}
Then, $\mathcal{H}$ contains every bounded and real valued $\mathcal{A}$-measurable function on $X$.
\end{theorem}

Saying that $\mathcal{A}$ is the $\sigma$-algebra generated by $\mathcal{K}$ means that it is the smallest $\sigma$-algebra containing the sets $f^{-1}(B)$ for $f\in\mathcal{K}$ and Borel subset $B$ of $\mathbb{R}$.

For example, letting $\mathcal{K}$ be as in theorem \ref{thm:2} and $\mu,\nu$ be finite measures on $(X,\mathcal{A})$ such that $\mu(X)=\nu(X)$ and $\int f\,d\mu=\int f\,d\nu$ for all $f\in\mathcal{K}$, then $\int f\,d\mu=\int f\,d\nu$ for all bounded and measurable real valued functions $f$. Therefore, $\mu=\nu$. In particular, a finite measure $\mu$ on $(\mathbb{R},\mathcal{B}(\mathbb{R}))$ is uniquely determined by its characteristic function $\chi(x)\equiv\int e^{ixy}\,d\mu(y)$ for $x\in\mathbb{R}$.
Similarly, a finite measure $\mu$ on a bounded interval is uniquely determined by the integrals $\int x^{n}\,d\mu(x)$ for $n\in\mathbb{Z}_+$.

%%%%%
%%%%%
\end{document}
