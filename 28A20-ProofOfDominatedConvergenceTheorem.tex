\documentclass[12pt]{article}
\usepackage{pmmeta}
\pmcanonicalname{ProofOfDominatedConvergenceTheorem}
\pmcreated{2013-03-22 13:30:02}
\pmmodified{2013-03-22 13:30:02}
\pmowner{paolini}{1187}
\pmmodifier{paolini}{1187}
\pmtitle{proof of dominated convergence theorem}
\pmrecord{4}{34077}
\pmprivacy{1}
\pmauthor{paolini}{1187}
\pmtype{Proof}
\pmcomment{trigger rebuild}
\pmclassification{msc}{28A20}
\pmrelated{SecondProofOfDominatedConvergenceTheorem}
\pmrelated{SecondProofOfDominatedConvergenceTheorem2}

\endmetadata

% this is the default PlanetMath preamble.  as your knowledge
% of TeX increases, you will probably want to edit this, but
% it should be fine as is for beginners.

% almost certainly you want these
\usepackage{amssymb}
\usepackage{amsmath}
\usepackage{amsfonts}

% used for TeXing text within eps files
%\usepackage{psfrag}
% need this for including graphics (\includegraphics)
%\usepackage{graphicx}
% for neatly defining theorems and propositions
%\usepackage{amsthm}
% making logically defined graphics
%%%\usepackage{xypic}

% there are many more packages, add them here as you need them

% define commands here
\begin{document}
It is not difficult to prove that $f$ is measurable. In fact we can write 
\[
  f(x)=\sup_n \inf_{k\ge n} f_k(x)
\]
and we know that measurable functions are closed under the $\sup$ and $\inf$ operation.

Consider the sequence $g_n(x)=2\Phi(x) - \vert f(x)-f_n(x)\vert$.
Clearly $g_n$ are nonnegative functions since $f-f_n\le 2\Phi$.
So, applying Fatou's Lemma, we obtain
\begin{eqnarray*}
\lefteqn{\lim_{n\to\infty} \int_X \vert f-f_n\vert\, d\mu 
\le \limsup_{n\to \infty} \int_X \vert f-f_n\vert\, d\mu}\\
& = & - \liminf_{n\to\infty} \int_X -\vert f-f_n\vert\, d\mu \\
& = & \int_X 2\Phi\, d\mu - \liminf_{n\to\infty}\int_X 2\Phi-\vert f-f_n\vert\,d\mu\\
& \le & \int_X 2\Phi\, d\mu - \int_X 2\Phi - \limsup_{n\to \infty}\vert f-f_n\vert\, d\mu \\
&= & \int_X 2\Phi\, d\mu - \int_X 2\Phi\, d\mu = 0.
\end{eqnarray*}
%%%%%
%%%%%
\end{document}
