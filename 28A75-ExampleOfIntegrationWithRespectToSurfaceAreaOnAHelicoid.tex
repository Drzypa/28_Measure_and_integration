\documentclass[12pt]{article}
\usepackage{pmmeta}
\pmcanonicalname{ExampleOfIntegrationWithRespectToSurfaceAreaOnAHelicoid}
\pmcreated{2013-03-22 14:58:01}
\pmmodified{2013-03-22 14:58:01}
\pmowner{rspuzio}{6075}
\pmmodifier{rspuzio}{6075}
\pmtitle{example of integration with respect to surface area on a helicoid}
\pmrecord{7}{36666}
\pmprivacy{1}
\pmauthor{rspuzio}{6075}
\pmtype{Example}
\pmcomment{trigger rebuild}
\pmclassification{msc}{28A75}

\endmetadata

% this is the default PlanetMath preamble.  as your knowledge
% of TeX increases, you will probably want to edit this, but
% it should be fine as is for beginners.

% almost certainly you want these
\usepackage{amssymb}
\usepackage{amsmath}
\usepackage{amsfonts}

% used for TeXing text within eps files
%\usepackage{psfrag}
% need this for including graphics (\includegraphics)
%\usepackage{graphicx}
% for neatly defining theorems and propositions
%\usepackage{amsthm}
% making logically defined graphics
%%%\usepackage{xypic}

% there are many more packages, add them here as you need them

% define commands here
\begin{document}
In this example, we shall consider itegration with respect to surface area on the helicoid.

The helicoid may be parameterized as follows:
 $$x = u \sin v$$
 $$y = u \cos v$$
 $$z = c v$$
(The constant $c$ may be thought of as the ``pitch of the screw''.)  Computing derivatives and appying trigonometric identities, we obtain
$$\frac{\partial (x, y)}{\partial (u,v)} =
\left| \begin{matrix}
\sin v & u \cos v \\
\cos v & - u \sin v
\end{matrix} \right| =
- u$$
$$\frac{\partial (y, z)}{\partial (u,v)} =
\left| \begin{matrix}
\cos v & - u \sin v \\
0 & c
\end{matrix} \right| =
c \cos v$$
$$\frac{\partial (z, x)}{\partial (u,v)} =
\left| \begin{matrix}
0 & c \\
\sin v & u \cos v
\end{matrix} \right| =
- c \sin v.$$
From this we have
 $$\sqrt{ \left(  \frac{\partial (x,y)}{\partial (u,v)} \right)^2 +  \left( \frac{\partial (y,z)}{\partial (u,v)} \right)^2 + \left( \frac{\partial (z,x)}{\partial (u,v)} \right)^2 } =$$
 $$\sqrt{ u^2 + c^2 \cos^2 v + c^2 \sin^2 v } = \sqrt { u^2 + c^2 }$$
so we can compute area integrals over helicoids as follows
 $$\int_S f(u,v) \, d^2 A = \int f(u,v) \sqrt{ c^2 + u^2 } \> du \, dv$$

To return to the main entry \PMlinkid{click here}{6660}
%%%%%
%%%%%
\end{document}
