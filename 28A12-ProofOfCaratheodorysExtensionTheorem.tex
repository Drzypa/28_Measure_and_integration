\documentclass[12pt]{article}
\usepackage{pmmeta}
\pmcanonicalname{ProofOfCaratheodorysExtensionTheorem}
\pmcreated{2013-03-22 18:33:28}
\pmmodified{2013-03-22 18:33:28}
\pmowner{gel}{22282}
\pmmodifier{gel}{22282}
\pmtitle{proof of Carath\'eodory's extension theorem}
\pmrecord{4}{41280}
\pmprivacy{1}
\pmauthor{gel}{22282}
\pmtype{Proof}
\pmcomment{trigger rebuild}
\pmclassification{msc}{28A12}
%\pmkeywords{measure}
%\pmkeywords{outer measure}
\pmrelated{CaratheodorysLemma}
\pmrelated{Measure}
\pmrelated{OuterMeasure2}

% this is the default PlanetMath preamble.  as your knowledge
% of TeX increases, you will probably want to edit this, but
% it should be fine as is for beginners.

% almost certainly you want these
\usepackage{amssymb}
\usepackage{amsmath}
\usepackage{amsfonts}

% used for TeXing text within eps files
%\usepackage{psfrag}
% need this for including graphics (\includegraphics)
%\usepackage{graphicx}
% for neatly defining theorems and propositions
%\usepackage{amsthm}
% making logically defined graphics
%%%\usepackage{xypic}

% there are many more packages, add them here as you need them

% define commands here

\begin{document}
\PMlinkescapeword{outer measure}
\PMlinkescapeword{outer measures}
% \PMlinkescapeword{countably additive}

The first step is to extend the set function $\mu_0$ to the power set $P(X)$. For any subset $S\subseteq X$ the value of $\mu^*(S)$ is defined by taking sequences $S_i$ in $A$ which cover $S$,
\begin{equation}\label{eq:1}
\mu^*(S)\equiv\inf\left\{\sum_{i=1}^\infty \mu_0(S_i): S_i\in A,\ S\subseteq\bigcup_{i=1}^\infty S_i\right\}.
\end{equation}
We show that this is an \PMlinkname{outer measure}{OuterMeasure2}. First, it is clearly non-negative. Secondly, if $S=\emptyset$ then we can take $S_i=\emptyset$ in (\ref{eq:1}) to obtain $\mu^*(S)\le\sum_i\mu_0(\emptyset)=0$, giving $\mu^*(\emptyset)$=0. It is also clear that $\mu^*$ is increasing, so that if $S\subseteq T$ then $\mu^*(S)\le\mu^*(T)$. The only remaining property to be proven is subadditivity. That is, if $S_i$ is a sequence in $P(X)$ then
\begin{equation}\label{eq:2}
\mu^*\left(\bigcup_i S_i\right)\le\sum_i\mu^*(S_i).
\end{equation}
To prove this inequality, choose any $\epsilon>0$ and, by the definition (\ref{eq:1}) of $\mu^*$, for each $i$ there exists a sequence $S_{i,j}\in A$ such that $S_i\subseteq\bigcup_j S_{i,j}$ and,
\begin{equation*}
\sum_{j=1}^\infty\mu_0(S_{i,j})\le\mu^*(S_i)+2^{-i}\epsilon.
\end{equation*}
As $\bigcup_iS_i\subseteq\bigcup_{i,j}S_{i,j}$, equation (\ref{eq:1}) defining $\mu^*$ gives
\begin{equation*}
\mu^*\left(\bigcup_iS_i\right)\le\sum_{i,j}\mu_0(S_{i,j})=\sum_i\sum_j\mu_0(S_{i,j})\le\sum_i(\mu^*(S_i)+2^{-i}\epsilon)=\sum_i\mu^*(S_i)+\epsilon.
\end{equation*}
As $\epsilon>0$ is arbitrary, this proves subadditivity (\ref{eq:2}). So, $\mu^*$ is indeed an outer measure.

The next step is to show that $\mu^*$ agrees with $\mu_0$ on $A$. So, choose any $S\in A$. The inequality $\mu^*(S)\le\mu_0(S)$ follows from taking $S_1=S$ and $S_i=\emptyset$ in (\ref{eq:1}), and it remains to prove the reverse inequality. So, let $S_i$ be a sequence in $A$ covering $S$, and set
\begin{equation*}
S^\prime_i=(S\cap S_i)\setminus\bigcup_{j=1}^{i-1}S_j\in A.
\end{equation*}
Then, $S^\prime_i$ are disjoint sets satisfying $\bigcup_{j=1}^iS^\prime_j=S\cap \bigcup_{j=1}^iS_j$ and, therefore, $\bigcup_iS^\prime_i=S$. By the countable additivity of $\mu_0$,
\begin{equation*}
\sum_i\mu_0(S_i)=\sum_i(\mu_0(S^\prime_i)+\mu_0(S_i\setminus S^\prime_i))\ge\sum_i\mu_0(S^\prime_i)=\mu_0(S).
\end{equation*}
As this inequality hold for any sequence $S_i\in A$ covering $S$, equation (\ref{eq:1}) gives $\mu^*(S)\ge\mu_0(S)$ and, by combining with the reverse inequality, shows that $\mu^*$ does indeed agree with $\mu_0$ on $A$.

We have shown that $\mu_0$ extends to an outer measure $\mu^*$ on the power set of $X$. The final step is to apply Carath\'eodory's lemma on the restriction of outer measures. A set $S\subseteq X$ is said to be $\mu^*$-measurable if the inequality
\begin{equation}\label{eq:3}
\mu^*(E)\ge\mu^*(E\cap S)+\mu^*(E\cap S^c)
\end{equation}
is satisfied for all subsets $E$ of $X$. Carath\'eodory's lemma then states that the collection $\mathcal{F}$ of $\mu^*$-measurable sets is a \PMlinkname{$\sigma$-algebra}{SigmaAlgebra} and that the restriction of $\mu^*$ to $\mathcal{F}$ is a measure.
To complete the proof of the theorem it only remains to be shown that every set in $A$ is $\mu^*$-measurable, as it will then follow that $\mathcal{F}$ contains $\mathcal{A}=\sigma(A)$ and the restriction of $\mu^*$ to $\mathcal{A}$ is a measure.

So, choosing any $S\in A$ and $E\subseteq X$, the proof will be complete once it is shown that (\ref{eq:3}) is satisfied.
Given any $\epsilon>0$, equation (\ref{eq:1}) says that there is a sequence $E_i$ in $A$ such that $E\subseteq\bigcup_iE_i$ and
\begin{equation*}
\sum_i\mu_0(E_i)\le\mu^*(E)+\epsilon.
\end{equation*}
As $E\cap S\subseteq\bigcup_i(E_i\cap S)$ and $E\cap S^c\subseteq\bigcup_i(E_i\cap S^c)$,
\begin{equation*}
\mu^*(E\cap S)+\mu^*(E\cap S^c) \le\sum_i\mu_0(E_i\cap S)+\sum_i\mu_0(E_i\cap S^c)=\sum_i\mu_0(E_i)\le\mu^*(E)+\epsilon.
\end{equation*}
Since $\epsilon$ is arbitrary, this shows that (\ref{eq:3}) is satisfied and $S$ is $\mu^*$-measurable.

%%%%%
%%%%%
\end{document}
