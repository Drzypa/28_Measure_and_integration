\documentclass[12pt]{article}
\usepackage{pmmeta}
\pmcanonicalname{ProofOfEquivalentDefinitionsOfAnalyticSetsForPavedSpaces}
\pmcreated{2013-03-22 18:48:36}
\pmmodified{2013-03-22 18:48:36}
\pmowner{gel}{22282}
\pmmodifier{gel}{22282}
\pmtitle{proof of equivalent definitions of analytic sets for paved spaces}
\pmrecord{4}{41612}
\pmprivacy{1}
\pmauthor{gel}{22282}
\pmtype{Proof}
\pmcomment{trigger rebuild}
\pmclassification{msc}{28A05}
%\pmkeywords{paved space}
%\pmkeywords{analytic set}
%\pmkeywords{Polish space}
%\pmkeywords{Baire space}
%\pmkeywords{Cantor space}

% almost certainly you want these
\usepackage{amssymb}
\usepackage{amsmath}
\usepackage{amsfonts}

% used for TeXing text within eps files
%\usepackage{psfrag}
% need this for including graphics (\includegraphics)
%\usepackage{graphicx}
% for neatly defining theorems and propositions
\usepackage{amsthm}
% making logically defined graphics
%%%\usepackage{xypic}

% there are many more packages, add them here as you need them

% define commands here
\newtheorem*{theorem*}{Theorem}
\newtheorem*{lemma*}{Lemma}
\newtheorem*{corollary*}{Corollary}
\newtheorem*{definition*}{Definition}
\newtheorem{theorem}{Theorem}
\newtheorem{lemma}{Lemma}
\newtheorem{corollary}{Corollary}
\newtheorem{definition}{Definition}

\begin{document}
\PMlinkescapeword{projections}
\PMlinkescapeword{onto}
\PMlinkescapeword{compact}
\PMlinkescapeword{equivalent}
\PMlinkescapeword{definitions}
\PMlinkescapeword{theorem}
\PMlinkescapeword{subset}
\PMlinkescapeword{projection}
\PMlinkescapeword{subsets}
\PMlinkescapeword{simple}
\PMlinkescapeword{state}
\PMlinkescapeword{analytic}
\PMlinkescapeword{collection}
\PMlinkescapeword{expression}
\PMlinkescapeword{closed}
\PMlinkescapeword{sequence}
\PMlinkescapeword{limit}
\PMlinkescapeword{satisfies}
\PMlinkescapeword{compact sets}
\PMlinkescapeword{scheme}
\PMlinkescapeword{consequence}

Let $(X,\mathcal{F})$ be a paved space with $\emptyset\in\mathcal{F}$, let $\mathcal{N}$ be Baire space, and let $Y$ be any uncountable Polish space. For a subset $A$ of $X$, we show that the following statements are equivalent.
\begin{enumerate}
\item\label{item:1} $A$ is $\mathcal{F}$-\PMlinkname{analytic}{AnalyticSet2}.
\item\label{item:2} There is a closed subset $S$ of $\mathcal{N}$ and $\theta\colon \mathbb{N}^2\to\mathcal{F}$ such that
\begin{equation*}
A=\bigcup_{s\in S}\bigcap_{n=1}^\infty \theta\left(n,s_n\right).
\end{equation*}
\item\label{item:3} There is a closed subset $S$ of $\mathcal{N}$ and $\theta\colon \mathbb{N}\to\mathcal{F}$ such that
\begin{equation*}
A=\bigcup_{s\in S}\bigcap_{n=1}^\infty \theta\left(s_n\right).
\end{equation*}
\item\label{item:4} $A$ is the result of a Souslin scheme on $\mathcal{F}$.
\item\label{item:5} $A$ is the projection of a set in $(\mathcal{F}\times\mathcal{K})_{\sigma\delta}$ onto $X$, where $\mathcal{K}$ is the collection of compact subsets of $Y$.
\item\label{item:6} $A$ is the projection of a set in $(\mathcal{F}\times\mathcal{G})_{\sigma\delta}$ onto $X$, where $\mathcal{G}$ is the collection of closed subsets of $Y$.
\end{enumerate}

\noindent{\bf (\ref{item:1}) implies (\ref{item:2})}:
As $A$ is analytic, there exists a compact paved space $(K,\mathcal{K})$ and a set $B\in(\mathcal{F}\times\mathcal{K})_{\sigma\delta}$ such that $A=\pi_X(B)$, where $\pi_X\colon X\times K\to X$ is the projection map.
Write
\begin{equation*}
B=\bigcap_{n=1}^\infty\bigcup_{m=1}^\infty A_{n,m}\times K_{n,m}
\end{equation*}
for $A_{n,m}\in\mathcal{F}$ and $K_{n,m}\in\mathcal{K}$.
Rearranging this expression,
\begin{equation*}
B=\bigcup_{s\in\mathcal{N}}\bigcap_{n=1}^\infty A_{n,s_n}\times K_{n,s_n}.
\end{equation*}
So, defining $S\subseteq\mathcal{N}$ by
\begin{equation*}
S=\left\{s\in\mathcal{N}\colon \bigcap_{n=1}^\infty K_{n,s_n}\not=\emptyset\right\}.
\end{equation*}
gives
\begin{equation*}
A=\pi_X(B)=\bigcup_{s\in S}\bigcap_{n=1}^\infty A_{n,s_n}.
\end{equation*}
Setting $\theta(n,m)=A_{n,m}$ gives the required expression, and it only remains to show that $S$ is closed.
So, let $s^1,s^2,\ldots$ be a sequence in $S$ converging to a limit $s\in\mathcal{N}$. For any $k\ge 0$ then $s^r_n=s_n$ for all $n\le k$ and large enough $r$. Hence,
\begin{equation*}
\bigcap_{n\le k}K_{n,s_n}=\bigcap_{n\le k}K_{n,s^r_n}\supseteq \bigcap_{n=1}^\infty K_{n,s_n}\not=\emptyset.
\end{equation*}
So, the collection of sets $K_{n,s_n}$ for $n=1,2,\ldots$ satisfies the finite intersection property, and \PMlinkname{compactness}{PavedSpace} of the paving $\mathcal{K}$ gives
\begin{equation*}
\bigcap_{n=1}^\infty K_{n,s_n}\not=\emptyset,
\end{equation*}
showing that $s\in S$ and that $S$ is indeed closed.

\noindent\textbf{(\ref{item:2}) implies (\ref{item:3})}:
Supposing that $A$ satisfies the required expression, choose any bijection $\phi\colon \mathbb{N}\to\mathbb{N}^2$. Then define $\tilde\theta\equiv \theta\circ\phi$ and $f\colon\mathcal{N}\to\mathcal{N}$ by $f(s)=t$ where $t_n=\phi^{-1}(n,s_n)$. As $S$ is closed, it follows that $\tilde S=f(S)$ will also be closed and,
\begin{equation*}
A=\bigcup_{s\in S}\bigcap_n\theta(n,s_n)
=\bigcup_{s\in S}\bigcap_n\tilde\theta(\phi^{-1}(n,s_n))
=\bigcup_{s\in\tilde S}\bigcap_n\tilde\theta(s_n)
\end{equation*}
as required.

\noindent\textbf{(\ref{item:3}) implies (\ref{item:4})}:
Suppose that $A$ satisfies the required expression and define a Souslin scheme $(A_s)$ as follows. For any $n\ge 1$ and $s\in\mathbb{N}^n$ let us set
\begin{equation*}
A_s=\begin{cases}
\theta(s_n),&\textrm{if $s=t|_n$ for some $t\in \mathcal{N}$},\\
\emptyset,&\textrm{otherwise}.
\end{cases}
\end{equation*}
Then, for $s\in\mathcal{N}$,
\begin{equation*}
\bigcap_{n=1}^\infty A_{s|_n}=\begin{cases}
\bigcap_{n=1}^\infty\theta(s_n),&\textrm{if $s\in S$},\\
\emptyset,&\textrm{otherwise}.
\end{cases}
\end{equation*}
Here, if $s\not\in S$, we have used the fact that $S$ is closed to deduce that for large $n$, there is no $t\in S$ with $t|_n=s|_n$ and, therefore, $A_{s|_n}=\emptyset$.
The result of the Souslin scheme $(A_s)$ is then
\begin{equation*}
\bigcup_{s\in\mathcal{N}}\bigcap_{n=1}^\infty A_{s|_n}=\bigcup_{s\in S}\bigcap_{n=1}^\infty\theta(s_n) = A
\end{equation*}
as required.

\noindent\textbf{(\ref{item:4}) implies (\ref{item:5})}:
Suppose that $A$ is the result of a Souslin scheme $(A_s)$. Let us first consider the case where $Y$ is Cantor space, $\mathcal{C}=\{0,1\}^\mathbb{N}$, which is a compact Polish space.
Then, for any $s\in\mathbb{N}^n$, let $K_s$ be the set of $t\in\mathcal{C}$ such that $t_k=1$ if $k=s_1+\cdots+s_m$ for some $m\le n$ and $t_k=0$ for all other $k< s_1+\cdots+s_n$.
These are closed and, therefore, compact sets.

Given any sequence $s^1\in\mathbb{N}^1,s^2\in\mathbb{N}^2,\ldots$ it is easily seen that $\bigcap_nK_{s^n}$ is nonempty if and only if there is an $s\in\mathcal{N}$ such that $s|_n=s^n$ for all $n$.
Define the set $B$ in $(\mathcal{F}\times\mathcal{K})_{\sigma\delta}$ by
\begin{equation*}\begin{split}
B&=\bigcap_{n=1}^\infty\bigcup_{s\in\mathbb{N}^n}A_s\times K_s\\
&=\bigcup_{s^1\in\mathbb{N}^1,s^2\in\mathbb{N}^2,\ldots}\bigcap_{n=1}^\infty A_{s^n}\times K_{s^n}\\
&=\bigcup_{s\in\mathcal{N}}\bigcap_{n=1}^\infty A_{s|_n}\times K_{s|_n}.
\end{split}\end{equation*}
The projection of $B$ onto $X$ is then
\begin{equation*}
\pi_X(S)=\bigcup_{s\in\mathcal{N}}\bigcap_{n=1}^\infty A_{s|_n},
\end{equation*}
which is the result $A$ of the scheme $(A_s)$ as required.
The result then generalizes to any uncountable Polish space $Y$, as all such spaces \PMlinkname{contain Cantor space}{UncountablePolishSpacesContainCantorSpace}.

\noindent\textbf{(\ref{item:5}) implies (\ref{item:6})}:
This is trivial, since all compact sets are closed.

\noindent\textbf{(\ref{item:6}) implies (\ref{item:1})}:
This is a consequence of the result that projections of analytic sets are analytic.

%%%%%
%%%%%
\end{document}
