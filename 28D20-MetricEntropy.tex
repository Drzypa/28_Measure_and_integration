\documentclass[12pt]{article}
\usepackage{pmmeta}
\pmcanonicalname{MetricEntropy}
\pmcreated{2013-03-22 14:31:59}
\pmmodified{2013-03-22 14:31:59}
\pmowner{Koro}{127}
\pmmodifier{Koro}{127}
\pmtitle{metric entropy}
\pmrecord{6}{36077}
\pmprivacy{1}
\pmauthor{Koro}{127}
\pmtype{Definition}
\pmcomment{trigger rebuild}
\pmclassification{msc}{28D20}
\pmclassification{msc}{37A35}
\pmsynonym{entropy}{MetricEntropy}
\pmsynonym{measure theoretic entropy}{MetricEntropy}

\endmetadata

% this is the default PlanetMath preamble.  as your knowledge
% of TeX increases, you will probably want to edit this, but
% it should be fine as is for beginners.

% almost certainly you want these
\usepackage{amssymb}
\usepackage{amsmath}
\usepackage{amsfonts}
\usepackage{mathrsfs}

% used for TeXing text within eps files
%\usepackage{psfrag}
% need this for including graphics (\includegraphics)
%\usepackage{graphicx}
% for neatly defining theorems and propositions
%\usepackage{amsthm}
% making logically defined graphics
%%%\usepackage{xypic}

% there are many more packages, add them here as you need them

% define commands here
\newcommand{\C}{\mathbb{C}}
\newcommand{\R}{\mathbb{R}}
\newcommand{\N}{\mathbb{N}}
\newcommand{\Z}{\mathbb{Z}}
\newcommand{\Per}{\operatorname{Per}}
\begin{document}
Let $(X,\mathscr{B},\mu)$ be a probability space, and $T\colon X\to X$ a measure-preserving transformation.
The entropy of $T$ with respect to a finite measurable partition $\mathcal{P}$ is
\[h_\mu(T,\mathcal{P})=\lim_{n\to\infty}H_\mu\left(\bigvee_{k=0}^{n-1} T^{-k}\mathcal{P}\right),\]
where $H_\mu$ is the entropy of a partition and $\vee$ denotes the join of partitions.
The above limit always exists, although it can be $+\infty$.
The entropy of $T$ is then defined as
\[h_\mu(T) = \sup_{\mathcal{P}} h_\mu(T,\mathcal{P}),\]
with the supremum taken over all finite measurable partitions.
Sometimes $h_\mu(T)$ is called the metric or measure theoretic entropy of $T$, to differentiate it from topological entropy.

\textbf{Remarks.}

\begin{enumerate}
        \item There is a natural correspondence between finite measurable partitions and finite
                sub-$\sigma$-algebras of $\mathscr{B}$. Each finite sub-$\sigma$-algebra is
                generated by a unique partition, and clearly each finite partition generates a finite $\sigma$-algebra.
                Because of this, sometimes $h_\mu(T,\mathcal{P})$ is called the entropy of $T$ with respect to
                the $\sigma$-algebra $\mathscr{P}$ generated by $\mathcal{P}$, and denoted by $h_\mu(T,\mathscr{P})$.
                This simplifies the notation in some instances.
\end{enumerate}
%%%%%
%%%%%
\end{document}
