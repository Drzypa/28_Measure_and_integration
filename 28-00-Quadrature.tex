\documentclass[12pt]{article}
\usepackage{pmmeta}
\pmcanonicalname{Quadrature}
\pmcreated{2013-03-22 12:07:35}
\pmmodified{2013-03-22 12:07:35}
\pmowner{rspuzio}{6075}
\pmmodifier{rspuzio}{6075}
\pmtitle{quadrature}
\pmrecord{13}{31286}
\pmprivacy{1}
\pmauthor{rspuzio}{6075}
\pmtype{Definition}
\pmcomment{trigger rebuild}
\pmclassification{msc}{28-00}
\pmclassification{msc}{65D32}
\pmclassification{msc}{41A55}
\pmclassification{msc}{26A42}
\pmdefines{cubature}

\usepackage{amssymb}
\usepackage{amsmath}
\usepackage{amsfonts}
\usepackage{graphicx}
%%%\usepackage{xypic}
\begin{document}
\emph{Quadrature} is the computation of a univariate definite integral.  It can refer to either numerical or analytic techniques; one must gather from context which is meant.  The term refers to the geometric origin of integration in determining the area of a plane figure by approximating it with squares.

\emph{Cubature} refers to higher-dimensional definite integral computation.  Likewise, this term refers to the geometric operation of approximating the volume of a solid by means of cubes (and has since been extended to higher dimensions).

The terms ``quadrature'' and ``cubature'' are typically used in numerical analysis
to denote the approximation of a definite integral, typically by a suitable
weighted sum.  Perhaps the simplest possibility is approximation by a sum of
values at equidistant points, i.e. approximate $\int_0^1 f(x) \, dx$ by
$\sum_{k=0}^n f(k/n) / n$.  More complicated approximations involve variable
weights and evaluation of the function at points which may not be spaced
equidistantly. Some such numerical quadrature methods are Simpson's rule, the trapezoidal rule, and Gaussian quadrature.
%%%%%
%%%%%
%%%%%
\end{document}
