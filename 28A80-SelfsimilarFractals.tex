\documentclass[12pt]{article}
\usepackage{pmmeta}
\pmcanonicalname{SelfsimilarFractals}
\pmcreated{2013-03-22 16:05:12}
\pmmodified{2013-03-22 16:05:12}
\pmowner{paolini}{1187}
\pmmodifier{paolini}{1187}
\pmtitle{self-similar fractals}
\pmrecord{11}{38146}
\pmprivacy{1}
\pmauthor{paolini}{1187}
\pmtype{Definition}
\pmcomment{trigger rebuild}
\pmclassification{msc}{28A80}

\endmetadata

% this is the default PlanetMath preamble.  as your knowledge
% of TeX increases, you will probably want to edit this, but
% it should be fine as is for beginners.

% almost certainly you want these
\usepackage{amssymb}
\usepackage{amsmath}
\usepackage{amsfonts}

% used for TeXing text within eps files
%\usepackage{psfrag}
% need this for including graphics (\includegraphics)
%\usepackage{graphicx}
% for neatly defining theorems and propositions
\usepackage{amsthm}
% making logically defined graphics
%%%\usepackage{xypic}

% there are many more packages, add them here as you need them

% define commands here
\newcommand{\R}{\mathbb R}
\newtheorem{theorem}{Theorem}
\newtheorem{definition}{Definition}
\theoremstyle{remark}
\newtheorem{example}{Example}
\begin{document}
Let $(X,d)$ be a metric space and let $T_1,\ldots,T_N$ be a finite number of contractions on $X$ i.e.\ each $T_i\colon X\to X$ enjoys the property
\[
  d(T_i(x),T_i(y)) \le \lambda_i d(x,y)
\]
($T_i$ is $\lambda_i$-Lipschitz) for some $\lambda_i<1$.

Given a set $A\subset X$ we can define
\[
  T(A) = \bigcup_{i=1}^N T_i(A).
\]

\begin{definition}
A set $K$ such that $T(K)=K$ (invariant set) is called a \emph{self-similar fractal} with respect to the contractions $\{T_1,\ldots,T_N\}$.
\end{definition}

The most famous example of self-similar fractal is the Cantor set.
This is constructed in $X=\R$ with the usual Euclidean metric structure, by
choosing $N=2$ contractions: $T_1(x)=x/3$, $T_2(x)=1-(1-x)/3$. 

A more interesting example is the Koch curve in $X=\R^2$. In this case we choose 
$N=4$ similitudes with factor $1/3$.

By choosing other appropriate transformations one can obtain the beautiful example of the Barnsley Fern, which shows how the fractal geometry can successfully describe nature.

An important result is given by the following Theorem.
\begin{theorem}
Let $X$ be a complete metric space and let
$T_1,\ldots, T_N\colon X \to X$ be a given set of contractions. 
Then there exists one and only one non empty compact set $K\subset X$ such that 
$T(K)=K$.
\end{theorem}

Notice that the empty set always satisfies the relation $T(\emptyset)=\emptyset$ 
and hence is not an interesting case. On the other hand, if at least one of the $T_i$ is surjective (as happens in the examples above), then the whole set $X$ 
satisfies $T(X)=X$.


%%%%%
%%%%%
\end{document}
