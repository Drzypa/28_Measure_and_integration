\documentclass[12pt]{article}
\usepackage{pmmeta}
\pmcanonicalname{ProjectionsOfAnalyticSetsAreAnalytic}
\pmcreated{2013-03-22 18:46:21}
\pmmodified{2013-03-22 18:46:21}
\pmowner{gel}{22282}
\pmmodifier{gel}{22282}
\pmtitle{projections of analytic sets are analytic}
\pmrecord{8}{41565}
\pmprivacy{1}
\pmauthor{gel}{22282}
\pmtype{Theorem}
\pmcomment{trigger rebuild}
\pmclassification{msc}{28A05}
%\pmkeywords{analytic set}
%\pmkeywords{product paving}
%\pmkeywords{projection map}
\pmrelated{MeasurableProjectionTheorem}

\endmetadata

% almost certainly you want these
\usepackage{amssymb}
\usepackage{amsmath}
\usepackage{amsfonts}

% used for TeXing text within eps files
%\usepackage{psfrag}
% need this for including graphics (\includegraphics)
%\usepackage{graphicx}
% for neatly defining theorems and propositions
\usepackage{amsthm}
% making logically defined graphics
%%%\usepackage{xypic}

% there are many more packages, add them here as you need them

% define commands here
\newtheorem*{theorem*}{Theorem}
\newtheorem*{lemma*}{Lemma}
\newtheorem*{corollary*}{Corollary}
\newtheorem*{definition*}{Definition}
\newtheorem{theorem}{Theorem}
\newtheorem{lemma}{Lemma}
\newtheorem{corollary}{Corollary}
\newtheorem{definition}{Definition}

\begin{document}
\PMlinkescapeword{property}
\PMlinkescapeword{projections}
\PMlinkescapeword{stable}
\PMlinkescapeword{compact}
\PMlinkescapeword{proof}
\PMlinkescapeword{analytic sets}
\PMlinkescapeword{paved space}
\PMlinkescapeword{satisfies}
\PMlinkescapeword{product}
\PMlinkescapeword{order}
\PMlinkescapeword{product paving}
\PMlinkescapeword{projection}
\PMlinkescapeword{onto}
\PMlinkescapeword{consequence}
\PMlinkescapeword{real numbers}
\PMlinkescapeword{compact paving}
\PMlinkescapeword{closed}
\PMlinkescapeword{collection}
\PMlinkescapeword{countable}
\PMlinkescapeword{isomorphic}

\subsection*{Projections along compact paved spaces}

Given sets $X$ and $K$, the projection map $\pi_X\colon X\times K\rightarrow X$ is defined by $\pi_X(x,y)=x$. An important property of \PMlinkname{analytic sets}{AnalyticSet2} is that they are stable under projections.

\begin{theorem}\label{thm:1}
Let $(X,\mathcal{F})$ be a \PMlinkname{paved space}{PavedSpace}, $(K,\mathcal{K})$ be a \PMlinkname{compact}{PavedSpace} paved space and $\pi_X\colon X\times K\rightarrow X$ be the projection map.

If $S\subseteq X\times K$ is $\mathcal{F}\times\mathcal{K}$-analytic then $\pi_X(S)$ is $\mathcal{F}$-analytic.
\end{theorem}

The proof of this follows easily from the definition of analytic sets.
First, there is a compact paved space $(K^\prime,\mathcal{K}^\prime)$ and a set $T\in\left(\mathcal{F}\times\mathcal{K}\times\mathcal{K}^\prime\right)_{\sigma\delta}$ such that $S=\pi_{X\times K}(T)$. Then,
\begin{equation*}
\pi_X(S)=\pi_X\left(\pi_{X\times K}(T)\right)=\pi_X(T).
\end{equation*}
However, $(K\times K^\prime,\mathcal{K}\times\mathcal{K}^\prime)$ is a compact paved space (see \PMlinkname{products of compact pavings are compact}{ProductsOfCompactPavingsAreCompact}), which shows that $\pi_X(S)$ satisfies the definition of $\mathcal{F}$-analytic sets.

\subsection*{Projections along Polish spaces}

Theorem \ref{thm:1} above can be used to prove the following result for projections from the product of a measurable space and a Polish space. For \PMlinkname{$\sigma$-algebras}{SigmaAlgebra} $\mathcal{F}$ and $\mathcal{B}$, we use the notation $\mathcal{F}\otimes\mathcal{B}$ for the \PMlinkname{product $\sigma$-algebra}{ProductSigmaAlgebra}, in order to distinguish it from the product paving $\mathcal{F}\times\mathcal{B}$.

\begin{theorem}\label{thm:2}
Let $(X,\mathcal{F})$ be a measurable space and $Y$ be a Polish space with \PMlinkname{Borel $\sigma$-algebra}{BorelSigmaAlgebra} $\mathcal{B}$.

If $S\subseteq X\times Y$ is $\mathcal{F}\otimes\mathcal{B}$-analytic, then its projection onto $X$ is $\mathcal{F}$-analytic.
\end{theorem}

An immediate consequence of this is the measurable projection theorem.

Although Theorem \ref{thm:2} applies to arbitrary Polish spaces, it is enough to just consider the case where $Y$ is the space of real numbers $\mathbb{R}$ with the standard topology. Indeed, all Polish spaces are Borel isomorphic to either the real numbers or a discrete subset of the reals (see Polish spaces up to Borel isomorphism), so the general case follows from this.

If $Y=\mathbb{R}$, then the Borel $\sigma$-algebra is generated by the compact paving $\mathcal{K}$ of closed and bounded intervals. The collection $a(\mathcal{F}\times\mathcal{K})$ of analytic sets is closed under countable unions and countable intersections so, by the monotone class theorem, includes the product $\sigma$-algebra $\mathcal{F}\otimes\mathcal{B}$. Then, as the analytic sets define a closure operator,
\begin{equation*}
a(\mathcal{F}\otimes\mathcal{B})\subseteq a(a(\mathcal{F}\times\mathcal{K}))=a(\mathcal{F}\times\mathcal{K}).
\end{equation*}
Thus every $\mathcal{F}\otimes\mathcal{B}$-analytic set is $\mathcal{F}\times\mathcal{K}$-analytic, and the result follows from Theorem \ref{thm:1}.

%%%%%
%%%%%
\end{document}
