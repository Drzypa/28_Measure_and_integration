\documentclass[12pt]{article}
\usepackage{pmmeta}
\pmcanonicalname{HausdorffDimension}
\pmcreated{2013-05-18 23:14:26}
\pmmodified{2013-05-18 23:14:26}
\pmowner{Mathprof}{13753}
\pmmodifier{unlord}{1}
\pmtitle{Hausdorff dimension}
\pmrecord{16}{32981}
\pmprivacy{1}
\pmauthor{Mathprof}{1}
\pmtype{Definition}
\pmcomment{trigger rebuild}
\pmclassification{msc}{28A80}
\pmrelated{Dimension3}
\pmrelated{HausdorffMeasure}
\pmdefines{countable r-cover}
\pmdefines{diameter}

% this is the default PlanetMath preamble.  as your knowledge
% of TeX increases, you will probably want to edit this, but
% it should be fine as is for beginners.

% almost certainly you want these
\usepackage{amssymb}
\usepackage{amsmath}
\usepackage{amsfonts}

% used for TeXing text within eps files
%\usepackage{psfrag}
% need this for including graphics (\includegraphics)
\usepackage{graphicx}
% for neatly defining theorems and propositions
%\usepackage{amsthm}
% making logically defined graphics
%%%\usepackage{xypic}

% there are many more packages, add them here as you need them

% define commands here
\newcommand{\R}{\mathbb{R}}

% dissapearing character hack %%%%%%%%%%%%%%%%%%%%%%%%%%%%%%%%%%%
\begin{document}
Let $\Theta$ be a bounded subset of $\R^n$
let $N_{\Theta}(\epsilon)$ be the minimum number of balls of radius $\epsilon$ required to cover $\Theta$.  Then define the \emph{Hausdorff dimension} 
$d_H$ of $\Theta$ to be 
\[ d_H (\Theta):= - \lim_{\epsilon \rightarrow 0} \frac{\log N_{\Theta}(\epsilon)}{\log \epsilon}.\]

Hausdorff dimension is easy to calculate for simple objects like the Sierpinski gasket or a Koch curve.  Each of these may be covered with a collection of scaled-down copies of itself.  In fact, in the case of the Sierpinski gasket, one can take the individual triangles in each approximation as balls in the covering.  At stage $n$, there are $3^n$ triangles of radius $\frac{1}{2^n}$, and so the Hausdorff dimension of the Sierpinski triangle is at most $- \frac{n \log 3}{n \log 1/2} = \frac{\log 3}{\log 2}$, and it can be shown that it is equal to $\frac{\log 3}{\log 2}$.

\paragraph{From some notes from Koro} This definition can be extended to a general metric space $X$ with distance function $d$.

Define the \emph{diameter} $|C|$ of a bounded subset $C$ of $X$ to be $\sup_{x,y\in C} d(x,y)$. 

Define a \emph{\PMlinkescapetext{countable} $r$-cover}
of $X$ to be a collection of subsets $C_i$ of $X$ indexed by some countable set $I$, such that $|C_i| < r$ and $X = \cup_{i\in I} C_i$.

We also define the  function $$H^D_r (X) = \inf \sum_{i\in I} |C_i|^D$$ where the infimum is over all countable $r$-covers of $X$.  
The \emph{Hausdorff dimension} of $X$ may then be defined as $$d_H (X)=\inf \{ D\mid \lim_{r\rightarrow 0} H^D_r(X)=0 \}.$$
When $X$ is a subset of $\R^n$ with any \PMlinkescapetext{restricted} norm-induced metric, then this definition reduces to that given above.

%Alternatively, with $N_{\Theta}(\epsilon)$ 
%as above, the Hausdorff dimension may 
%be defined as the unique real number 
%$d$ (if it exists) such that the limit
%\[ \lim_{\epsilon \rightarrow 0} N(\epsilon) \epsilon^d \]
%is neither zero nor infinity.  Let $x$ be a 
%point in $\Theta$, and let $B(x,r)$ be a 
%ball of radius $r$ cetered at $x$.  Then 
%the limit
%$d_H (\Theta; x) \lim_{r\rightarrow 0} d_H (\Theta \cap B(x,r))$ 
%(when it exists) can be thought of as the 
%dimension of $\Theta$ at$x$.  Then in this 
%sense, the dimension of $\Theta$ is a kind 
%of essential supremum over $\Theta$ of 
%$d_H (\Theta;x).............
%%%%%
%%%%%.
\end{document}
