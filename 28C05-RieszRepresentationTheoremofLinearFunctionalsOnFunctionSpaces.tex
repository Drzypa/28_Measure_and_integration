\documentclass[12pt]{article}
\usepackage{pmmeta}
\pmcanonicalname{RieszRepresentationTheoremofLinearFunctionalsOnFunctionSpaces}
\pmcreated{2013-03-22 17:28:18}
\pmmodified{2013-03-22 17:28:18}
\pmowner{asteroid}{17536}
\pmmodifier{asteroid}{17536}
\pmtitle{Riesz representation theorem (of linear functionals on function spaces)}
\pmrecord{13}{39857}
\pmprivacy{1}
\pmauthor{asteroid}{17536}
\pmtype{Theorem}
\pmcomment{trigger rebuild}
\pmclassification{msc}{28C05}
\pmclassification{msc}{46A99}
\pmdefines{Riesz-Markov theorem}

% this is the default PlanetMath preamble.  as your knowledge
% of TeX increases, you will probably want to edit this, but
% it should be fine as is for beginners.

% almost certainly you want these
\usepackage{amssymb}
\usepackage{amsmath}
\usepackage{amsfonts}

% used for TeXing text within eps files
%\usepackage{psfrag}
% need this for including graphics (\includegraphics)
%\usepackage{graphicx}
% for neatly defining theorems and propositions
%\usepackage{amsthm}
% making logically defined graphics
%%%\usepackage{xypic}

% there are many more packages, add them here as you need them

% define commands here

\begin{document}
\PMlinkescapeword{positive linear functional}

This entry should not be mistaken with the entry on the Riesz representation theorem of \PMlinkname{bounded}{BoundedOperator} linear functionals on an Hilbert space.

The Riesz \PMlinkescapetext{representation theorem(s)} provided here basically \PMlinkescapetext{state} that linear functionals on certain spaces of functions can be seen as integration against measures. In other \PMlinkescapetext{words}, for some spaces of functions all linear functionals have the form
\begin{displaymath}
f \longmapsto \int f\; d\mu
\end{displaymath}
for some measure $\mu$.

There are many versions of these Riesz \PMlinkescapetext{representation theorems}, and which version is used depends upon the generality \PMlinkescapetext{one} wishes to achieve, the difficulty of proof, the \PMlinkescapetext{type} of space of functions involved, the \PMlinkescapetext{type} of linear functionals involved, the \PMlinkescapetext{type} of the "\PMlinkescapetext{base}" space involved, and also the \PMlinkescapetext{type} of measures involved.

We present here some possible Riesz \PMlinkescapetext{representation theorems} of general use.

{\bf Notation -} In the following we adopt the following conventions:
\begin{itemize}
\item $X$ is a locally compact Hausdorff space.
\item $C_c(X)$ denotes the space of real valued continuous functions on $X$ with compact support.
\item $C_0(X)$ denotes the space of real valued continuous functions on $X$ that vanish at infinity.
\item all function spaces are endowed with the sup-norm $\|.\|_{\infty}$
\item a linear functional $L$ is said to be \PMlinkescapetext{positive} if $0 \leq L(f)$ whenever $0 \leq f$.
\end{itemize}

{\bf Theorem 1 (Riesz-Markov) -} Let $L$ be a positive linear functional on $C_c(X)$. There exists a unique Radon measure $\mu$ on $X$, whose underlying \PMlinkname{$\sigma$-algebra}{SigmaAlgebra} is the $\sigma$-algebra generated by all compact sets, such that
\begin{displaymath}
L(f) = \int_X f \; d\mu
\end{displaymath}
Moreover, $\mu$ is finite if and only if $L$ is bounded.

Notice that when $X$ is \PMlinkname{$\sigma$-compact}{SigmaCompact} the underlying $\sigma$-algebra for these measures is precisely the \PMlinkname{Borel $\sigma$-algebra}{BorelSigmaAlgebra} of $X$.

$\,$

{\bf Theorem 2 -} Let $L$ be a positive linear functional on $C_0(X)$. There exists a unique finite Radon measure $\mu$ on $X$ such that
\begin{displaymath}
L(f) = \int_X f \; d\mu
\end{displaymath}


{\bf Theorem 3 (Dual of $C_0(X)$) -} Let $L$ be a \PMlinkescapetext{bounded} linear functional on $C_0(X)$. There exists a unique finite \PMlinkname{signed}{SignedMeasure} Borel measure on $X$ such that
\begin{displaymath}
L(f) = \int_X f \; d\mu
\end{displaymath}

\subsubsection{Complex version:}

Here $C_0(X)$ denotes the space of complex valued continuous functions on $X$ that vanish at infinity.

{\bf Theorem 4 -} Let $L$ be a \PMlinkescapetext{bounded} linear functional on $C_0(X)$. There exists a unique finite complex \PMlinkescapetext{regular} Borel measure $\mu$ on $X$ such that
\begin{displaymath}
L(f) = \int_X f \; d\mu
\end{displaymath}
%%%%%
%%%%%
\end{document}
