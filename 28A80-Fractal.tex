\documentclass[12pt]{article}
\usepackage{pmmeta}
\pmcanonicalname{Fractal}
\pmcreated{2013-03-22 12:41:51}
\pmmodified{2013-03-22 12:41:51}
\pmowner{mathcam}{2727}
\pmmodifier{mathcam}{2727}
\pmtitle{fractal}
\pmrecord{17}{32979}
\pmprivacy{1}
\pmauthor{mathcam}{2727}
\pmtype{Definition}
\pmcomment{trigger rebuild}
\pmclassification{msc}{28A80}
\pmrelated{MengerSponge}

\endmetadata

\usepackage{amssymb}
\usepackage{amsmath}
\usepackage{amsfonts}

% used for TeXing text within eps files
%\usepackage{psfrag}
% need this for including graphics (\includegraphics)
%\usepackage{graphicx}
% for neatly defining theorems and propositions
%\usepackage{amsthm}
% making logically defined graphics
%%%\usepackage{xypic}

% there are many more packages, add them here as you need them

% define commands here

\newcommand{\R}{\mathbb R}
\newcommand{\Q}{\mathbb Q}
\begin{document}
\PMlinkescapeword{class}
\PMlinkescapeword{equivalence}
\PMlinkescapeword{equivalent}
\PMlinkescapeword{region}
\PMlinkescapeword{simple}
\PMlinkescapeword{type}

There are several ways of defining a fractal, and a reader will need to reference their source to see which definition is being used.  

Perhaps the simplest definition is to define a \emph{fractal} to be a subset of $\mathbb{R}^n$ with Hausdorff dimension greater than its Lebesgue covering dimension.  It is worth noting that typically (but not always), fractals have non-integer Hausdorff dimension.  See, for example, the Koch snowflake and the Mandelbrot set (named after Benoit Mandelbrot, who also coined the term ``fractal'' for these objects).

A looser definition of a \emph{fractal} is a ``self-similar object''.  That is, a subset or $\R^n$ which can be covered by copies of itself using a set of (usually two or more) transformation mappings.  Another way to say this would be ``an object with a discrete approximate scaling symmetry''. 

See also the discussion near the end of the entry \PMlinkname{Hausdorff dimension}{HausdorffDimension}.
%%%%%
%%%%%
\end{document}
