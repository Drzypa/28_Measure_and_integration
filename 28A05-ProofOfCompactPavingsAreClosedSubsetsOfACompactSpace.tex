\documentclass[12pt]{article}
\usepackage{pmmeta}
\pmcanonicalname{ProofOfCompactPavingsAreClosedSubsetsOfACompactSpace}
\pmcreated{2013-03-22 18:45:07}
\pmmodified{2013-03-22 18:45:07}
\pmowner{gel}{22282}
\pmmodifier{gel}{22282}
\pmtitle{proof of compact pavings are closed subsets of a compact space}
\pmrecord{4}{41529}
\pmprivacy{1}
\pmauthor{gel}{22282}
\pmtype{Proof}
\pmcomment{trigger rebuild}
\pmclassification{msc}{28A05}
%\pmkeywords{compact paving}
%\pmkeywords{ultrafilter}

\endmetadata

% almost certainly you want these
\usepackage{amssymb}
\usepackage{amsmath}
\usepackage{amsfonts}

% used for TeXing text within eps files
%\usepackage{psfrag}
% need this for including graphics (\includegraphics)
%\usepackage{graphicx}
% for neatly defining theorems and propositions
\usepackage{amsthm}
% making logically defined graphics
%%%\usepackage{xypic}

% there are many more packages, add them here as you need them

% define commands here
\newtheorem*{theorem*}{Theorem}
\newtheorem*{lemma*}{Lemma}
\newtheorem*{corollary*}{Corollary}
\newtheorem*{definition*}{Definition}
\newtheorem{theorem}{Theorem}
\newtheorem{lemma}{Lemma}
\newtheorem{corollary}{Corollary}
\newtheorem{definition}{Definition}

\begin{document}
\PMlinkescapeword{compact}
\PMlinkescapeword{compact paving}
\PMlinkescapeword{closed under}
\PMlinkescapeword{satisfy}
\PMlinkescapeword{paving}
\PMlinkescapeword{contained}
\PMlinkescapeword{compactness}
\PMlinkescapeword{implies}
\PMlinkescapeword{contains}
\PMlinkescapeword{ultrafilter}
\PMlinkescapeword{lemma}
\PMlinkescapeword{satisfies}
\PMlinkescapeword{filter}

Let $(K,\mathcal{K})$ be a \PMlinkname{compact paved space}{paved space}. We use the \PMlinkname{ultrafilter lemma}{EveryFilterIsContainedInAnUltrafilter} to show that there is a compact paving $\mathcal{K}^\prime$ containing $\mathcal{K}$ that is closed under arbitrary intersections and finite unions.

We first show that the paving $\mathcal{K}_1$ consisting of all finite unions of elements of $\mathcal{K}$ is compact.
Let $\mathcal{F}\subseteq\mathcal{K}_1$ satisfy the finite intersection property. It then follows that the collection of finite intersections of $\mathcal{F}$ is a \PMlinkname{filter}{Filter}. The ultrafilter lemma says that $\mathcal{F}$ is contained in an ultrafilter $\mathcal{U}$.

By definition, the ultrafilter satisfies the finite intersection property. So, the compactness of $\mathcal{K}$ implies that $\mathcal{F}^\prime\equiv\mathcal{U}\cap\mathcal{K}$ has nonempty intersection.
Also, every element $S$ of $\mathcal{F}$ is a union of finitely many elements of $\mathcal{K}$, one of which must be in $\mathcal{U}$ (see \PMlinkname{alternative characterization of ultrafilter}{AlternativeCharacterizationOfUltrafilter}). In particular, $S$ contains the intersection of $\mathcal{F}^\prime$ and,
\begin{equation*}
\bigcap\mathcal{F}\supseteq\bigcap\mathcal{F}^\prime\not=\emptyset.
\end{equation*}
Consequently, $\mathcal{K}_1$ is compact.

Finally, we let $\mathcal{K}^\prime$ be the set of arbitrary intersections of $\mathcal{K}_1$. This is closed under all arbitrary intersections and finite unions. Furthermore, if $\mathcal{F}\subseteq\mathcal{K}^\prime$ satisfies the finite intersection property then so does
\begin{equation*}
\mathcal{F}^\prime\equiv\left\{A\in \mathcal{K}_1\colon B\subseteq A\text{ for some }B\in\mathcal{F}\right\}.
\end{equation*}
The compactness of $\mathcal{K}_1$ gives
\begin{equation*}
\bigcap\mathcal{F}=\bigcap\mathcal{F}^\prime\not=\emptyset
\end{equation*}
as required.

%%%%%
%%%%%
\end{document}
