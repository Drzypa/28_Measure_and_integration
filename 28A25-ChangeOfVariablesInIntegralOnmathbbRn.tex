\documentclass[12pt]{article}
\usepackage{pmmeta}
\pmcanonicalname{ChangeOfVariablesInIntegralOnmathbbRn}
\pmcreated{2013-03-22 15:29:32}
\pmmodified{2013-03-22 15:29:32}
\pmowner{stevecheng}{10074}
\pmmodifier{stevecheng}{10074}
\pmtitle{change of variables in integral on $\mathbb{R}^n$}
\pmrecord{8}{37349}
\pmprivacy{1}
\pmauthor{stevecheng}{10074}
\pmtype{Theorem}
\pmcomment{trigger rebuild}
\pmclassification{msc}{28A25}
\pmclassification{msc}{26B15}
\pmclassification{msc}{26B10}
\pmsynonym{integral substitution formula}{ChangeOfVariablesInIntegralOnmathbbRn}
\pmsynonym{integral substitution rule}{ChangeOfVariablesInIntegralOnmathbbRn}
\pmsynonym{change-of-variables formula}{ChangeOfVariablesInIntegralOnmathbbRn}
\pmrelated{JacobiDeterminant}
\pmrelated{LebesgueMeasure}
\pmrelated{AreaFormula}
\pmrelated{PotentialOfHollowBall}
\pmrelated{ExampleOfRiemannTripleIntegral}
\pmrelated{ExampleOfRiemannDoubleIntegral}

% The standard font packages
\usepackage{amssymb}
\usepackage{amsmath}
\usepackage{amsfonts}

% For neatly defining theorems and definitions
\usepackage{amsthm}

% Including EPS/PDF graphics (\includegraphics)
\usepackage{graphicx}

% Making matrix-based graphics
%%%\usepackage{xypic}

% Enumeration lists with different styles
%\usepackage{enumerate}

% Set up the theorem environments
\newtheorem{thm}{Theorem}
%\newtheorem*{thm*}{Theorem}

\newcommand{\defnterm}[1]{\emph{#1}}

% The standard number systems
\newcommand{\real}{\mathbb{R}}

% Absolute values and norms
% Normal, wide, and big versions of the delimeters
\newcommand{\abs}[1]{\lvert#1\rvert}
\newcommand{\absW}[1]{\left\lvert#1\right\rvert}
\newcommand{\absB}[1]{\Bigl\lvert#1\Bigr\rvert}
\newcommand{\norm}[1]{\lVert#1\rVert}
\newcommand{\normW}[1]{\left\lVert#1\right\rVert}
\newcommand{\normB}[1]{\Bigl\lVert#1\Bigr\rVert}

% Inverse functions
\newcommand{\inv}[1]{{#1}^{-1}}

% Operators and functions occassionally used in my articles
\DeclareMathOperator{\D}{D}

\begin{document}
\begin{thm}
Let $g\colon X \to Y$ be a diffeomorphism between 
open subsets $X$ and $Y$ of $\real^n$.
Then for any measurable function $f\colon Y \to \real$, and any measurable 
set $E \subseteq X$,
\[
 \int_E f(g(x)) \, \abs{\det \D g(x) } dx = \int_{g(E)} f(y) \, dy\,.
\]
Also, if one of these integrals does not exist, then neither does the other.
\end{thm}

This theorem is a generalization of the substitution rule 
for integrals from one-variable calculus.

To go from the left-hand side to the right-hand side or vice versa,
we can perform the formal substitutions:
\[
y = g(x) \,, \quad dy = g(dx) = \abs{ \det \D g(x) } dx\,.
\]
The volume scaling factor $\abs{\det \D g(x)}$ is sometimes called
the \emph{Jacobian} or \emph{Jacobian determinant}.

Theorem 1 is typically applied 
when integrating over $\real^2$ using polar coordinates,
or when integrating over $\real^3$ using cylindrical or spherical coordinates.

Intuitively speaking, the image of a small cube centered at $x$,
under a differentiable map $g$ is approximately
the parallelogram resulting from the linear mapping $\D g(x)$ 
applied on that cube. If the volume of the original cube is $dx$,
then the volume of the image parallelogram is $dy = \abs{\det \D g(x)} dx$.
The integral formula in Theorem 1 follows for an
arbitrary set by approximating it by many numbers of small cubes,
and taking limits.  

\begin{figure}
\begin{center}
\includegraphics{jacobian.eps}
\end{center}
\caption{Illustration of 
linear approximation to $g(Q)$ by $x + \D g(x) (Q-x)$.
{\small \PMlinkexternal{Source program in Python for diagram}{http://aux.planetmath.org/files/objects/7349/jacobian.py}}
}
\end{figure}

Proofs of Theorem 1 can be obtained
by making this procedure rigorous;
see \cite{Schwartz}, \cite{Flett}, or \cite{Guzman}.

A slightly stronger version of the theorem that does not require 
$g$ to be a diffeomorphism
(i.e. that $g$ is a bijection and has non-singular derivative) is:

\begin{thm}
Let $g\colon X \to \real^n$ be continuously differentiable 
on an open subset $X$ of $\real^n$.
Then for any measurable function $f\colon Y \to \real$, and 
any measurable set $E \subseteq X$,
\[
 \int_E f(g(x)) \, \abs{\det \D g(x) } \, dx = 
 \int_{g(E)} f(y) \, \# g|_E^{-1}(y) \, dy\,,
\]
where $\# g|_E^{-1}(y) \in \{ 1, 2, \dotsc, \infty \}$
counts the number of pre-images in $E$ of $y$.
\end{thm}
Observe that Theorem 2 (as well as its proof) includes 
a special case of Sard's Theorem.

The idea of Theorem 2 is that we may ignore those pieces of the set $E$
that transform to zero volumes, and if the map $g$ is not one-to-one,
then some pieces of the image $g(E)$ may be counted multiple times
in the left-hand integral.

These formulas can also be generalized for 
\PMlinkname{Hausdorff measures}{AreaFormula} on $\real^n$,
and non-differentiable, but Lipschitz, functions $g$.  See \cite{Morgan}
or other geometric measure theory books for details.

\begin{thebibliography}{3}
\bibitem{Flett}
T. M. Flett. ``On Transformations in $\real^n$ and a Theorem of Sard''.
\emph{American Mathematical Monthly}, Vol. 71, No. 6 (Jun--Jul 1964), 
p. 623--629.
\bibitem{Folland}
Gerald B. Folland. 
\emph{Real Analysis: Modern Techniques and Their Applications}, 
second ed. Wiley-Interscience, 1999.
\bibitem{Guzman}
Miguel De Guzman. ``Change-of-Variables Formula Without Continuity''.
\emph{American Mathematical Monthly}, Vol. 87, No. 9 (Nov 1980), p. 736--739.
\bibitem{Morgan}
Frank Morgan. \emph{Geometric Measure Theory: A Beginner's Guide}, second ed. 
Academic Press, 1995.
\bibitem{Munkres}
James R. Munkres. \emph{Analysis on Manifolds}. Westview Press, 1991.
\bibitem{Sard}
Arthur Sard. ``\PMlinkexternal{The Measure of the Critical Values of Differentiable Maps}{http://www.ams.org/bull/1942-48-12/S0002-9904-1942-07812-8/S0002-9904-1942-07812-8.pdf}''.
\emph{Bulletins of the American Mathematical Society}, Vol. 48 (1942), No. 12, p. 883-890.
\bibitem{Schwartz}
J. Schwartz. ``The Formula for Change in Variables in a Multiple Integral''.
\emph{American Mathematical Monthly}, Vol. 61, No. 2 (Feb 1954), p. 81--95.
\bibitem{Spivak}
Michael Spivak. \emph{Calculus on Manifolds}. Perseus Books, 1998.
\end{thebibliography}

%%%%%
%%%%%
\end{document}
