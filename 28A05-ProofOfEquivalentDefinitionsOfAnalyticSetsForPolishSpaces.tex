\documentclass[12pt]{article}
\usepackage{pmmeta}
\pmcanonicalname{ProofOfEquivalentDefinitionsOfAnalyticSetsForPolishSpaces}
\pmcreated{2013-03-22 18:48:48}
\pmmodified{2013-03-22 18:48:48}
\pmowner{gel}{22282}
\pmmodifier{gel}{22282}
\pmtitle{proof of equivalent definitions of analytic sets for Polish spaces}
\pmrecord{5}{41615}
\pmprivacy{1}
\pmauthor{gel}{22282}
\pmtype{Proof}
\pmcomment{trigger rebuild}
\pmclassification{msc}{28A05}
%\pmkeywords{analytic set}
%\pmkeywords{Polish space}
%\pmkeywords{Borel measurable}
%\pmkeywords{paving}

% almost certainly you want these
\usepackage{amssymb}
\usepackage{amsmath}
\usepackage{amsfonts}

% used for TeXing text within eps files
%\usepackage{psfrag}
% need this for including graphics (\includegraphics)
%\usepackage{graphicx}
% for neatly defining theorems and propositions
\usepackage{amsthm}
% making logically defined graphics
%%%\usepackage{xypic}

% there are many more packages, add them here as you need them

% define commands here
\newtheorem*{theorem*}{Theorem}
\newtheorem*{lemma*}{Lemma}
\newtheorem*{corollary*}{Corollary}
\newtheorem*{definition*}{Definition}
\newtheorem{theorem}{Theorem}
\newtheorem{lemma}{Lemma}
\newtheorem{corollary}{Corollary}
\newtheorem{definition}{Definition}

\begin{document}
\PMlinkescapeword{subset}
\PMlinkescapeword{projection}
\PMlinkescapeword{onto}
\PMlinkescapeword{image}
\PMlinkescapeword{collection}
\PMlinkescapeword{contains}
\PMlinkescapeword{analytic}
\PMlinkescapeword{subsets}
\PMlinkescapeword{function}
\PMlinkescapeword{expression}
\PMlinkescapeword{closed}
\PMlinkescapeword{continuous}
\PMlinkescapeword{isomorphic}
\PMlinkescapeword{borel isomorphism}
\PMlinkescapeword{isomorphism}
\PMlinkescapeword{borel set}
\PMlinkescapeword{generating}
\PMlinkescapeword{integers}
\PMlinkescapeword{radius}
\PMlinkescapeword{borel sets}
\PMlinkescapeword{consequence}

Let $A$ be a nonempty subset of a Polish space $X$. Then, letting $\mathcal{N}$ denote Baire space and $Y$ be any uncountable Polish space, we show that the following are equivalent.
\begin{enumerate}
\item\label{item:1} $A$ is $\mathcal{F}$-\PMlinkname{analytic}{AnalyticSet2}.
\item\label{item:2} $A$ is the \PMlinkname{projection}{GeneralizedCartesianProduct} of a closed subset of $X\times\mathcal{N}$ onto $X$.
\item\label{item:3} $A$ is the \PMlinkname{image}{DirectImage} of a continuous function $f\colon Z\to X$ for some Polish space $Z$.
\item\label{item:4} $A$ is the image of a continuous function $f\colon \mathcal{N}\to X$.
\item\label{item:5} $A$ is the image of a Borel measurable function $f\colon Y\to X$.
\item\label{item:6} $A$ is the projection of a Borel subset of $X\times Y$ onto $X$.
\end{enumerate}

\noindent\textbf{(\ref{item:1}) implies (\ref{item:2})}:
Let $\mathcal{F}$ be the paving consisting of closed subsets of $X$. The collection of $\mathcal{K}$-analytic sets contains the Borel $\sigma$-algebra of $X$ (see countable unions and intersections of analytic sets are analytic) and, as the \PMlinkname{analytic sets are given by a closure operator}{AnalyticSetsDefineAClosureOperator} it follows that it contains all analytic subsets of $X$. So, any analytic subset $A$ of $X$ is $\mathcal{F}$-analytic.
Then, there is a closed $S\subseteq\mathcal{N}$ and a function $\theta\colon\mathbb{N}\to\mathcal{F}$ such that
\begin{equation*}
A=\bigcup_{s\in S}\bigcap_{n=1}^\infty\theta(s_n)
\end{equation*}
(see proof of equivalent definitions of analytic sets for paved spaces).
For each $m,n\in\mathbb{N}$ let $K_{m,n}$ denote the closed subset of $s\in\mathcal{N}$ with $s_n=m$. Then, we can rearrange the above expression to get $A=\pi_X(B)$ where $\pi_X\colon X\times\mathcal{N}\to X$ is the projection map and
\begin{equation*}
B=(X\times S)\cap \bigcap_{n=1}^\infty\bigcup_{m=1}^\infty\theta(m)\times K_{m,n}.
\end{equation*}
It is easily seen that $\bigcup_m\theta(m)\times K_{m,n}$ is a closed subset of $X\times\mathcal{N}$ for each $n$, and therefore $B$ is closed, as required.


\noindent\textbf{(\ref{item:2}) implies (\ref{item:3})}:
Suppose that $A=\pi_X(S)$ for a closed subset $S$ of $X\times \mathcal{N}$, where $\pi_X\colon X\times\mathcal{N}\to X$ is the projection map. As the product of Polish spaces is Polish, and every closed subset of a Polish space is Polish, then $S$ will be a Polish space under the subspace topology. So, we can take $Z=S$ and let $f\colon Z\to X$ be the restriction of $\pi_X$ to $Z$.

\noindent\textbf{(\ref{item:3}) implies (\ref{item:4})}:
Suppose that $A$ is the image of a continuous function $g\colon Z\to X$, for a Polish space $Z$. As Baire space is universal for Polish spaces, there exists a continuous and \PMlinkname{onto}{Surjective} function $h\colon\mathcal{N}\to Z$. The result follows by taking $f=g\circ h$.

\noindent\textbf{(\ref{item:4}) implies (\ref{item:5})}:
Suppose that $A$ is the image of a continuous function $g\colon \mathcal{N}\to X$. Since uncountable Polish spaces are all Borel isomorphic (see Polish spaces up to Borel isomorphism), there is a Borel isomorphism $h\colon Y\to\mathcal{N}$. The result follows by taking $f=g\circ h$.

\noindent\textbf{(\ref{item:5}) implies (\ref{item:6})}:
Suppose that $A$ is the image of a Borel measurable function $f\colon Y\to X$, and let $\Gamma$ be its \PMlinkname{graph}{Graph2}
\begin{equation*}
\Gamma\equiv\left\{(f(y),y)\colon y\in Y\right\}\subseteq X\times Y.
\end{equation*}
The projection of $\Gamma$ onto $X$ is equal to $f(Y)=A$, so the result will follow once it is shown that $\Gamma$ is a Borel set.

Choose a countable and dense subset $\{x_1,x_2,\ldots\}$ of $X$, and let $d$ be a metric generating the topology on $X$. Then, for integers $m,n\ge 1$, denote the open ball about $x_m$ of radius $1/n$ by $B_{m,n}$. Since the $x_m$ form a dense set, $\bigcup_mB_{m,n}=X$ for each $n$. Let us define
\begin{equation*}
\Gamma_n\equiv\bigcup_{m=1}^\infty B_{m,n}\times f^{-1}(B_{m,n})\subseteq X\times Y,
\end{equation*}
which contains $\Gamma$. Furthermore, since $f^{-1}(B_{m,n})$ are Borel, $\Gamma_n$ are Borel sets.
Suppose that $(x,y)\in\bigcap_n\Gamma_n$. Then, for each $n$, there is an $m$ such that $x\in B_{m,n}$ and $y\in f^{-1}(B_{m,n})$. So,
\begin{equation*}
d(x,f(y))\le d(x,x_m)+d(x_m,f(y))\le 2/n.
\end{equation*}
This holds for all $n$, showing that $y=f(x)$ and so $(x,y)\in \Gamma$. We have shown that $\Gamma=\bigcap_n\Gamma_n$ is Borel, as required.

\noindent\textbf{(\ref{item:6}) implies (\ref{item:1})}:
This is an immediate consequence of the result that projections of analytic sets are analytic.

%%%%%
%%%%%
\end{document}
