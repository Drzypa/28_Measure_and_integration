\documentclass[12pt]{article}
\usepackage{pmmeta}
\pmcanonicalname{MonotoneClassTheorem}
\pmcreated{2013-03-22 17:07:34}
\pmmodified{2013-03-22 17:07:34}
\pmowner{fernsanz}{8869}
\pmmodifier{fernsanz}{8869}
\pmtitle{monotone class theorem}
\pmrecord{8}{39429}
\pmprivacy{1}
\pmauthor{fernsanz}{8869}
\pmtype{Theorem}
\pmcomment{trigger rebuild}
\pmclassification{msc}{28A05}
%\pmkeywords{algebra}
%\pmkeywords{sigma algebra}
%\pmkeywords{monotone class}
\pmrelated{MonotoneClass}
\pmrelated{SigmaAlgebra}
\pmrelated{Algebra}
\pmrelated{FunctionalMonotoneClassTheorem}

% this is the default PlanetMath preamble.  as your knowledge
% of TeX increases, you will probably want to edit this, but
% it should be fine as is for beginners.

% almost certainly you want these
\usepackage{amssymb}
\usepackage{amsmath}
\usepackage{amsfonts}
\usepackage{amsthm}

% define commands here
\newtheorem*{thm}{Theorem}
\theoremstyle{remark}
\newtheorem{exmpl}{Example}
\newtheorem{rem}{Remark}
\numberwithin{equation}{section}
\newcommand{\M}{\mathcal{M}}
\newcommand{\F}{\mathcal{F}}
\begin{document}
\title{Monotone Class theorem}%
\author{Fernando Sanz Gamiz}%

\begin{thm}
Let $\F_0$ an algebra of subsets of $\Omega$. Let $\M$ be the
smallest monotone class such that $\F_0 \subset \M$ and
$\sigma(\F_0)$ be the sigma algebra generated by $\F_0$. Then
$\M=\sigma(\F_0)$.
\end{thm}

\bigskip

\begin{proof}
It is enough to prove that $\M$ is an algebra, because an algebra
which is a monotone class is obviously a $\sigma$-algebra.

Let $\M_A=\{B \in \M |A \cap B, A \cap B^\complement \mbox{ and } A^\complement \cap B
\in \M \}$. Then is clear that $\M_A$ is a monotone class and, in
fact, $\M_A=\M$, for if $A \in \F_0$, then $\F_0 \subset \M_A$ since
$\F_0$ is a field, hence  $\M \subset \M_A$ by minimality of $\M$;
consequently $\M=\M_A$ by definition of $\M_A$. But this shows that
for any $B \in \M$ we have $A \cap B, A \cap B^\complement \mbox{ and } A^\complement
\cap B \in \M$ for any $A \in \F_0$, so that $\F_0 \subset \M_B$ and
again by minimality $\M=\M_B$. But what we have just proved is that
$\M$ is an algebra, for if $A,B \in \M=\M_A$ we have showed that $A
\cap B, A \cap B^\complement \mbox{ and } A^\complement \cap B \in \M$, and, of course,
$\Omega \in \M$.
\end{proof}

\bigskip

\begin{rem}
One of the main applications of the Monotone Class Theorem is that
of showing that certain property is satisfied by all sets in an
$\sigma$-algebra, generally starting by the fact that the field
generating the $\sigma$-algebra satisfies such property and that the
sets that satisfies it constitutes a monotone class.
\end{rem}

\bigskip

\begin{exmpl}
Consider an infinite sequence of independent random variables
$\{X_n, n \in \mathbb N\}$. The definition of independence is
$$P(X_1 \in A_1, X_2 \in A_2,...,X_n \in A_n)=P(X_1 \in A_1)P(X_2 \in A_2)\cdots P(X_n
\in A_n)$$ for any Borel sets $A_1, A_2,.., A_n$ and any finite $n$.
Using the Monotone Class Theorem one can show, for example, that any
event in $\sigma(X_1,X_2,...,X_n)$ is independent of any event in
$\sigma(X_{n+1},X_{n+2},...)$. For, by independence
$$P((X_1,X_2,...,X_n)\in A, (X_{n+1},X_{n+2},...)\in B)=P((X_1,X_2,...,X_n)\in A)P((X_{n+1},X_{n+2},...)\in B)$$
when A and
B are measurable rectangles in $\mathcal B^n$ and $\mathcal
B^{\infty}$ respectively. Now it is clear that the sets A which
satisfies the above relation form a monotone class. So
$$P((X_1,X_2,...,X_n)\in A, (X_{n+1},X_{n+2},...)\in B)=P((X_1,X_2,...,X_n)\in A)P((X_{n+1},X_{n+2},...)\in B)$$
for every $A \in \sigma(X_1,X_2,...,X_n)$ and any measurable
rectangle $B \in \mathcal B^{\infty}$. A second application of the
theorem shows finally that the above relation holds for any $A \in
\sigma(X_1,X_2,...,X_n)$ and $B \in \sigma(X_{n+1},X_{n+2},...)$


\end{exmpl}
%%%%%
%%%%%
\end{document}
