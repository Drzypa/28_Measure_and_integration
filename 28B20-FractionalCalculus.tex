\documentclass[12pt]{article}
\usepackage{pmmeta}
\pmcanonicalname{FractionalCalculus}
\pmcreated{2013-03-22 16:18:27}
\pmmodified{2013-03-22 16:18:27}
\pmowner{bchui}{10427}
\pmmodifier{bchui}{10427}
\pmtitle{fractional calculus}
\pmrecord{14}{38430}
\pmprivacy{1}
\pmauthor{bchui}{10427}
\pmtype{Definition}
\pmcomment{trigger rebuild}
\pmclassification{msc}{28B20}
\pmsynonym{fractional calculus}{FractionalCalculus}
%\pmkeywords{fractional Calculus}
\pmdefines{fractional calculus}

\endmetadata

% this is the default PlanetMath preamble.  as your knowledge
% of TeX increases, you will probably want to edit this, but
% it should be fine as is for beginners.

% almost certainly you want these
\usepackage{amssymb}
\usepackage{amsmath}
\usepackage{amsfonts}

% used for TeXing text within eps files
%\usepackage{psfrag}
% need this for including graphics (\includegraphics)
%\usepackage{graphicx}
% for neatly defining theorems and propositions
%\usepackage{amsthm}
% making logically defined graphics
%%%\usepackage{xypic}

% there are many more packages, add them here as you need them

% define commands here

\begin{document}
The idea of calculus in fractional order is nearly as old as its integer counterpart. In a letter dated September 30th 1650, l'H\^{o}pital posed the question of the meaning of $\displaystyle{\frac{d^\alpha f}{dx^\alpha}}$ if $\displaystyle{\alpha=\frac{1}{2}}$ to Leibniz. 
There are different approaches to define calculus of fractional order. The following approaches are the most common and we can prove that they are equivalent 

(1) Riemann-Liouville approach of fractional integration

(2) Grunwald-Letnikov approach of fractional differentiation

%%%%%
%%%%%
\end{document}
