\documentclass[12pt]{article}
\usepackage{pmmeta}
\pmcanonicalname{ConditionalExpectationsAreUniformlyIntegrable}
\pmcreated{2013-03-22 18:40:08}
\pmmodified{2013-03-22 18:40:08}
\pmowner{gel}{22282}
\pmmodifier{gel}{22282}
\pmtitle{conditional expectations are uniformly integrable}
\pmrecord{5}{41413}
\pmprivacy{1}
\pmauthor{gel}{22282}
\pmtype{Theorem}
\pmcomment{trigger rebuild}
\pmclassification{msc}{28A20}
\pmclassification{msc}{60A10}
%\pmkeywords{conditional expectation}
%\pmkeywords{uniformly continuous}
\pmrelated{ConditionalExpectation}
\pmrelated{UniformlyIntegrable}

% almost certainly you want these
\usepackage{amssymb}
\usepackage{amsmath}
\usepackage{amsfonts}

% used for TeXing text within eps files
%\usepackage{psfrag}
% need this for including graphics (\includegraphics)
%\usepackage{graphicx}
% for neatly defining theorems and propositions
\usepackage{amsthm}
% making logically defined graphics
%%%\usepackage{xypic}

% there are many more packages, add them here as you need them

% define commands here
\newtheorem*{theorem*}{Theorem}
\newtheorem*{lemma*}{Lemma}
\newtheorem*{corollary*}{Corollary}
\newtheorem*{definition*}{Definition}
\newtheorem{theorem}{Theorem}
\newtheorem{lemma}{Lemma}
\newtheorem{corollary}{Corollary}
\newtheorem{definition}{Definition}

\begin{document}
\PMlinkescapeword{constant}
\PMlinkescapeword{collection}

The collection of all conditional expectations of an integrable random variable forms a uniformly integrable set. More generally, we have the following result.


\begin{theorem*}
Let $S$ be a uniformly integrable set of random variables defined on a probability space $(\Omega,\mathcal{F},\mathbb{P})$. Then, the set
\begin{equation*}
\left\{\mathbb{E}[X\mid\mathcal{G}]:\textrm{$X\in S$ and $\mathcal{G}$ is a sub-$\sigma$-algebra of $\mathcal{F}$}\right\}
\end{equation*}
is also uniformly integrable.
\end{theorem*}

To prove the result, we first use the fact that uniform integrability implies that $S$ is $L^1$-bounded. That is, there is a constant $L>0$ such that $\mathbb{E}[|X|]\le L$ for every $X\in S$.
Also, choosing any $\epsilon>0$, there is a $\delta>0$ so that
\begin{equation*}
\mathbb{E}[|X|1_A]<\epsilon
\end{equation*}
for all $X\in S$ and $A\in\mathcal{F}$ with $\mathbb{P}(A)\le\delta$.

Set $K=L/\delta$. Then, if $Y=\mathbb{E}[X\mid\mathcal{G}]$ for any $X\in S$ and $\mathcal{G}\subseteq\mathcal{F}$, Jensen's inequality gives
\begin{equation*}
|Y|\le\mathbb{E}[|X|\mid\mathcal{G}].
\end{equation*}
So, applying Markov's inequality,
\begin{equation*}
\mathbb{P}(|Y|>K)\le K^{-1}\mathbb{E}[|Y|]\le K^{-1}\mathbb{E}[|X|]\le L/K=\delta
\end{equation*}
and, therefore
\begin{equation*}
\mathbb{E}[|Y|1_{\{|Y|>K\}}]\le \mathbb{E}[|X|1_{\{|Y|>K\}}]<\epsilon.
\end{equation*}

%%%%%
%%%%%
\end{document}
