\documentclass[12pt]{article}
\usepackage{pmmeta}
\pmcanonicalname{CountableUnionsAndIntersectionsOfAnalyticSetsAreAnalytic}
\pmcreated{2013-03-22 18:45:18}
\pmmodified{2013-03-22 18:45:18}
\pmowner{gel}{22282}
\pmmodifier{gel}{22282}
\pmtitle{countable unions and intersections of analytic sets are analytic}
\pmrecord{4}{41533}
\pmprivacy{1}
\pmauthor{gel}{22282}
\pmtype{Theorem}
\pmcomment{trigger rebuild}
\pmclassification{msc}{28A05}
%\pmkeywords{paving}
%\pmkeywords{analytic set}
%\pmkeywords{sigma algebra}

\endmetadata

% almost certainly you want these
\usepackage{amssymb}
\usepackage{amsmath}
\usepackage{amsfonts}

% used for TeXing text within eps files
%\usepackage{psfrag}
% need this for including graphics (\includegraphics)
%\usepackage{graphicx}
% for neatly defining theorems and propositions
\usepackage{amsthm}
% making logically defined graphics
%%%\usepackage{xypic}

% there are many more packages, add them here as you need them

% define commands here
\newtheorem*{theorem*}{Theorem}
\newtheorem*{lemma*}{Lemma}
\newtheorem*{corollary*}{Corollary}
\newtheorem*{definition*}{Definition}
\newtheorem{theorem}{Theorem}
\newtheorem{lemma}{Lemma}
\newtheorem{corollary}{Corollary}
\newtheorem{definition}{Definition}

\begin{document}
\PMlinkescapeword{property}
\PMlinkescapeword{theory}
\PMlinkescapeword{applications}
\PMlinkescapeword{closed under}
\PMlinkescapeword{countable}
\PMlinkescapeword{unions}
\PMlinkescapeword{intersections}
\PMlinkescapeword{sequence}
\PMlinkescapeword{consequence}
\PMlinkescapeword{analytic}
\PMlinkescapeword{complements}
\PMlinkescapeword{corollary}
\PMlinkescapeword{theorem}
\PMlinkescapeword{simple}
\PMlinkescapeword{application}
\PMlinkescapeword{collection}
\PMlinkescapeword{finite}
\PMlinkescapeword{contain}
\PMlinkescapeword{intersections of sets}
\PMlinkescapeword{limits}
\PMlinkescapeword{increasing}
\PMlinkescapeword{decreasing}
\PMlinkescapeword{contains}

A property of \PMlinkname{analytic sets}{AnalyticSet2} which makes them particularly suited to applications in measure theory is that, in common with \PMlinkname{$\sigma$-algebras}{SigmaAlgebra}, they are closed under countable unions and intersections.

\begin{theorem}\label{thm:1}
Let $(X,\mathcal{F})$ be a paved space and $(A_n)_{n\in\mathbb{N}}$ be a sequence of $\mathcal{F}$-analytic sets. Then, $\bigcup_nA_n$ and $\bigcap_n A_n$ are $\mathcal{F}$-analytic.
\end{theorem}

A consequence of this is that measurable sets are analytic, as follows.

\begin{corollary*}
Let $\mathcal{F}$ be a nonempty paving on a set $X$ such that the \PMlinkname{complement}{Complement} of any $S\in\mathcal{F}$ is a union of countably many sets in $\mathcal{F}$.

Then, every set $A$ in the $\sigma$-algebra generated by $\mathcal{F}$ is $\mathcal{F}$-analytic.
\end{corollary*}

For example, every closed subset of a metric space $X$ is a union of countably many open sets. Therefore, the corollary shows that all Borel sets are analytic with respect to the open subsets of $X$.

That the corollary does indeed follow from Theorem \ref{thm:1} is a simple application of the monotone class theorem.
First, as the collection $a(\mathcal{F})$ of $\mathcal{F}$-analytic sets is closed under countable unions and finite intersections, it will contain all finite unions of finite intersections of sets in $\mathcal{F}$ and their complements, which is an \PMlinkname{algebra}{RingOfSets}. Then, Theorem \ref{thm:1} says that $a(\mathcal{F})$ is closed under taking limits of increasing and decreasing sequences of sets. So, by the monotone class theorem, it contains the $\sigma$-algebra generated by $\mathcal{F}$.

%%%%%
%%%%%
\end{document}
