\documentclass[12pt]{article}
\usepackage{pmmeta}
\pmcanonicalname{ExtendedRealNumbers}
\pmcreated{2013-03-22 13:44:44}
\pmmodified{2013-03-22 13:44:44}
\pmowner{matte}{1858}
\pmmodifier{matte}{1858}
\pmtitle{extended real numbers}
\pmrecord{21}{34441}
\pmprivacy{1}
\pmauthor{matte}{1858}
\pmtype{Definition}
\pmcomment{trigger rebuild}
\pmclassification{msc}{28-00}
\pmclassification{msc}{12D99}
\pmrelated{ImproperLimits}
\pmrelated{IntermediateValueTheoremForExtendedRealNumbers}
\pmrelated{ExampleOfNonCompleteLatticeHomomorphism}
\pmdefines{plus infinity}
\pmdefines{minus infinity}
\pmdefines{$\overline{\mathbb{R}}$}
\pmdefines{infinite}
\pmdefines{infinity}
\pmdefines{finite}

% almost certainly you want these
\usepackage{amssymb}
\usepackage{amsmath}
\usepackage{amsfonts}

% used for TeXing text within eps files
%\usepackage{psfrag}
% need this for including graphics (\includegraphics)
%\usepackage{graphicx}
% for neatly defining theorems and propositions
%\usepackage{amsthm}
% making logically defined graphics
%%%\usepackage{xypic}

% there are many more packages, add them here as you need them

% define commands here

\newcommand{\sR}[0]{\mathbb{R}}
\newcommand{\sC}[0]{\mathbb{C}}
\newcommand{\sN}[0]{\mathbb{N}}
\newcommand{\sZ}[0]{\mathbb{Z}}
\begin{document}
\PMlinkescapeword{order}
\PMlinkescapeword{areas}

The \emph{extended real numbers} are the real numbers together with 
$+\infty$ (or simply $\infty$) and $-\infty$.\, 
This set is usually denoted by $\overline{\sR}$ or\, $[-\infty,\,\infty]$,\,
and the elements $+\infty$ and $-\infty$ are called
\emph{plus} and \emph{minus infinity}, respectively.\, (N.B.,\, ``$\overline{\sR}$'' may sometimes \PMlinkescapetext{mean} the algebraic closure of $\mathbb{R}$; see the special notations in algebra.) 

The real numbers are in certain contexts called {\em finite} as contrast to $\infty$.

\subsubsection{Order on $\overline{\sR}$}
The \PMlinkname{order}{TotalOrder} relation on $\sR$ extends to $\overline{\sR}$ by
defining that for any $x\in \sR$, we have
\begin{eqnarray*}
-\infty&<& x, \\
x &<& \infty,
\end{eqnarray*}
and that $-\infty < \infty$.\, For\, $a\in\sR$, let us also define intervals 
\begin{eqnarray*}
(a,\,\infty{]} &=& \{x\in \sR: x>a \}, \\
{[}{-\infty},\,a) &=& \{x\in \sR: x<a \}.
\end{eqnarray*}

\subsubsection{Addition}
For any real number $x$, we define
\begin{eqnarray*}
 x + (\pm\infty) &=& (\pm\infty) + x = \pm\infty,
\end{eqnarray*}
and for $+\infty$ and $-\infty$, we define
\begin{eqnarray*}
(\pm \infty) + (\pm \infty) &=& \pm \infty.
\end{eqnarray*}
It should be pointed out that sums like $(+\infty) + (-\infty)$ 
are left undefined.\, Thus $\overline{\sR}$ is not an ordered ring
although $\sR$ is. 

\subsubsection{Multiplication}
If $x$ is a positive real number, then 
\begin{eqnarray*}
 x \cdot (\pm \infty) &=& (\pm\infty)\cdot x  = \pm\infty.
\end{eqnarray*}
Similarly, if $x$ is a negative real number, then 
\begin{eqnarray*}
 x \cdot (\pm \infty) &=& (\pm \infty)\cdot x = \mp\infty.
\end{eqnarray*}
Furthermore, for $\infty$ and $-\infty$, we define
\begin{eqnarray*}
(+\infty) \cdot(+\infty) &=& (-\infty)\cdot (-\infty) = +\infty, \\
(+\infty) \cdot (- \infty) &=& (-\infty)\cdot (+\infty) = -\infty. 
\end{eqnarray*}

In many areas of mathematics, products like $0\cdot \infty$
are left undefined.\, However, a special case is 
measure theory, where it is convenient to define
\begin{eqnarray*}
0\cdot (\pm \infty) &=& (\pm \infty) \cdot 0 = 0.
\end{eqnarray*}

\subsubsection{Absolute value}
For $\infty$ and $-\infty$, the absolute value is defined as
$$
  |\pm \infty| = +\infty.
$$

\subsubsection{Topology}
The topology of $\overline{R}$ is given by the usual base of $\sR$
together with with intervals of type\, $[-\infty,\,a)$,\, $(a,\,\infty]$.\, 
This makes $\overline{\sR}$ into a compact topological space.
$\overline{\sR}$ can also be seen to be homeomorphic to the interval\, $[-1,\,1]$, via 
the map $x \mapsto (2/\pi) \arctan x$.
Consequently, every 
continuous function $f\colon \overline{\sR}\to \overline{\sR}$ has
a minimum and maximum. 

\subsubsection{Examples}
\begin{enumerate}
\item By taking\, $x = -1$\, in the \PMlinkescapetext{product rule}, we obtain
the relations
\begin{eqnarray*}
(-1)\cdot (\pm\infty) &=& \mp \infty.
\end{eqnarray*}
\end{enumerate}
%%%%%
%%%%%
\end{document}
