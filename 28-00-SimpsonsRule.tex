\documentclass[12pt]{article}
\usepackage{pmmeta}
\pmcanonicalname{SimpsonsRule}
\pmcreated{2013-03-22 13:40:12}
\pmmodified{2013-03-22 13:40:12}
\pmowner{drini}{3}
\pmmodifier{drini}{3}
\pmtitle{Simpson's rule}
\pmrecord{9}{34335}
\pmprivacy{1}
\pmauthor{drini}{3}
\pmtype{Theorem}
\pmcomment{trigger rebuild}
\pmclassification{msc}{28-00}
\pmclassification{msc}{26A06}
\pmclassification{msc}{41A55}
\pmclassification{msc}{65D32}
\pmrelated{LagrangeInterpolationFormula}
\pmrelated{NewtonAndCotesFormulas}
\pmrelated{Prismatoid}

\endmetadata

\usepackage{amssymb}
\usepackage{amsmath}
\usepackage{amsfonts}

%\usepackage{psfrag}
%\usepackage{graphicx}
%%%\usepackage{xypic}
\begin{document}
\emph{Simpson's rule} is a method of (approximate) numerical definite integration (or quadrature).  Simpson's rule is based on a parabolic model of the function to be integrated (in contrast to the trapezoidal model of the trapezoidal rule).  Thus, a minimum of three points and three function values are required. Here we take three equidistant points: $x_0x_2$ the interval endpoints, $x_1=(x_0+x_2)/2$ the midpoint, and let $h=|b-a|/2$ the distance between each. The definite integral is then approximated by:
\[
\int_{x_0}^{x_2} f(x) dx \approx I = \frac{h}{3} (f(x_0) + 4f(x_1) + f(x_2))
\]

We can extend this to greater precision by breaking our target domain into $n$ equal-length fragments.  The quadrature is then the weighted sum of the above formula for every pair of adjacent regions, which works out for even $n$ to
\[
 I = \frac{h}{3} (f(x_0) + 4f(x_1) + 2f(x_2) + 4f(x_3) + \cdots + 4f(x_{n-3}) + 2f(x_{n-2}) + 4f(x_{n-1}) + f(x_n)) 
\]
%%%%%
%%%%%
\end{document}
