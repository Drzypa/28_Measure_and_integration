\documentclass[12pt]{article}
\usepackage{pmmeta}
\pmcanonicalname{HausdorffMeasure}
\pmcreated{2013-03-22 14:27:26}
\pmmodified{2013-03-22 14:27:26}
\pmowner{paolini}{1187}
\pmmodifier{paolini}{1187}
\pmtitle{Hausdorff measure}
\pmrecord{8}{35976}
\pmprivacy{1}
\pmauthor{paolini}{1187}
\pmtype{Definition}
\pmcomment{trigger rebuild}
\pmclassification{msc}{28A78}
\pmrelated{HausdorffDimension}
\pmrelated{LebesgueMeasure}

% this is the default PlanetMath preamble.  as your knowledge
% of TeX increases, you will probably want to edit this, but
% it should be fine as is for beginners.

% almost certainly you want these
\usepackage{amssymb}
\usepackage{amsmath}
\usepackage{amsfonts}

% used for TeXing text within eps files
%\usepackage{psfrag}
% need this for including graphics (\includegraphics)
%\usepackage{graphicx}
% for neatly defining theorems and propositions
%\usepackage{amsthm}
% making logically defined graphics
%%%\usepackage{xypic}

% there are many more packages, add them here as you need them

% define commands here
\renewcommand{\H}{\mathcal H}
\newcommand{\R}{\mathbb R}
\newcommand{\diam}{\mathrm{diam}}
\begin{document}
\section*{Introduction}

Given a real number $\alpha\ge 0$ we are going to define a Borel external measure $\H^\alpha$ on $\R^n$ with values in $[0,+\infty]$ which will comprehend and generalize the concepts of length (for $\alpha=1$), area ($\alpha=2$) and volume ($\alpha=3$) of sets in $\R^n$.
In particular if $M\subset \R^n$ is an $m$-dimensional regular surface then one will show that 
$\H^m(M)$ is the $m$-dimensional area of $M$.
However, being an external measure, $\H^m$ is defined not only on regular surfaces but on every subset of $\R^n$ thus generalizing the concepts of length, area and volume. In particular, for $m=n$, it turns out that the Hausdorff measure $\H^n$ is nothing else than the Lebesgue measure of $\R^n$.

Given any fixed set $E\subset \R^n$ one can consider the measures $\H^\alpha(E)$ with $\alpha$ varying in $[0,+\infty)$. We will see that for a fixed set $E$ there exists at most one value $\alpha$ such that $\H^\alpha(E)$ is finite and positive; while for every other value $\beta$ one will have $\H^\beta(E)=0$ if $\beta>\alpha$ and $\H^\beta(E)=+\infty$ if $\beta<\alpha$.
For example, if $E$ is a regular $2$-dimensional surface then only $\H^2(E)$ (which is the area of the surface) may possibly be finite and different from $0$ while, for example, the volume of $E$ will be $0$ and the length of $E$ will be infinite.

This can be used to define the dimension of a set $E$ (this is called the Hausdorff dimension). A very interesting fact is the existence of sets with dimension $\alpha$ which is not integer, as happens for most \emph{fractals}.

Also, the measure $\H^\alpha$ is naturally defined on every metric space $(X,d)$, not only on $\R^n$. 

\section*{Definition}
Let $(X,d)$ be a metric space. 
Given $E\subset X$ we define the diameter of $E$ as
\[
  \mathrm{diam}(E) := \sup_{x,y\in E} d(x,y).
\]
Given a real number $\alpha$ we consider the conventional constant
\[
  \omega_\alpha=\frac{\pi^{\alpha/2}}{\Gamma(\alpha/2+1)}
\]
where $\Gamma(x)$ is the gamma function.

For all $\delta>0$, $\alpha\ge 0$ and $E\subset X$ let us define
\begin{equation}\label{defHdelta}
  \H^\alpha_\delta(E) := \inf \left\{\sum_{j=0}^\infty \omega_\alpha \left(\frac{\diam (B_j)} 2\right)^\alpha
\colon B_j\subset X,\ \bigcup_{j=0}^\infty B_j\supset E,\ 
\diam (B_j) \le \delta\ \forall j=0,1,\ldots
\right\}.
\end{equation}
The \emph{infimum} is taken over all possible enumerable families of sets $B_0, B_1, \ldots, B_j,\ldots$ which are sufficiently small ($\diam B_j \le \delta$) and which cover $E$.

Notice that the function $\H^\alpha_\delta(E)$ is decreasing in $\delta$. In fact given $\delta'>\delta$ the family of sequences $B_j$ considered in the definition of $\H^\alpha_{\delta'}$ contains  the family of sequences considered  in the definition of $\H^\alpha_\delta$ and hence the infimum is smaller. 
So the limit in the following definition exists:
\begin{equation}\label{defH}
\H^\alpha(E):=\lim_{\delta\to 0^+} \H^\alpha_\delta(E).
\end{equation}
The number $\H^\alpha(E)\in [0,+\infty]$ is called \emph{$\alpha$-dimensional Hausdorff measure} of the set $E\subset X$.
%%%%%
%%%%%
\end{document}
