\documentclass[12pt]{article}
\usepackage{pmmeta}
\pmcanonicalname{ProofOfExistenceAndUnicityOfSelfsimilarFractals}
\pmcreated{2013-03-22 16:05:30}
\pmmodified{2013-03-22 16:05:30}
\pmowner{paolini}{1187}
\pmmodifier{paolini}{1187}
\pmtitle{proof of existence and unicity of self-similar fractals}
\pmrecord{5}{38153}
\pmprivacy{1}
\pmauthor{paolini}{1187}
\pmtype{Proof}
\pmcomment{trigger rebuild}
\pmclassification{msc}{28A80}

% this is the default PlanetMath preamble.  as your knowledge
% of TeX increases, you will probably want to edit this, but
% it should be fine as is for beginners.

% almost certainly you want these
\usepackage{amssymb}
\usepackage{amsmath}
\usepackage{amsfonts}

% used for TeXing text within eps files
%\usepackage{psfrag}
% need this for including graphics (\includegraphics)
%\usepackage{graphicx}
% for neatly defining theorems and propositions
\usepackage{amsthm}
% making logically defined graphics
%%%\usepackage{xypic}

% there are many more packages, add them here as you need them

% define commands here
\newcommand{\R}{\mathbb R}
\newtheorem{theorem}{Theorem}
\newtheorem{definition}{Definition}
\theoremstyle{remark}
\newtheorem{example}{Example}
\begin{document}
We consider the space $\mathcal K(X)=\{K\subset X\colon K\mathrm{\ compact\ and\ non\ empty}\}$ endowed with the Hausdorff distance $\delta$.
Since Hausdorff metric inherits completeness, being $X$ complete, $(\mathcal K(X),\delta)$ is complete too. We then consider the mapping $T\colon \mathcal K(X) \to \mathcal K(X)$ defined by
\[
  T(A) = \bigcup_{i=1}^N T_i(A).
\]
We claim that $T$ is a contraction. In fact, recalling that $\delta(A_1\cup A_2, B_1\cup B_2) \le \max\{\delta(A_1,B_1),\delta(A_2,B_2)\}$ while $\delta(T_i(A),T_i(B))\le \lambda_i \delta(A,B)$ if $T_i$ is $\lambda_i$-Lipschitz, we have
\begin{align*}
  \delta(T(A),T(B))
  &= \delta(\bigcup_i T_i(A), \bigcup_i T_i(B))
  \le \max_i \delta(T_i(A),T_i(B))\\
  &\le \max_i \lambda_i \delta(A,B) = \lambda \delta(A,B)
\end{align*}
with $\lambda=\max_i \lambda_i <1$. 

So $T$ is a contraction on the complete metric space $\mathcal K(X)$ and hence,
by Banach Fixed Point Theorem, there exists one and only one $K\in\mathcal K(X)$ such that $T(K)=K$.




%%%%%
%%%%%
\end{document}
