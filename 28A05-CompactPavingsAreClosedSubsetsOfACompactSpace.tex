\documentclass[12pt]{article}
\usepackage{pmmeta}
\pmcanonicalname{CompactPavingsAreClosedSubsetsOfACompactSpace}
\pmcreated{2013-03-22 18:45:01}
\pmmodified{2013-03-22 18:45:01}
\pmowner{gel}{22282}
\pmmodifier{gel}{22282}
\pmtitle{compact pavings are closed subsets of a compact space}
\pmrecord{6}{41527}
\pmprivacy{1}
\pmauthor{gel}{22282}
\pmtype{Theorem}
\pmcomment{trigger rebuild}
\pmclassification{msc}{28A05}
%\pmkeywords{compact paving}
%\pmkeywords{compact space.}

% almost certainly you want these
\usepackage{amssymb}
\usepackage{amsmath}
\usepackage{amsfonts}

% used for TeXing text within eps files
%\usepackage{psfrag}
% need this for including graphics (\includegraphics)
%\usepackage{graphicx}
% for neatly defining theorems and propositions
\usepackage{amsthm}
% making logically defined graphics
%%%\usepackage{xypic}

% there are many more packages, add them here as you need them

% define commands here
\newtheorem*{theorem*}{Theorem}
\newtheorem*{lemma*}{Lemma}
\newtheorem*{corollary*}{Corollary}
\newtheorem*{definition*}{Definition}
\newtheorem{theorem}{Theorem}
\newtheorem{lemma}{Lemma}
\newtheorem{corollary}{Corollary}
\newtheorem{definition}{Definition}

\begin{document}
\PMlinkescapeword{compact}
\PMlinkescapeword{collection}
\PMlinkescapeword{compact paving}
\PMlinkescapeword{way}
\PMlinkescapeword{intersections of sets}
\PMlinkescapeword{subsets}
\PMlinkescapeword{closed under}
\PMlinkescapeword{topology}
\PMlinkescapeword{compact pavings}
\PMlinkescapeword{theorem}

Recall that a paving $\mathcal{K}$ is compact if every subcollection satisfying the finite intersection property has nonempty intersection. In particular, a topological space is \PMlinkname{compact}{Compact} if and only if its collection of closed subsets forms a compact paving. Compact paved spaces can therefore be constructed by taking closed subsets of a compact topological space. In fact, all compact pavings arise in this way, as we now show.

Given any compact paving $\mathcal{K}$ the following result says that the collection $\mathcal{K}^\prime$ of all intersections of finite unions of sets in $\mathcal{K}$ is also compact.

\begin{theorem}\label{thm:1}
Suppose that $(K,\mathcal{K})$ is a compact paved space. Let $\mathcal{K}^\prime$ be the smallest collection of subsets of $X$ such that $\mathcal{K}\subseteq\mathcal{K}^\prime$ and which is closed under arbitrary intersections and finite unions.
Then, $\mathcal{K}^\prime$ is a compact paving.
\end{theorem}
In particular,
\begin{equation*}
\mathcal{T}\equiv\left\{K\setminus C\colon C\in\mathcal{K}^\prime\right\}\cup\left\{\emptyset,K\right\}
\end{equation*}
is closed under arbitrary unions and finite intersections, and hence is a topology on $K$.
The collection of closed sets defined with respect to this topology is $\mathcal{K}^\prime\cup\{\emptyset,K\}$ which, by Theorem \ref{thm:1}, is a compact paving. So, the following is obtained.
\begin{corollary*}
A paving $(K,\mathcal{K})$ is compact if and only if there exists a topology on $K$ with respect to which $\mathcal{K}$ are closed sets and $K$ is compact.
\end{corollary*}

%%%%%
%%%%%
\end{document}
