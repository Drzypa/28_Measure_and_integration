\documentclass[12pt]{article}
\usepackage{pmmeta}
\pmcanonicalname{ProofOfFatousLemma}
\pmcreated{2013-03-22 13:29:59}
\pmmodified{2013-03-22 13:29:59}
\pmowner{paolini}{1187}
\pmmodifier{paolini}{1187}
\pmtitle{proof of Fatou's lemma}
\pmrecord{4}{34076}
\pmprivacy{1}
\pmauthor{paolini}{1187}
\pmtype{Proof}
\pmcomment{trigger rebuild}
\pmclassification{msc}{28A20}

% this is the default PlanetMath preamble.  as your knowledge
% of TeX increases, you will probably want to edit this, but
% it should be fine as is for beginners.

% almost certainly you want these
\usepackage{amssymb}
\usepackage{amsmath}
\usepackage{amsfonts}

% used for TeXing text within eps files
%\usepackage{psfrag}
% need this for including graphics (\includegraphics)
%\usepackage{graphicx}
% for neatly defining theorems and propositions
%\usepackage{amsthm}
% making logically defined graphics
%%%\usepackage{xypic}

% there are many more packages, add them here as you need them

% define commands here
\begin{document}
Let $f(x)=\liminf_{n\to\infty} f_n(x)$ and let $g_n(x)=\inf_{k\ge n} f_k(x)$
so that we have
\[
  f(x) = \sup_n g_n(x).
\]

As $g_n$ is an increasing sequence of measurable nonnegative functions we can apply the monotone convergence Theorem to obtain
\[
  \int_X f\, d\mu = \lim_{n\to\infty} \int_X g_n\, d\mu.
\]
On the other hand, being $g_n\le f_n$, we conclude by observing
\[
  \lim_{n\to\infty} \int_X g_n\, d\mu 
= \liminf_{n\to\infty}\int_X g_n\, d\mu 
\le \liminf_{n\to\infty}\int_X f_n\, d\mu.
\]
%%%%%
%%%%%
\end{document}
