\documentclass[12pt]{article}
\usepackage{pmmeta}
\pmcanonicalname{ProofOfDominatedConvergenceTheorem1}
\pmcreated{2013-03-22 14:33:58}
\pmmodified{2013-03-22 14:33:58}
\pmowner{rspuzio}{6075}
\pmmodifier{rspuzio}{6075}
\pmtitle{proof of dominated convergence theorem}
\pmrecord{5}{36123}
\pmprivacy{1}
\pmauthor{rspuzio}{6075}
\pmtype{Proof}
\pmcomment{trigger rebuild}
\pmclassification{msc}{28A20}
%\pmkeywords{convergence}
%\pmkeywords{integral}
\pmrelated{ProofOfDominatedConvergenceTheorem}

% this is the default PlanetMath preamble.  as your knowledge
% of TeX increases, you will probably want to edit this, but
% it should be fine as is for beginners.

% almost certainly you want these
\usepackage{amssymb}
\usepackage{amsmath}
\usepackage{amsfonts}

% used for TeXing text within eps files
%\usepackage{psfrag}
% need this for including graphics (\includegraphics)
%\usepackage{graphicx}
% for neatly defining theorems and propositions
%\usepackage{amsthm}
% making logically defined graphics
%%%\usepackage{xypic}

% there are many more packages, add them here as you need them

% define commands here
\begin{document}
Define the functions $h_n^+$ and $h_n^-$ as follows:
 $$h_n^+ (x) = \sup \{f_m (x) \colon m \ge n\}$$
 $$h_n^- (x) = \inf \{f_m (x) \colon m \ge n\}$$
These suprema and infima exist because, for every $x$, $|f_n (x)|
\le g(x)$.  These functions enjoy the following properties:

For every $n$, $|h_n^\pm| \le g$

The sequence $h_n^+$ is decreasing and the sequence $h_n^-$ is increasing.

For every $x$, $\lim_{n \to \infty} h_n^\pm (x) = f(x)$

Each $h_n^\pm$ is measurable.

The first property follows from immediately from the definition of
supremum.  The second property follows from the fact that the
supremum or infimum is being taken over a larger set to define
$h_n^\pm (x)$ than to define $h_m^\pm (x)$ when $n > m$.  The third
property is a simple consequence of the fact that, for any sequence
of real numbers, if the sequence converges, then the sequence has an
upper limit and a lower limit which equal each other and equal the
limit.  As for the fourth statement, it means that, for every real
number $y$ and every integer $n$, the sets
 $$\{x \mid h_n^- (x) \ge y\} \hbox{\hskip 0.5in and \hskip 0.5in}
 \{x \mid h_n^+ (x) \le y\}$$
are measurable.  However, by the definition of $h_n^\pm$, these sets
can be expressed as
 $$\bigcup_{m \le n} \{x \mid f_n (x) \le y\} \hbox{\hskip 0.5in and
  \hskip 0.5in} \bigcup_{m \ge n} \{x \mid f_n (x) \le y\}$$
respectively. Since each $f_n$ is assumed to be measurable, each set
in either union is measurable.  Since the union of a countable
number of measurable sets is itself measurable, these unions are
measurable, and hence the functions $h_n^\pm$ are measurable.

Because of properties 1 and 4 above and the assumption that $g$ is
integrable, it follows that each $h_n^\pm$ is integrable.  This
conclusion and property 2 mean that the monotone convergence theorem
is applicable so one can conclude that $f$ is integrable and that
 $$\lim_{n \to \infty} \int h_n^\pm (x) \,d\mu(x) = \int \lim_{n \to \infty}
  h_n^\pm (x) \,d\mu(x)$$
By property 3, the right hand side equals $\int f(x) \,d\mu(x)$.

By construction, $h_n^- \le f_n \le h_n^+$ and hence
 $$\int h_n^- \le \int f_n \le \int h_n^+$$
Because the outer two terms in the above inequality tend towards the
same limit as $n \to \infty$, the middle term is squeezed into
converging to the same limit.  Hence
 $$\lim_{n \to \infty} \int f_n (x) \,d\mu(x) = \int f (x) \,d\mu(x)$$
%%%%%
%%%%%
\end{document}
