\documentclass[12pt]{article}
\usepackage{pmmeta}
\pmcanonicalname{ExampleOfANonRiemannIntegrableFunction}
\pmcreated{2013-03-22 15:03:28}
\pmmodified{2013-03-22 15:03:28}
\pmowner{paolini}{1187}
\pmmodifier{paolini}{1187}
\pmtitle{example of a non Riemann integrable function}
\pmrecord{4}{36776}
\pmprivacy{1}
\pmauthor{paolini}{1187}
\pmtype{Example}
\pmcomment{trigger rebuild}
\pmclassification{msc}{28-XX}
\pmclassification{msc}{26-XX}

% this is the default PlanetMath preamble.  as your knowledge
% of TeX increases, you will probably want to edit this, but
% it should be fine as is for beginners.

% almost certainly you want these
\usepackage{amssymb}
\usepackage{amsmath}
\usepackage{amsfonts}

% used for TeXing text within eps files
%\usepackage{psfrag}
% need this for including graphics (\includegraphics)
%\usepackage{graphicx}
% for neatly defining theorems and propositions
%\usepackage{amsthm}
% making logically defined graphics
%%%\usepackage{xypic}

% there are many more packages, add them here as you need them

% define commands here
\begin{document}
Let $[a,b]$ be any closed interval and
consider the Dirichlet's function $f\colon [a,b]\to\mathbb R$
\[
  f(x) = \begin{cases} 1 &\text{if $x$ is rational}\\
0 &\text{otherwise}.
\end{cases}
\]

Then $f$ is not Riemann integrable. In fact given any interval $[x_1,x_2]\subset [a,b]$ with $x_1<x_2$ one has
\[
 \sup_{[x_1,x_2]} f(x) = 1,\qquad
 \inf_{[x_1,x_2]} f(x) = 0
\]
because every interval contains both rational and irrational points.
So all upper Riemann sums are equal to $1$ and all lower Riemann sums are equal to $0$.
%%%%%
%%%%%
\end{document}
