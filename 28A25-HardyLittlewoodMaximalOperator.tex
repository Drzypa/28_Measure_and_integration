\documentclass[12pt]{article}
\usepackage{pmmeta}
\pmcanonicalname{HardyLittlewoodMaximalOperator}
\pmcreated{2013-03-22 13:27:30}
\pmmodified{2013-03-22 13:27:30}
\pmowner{azdbacks4234}{14155}
\pmmodifier{azdbacks4234}{14155}
\pmtitle{Hardy-Littlewood maximal operator}
\pmrecord{8}{34024}
\pmprivacy{1}
\pmauthor{azdbacks4234}{14155}
\pmtype{Definition}
\pmcomment{trigger rebuild}
\pmclassification{msc}{28A25}
\pmclassification{msc}{28A15}
\pmrelated{HardyLittlewoodMaximalTheorem}
\pmdefines{Hardy-Littlewood maximal function}

% this is the default PlanetMath preamble.  as your knowledge
% of TeX increases, you will probably want to edit this, but
% it should be fine as is for beginners.

% almost certainly you want these
\usepackage{amssymb}
\usepackage{amsmath}
\usepackage{amsfonts}

% used for TeXing text within eps files
%\usepackage{psfrag}
% need this for including graphics (\includegraphics)
%\usepackage{graphicx}
% for neatly defining theorems and propositions
%\usepackage{amsthm}
% making logically defined graphics
%%%\usepackage{xypic}

% there are many more packages, add them here as you need them

% define commands here
\begin{document}
The \emph{Hardy-Littlewood maximal operator} in $\mathbb{R}^n$ is an operator defined on $L^1_{\textnormal{loc}}(\mathbb{R}^n)$ (the space of locally integrable functions in $\mathbb{R}^n$ with the Lebesgue measure) which maps each locally integrable function $f$ to another function $Mf$, defined for each $x\in \mathbb{R}^n$ by 
\[Mf(x) = \sup_Q \frac{1}{m(Q)}\int_Q |f(y)|dy,\]
where the supremum is taken over all cubes $Q$ containing $x$. 
This function is lower semicontinuous (and hence measurable), and it is called the \emph{Hardy-Littlewood maximal function} of $f$.

The operator $M$ is sublinear, which means that 
\[M(af + bg) \leq |a|Mf + |b|Mg\]
for each pair of locally integrable functions $f,g$ and scalars $a,b$.
%%%%%
%%%%%
\end{document}
