\documentclass[12pt]{article}
\usepackage{pmmeta}
\pmcanonicalname{LpnormIsDualToLq}
\pmcreated{2013-03-22 18:38:13}
\pmmodified{2013-03-22 18:38:13}
\pmowner{gel}{22282}
\pmmodifier{gel}{22282}
\pmtitle{$L^p$-norm is dual to $L^q$}
\pmrecord{5}{41377}
\pmprivacy{1}
\pmauthor{gel}{22282}
\pmtype{Theorem}
\pmcomment{trigger rebuild}
\pmclassification{msc}{28A25}
\pmclassification{msc}{46E30}
%\pmkeywords{measure space}
%\pmkeywords{$L^p$-space}
\pmrelated{LpSpace}
\pmrelated{HolderInequality}
\pmrelated{BoundedLinearFunctionalsOnLinftymu}
\pmrelated{BoundedLinearFunctionalsOnLpmu}

\endmetadata

% almost certainly you want these
\usepackage{amssymb}
\usepackage{amsmath}
\usepackage{amsfonts}

% used for TeXing text within eps files
%\usepackage{psfrag}
% need this for including graphics (\includegraphics)
%\usepackage{graphicx}
% for neatly defining theorems and propositions
\usepackage{amsthm}
% making logically defined graphics
%%%\usepackage{xypic}

% there are many more packages, add them here as you need them

% define commands here
\newtheorem*{theorem*}{Theorem}
\newtheorem*{lemma*}{Lemma}
\newtheorem*{corollary*}{Corollary}
\newtheorem*{definition*}{Definition}
\newtheorem{theorem}{Theorem}
\newtheorem{lemma}{Lemma}
\newtheorem{corollary}{Corollary}
\newtheorem{definition}{Definition}

\begin{document}
If $(X,\mathfrak{M},\mu)$ is any measure space and $1\le p,q\le \infty$ are \PMlinkname{H\"older conjugates}{ConjugateIndex} then, for $f\in L^p$, the following linear function can be defined
\begin{align*}
&\Phi_f\colon L^q\rightarrow\mathbb{C},\\
&g\mapsto\Phi_f(g)\equiv\int fg\,d\mu.
\end{align*}
The \PMlinkname{H\"older inequality}{HolderInequality} shows that this gives a well defined and bounded linear map. Its operator norm is given by
\begin{equation*}
\Vert\Phi_f\Vert=\left\{\Vert fg\Vert_1:g\in L^q, \Vert g\Vert_q=1\right\}.
\end{equation*}
The following theorem shows that the operator norm of $\Phi_f$ is equal to the $L^p$-norm of $f$.

\begin{theorem*}
Let $(X,\mathfrak{M},\mu)$ be a $\sigma$-finite measure space and $p,q$ be H\"older conjugates. Then, any measurable function $f\colon X\rightarrow\mathbb{C}$ has $L^p$-norm
\begin{equation}\label{eq:1}
\Vert f\Vert_p=\sup\left\{\Vert fg\Vert_1: g\in L^q, \Vert g\Vert_q=1\right\}.
\end{equation}
Furthermore, if either $p<\infty$ and $\Vert f\Vert_p<\infty$ or $p=1$ then $\mu$ is not required to be $\sigma$-finite.
\end{theorem*}

Note that the $\sigma$-finite condition is required, except in the cases mentioned. For example, if $\mu$ is the measure satisfying $\mu(A)=\infty$ for every nonempty set $A$, then $L^p(\mu)=\{0\}$ for $p<\infty$ and it is easily checked that equality (\ref{eq:1}) fails whenever $f=1$ and $p>1$.

%%%%%
%%%%%
\end{document}
