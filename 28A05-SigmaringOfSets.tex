\documentclass[12pt]{article}
\usepackage{pmmeta}
\pmcanonicalname{SigmaringOfSets}
\pmcreated{2013-03-22 17:04:34}
\pmmodified{2013-03-22 17:04:34}
\pmowner{Mathprof}{13753}
\pmmodifier{Mathprof}{13753}
\pmtitle{sigma--ring of sets}
\pmrecord{7}{39369}
\pmprivacy{1}
\pmauthor{Mathprof}{13753}
\pmtype{Definition}
\pmcomment{trigger rebuild}
\pmclassification{msc}{28A05}

% this is the default PlanetMath preamble.  as your knowledge
% of TeX increases, you will probably want to edit this, but
% it should be fine as is for beginners.

% almost certainly you want these
\usepackage{amssymb}
\usepackage{amsmath}
\usepackage{amsfonts}

% used for TeXing text within eps files
%\usepackage{psfrag}
% need this for including graphics (\includegraphics)
%\usepackage{graphicx}
% for neatly defining theorems and propositions
%\usepackage{amsthm}
% making logically defined graphics
%%%\usepackage{xypic}

% there are many more packages, add them here as you need them

% define commands here

\begin{document}
A \emph{$\sigma$-ring of sets} is a nonempty collection $\mathcal{S}$ of sets such that
\begin{itemize}
\item
if $A \in \mathcal{S}$ and $B \in \mathcal{S}$ then $A-B \in \mathcal{S}$ and
\item
if $A_i \in \mathcal{S}$ for $i=1,2 \ldots,$ then $\cup_{i=1}^{\infty} A_i \in \mathcal{S}$
\end{itemize}

A $\sigma$-ring is also closed under countable intersections since
$$
\cap_{i=1}^{\infty} A_i = A - \cup_{i=1}^{\infty}(A-A_i)
$$
where $A = \cup_{i=1}^{\infty} A_i$.

$\sigma$-rings are  used in measure theory.

%%%%%
%%%%%
\end{document}
