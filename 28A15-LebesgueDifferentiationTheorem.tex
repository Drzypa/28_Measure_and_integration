\documentclass[12pt]{article}
\usepackage{pmmeta}
\pmcanonicalname{LebesgueDifferentiationTheorem}
\pmcreated{2013-03-22 13:27:36}
\pmmodified{2013-03-22 13:27:36}
\pmowner{Koro}{127}
\pmmodifier{Koro}{127}
\pmtitle{Lebesgue differentiation theorem}
\pmrecord{9}{34026}
\pmprivacy{1}
\pmauthor{Koro}{127}
\pmtype{Theorem}
\pmcomment{trigger rebuild}
\pmclassification{msc}{28A15}

\endmetadata

% this is the default PlanetMath preamble.  as your knowledge
% of TeX increases, you will probably want to edit this, but
% it should be fine as is for beginners.

% almost certainly you want these
\usepackage{amssymb}
\usepackage{amsmath}
\usepackage{amsfonts}

% used for TeXing text within eps files
%\usepackage{psfrag}
% need this for including graphics (\includegraphics)
%\usepackage{graphicx}
% for neatly defining theorems and propositions
%\usepackage{amsthm}
% making logically defined graphics
%%%\usepackage{xypic}

% there are many more packages, add them here as you need them

% define commands here
\begin{document}
Let $f$ be a locally integrable function on $\mathbb{R}^n$ with Lebesgue measure $m$, i.e. $f\in L^1_\textnormal{loc}(\mathbb{R}^n)$. \emph{Lebesgue's differentiation theorem} basically says that for almost every $x$, the averages 
\[\frac{1}{m(Q)}\int_Q |f(y)-f(x)|dy\]
converge to $0$ when $Q$ is a cube containing $x$ and $m(Q)\rightarrow 0$. 

Formally, this means that there is a set $N\subset \mathbb{R}^n$ with $\mu(N)=0$, such that for every $x\notin N$ and $\varepsilon>0$, there exists 
$\delta>0$ such that, for each cube $Q$ with $x\in Q$ and $m(Q)<\delta$, we have
\[\frac{1}{m(Q)}\int_Q|f(y)-f(x)|dy<\varepsilon.\]

For $n=1$, this can be restated as an analogue of the fundamental theorem of calculus for Lebesgue integrals. Given a $x_0\in \mathbb{R}$,
\[\frac{d}{dx}\int_{x_0}^x f(t)dt = f(x)\]
for almost every $x$.
%%%%%
%%%%%
\end{document}
