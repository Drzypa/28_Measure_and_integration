\documentclass[12pt]{article}
\usepackage{pmmeta}
\pmcanonicalname{LusinsTheorem}
\pmcreated{2013-03-22 16:13:48}
\pmmodified{2013-03-22 16:13:48}
\pmowner{Wkbj79}{1863}
\pmmodifier{Wkbj79}{1863}
\pmtitle{Lusin's theorem}
\pmrecord{6}{38330}
\pmprivacy{1}
\pmauthor{Wkbj79}{1863}
\pmtype{Theorem}
\pmcomment{trigger rebuild}
\pmclassification{msc}{28A20}
\pmrelated{MeasurableFunctions}
\pmrelated{AlmostContinuousFunction}

\usepackage{amssymb}
\usepackage{amsmath}
\usepackage{amsfonts}

\usepackage{psfrag}
\usepackage{graphicx}
\usepackage{amsthm}
%%\usepackage{xypic}

\begin{document}
Let $A$ be a Lebesgue measurable subset of $\mathbb{R}$ and $f:A \to \mathbb{C}$ (or $f:A \to \mathbb{R}$). Then $f$ is \PMlinkname{Lebesgue measurable}{MeasurableFunctions} if and only if $f$ is almost continuous.
%%%%%
%%%%%
\end{document}
