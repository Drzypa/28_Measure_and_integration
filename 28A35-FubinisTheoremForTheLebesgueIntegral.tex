\documentclass[12pt]{article}
\usepackage{pmmeta}
\pmcanonicalname{FubinisTheoremForTheLebesgueIntegral}
\pmcreated{2013-03-22 14:16:09}
\pmmodified{2013-03-22 14:16:09}
\pmowner{jirka}{4157}
\pmmodifier{jirka}{4157}
\pmtitle{Fubini's theorem for the Lebesgue integral}
\pmrecord{9}{35719}
\pmprivacy{1}
\pmauthor{jirka}{4157}
\pmtype{Theorem}
\pmcomment{trigger rebuild}
\pmclassification{msc}{28A35}
\pmsynonym{Fubini's theorem}{FubinisTheoremForTheLebesgueIntegral}
\pmrelated{FubinisTheorem}
\pmrelated{TonellisTheorem}

% this is the default PlanetMath preamble.  as your knowledge
% of TeX increases, you will probably want to edit this, but
% it should be fine as is for beginners.

% almost certainly you want these
\usepackage{amssymb}
\usepackage{amsmath}
\usepackage{amsfonts}

% used for TeXing text within eps files
%\usepackage{psfrag}
% need this for including graphics (\includegraphics)
%\usepackage{graphicx}
% for neatly defining theorems and propositions
\usepackage{amsthm}
% making logically defined graphics
%%%\usepackage{xypic}

% there are many more packages, add them here as you need them

% define commands here
\theoremstyle{theorem}
\newtheorem*{thm}{Theorem}
\newtheorem*{lemma}{Lemma}
\newtheorem*{conj}{Conjecture}
\newtheorem*{cor}{Corollary}
\newtheorem*{example}{Example}
\theoremstyle{definition}
\newtheorem*{defn}{Definition}
\begin{document}
This is the version of the Fubini's Theorem for the Lebesgue integral.  For
the Riemann integral, see \PMlinkname{the standard calculus version}{FubinisTheorem}.

In the following suppose we will by convention define $\int_X f d\mu := 0$ in case $f$
is not integrable.  This simplifies notation and does not affect the results since it will turn out that such cases happen on a set of measure 0.

Also if we have a function $f\colon X \times Y \to {\mathbb{F}}$ then define $f_x(y) := f(x,y)$ and $f^y(x) := f(x,y)$.

\begin{thm}[Fubini]
Suppose $(X,{\mathcal{M}},\mu)$ and $(Y,{\mathcal{N}},\nu)$ are \PMlinkname{$\sigma$-finite}{SigmaFinite}
measure spaces. If $f \in L^1(X \times Y)$
then $f_x \in L^1(\nu)$ for $\mu$-almost every $x$ and
$f^y \in L^1(\mu)$ for $\nu$-almost every $y$.  Further
the functions
$x \mapsto \int_Y f_x d\nu$ and $y \mapsto \int_X f^y d\nu$ are
in $L^1(\mu)$ and $L^1(\nu)$ respectively and
\begin{equation*}
\int_{X \times Y} f \, d(\mu \times \nu) =
\int_X \left[ \int_Y f(x,y) \,d\nu(y) \right] d\mu(x) =
\int_Y \left[ \int_X f(x,y) \,d\mu(x) \right] d\nu(y) .
\end{equation*}
\end{thm}

You can now see the reason for defining the integral even where $f_x$ and
$f^y$ are not integrable since the functions 
$x \mapsto \int_Y f_x d\nu$ and $y \mapsto \int_X f^y d\nu$
are normally only almost everywhere defined, and we'd like to define them everywhere.  Since we have changed the definition only on a set of measure
zero, this does not change the final result and we can interchange the
integrals freely without having to worry about where the functions are
actually defined.

Note the \PMlinkescapetext{difference} of this theorem and Tonelli's theorem for non-negative functions.  Here you actually need to check some integrability before switching
the integral \PMlinkescapetext{order}.  A \PMlinkescapetext{simple} application of Tonelli's theorem actually shows that you can prove any one of these equations to show that $f \in L^1(X \times Y)$
\begin{gather*}
\int_{X \times Y} \lvert f \rvert \, d(\mu \times \nu) < \infty ,
\\
\int_X \left[ \int_Y \lvert f(x,y) \rvert \,d\nu(y) \right] d\mu(x) < \infty, \text{ or }
\\
\int_Y \left[ \int_X \lvert f(x,y) \rvert \,d\mu(x) \right] d\nu(y) < \infty .
\end{gather*}

If we take the counting measure on ${\mathbb{N}}$, then one can \PMlinkescapetext{state} the Fubini theorem for sums.

\begin{thm}[Fubini for sums]
Suppose that $f_{ij}$ is absolutely summable, that is
$\sum_{i,j \in {\mathbb{N}}} \lvert f_{ij}\rvert < \infty$, then
\begin{equation*}
\sum_{i,j \in {\mathbb{N}}} f_{ij}
=
\sum_{i=1}^\infty \sum_{j=1}^\infty f_{ij}
=
\sum_{j=1}^\infty \sum_{i=1}^\infty f_{ij} .
\end{equation*}
\end{thm}

In the above theorem we have used ${\mathbb{N}}$ as our \PMlinkescapetext{index} set for simplicity and familiarity of notation.  Any summable function $f_{ij}$ will have only a countable number of non-zero elements and thus the theorem for arbitrary \PMlinkescapetext{index}
sets just reduces to the above case.

\begin{thebibliography}{9}
\bibitem{folland}
Gerald B.\@ Folland. \emph{\PMlinkescapetext{Real Analysis, Modern Techniques
and Their Applications}}. John Wiley \& Sons, Inc., New York, New York, 1999
\end{thebibliography}
%%%%%
%%%%%
\end{document}
