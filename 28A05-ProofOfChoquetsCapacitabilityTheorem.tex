\documentclass[12pt]{article}
\usepackage{pmmeta}
\pmcanonicalname{ProofOfChoquetsCapacitabilityTheorem}
\pmcreated{2013-03-22 18:47:51}
\pmmodified{2013-03-22 18:47:51}
\pmowner{gel}{22282}
\pmmodifier{gel}{22282}
\pmtitle{proof of Choquet's capacitability theorem}
\pmrecord{4}{41597}
\pmprivacy{1}
\pmauthor{gel}{22282}
\pmtype{Proof}
\pmcomment{trigger rebuild}
\pmclassification{msc}{28A05}
\pmclassification{msc}{28A12}
%\pmkeywords{paved space}
%\pmkeywords{analytic set}
%\pmkeywords{capacitable}

\endmetadata

% almost certainly you want these
\usepackage{amssymb}
\usepackage{amsmath}
\usepackage{amsfonts}

% used for TeXing text within eps files
%\usepackage{psfrag}
% need this for including graphics (\includegraphics)
%\usepackage{graphicx}
% for neatly defining theorems and propositions
\usepackage{amsthm}
% making logically defined graphics
%%%\usepackage{xypic}

% there are many more packages, add them here as you need them

% define commands here
\newtheorem*{theorem*}{Theorem}
\newtheorem*{lemma*}{Lemma}
\newtheorem*{corollary*}{Corollary}
\newtheorem*{definition*}{Definition}
\newtheorem{theorem}{Theorem}
\newtheorem{lemma}{Lemma}
\newtheorem{corollary}{Corollary}
\newtheorem{definition}{Definition}

\begin{document}
\PMlinkescapeword{finite}
\PMlinkescapeword{states}
\PMlinkescapeword{collection}
\PMlinkescapeword{countable}
\PMlinkescapeword{analytic}
\PMlinkescapeword{capacitable}
\PMlinkescapeword{lemma}
\PMlinkescapeword{compact}
\PMlinkescapeword{paving}
\PMlinkescapeword{closure}
\PMlinkescapeword{onto}
\PMlinkescapeword{projection}
\PMlinkescapeword{positive}
\PMlinkescapeword{integers}
\PMlinkescapeword{capacity}
\PMlinkescapeword{infinity}
\PMlinkescapeword{infinite}
\PMlinkescapeword{induction}
\PMlinkescapeword{inequality}
\PMlinkescapeword{contains}
\PMlinkescapeword{analytic sets}

Let $(X,\mathcal{F})$ be a paved space such that $\mathcal{F}$ is closed under finite unions and finite intersections, and let $I$ be an $\mathcal{F}$-\PMlinkname{capacity}{ChoquetCapacity}. We prove the capacitability theorem, which states that all $\mathcal{F}$-\PMlinkname{analytic}{AnalyticSet2} sets are $(\mathcal{F},I)$-\PMlinkname{capacitable}{ChoquetCapacity}. The idea is to deduce it from the following special case.

\begin{lemma*}
With the above notation, every set in $\mathcal{F}_{\sigma\delta}$ is $(\mathcal{F},I)$-capacitable.
\end{lemma*}

Recall that $\mathcal{F}_{\sigma\delta}$ is the collection of countable intersections of countable unions in $\mathcal{F}$ and, since countable unions and intersections of analytic sets are analytic, all such sets are analytic. According to the capacitability theorem they should then be capacitable, and the lemma is indeed a special case.

Supposing that the lemma is true, then the capacitability theorem can be deduced as follows. For an $\mathcal{F}$-analytic set $A\subseteq X$, there is a \PMlinkname{compact paved space}{PavedSpace} $(K,\mathcal{K})$ and $S\in(\mathcal{F}\times\mathcal{K})_{\sigma\delta}$ such that $A=\pi_X(S)$, where $\pi_X$ is the projection map from $X\times K$ to $X$. Letting $\mathcal{G}$ be the closure under finite unions and finite intersections of the paving $\mathcal{F}\times\mathcal{K}$, then the composition $I\circ\pi_X$ is a $\mathcal{G}$-capacity, and the projection of any $(\mathcal{G},I\circ\pi_X)$-capacitable set onto $X$ is itself $(\mathcal{F},I)$-capacitable (see \PMlinkname{extending a capacity to a Cartesian product}{ExtendingACapacityToACartesianProduct}). In particular, $S\in\mathcal{G}_{\sigma\delta}$ so, by the lemma, is $(\mathcal{G},I\circ\pi_X)$-capacitable. Therefore, $A=\pi_X(S)$ is $(\mathcal{F},I)$-capacitable. It only remains to prove the lemma.


\begin{proof}[Proof of lemma]
If $S\in\mathcal{F}_{\sigma\delta}$ then there exists $S_{m,n}\in\mathcal{F}$ such that
\begin{equation*}
S =\bigcap_n\bigcup_mS_{m,n}.
\end{equation*}
For any positive integers $m_1,m_2,\ldots,m_k$ let us write
\begin{equation*}
S(m_1,m_2,\ldots,m_k)\equiv\left(\bigcap_{n\le k}\bigcup_{m\le m_n}S_{m,n}\right)\bigcap\left(\bigcap_{n>k}\bigcup_mS_{m,n}\right).
\end{equation*}
In particular, $S()=S$ and, $I(S())=I(S)$. For any $\epsilon>0$ and $k\in\mathbb{N}$ suppose that we have chosen positive integers $m_1,\dots,m_{k-1}$ such that $I(S(m_1,\ldots,m_{k-1}))>I(S)-\epsilon$. Since $I$ is a capacity and $S(m_1,\ldots,m_k)$ increases to $S(m_1,\ldots,m_{k-1})$ as $m_k$ increases to infinity,
\begin{equation*}
I(S(m_1,\ldots,m_{k}))\rightarrow I(S(m_1,\ldots,m_{k-1}))
\end{equation*}
as $m_k$ tends to infinity. So, by choosing $m_k$ large enough, we have
\begin{equation*}
I(S(m_1,\ldots,m_k))>I(S)-\epsilon.
\end{equation*}
Then, by induction, we can find an infinite sequence $m_1,m_2,\ldots$ such that this inequality holds for every $k$.
Setting
\begin{align*}
&A_k\equiv\bigcap_{n\le k}\bigcup_{m\le m_n}S_{m,n}\in\mathcal{F},\\
&A\equiv\bigcap_n\bigcup_{m\le m_n}S_{m,n}=\bigcap_kA_k\in\mathcal{F}_\delta,
\end{align*}
then $A\subseteq S$. Furthermore, $A_k$ contains $S(m_1,\ldots,m_k)$ and decreases to $A$ as $k$ tends to infinity.  As $I$ is an $\mathcal{F}$-capacity this gives
\begin{equation*}
I(A)=\lim_{k\rightarrow\infty}I(A_k)\ge\lim_{k\rightarrow\infty}I(S(m_1,\ldots,m_k))\ge I(S)-\epsilon.
\end{equation*}
So $S$ is $(\mathcal{F},I)$-capacitable, as required.
\end{proof}

%%%%%
%%%%%
\end{document}
