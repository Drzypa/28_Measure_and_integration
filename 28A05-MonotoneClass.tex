\documentclass[12pt]{article}
\usepackage{pmmeta}
\pmcanonicalname{MonotoneClass}
\pmcreated{2013-03-22 17:07:25}
\pmmodified{2013-03-22 17:07:25}
\pmowner{fernsanz}{8869}
\pmmodifier{fernsanz}{8869}
\pmtitle{monotone class}
\pmrecord{5}{39426}
\pmprivacy{1}
\pmauthor{fernsanz}{8869}
\pmtype{Definition}
\pmcomment{trigger rebuild}
\pmclassification{msc}{28A05}
%\pmkeywords{Monotone class theorem}
%\pmkeywords{sigma algebra}
\pmrelated{SigmaAlgebra}
\pmrelated{MonotoneClassTheorem}

% this is the default PlanetMath preamble.  as your knowledge
% of TeX increases, you will probably want to edit this, but
% it should be fine as is for beginners.

% almost certainly you want these
\usepackage{amssymb}
\usepackage{amsmath}
\usepackage{amsfonts}
\usepackage{amsthm}

% define commands here
\theoremstyle{definition}
\newtheorem*{defn}{Definition}
\theoremstyle{remark}
\newtheorem*{rem}{Remark}
\numberwithin{equation}{section}
\begin{document}
\title{Monotone Class}%
\author{Fernando Sanz Gámiz}%

\begin{defn}
A collection $\mathcal M$ of subsets of $\Omega$ is a \emph{monotone class}
if it is closed under increasing and decreasing sequences of sets,
i.e.

\begin{enumerate}
\item $A_1 \subseteq A_2 \subseteq A_3 ,...\in \mathcal M \Rightarrow
\bigcup A_n \in \mathcal M$
\item $A_1 \supseteq A_2 \supseteq A_3 ,...\in \mathcal M \Rightarrow
\bigcap A_n \in \mathcal M$
\end{enumerate}

\end{defn}
%%%%%
%%%%%
\end{document}
