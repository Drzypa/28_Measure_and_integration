\documentclass[12pt]{article}
\usepackage{pmmeta}
\pmcanonicalname{ProofOfPseudoparadoxInMeasureTheory}
\pmcreated{2013-03-22 14:38:43}
\pmmodified{2013-03-22 14:38:43}
\pmowner{rspuzio}{6075}
\pmmodifier{rspuzio}{6075}
\pmtitle{proof of pseudoparadox in measure theory}
\pmrecord{10}{36233}
\pmprivacy{1}
\pmauthor{rspuzio}{6075}
\pmtype{Proof}
\pmcomment{trigger rebuild}
\pmclassification{msc}{28E99}
\pmrelated{ProofOfVitalisTheorem}

% this is the default PlanetMath preamble.  as your knowledge
% of TeX increases, you will probably want to edit this, but
% it should be fine as is for beginners.

% almost certainly you want these
\usepackage{amssymb}
\usepackage{amsmath}
\usepackage{amsfonts}

% used for TeXing text within eps files
%\usepackage{psfrag}
% need this for including graphics (\includegraphics)
%\usepackage{graphicx}
% for neatly defining theorems and propositions
%\usepackage{amsthm}
% making logically defined graphics
%%%\usepackage{xypic}

% there are many more packages, add them here as you need them

% define commands here
\begin{document}
Since this paradox depends crucially on the axiom of choice, we will
place the application of this controversial axiom at the head of the
proof rather than bury it deep within the bowels of the argument.

One can define an equivalence relation $\sim$ on $\mathbb{R}$ by the
condition that $x \sim y$ if and only if $x - y$ is rational.  By
the Archimedean property of the real line, for every $x \in \mathbb{R}$ there will exist a number $y \in [0,1)$ such that $y \sim x$.
Therefore, by the axiom of choice, there will exist a choice
function $f \colon \mathbb{R} \to [0,1)$ such that $f(x) = f(y)$ if
and only if $x \sim y$.

We shall use our choice function $f$ to exhibit a bijection between
$[0,1)$ and $[0,2)$.  Let $w$ be the ``wrap-around function'' which
is defined as $w(x) = x$ when $x \ge 0$ and $w(x) = x + 2$ when $x <
0$.  Define $g \colon [0,1) \to \mathbb{R}$ by
 $$g(x) = w(2x - f(x))$$
From the definition, it is clear that, since $x \sim f(x)$ and $w(x)
\sim x$, $g(x) \sim x$.  Also, it is easy to see that $g$ maps
$[0,1)$ into $[0,2)$.  If $f(x) \le 2x$, then $g(x) = 2x - f(x) \le
2x < 2$.  On the other hand, if $f(x) > 2x$, then $g(x) = 2x + 2 -
f(x)$.  Since $2x - f(x)$ is strictly negative, $g(x) < 2$.  Since
$f(x) < 1$, $g(x) > 0$.

Next, we will show that $g$ is injective.  Suppose that $g(x) =
g(y)$ and $x < y$.  By what we already observed, $x \sim y$, so $y -
x$ is a non-negative rational number and $f(x) = f(y)$.  There are 3
possible cases: 1) $f(x) \le 2x \le 2y$  In this case, $g(x) = g(y)$
implies that $2x - f(x) = 2y - f(x)$, which would imply that $x =
y$. 2) $2x < f(x) < 2y$  In this case, $g(x) = g(y)$ implies $2 + 2x
- f(x) = 2y - f(x)$ which, in turn, implies that $y = x + 1$, which
is impossible if both $x$ and $y$ belong to $[0,1)$.  3) $2x < 2y <
f(x)$ In this case, $g(x) = g(y)$ implies that $2x + 2 - f(x) = 2y +
2 - f(x)$, which would imply that $x = y$.  The only remaining
possibility is that $x=y$, so $g(x) = g(y)$ implies that $x = y$.

Next, we show that $g$ is surjective.  Pick a number $y$ in $[0,2)$.
We need to find a number $x \in [0,1)$ such that $w(2x - f(y)) = y$.
If $f(y) + y < 2$, we can choose $x = (f(y) + y)/2$.  If $2 \le f(y)
+ y$, we can choose $x = (f(y) + y)/2 - 1$.

Having shown that $g$ is a bijection between $[0,1)$ and $[0,2)$, we
shall now complete the proof by examining the action of $g$.  As we
already noted, $g(x) - x$ is a rational number.   Since the rational
numbers are countable, we can arrange them in a series $r_0, r_1,
r_2 \ldots$ such that no number is counted twice.  Define $A_i
\subset C_1$ as
 $$A_i = \{ x \in [0,1) \mid g(x) = r_i \}$$
It is obvious from this definition that the $A_i$ are mutually
disjoint.  Furthermore, $\bigcup_{i=1}^\infty A_i = [0,1)$ and
$\bigcup_{i=1}^\infty B_i = [0,2)$ where $B_i$ is the translate of
$A_i$ by $r_i$.
%%%%%
%%%%%
\end{document}
