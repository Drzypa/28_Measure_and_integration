\documentclass[12pt]{article}
\usepackage{pmmeta}
\pmcanonicalname{BoundedLinearFunctionalsOnLpmu}
\pmcreated{2013-03-22 18:32:57}
\pmmodified{2013-03-22 18:32:57}
\pmowner{azdbacks4234}{14155}
\pmmodifier{azdbacks4234}{14155}
\pmtitle{bounded linear functionals on $L^p(\mu)$}
\pmrecord{15}{41269}
\pmprivacy{1}
\pmauthor{azdbacks4234}{14155}
\pmtype{Theorem}
\pmcomment{trigger rebuild}
\pmclassification{msc}{28B15}
%\pmkeywords{linear functional}
%\pmkeywords{dual space}
%\pmkeywords{conjugate exponent}
\pmrelated{LpSpace}
\pmrelated{HolderInequality}
\pmrelated{ContinuousLinearMapping}
\pmrelated{BanachSpace}
\pmrelated{DualSpace}
\pmrelated{ConjugateIndex}
\pmrelated{RadonNikodymTheorem}
\pmrelated{BoundedLinearFunctionalsOnLinftymu}
\pmrelated{LpNormIsDualToLq}

%packages
\usepackage{amsmath,mathrsfs,amsfonts,amsthm}
%theorem environments
\theoremstyle{plain}
\newtheorem*{thm*}{Theorem}
\newtheorem*{lem*}{Lemma}
\newtheorem*{cor*}{Corollary}
\newtheorem*{prop*}{Proposition}
%delimiters
\newcommand{\set}[1]{\{#1\}}
\newcommand{\medset}[1]{\big\{#1\big\}}
\newcommand{\bigset}[1]{\bigg\{#1\bigg\}}
\newcommand{\Bigset}[1]{\Bigg\{#1\Bigg\}}
\newcommand{\abs}[1]{\vert#1\vert}
\newcommand{\medabs}[1]{\big\vert#1\big\vert}
\newcommand{\bigabs}[1]{\bigg\vert#1\bigg\vert}
\newcommand{\Bigabs}[1]{\Bigg\vert#1\Bigg\vert}
\newcommand{\norm}[1]{\Vert#1\Vert}
\newcommand{\mednorm}[1]{\big\Vert#1\big\Vert}
\newcommand{\bignorm}[1]{\bigg\Vert#1\bigg\Vert}
\newcommand{\Bignorm}[1]{\Bigg\Vert#1\Bigg\Vert}
\newcommand{\vbrack}[1]{\langle#1\rangle}
\newcommand{\medvbrack}[1]{\big\langle#1\big\rangle}
\newcommand{\bigvbrack}[1]{\bigg\langle#1\bigg\rangle}
\newcommand{\Bigvbrack}[1]{\Bigg\langle#1\Bigg\rangle}
\newcommand{\sbrack}[1]{[#1]}
\newcommand{\medsbrack}[1]{\big[#1\big]}
\newcommand{\bigsbrack}[1]{\bigg[#1\bigg]}
\newcommand{\Bigsbrack}[1]{\Bigg[#1\Bigg]}
%operators
\DeclareMathOperator{\Hom}{Hom}
\DeclareMathOperator{\Tor}{Tor}
\DeclareMathOperator{\Ext}{Ext}
\DeclareMathOperator{\Aut}{Aut}
\DeclareMathOperator{\End}{End}
\DeclareMathOperator{\Inn}{Inn}
\DeclareMathOperator{\lcm}{lcm}
\DeclareMathOperator{\ord}{ord}
\DeclareMathOperator{\rank}{rank}
\DeclareMathOperator{\tr}{tr}
\DeclareMathOperator{\Mat}{Mat}
\DeclareMathOperator{\Gal}{Gal}
\DeclareMathOperator{\GL}{GL}
\DeclareMathOperator{\SL}{SL}
\DeclareMathOperator{\SO}{SO}
\DeclareMathOperator{\ann}{ann}
\DeclareMathOperator{\im}{im}
\DeclareMathOperator{\Char}{char}
\DeclareMathOperator{\Spec}{Spec}
\DeclareMathOperator{\supp}{supp}
\DeclareMathOperator{\diam}{diam}
\DeclareMathOperator{\Ind}{Ind}
\DeclareMathOperator{\vol}{vol}

\begin{document}
If $\mu$ is a positive measure on a set $X$, $1\leq p\leq\infty$, and $g\in L^q(\mu)$, where $q$ is the H\"{o}lder conjugate of $p$, then H\"{o}lder's inequality implies that the map $f\mapsto\int_Xfgd\mu$ is a bounded linear functional on $L^p(\mu)$. It is therefore natural to ask whether or not all such functionals on $L^p(\mu)$ are of this form for some $g\in L^q(\mu)$. Under fairly mild hypotheses, and excepting the case $p=\infty$, the Radon-Nikodym Theorem answers this question affirmatively.
\begin{thm*}
Let $(X,\mathfrak{M},\mu)$ be a $\sigma$-finite measure space, $1\leq p<\infty$, and $q$ the H\"{o}lder conjugate of $p$. If $\Phi$ is a bounded linear functional on $L^p(\mu)$, then there exists a unique $g\in L^q(\mu)$ such that
\begin{equation}
\Phi(f)=\int_Xfgd\mu
\end{equation}
for all $f\in L^p(\mu)$. Furthermore, $\norm{\Phi}=\norm{g}_q$. Thus, under the stated hypotheses, $L^q(\mu)$ is isometrically isomorphic to the dual space of $L^p(\mu)$.
\end{thm*}
If $1<p<\infty$, then the assertion of the theorem remains valid without the assumption that $\mu$ is $\sigma$-finite; however, even with this hypothesis, the result can fail in the case that $p=\infty$. In particular, the bounded linear functionals on $L^\infty(m)$, where $m$ is Lebesgue measure on $[0,1]$, are not all obtained in the above manner via members of $L^1(m)$. An explicit example illustrating this is constructed as follows: the assignment $f\mapsto f(0)$ defines a bounded linear functional on $C([0,1])$, which, by the Hahn-Banach Theorem, may be extended to a bounded linear functional $\Phi$ on $L^\infty(m)$. Assume for the sake of contradiction that there exists $g\in L^1(m)$ such that $\Phi(f)=\int_{[0,1]}fgdm$ for every $f\in L^\infty(m)$, and for $n\in\mathbb{Z}^+$, define $f_n:[0,1]\rightarrow\mathbb{C}$ by $f_n(x)=\max\set{1-nx,0}$. As each $f_n$ is continuous, we have $\Phi(f_n)=\varphi(f_n)=1$ for all $n$; however, because $f_n\rightarrow 0$ almost everywhere and $\abs{f_n}\leq 1$, the Dominated Convergence Theorem, together with our hypothesis on $g$, gives
\begin{equation*}
1=\lim_{n\rightarrow\infty}\Phi(f_n)=\lim_{n\rightarrow\infty}\int_{[0,1]}f_ngdm=0\text{,}
\end{equation*}
a contradiction. It follows that no such $g$ can exist.

%%%%%
%%%%%
\end{document}
