\documentclass[12pt]{article}
\usepackage{pmmeta}
\pmcanonicalname{HaarIntegral}
\pmcreated{2013-03-22 13:39:56}
\pmmodified{2013-03-22 13:39:56}
\pmowner{rspuzio}{6075}
\pmmodifier{rspuzio}{6075}
\pmtitle{Haar integral}
\pmrecord{9}{34324}
\pmprivacy{1}
\pmauthor{rspuzio}{6075}
\pmtype{Definition}
\pmcomment{trigger rebuild}
\pmclassification{msc}{28C05}
\pmdefines{normalized Haar integral}

\endmetadata

% this is the default PlanetMath preamble.  as your knowledge
% of TeX increases, you will probably want to edit this, but
% it should be fine as is for beginners.

% almost certainly you want these
\usepackage{amssymb}
\usepackage{amsmath}
\usepackage{amsfonts}

% used for TeXing text within eps files
%\usepackage{psfrag}
% need this for including graphics (\includegraphics)
%\usepackage{graphicx}
% for neatly defining theorems and propositions
%\usepackage{amsthm}
% making logically defined graphics
%%%\usepackage{xypic} 

% there are many more packages, add them here as you need them

% define commands here
\begin{document}
Let $\Gamma$ be a locally compact topological group and $\mathcal{C}$ be the algebra of all continuous real-valued functions on $\Gamma$ with compact support.  In addition we define $\mathcal{C}^{+}$ to be the set of non-negative functions that belong to $\mathcal{C}$.  The \emph{Haar integral} is a real linear map $I$ of $\mathcal{C}$ into the field of the real number for $\Gamma$ if it satisfies:
\begin{itemize}
\item $I$ is not the zero map
\item $I$ only takes non-negative values on $\mathcal{C}^{+}$
\item $I$ has the following property $I(\gamma\cdot f) = I(f)$ for all elements $f$ of $\mathcal{C}$ and all element $\gamma$ of $\Gamma$.
\end{itemize}
The \emph{Haar integral} may be denoted in the following way \textit{(there are also other ways)}:
\begin{center}
$\int_{\gamma \in \Gamma} f(\gamma)$ or $\int_\Gamma f$ or $\int_\Gamma f d\gamma$ or $I(f)$
\end{center}
The following are necessary and sufficient conditions for the existence of
a unique Haar integral:
There is a real-valued function $I^+$
\begin{enumerate}
\item \textit{(Linearity)}.$I^+ (\lambda f + \mu g) = \lambda I^+(f) + \mu I^+(g)$ where $f,g \in \mathcal{C}^+$ and $\lambda, \mu \in \mathbb{R}_+$. 
\item \textit{(Positivity)}.  If $f(\gamma ) \geq 0$ for all $\gamma \in \Gamma$ then $I^+(f(\gamma)) \geq 0$.
\item \textit{(Translation-Invariance)}.  $I(f(\delta \gamma )) = I(f(\gamma ))$ for any fixed $\delta \in \Gamma$ and every $f$ in $\mathcal{C}^+$.
\end{enumerate}
An additional property is if $\Gamma$ is a compact group then the Haar integral  has right translation-invariance:  $\int_{\gamma \in \Gamma} f(\gamma \delta) = \int_{\gamma \in \Gamma} f(\gamma )$ for any fixed $\delta \in \Gamma$.
In addition we can define \emph{normalized Haar integral} to be $\int_\Gamma 1 = 1$ since $\Gamma$ is compact, it implies that $\int_\Gamma 1$ is finite.\\
\textit{(The proof for existence and uniqueness of the Haar integral is presented in \cite{2} on page 9.)}\\\\
\small\textit{(the information of this entry is in part quoted and paraphrased from \cite{1})}
\begin{thebibliography}{2}
\bibitem[GSS]{1} Golubsitsky, Martin. Stewart, Ian. Schaeffer, G. David.: Singularities and Groups in Bifurcation Theory \textit{(Volume II)}. Springer-Verlag, New York, 1988.
\bibitem[HG]{2} Gochschild, G.: The Structure of Lie Groups. Holden-Day, San Francisco, 1965.
\end{thebibliography}
%%%%%
%%%%%
\end{document}
