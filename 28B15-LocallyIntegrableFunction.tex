\documentclass[12pt]{article}
\usepackage{pmmeta}
\pmcanonicalname{LocallyIntegrableFunction}
\pmcreated{2013-03-22 13:44:19}
\pmmodified{2013-03-22 13:44:19}
\pmowner{matte}{1858}
\pmmodifier{matte}{1858}
\pmtitle{locally integrable function}
\pmrecord{11}{34430}
\pmprivacy{1}
\pmauthor{matte}{1858}
\pmtype{Definition}
\pmcomment{trigger rebuild}
\pmclassification{msc}{28B15}

% this is the default PlanetMath preamble.  as your knowledge
% of TeX increases, you will probably want to edit this, but
% it should be fine as is for beginners.

% almost certainly you want these
\usepackage{amssymb}
\usepackage{amsmath}
\usepackage{amsfonts}

% used for TeXing text within eps files
%\usepackage{psfrag}
% need this for including graphics (\includegraphics)
%\usepackage{graphicx}
% for neatly defining theorems and propositions
%\usepackage{amsthm}
% making logically defined graphics
%%%\usepackage{xypic}

% there are many more packages, add them here as you need them

% define commands here

\newcommand{\sR}[0]{\mathbb{R}}
\newcommand{\sC}[0]{\mathbb{C}}
\newcommand{\sN}[0]{\mathbb{N}}
\newcommand{\sZ}[0]{\mathbb{Z}}
\begin{document}
{\bf Definition} 
Suppose that $U$ is an open set in $\sR^n$, and 
$f\colon U\to\sC$ is a Lebesgue measurable function. 
If the Lebesgue integral
$$
   \int_K |f| dx
$$
is finite for all compact subsets $K$ in $U$, then $f$ is
\emph{locally integrable}. The set of all
such functions is denoted by 
$L^1_{\scriptsize{\mbox{loc}}}(U)$.

\subsubsection*{Example}
\begin{enumerate}
\item $L^1(U)\subset L^1_{\scriptsize{\mbox{loc}}}(U)$, where
$L^1(U)$ is the set of (globally) integrable functions. 
\item Continuous functions are locally integrable.
\item The function $f(x)=1/x$ for $x\neq 0$ and $f(0)=0$ 
is not locally integrable. 
\end{enumerate}
%%%%%
%%%%%
\end{document}
