\documentclass[12pt]{article}
\usepackage{pmmeta}
\pmcanonicalname{SanMarcoDragon}
\pmcreated{2013-03-22 17:15:54}
\pmmodified{2013-03-22 17:15:54}
\pmowner{PrimeFan}{13766}
\pmmodifier{PrimeFan}{13766}
\pmtitle{San Marco dragon}
\pmrecord{5}{39603}
\pmprivacy{1}
\pmauthor{PrimeFan}{13766}
\pmtype{Example}
\pmcomment{trigger rebuild}
\pmclassification{msc}{28A80}

% this is the default PlanetMath preamble.  as your knowledge
% of TeX increases, you will probably want to edit this, but
% it should be fine as is for beginners.

% almost certainly you want these
\usepackage{amssymb}
\usepackage{amsmath}
\usepackage{amsfonts}

% used for TeXing text within eps files
%\usepackage{psfrag}

% need this for including graphics (\includegraphics)
\usepackage{graphicx}

% for neatly defining theorems and propositions
%\usepackage{amsthm}
% making logically defined graphics
%%%\usepackage{xypic}

% there are many more packages, add them here as you need them

% define commands here

\begin{document}
The {\em San Marco dragon} is a \PMlinkname{Julia set}{SetDeJulia} produced by $$c = -\frac{3}{4} + 0i.$$

\begin{center}
\includegraphics{SanMarcoDragon}
\end{center}

Like other \PMlinkescapetext{Julia sets} on the horizontal center line of the Mandelbrot set, the San Marco dragon is symmetrical around its horizontal axis, but this particular set reminded Beno\^it Mandelbrot of St. Mark's cathedral in Venice (and its reflection in the canal) more than the others.

\begin{thebibliography}{1}
\bibitem{hl} H. Lauwerier, translated by Sophia Gill-Hoffst\"adt. {\it Fractals: Endlessly Repeated Geometric Figures} Princeton: Princeton University Press (1991): 144
\end{thebibliography}
%%%%%
%%%%%
\end{document}
