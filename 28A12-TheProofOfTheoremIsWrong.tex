\documentclass[12pt]{article}
\usepackage{pmmeta}
\pmcanonicalname{TheProofOfTheoremIsWrong}
\pmcreated{2013-03-22 19:16:05}
\pmmodified{2013-03-22 19:16:05}
\pmowner{tomprimozic}{26284}
\pmmodifier{tomprimozic}{26284}
\pmtitle{The proof of theorem is wrong}
\pmrecord{4}{42199}
\pmprivacy{1}
\pmauthor{tomprimozic}{26284}
\pmtype{Example}
\pmcomment{trigger rebuild}
\pmclassification{msc}{28A12}

\endmetadata

% this is the default PlanetMath preamble.  as your knowledge
% of TeX increases, you will probably want to edit this, but
% it should be fine as is for beginners.

% almost certainly you want these
\usepackage{amssymb}
\usepackage{amsmath}
\usepackage{amsfonts}

% used for TeXing text within eps files
%\usepackage{psfrag}
% need this for including graphics (\includegraphics)
%\usepackage{graphicx}
% for neatly defining theorems and propositions
%\usepackage{amsthm}
% making logically defined graphics
%%%\usepackage{xypic}

% there are many more packages, add them here as you need them

% define commands here

\begin{document}
Let's create a very simple measurable space: $X=\{a,b\}$, $\mathcal{A}=\{\emptyset, \{a\}, \{b\}, X\}$. 

Let's take the $\pi$-system $P=\{\{a\}\}$ containing only one subset of $X$. 

Let's create two measures $\mu = \delta_a + \delta_b$ and $\nu = \delta_a + 2\delta_b$. Then obviously $\mu$ and $\nu$ agree on $P$ and are finite, but they obviously are not equal on $\mathcal{A}$.

The proof, however, claims that it is sufficient if $\mu$ and $\nu$ are finite. I believe that $\mu(X)=\nu(X)$ is a necessary condition.
%%%%%
%%%%%
\end{document}
