\documentclass[12pt]{article}
\usepackage{pmmeta}
\pmcanonicalname{ExampleOfIntegrationOverSphereWithRespectToSurfaceArea}
\pmcreated{2013-03-22 14:57:58}
\pmmodified{2013-03-22 14:57:58}
\pmowner{rspuzio}{6075}
\pmmodifier{rspuzio}{6075}
\pmtitle{example of integration over sphere with respect to surface area}
\pmrecord{5}{36665}
\pmprivacy{1}
\pmauthor{rspuzio}{6075}
\pmtype{Example}
\pmcomment{trigger rebuild}
\pmclassification{msc}{28A75}

\endmetadata

% this is the default PlanetMath preamble.  as your knowledge
% of TeX increases, you will probably want to edit this, but
% it should be fine as is for beginners.

% almost certainly you want these
\usepackage{amssymb}
\usepackage{amsmath}
\usepackage{amsfonts}

% used for TeXing text within eps files
%\usepackage{psfrag}
% need this for including graphics (\includegraphics)
%\usepackage{graphicx}
% for neatly defining theorems and propositions
%\usepackage{amsthm}
% making logically defined graphics
%%%\usepackage{xypic}

% there are many more packages, add them here as you need them

% define commands here
\begin{document}
As an example of how to use the formula derived in \PMlinkid{example 1}{6664}, let us consider the following example:
 $$\int_S (\sin u \cos v - \sin v)^2 d^2 A = \int_0^{2 \pi} \int_0^\pi (\sin u \cos v - \sin v)^2 \sin u \> du \, dv =$$
$$\int_0^{2 \pi} \int_0^\pi \left( \sin^3 u \cos^2 v - 2 \sin^2 u \sin v \cos v + \sin u \sin^2 v \right) \> du \, dv =$$
$$\int_0^{2 \pi} \left( 2 \cos^2 v - \frac{2}{3} \cos^2 v - \pi \sin v \cos v  + 2 \sin^2 v \right) \, dv =$$
$$2 \pi - \frac{2 \pi}{3} - 0 + 2 \pi = \frac{10 \pi}{3}$$

To return to the main entry \PMlinkid{click here.}{6660}
%%%%%
%%%%%
\end{document}
