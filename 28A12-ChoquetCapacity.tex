\documentclass[12pt]{article}
\usepackage{pmmeta}
\pmcanonicalname{ChoquetCapacity}
\pmcreated{2013-03-22 18:47:26}
\pmmodified{2013-03-22 18:47:26}
\pmowner{gel}{22282}
\pmmodifier{gel}{22282}
\pmtitle{Choquet capacity}
\pmrecord{5}{41587}
\pmprivacy{1}
\pmauthor{gel}{22282}
\pmtype{Definition}
\pmcomment{trigger rebuild}
\pmclassification{msc}{28A12}
\pmclassification{msc}{28A05}
\pmsynonym{capacity}{ChoquetCapacity}
%\pmkeywords{paved space}
%\pmkeywords{set function}
\pmdefines{capacitable}

% almost certainly you want these
\usepackage{amssymb}
\usepackage{amsmath}
\usepackage{amsfonts}

% used for TeXing text within eps files
%\usepackage{psfrag}
% need this for including graphics (\includegraphics)
%\usepackage{graphicx}
% for neatly defining theorems and propositions
\usepackage{amsthm}
% making logically defined graphics
%%%\usepackage{xypic}

% there are many more packages, add them here as you need them

% define commands here
\newtheorem*{theorem*}{Theorem}
\newtheorem*{lemma*}{Lemma}
\newtheorem*{corollary*}{Corollary}
\newtheorem*{definition*}{Definition}
\newtheorem{theorem}{Theorem}
\newtheorem{lemma}{Lemma}
\newtheorem{corollary}{Corollary}
\newtheorem{definition}{Definition}

\begin{document}
\PMlinkescapeword{set function}
\PMlinkescapeword{collection}
\PMlinkescapeword{subsets}
\PMlinkescapeword{increasing}
\PMlinkescapeword{decreasing}
\PMlinkescapeword{subadditivity}
\PMlinkescapeword{finite}
\PMlinkescapeword{mapping}
\PMlinkescapeword{sequence}
\PMlinkescapeword{outer}
\PMlinkescapeword{theory}
\PMlinkescapeword{application}
\PMlinkescapeword{theorem}
\PMlinkescapeword{subset}

A \emph{Choquet capacity}, or just \emph{capacity}, on a set $X$ is a kind of set function, mapping the power set $\mathcal{P}(X)$ to the real numbers.

\begin{definition*}
Let $\mathcal{F}$ be a collection of subsets of $X$. Then, an $\mathcal{F}$-capacity is an increasing set function
\begin{equation*}
I\colon\mathcal{P}(X)\rightarrow\mathbb{R}_+
\end{equation*}
satisfying the following.
\begin{enumerate}
\item If $(A_n)_{n\in\mathbb{N}}$ is an increasing sequence of subsets of $X$ then $I(A_n)\rightarrow I\left(\bigcup_mA_m\right)$ as $n\rightarrow\infty$.
\item If $(A_n)_{n\in\mathbb{N}}$ is a decreasing sequence of subsets of $X$ such that $A_n\in\mathcal{F}$ for each $n$, then $I(A_n)\rightarrow I\left(\bigcap_mA_m\right)$ as $n\rightarrow\infty$.
\end{enumerate}
\end{definition*}
The condition that $I$ is increasing means that $I(A)\le I(B)$ whenever $A\subseteq B$.
Note that capacities differ from the concepts of measures and outer measures, as no additivity or subadditivity conditions are imposed. However, for any finite measure, there is a \PMlinkname{corresponding capacity}{CapacityGeneratedByAMeasure}. An important application to the theory of measures and analytic sets is given by the capacitability theorem.

The \emph{$(\mathcal{F},I)$-capacitable} sets are defined as follows. Recall that $\mathcal{F}_\delta$ denotes the collection of countable intersections of sets in the paving $\mathcal{F}$.

\begin{definition*}
Let $I$ be an $\mathcal{F}$-capacity on a set $X$. Then a subset $A\subseteq X$ is \emph{$(\mathcal{F},I)$-capacitable} if, for each $\epsilon >0$, there exists a $B\in\mathcal{F}_\delta$ such that $B\subseteq A$ and $I(B)\ge I(A)-\epsilon$.
\end{definition*}
Alternatively, such sets are called $I$-capacitable or, simply, capacitable.

%%%%%
%%%%%
\end{document}
