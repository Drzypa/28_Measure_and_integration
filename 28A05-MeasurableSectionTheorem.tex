\documentclass[12pt]{article}
\usepackage{pmmeta}
\pmcanonicalname{MeasurableSectionTheorem}
\pmcreated{2013-03-22 18:48:21}
\pmmodified{2013-03-22 18:48:21}
\pmowner{gel}{22282}
\pmmodifier{gel}{22282}
\pmtitle{measurable section theorem}
\pmrecord{7}{41607}
\pmprivacy{1}
\pmauthor{gel}{22282}
\pmtype{Theorem}
\pmcomment{trigger rebuild}
\pmclassification{msc}{28A05}
\pmsynonym{measurable cross section theorem}{MeasurableSectionTheorem}
%\pmkeywords{universally complete}
%\pmkeywords{analytic set}

% almost certainly you want these
\usepackage{amssymb}
\usepackage{amsmath}
\usepackage{amsfonts}

% used for TeXing text within eps files
%\usepackage{psfrag}
% need this for including graphics (\includegraphics)
\usepackage{graphicx,float}
% for neatly defining theorems and propositions
\usepackage{amsthm}
% making logically defined graphics
%%%\usepackage{xypic}

% there are many more packages, add them here as you need them

% define commands here
\newtheorem*{theorem*}{Theorem}
\newtheorem*{lemma*}{Lemma}
\newtheorem*{corollary*}{Corollary}
\newtheorem*{definition*}{Definition}
\newtheorem{theorem}{Theorem}
\newtheorem{lemma}{Lemma}
\newtheorem{corollary}{Corollary}
\newtheorem{definition}{Definition}

\begin{document}
\PMlinkescapeword{subset}
\PMlinkescapeword{cross section}
\PMlinkescapeword{map}
\PMlinkescapeword{theorem}
\PMlinkescapeword{states}
\PMlinkescapeword{way}
\PMlinkescapeword{cross sections}
\PMlinkescapeword{order}
\PMlinkescapeword{simple}
\PMlinkescapeword{application}
\PMlinkescapeword{force}
\PMlinkescapeword{basic}
\PMlinkescapeword{universal}
\PMlinkescapeword{index set}
\PMlinkescapeword{real numbers}
\PMlinkescapeword{equivalent}
\PMlinkescapeword{positive}
\PMlinkescapeword{projection}
\PMlinkescapeword{equivalence}
\PMlinkescapeword{theorems}
\PMlinkescapeword{section}
\PMlinkescapeword{applications}

Consider a Cartesian product $X\times Y$ with projection map $\pi_X\colon X\times Y\to X$. Then, a \emph{cross section} of a subset $S\subseteq X\times Y$ is a map $\tau\colon\pi_X(S)\to Y$ satisfying $(x,\tau(x))\in S$.
The following theorem states that such cross sections can be chosen in a \PMlinkname{measurable}{MeasurableFunctions} way. We use the notation $\mathcal{F}\otimes\mathcal{B}$ for the \PMlinkname{product $\sigma$-algebra}{ProductSigmaAlgebra} in order to distinguish it from the product paving $\mathcal{F}\times\mathcal{B}$.

\begin{theorem*}
Let $(X,\mathcal{F})$ be a universally complete \PMlinkname{measurable space}{MeasurableSpace}, $Y$ be a Polish space with \PMlinkname{Borel $\sigma$-algebra}{BorelSigmaAlgebra} $\mathcal{B}$, and $\pi_X\colon X\times Y\to X$ be the projection map.

For any $\mathcal{F}\otimes\mathcal{B}$-analytic set $S$, then $\pi_X(S)\in \mathcal{F}$ and there is a measurable map $\tau\colon \pi_X(S)\to Y$ such that $(x,\tau(x))\in S$ for all $x\in \pi_X(S)$.
\end{theorem*}


The existence of a map satisfying $(x,\tau(x))\in S$ is no surprise, and indeed must exist by a simple application of the axiom of choice. The main force of the theorem is that $\tau$ may be taken to be measurable, which is not something that could be achieved by such basic use of choice.
In fact, the result fails to be true if either the universal completeness assumption for $(X,\mathcal{F})$ or the requirement that $Y$ be a Polish space is dropped.

\begin{figure}[H]
\centering
\includegraphics[scale=1.1]{sectionpic}
\end{figure}

This result has important applications to continuous-time stochastic processes, where the space $Y$ is taken to be the time index set, usually the set $\mathbb{R}_+$ of nonnegative real numbers.

Consider stochastic processes $(U_t)_{t\in\mathbb{R}_+}$ and $(V_t)_{t\in\mathbb{R}_+}$ defined on a probability space $(\Omega,\mathcal{F},\mathbb{P})$. Suppose that they were not equivalent so that, with positive probability, there is a time $t$ at which they differ. Then the projection $\pi_\Omega(S)$ of the set of times $S=\{(t,\omega)\colon U_t(\omega)\not=V_t(\omega)\}$ has positive probability. Consequently, the measurable section theorem guarantees the existence of a random time $\tau$ for which $U_\tau\not=V_\tau$ with positive probability.

\begin{corollary*}
Let $(U_t)_{t\in\mathbb{R}_+}$ and $(V_t)_{t\in\mathbb{R}_+}$ be jointly measurable stochastic processes defined on the probability space $(\Omega,\mathcal{F},\mathbb{P})$. If
\begin{equation*}
\mathbb{P}(U_\tau=V_\tau)=1
\end{equation*}
for all random times $\tau\colon\Omega\to\mathbb{R}_+$, then $U$ and $V$ are equivalent processes.
\end{corollary*}

So, continuous-time stochastic processes are fully  determined up to equivalence by their values, neglecting zero probability sets, at random times. Note that the condition that $(\Omega,\mathcal{F})$ be universally complete can be dropped from the statement, due to the fact that any random variable $\tau$ on the completion of a probability space can be replaced by a random variable $\tau^\prime$ satisfying $\mathbb{P}(\tau=\tau^\prime)=1$.

More useful statements which apply to adapted processes, and allow $\tau$ to be either a stopping time or a predictable stopping time, are given by the optional and predictable section theorems.

%%%%%
%%%%%
\end{document}
