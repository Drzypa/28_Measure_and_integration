\documentclass[12pt]{article}
\usepackage{pmmeta}
\pmcanonicalname{LebesgueDecompositionTheorem}
\pmcreated{2013-03-22 13:26:28}
\pmmodified{2013-03-22 13:26:28}
\pmowner{Koro}{127}
\pmmodifier{Koro}{127}
\pmtitle{Lebesgue decomposition theorem}
\pmrecord{8}{34003}
\pmprivacy{1}
\pmauthor{Koro}{127}
\pmtype{Theorem}
\pmcomment{trigger rebuild}
\pmclassification{msc}{28A12}

\endmetadata

% this is the default PlanetMath preamble.  as your knowledge
% of TeX increases, you will probably want to edit this, but
% it should be fine as is for beginners.

% almost certainly you want these
\usepackage{amssymb}
\usepackage{amsmath}
\usepackage{amsfonts}
\usepackage{mathrsfs}
% used for TeXing text within eps files
%\usepackage{psfrag}
% need this for including graphics (\includegraphics)
%\usepackage{graphicx}
% for neatly defining theorems and propositions
%\usepackage{amsthm}
% making logically defined graphics
%%%\usepackage{xypic}

% there are many more packages, add them here as you need them

% define commands here
\begin{document}
Let $\mu$ and $\nu$ be two $\sigma$-finite signed measures in the measurable space $(\Omega,\mathscr{S})$. There exist two \PMlinkname{$\sigma$-finite}{SigmaFinite} signed measures $\nu_0$ and $\nu_1$ such that:
\begin{enumerate}
\item $\nu=\nu_0+\nu_1$;
\item $\nu_0\ll\mu$ (i.e. $\nu_0$ is absolutely continuous with respect to $\mu$;)
\item $\nu_1\perp\mu$ (i.e. $\nu_1$ and $\mu$ are singular.)
\end{enumerate}
These two measures are uniquely determined.
%%%%%
%%%%%
\end{document}
