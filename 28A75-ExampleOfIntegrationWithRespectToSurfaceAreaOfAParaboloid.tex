\documentclass[12pt]{article}
\usepackage{pmmeta}
\pmcanonicalname{ExampleOfIntegrationWithRespectToSurfaceAreaOfAParaboloid}
\pmcreated{2013-03-22 14:58:20}
\pmmodified{2013-03-22 14:58:20}
\pmowner{yark}{2760}
\pmmodifier{yark}{2760}
\pmtitle{example of integration with respect to surface area of a paraboloid}
\pmrecord{15}{36672}
\pmprivacy{1}
\pmauthor{yark}{2760}
\pmtype{Example}
\pmcomment{trigger rebuild}
\pmclassification{msc}{28A75}

\endmetadata


\begin{document}
In this example we examine the paraboloid given by the equation $z = x^2 + 3 y^2$.  Putting $g(x,y) = x^2 + 3 y^2$, we have
 $$\sqrt{1 + \left( \frac{\partial g}{\partial x} \right)^{\!2} + \left( \frac{\partial g}{\partial y} \right)^{\!2}}
= \sqrt{1 + \left( 2 x \right)^2 + \left( 6 y \right)^2}
= \sqrt{1 + 4 x^2 + 36 y^2 }$$
and hence
 $$\int_S f(x,y) \, d^2 A = \int f(x,y) \sqrt{ 1 + 4 x^2 + 36 y^2 } \, dx \, dy.$$

\PMlinkescapetext{\sl Quick links:}
\begin{itemize}
\item \PMlinkid{main entry}{6660}
\item \PMlinkid{previous example}{6669}
\item \PMlinkid{next example}{6673}
\end{itemize}

%%%%%
%%%%%
\end{document}
