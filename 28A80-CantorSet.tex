\documentclass[12pt]{article}
\usepackage{pmmeta}
\pmcanonicalname{CantorSet}
\pmcreated{2013-03-22 12:21:37}
\pmmodified{2013-03-22 12:21:37}
\pmowner{yark}{2760}
\pmmodifier{yark}{2760}
\pmtitle{Cantor set}
\pmrecord{34}{32083}
\pmprivacy{1}
\pmauthor{yark}{2760}
\pmtype{Definition}
\pmcomment{trigger rebuild}
\pmclassification{msc}{28A80}
\pmrelated{Countable}
\pmrelated{Measure}
\pmrelated{SingularFunction}
\pmrelated{CantorFunction}
\pmrelated{MengerSponge}
\pmrelated{RepresentationTheoremForCompactMetricSpaces}

\usepackage{amssymb}
\usepackage{amsmath}
\usepackage{amsfonts}

% used for TeXing text within eps files
%\usepackage{psfrag}
% need this for including graphics (\includegraphics)
\usepackage{graphicx}
% for neatly defining theorems and propositions
%\usepackage{amsthm}
% making logically defined graphics
%%%\usepackage{xypic} 

% there are many more packages, add them here as you need them

% define commands here
\newcommand{\md}{d}
\newcommand{\mv}[1]{\mathbf{#1}}	% matrix or vector
\newcommand{\mvt}[1]{\mv{#1}^{\mathrm{T}}}
\newcommand{\mvi}[1]{\mv{#1}^{-1}}
\newcommand{\mderiv}[1]{\frac{\md}{\md {#1}}} %d/dx
\newcommand{\mnthderiv}[2]{\frac{\md^{#2}}{\md {#1}^{#2}}} %d^n/dx
\newcommand{\mpderiv}[1]{\frac{\partial}{\partial {#1}}} %partial d^n/dx
\newcommand{\mnthpderiv}[2]{\frac{\partial^{#2}}{\partial {#1}^{#2}}} %partial d^n/dx
\newcommand{\borel}{\mathfrak{B}}
\newcommand{\integers}{\mathbb{Z}}
\newcommand{\rationals}{\mathbb{Q}}
\newcommand{\reals}{\mathbb{R}}
\newcommand{\complexes}{\mathbb{C}}
\newcommand{\naturals}{\mathbb{N}}
\newcommand{\defined}{:=}
\newcommand{\var}{\mathrm{var}}
\newcommand{\cov}{\mathrm{cov}}
\newcommand{\corr}{\mathrm{corr}}
\newcommand{\set}[1]{\left\{#1\right\}}
\newcommand{\powerset}[1]{\mathcal{P}(#1)}
\newcommand{\bra}[1]{\langle#1 \vert}
\newcommand{\ket}[1]{\vert \hspace{1pt}#1\rangle}
\newcommand{\braket}[2]{\langle #1 \ket{#2}}
\newcommand{\abs}[1]{\left|#1\right|}
\newcommand{\norm}[1]{\left|\left|#1\right|\right|}
\newcommand{\esssup}{\mathrm{ess\ sup}}
\newcommand{\Lspace}[1]{L^{#1}}
\newcommand{\Lone}{\Lspace{1}}
\newcommand{\Ltwo}{\Lspace{2}}
\newcommand{\Lp}{\Lspace{p}}
\newcommand{\Lq}{\Lspace{q}}
\newcommand{\Linf}{\Lspace{\infty}}

% character creep scotch tape fix ...........................
\begin{document}
\PMlinkescapeword{between}
\PMlinkescapeword{canonical}
\PMlinkescapeword{entire}
\PMlinkescapeword{even}
\PMlinkescapeword{infinite}
\PMlinkescapeword{moment}
\PMlinkescapeword{observation}
\PMlinkescapeword{right}
\PMlinkescapeword{rights}
\PMlinkescapeword{similar}

The Cantor set $C$ is the canonical example of an uncountable set of measure zero.  We construct $C$ as follows.

Begin with the unit interval $C_0 = [0,1]$, and remove the open segment $R_1 \defined (\frac{1}{3},\frac{2}{3})$ from the middle.  We 
define $C_1$ as the two remaining pieces
\begin{equation}
C_1 \defined C_0 \setminus R_1 = \left[0,\frac{1}{3}\right] \bigcup \left[\frac{2}{3},1\right]
\end{equation}
Now repeat the process on each remaining segment, removing the open set
\begin{equation}
R_2 \defined \left(\frac{1}{9},\frac{2}{9}\right) \bigcup \left(\frac{7}{9},\frac{8}{9}\right)
\end{equation}
to form the four-piece set
\begin{equation}
C_2 \defined C_1 \setminus R_2 = \left[0,\frac{1}{9}\right] \bigcup \left[\frac{2}{9},\frac{1}{3}\right] \bigcup \left[\frac{2}{3},\frac{7}{9}\right] \bigcup \left[\frac{8}{9},1\right]
\end{equation}
Continue the process, forming $C_3, C_4, \ldots$  Note that $C_k$ has $2^k$ pieces.
\newline
%Commented out, as the picture has disappeared!
%\begin{figure}[hhp]
%\begin{center}
%\caption{The sets $C_0$ through $C_5$ in the construction of the Cantor set}
%\includegraphics[width=\textwidth]{cantor-set.eps}
%\end{center}
%\end{figure}

Also note that at each step, the endpoints of each closed segment will stay in the set forever---e.g., the point $\frac{2}{3}$ isn't 
touched as we remove sets.

The \emph{Cantor set} is defined as
\begin{equation}
C \defined \bigcap_{k=1}^{\infty} C_k = C_0 \setminus \bigcup_{n=1}^{\infty}R_n
\end{equation}

\paragraph{Cardinality of the Cantor set}
To establish cardinality, we want a bijection between the Cantor set
and some set whose cardinality we know .

Start at $C_1$, which has two pieces.  Mark the left-hand segment ``0'' and the right-hand segment ``1''.  Then continue to $C_2$, 
and consider only the leftmost pair.  Again, mark the segments ``0'' and ``1'', and do the same for the rightmost pair.

Keep doing this all the way down the $C_k$, starting at the left side and marking the segments 0, 1, 0, 1, 0, 1 as you encounter them, 
until you've labeled the entire Cantor set.

Now, pick a path through the tree starting at $C_0$ and going left-left-right-left\ldots and so on.  Mark a decimal point for $C_0$, and 
record the zeros and ones as you proceed.  Each path has a unique number based on your decision at each step.
%Commented out, as the picture has disappeared!
%For example, the 
%figure represents your choice of left-left-right-left-right at the first five %steps, representing the number beginning 0.00101...
%\newline
%\begin{figure}[!hbp]
%\begin{center}
%\caption{One possible path through $C_5: 0.00101$}
%\includegraphics[width=\textwidth]{cantor-set-ladder.eps}
%\end{center}
%\end{figure}
Every point in the Cantor set will have a unique address dependent solely on the pattern of lefts and rights, 0's and 1's, required to 
reach it.
The Cantor set therefore has the same cardinality as the set of sequences of 0's and 1's, which is $2^{\aleph_0}$, the cardinality of the continuum.

\paragraph{The Cantor set and ternary expansions}
Return, for a moment, to the earlier observation that numbers such as $\frac{1}{3}$, $\frac{2}{9}$, the endpoints of deleted intervals, 
are themselves never deleted.  In particluar, consider the first deleted interval:  the ternary expansions of its constituent numbers are 
precisely those that begin $0.1$, and proceed thence with at least one non-zero ``ternary'' digit further along.  Note 
also that the point $\frac{1}{3}$, with ternary expansion $0.1$, may also be written $0.0\dot{2}$ (or $0.0\bar{2}$), which has no ternary digit $1$.
Similar descriptions apply to further deleted intervals.
The result is that the Cantor set is precisely those numbers in the set 
$[0,1]$ that have a ternary expansion contains no digits $1$.

\paragraph{Measure of the Cantor set}
Let $\mu$ be Lebesgue measure.  The measure of the sets $R_k$ that we remove during the construction of the Cantor set are
\begin{align}
\mu(R_1) &= \frac{2}{3} - \frac{1}{3} = \frac{1}{3}\\
\mu(R_2) &= \left(\frac{2}{9} - \frac{1}{9}\right) + \left(\frac{8}{9} - \frac{7}{9}\right) = \frac{2}{9}\\
&\ \vdots\\
\mu(R_k) &= \sum_{n=1}^{k}\frac{2^{n-1}}{3^n}
\end{align}
Note that the $R$'s are disjoint, which will allow us to sum their measures without worry.  In the limit $k \to \infty$, this gives us
\begin{equation}
\mu\left(\bigcup_{n=1}^{\infty}R_n\right) = \sum_{n=1}^{\infty} \frac{2^{n-1}}{3^n} = 1.
\end{equation}
But we have $\mu(C_0) = 1$ as well, so this means
\begin{equation}
\mu(C) = \mu\left(C_0 \setminus \bigcup_{n=1}^{\infty}R_n\right) = \mu(C_0) - \sum_{n=1}^{\infty} \frac{1}{2^n} = 1-1 = 0.
\end{equation}

Thus we have seen that the measure of $C$ is zero (though see below for more on this topic).  How many points are there in $C$?  
Lots, as we shall see.


So we have a set of measure zero (very tiny) with uncountably many points (very big).  This non-intuitive result is what 
makes Cantor sets so interesting.

\paragraph{Cantor sets with positive measure}
Clearly, Cantor sets can be constructed for all sorts of ``removals''---we can remove middle halves, or thirds, or any amount 
$\frac{1}{r}, r>1$ we like.  All of these Cantor sets have measure zero, since at each step $n$ we end up with
\begin{equation}
L_n = \left(1 - \frac{1}{r}\right)^n
\end{equation}
of what we started with, and $\lim_{n \to \infty} L_n = 0$ for any $r>1$.
%With apologies, the figure above is drawn for the case $r=2$,
%rather than the $r=3$ which seems to be the publicly favored example.

However, it is possible to construct Cantor sets with positive 
measure as well; the key is to remove less and less as we proceed.  These Cantor sets have the same ``shape'' (topology) as the 
Cantor set we first constructed, and the same cardinality, but a different ``size.''

Again, start with the unit interval for $C_0$, and choose a number $0 < p < 1$.  Let
\begin{equation}
R_1 \defined \left(\frac{2-p}{4},\frac{2+p}{4}\right)
\end{equation}
which has length (measure) $\frac{p}{2}$.  Again, define $C_1 \defined C_0 \setminus R_1$.  Now define
\begin{equation}
R_2 \defined \left(\frac{2-p}{16},\frac{2+p}{16}\right) \bigcup \left(\frac{14-p}{16},\frac{14+p}{16}\right)
\end{equation}
which has measure $\frac{p}{4}$.  Continue as before, such that each $R_k$ has measure $\frac{p}{2^k}$; note again that all the 
$R_k$ are disjoint.  The resulting Cantor set has measure
\begin{equation*}
\mu\left(C_0 \setminus \bigcup_{n=1}^{\infty}R_n \right) = 1 - \sum_{n=1}^{\infty} \mu(R_n) = 1 - \sum_{n=1}^{\infty} p\ 2^{-n} = 1-p > 0.
\end{equation*}
Thus we have a whole family of Cantor sets of positive measure to accompany their vanishing brethren.

% character creep scotch-tape fix  ...................................
%%%%%
%%%%%
\end{document}
