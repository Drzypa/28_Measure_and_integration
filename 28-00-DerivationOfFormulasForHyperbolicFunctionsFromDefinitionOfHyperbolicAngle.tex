\documentclass[12pt]{article}
\usepackage{pmmeta}
\pmcanonicalname{DerivationOfFormulasForHyperbolicFunctionsFromDefinitionOfHyperbolicAngle}
\pmcreated{2013-03-22 17:53:58}
\pmmodified{2013-03-22 17:53:58}
\pmowner{Wkbj79}{1863}
\pmmodifier{Wkbj79}{1863}
\pmtitle{derivation of formulas for hyperbolic functions from definition of hyperbolic angle}
\pmrecord{6}{40390}
\pmprivacy{1}
\pmauthor{Wkbj79}{1863}
\pmtype{Derivation}
\pmcomment{trigger rebuild}
\pmclassification{msc}{28-00}
\pmclassification{msc}{26A09}
\pmrelated{HyperbolicFunctions}
\pmrelated{HyperbolicIdentities}

\endmetadata

\usepackage{amssymb}
\usepackage{amsmath}
\usepackage{amsfonts}
\usepackage{pstricks}
\usepackage{pst-plot}
\usepackage{psfrag}
\usepackage{graphicx}
\usepackage{amsthm}
%%\usepackage{xypic}

\begin{document}
\PMlinkescapeword{formula}
\PMlinkescapeword{formulas}
\PMlinkescapephrase{integration formula}
\PMlinkescapeword{order}

Let $H_1$ be the branch (connected component) of the unit hyperbola $x^2-y^2=1$ with $x>0$.

\begin{center}
\psset{unit=1.5cm}
\begin{pspicture}(-1,-2.25)(2.5,2.25)
\psaxes[Dx=10,Dy=10]{<->}(0,0)(-1,-2.25)(2.5,2.25)
\rput(-0.2,2.25){$y$}
\rput(2.5,-0.2){$x$}
\rput(1.8,1.9){$H_1$}
\psplot{1}{2}{x 2 exp -1 add 0.5 exp}
\psplot{1}{2}{x 2 exp -1 add 0.5 exp -1 mul}
\rput[r](-1,0){.}
\rput[a](0,-2.25){.}
\end{pspicture}
\end{center}

Let $\alpha>0$.  Then $(\cosh\alpha,\sinh\alpha)$ is the point on $H_1$ with hyperbolic angle $\alpha$.

In order to draw the hyperbolic angle, the line passing through $(0,0)$ and $(\cosh\alpha,\sinh\alpha)$ must be drawn.  Recall that $\tanh\alpha$ is defined by
\[
\tanh\alpha :=\frac{\sinh\alpha}{\cosh\alpha}.
\]
Thus, the equation of the line passing through $(0,0)$ and $(\cosh\alpha,\sinh\alpha)$ is $y=(\tanh\alpha)x$.

Below is the \PMlinkname{graph}{Graph2} of $H_1$ and the line $y=(\tanh\alpha)x$.

\begin{center}
\psset{unit=2cm}
\begin{pspicture}(-1,-2.25)(2.5,2.25)
\psaxes[Dx=10,Dy=10]{<->}(0,0)(-1,-2.25)(2.5,2.25)
\rput(-0.2,2.25){$y$}
\rput(2.5,-0.2){$x$}
\rput(1.8,1.9){$H_1$}
\rput[l](1.32,0.6){$(\cosh\alpha,\sinh\alpha)$}
\rput[l](2.1,1.25){$y=(\tanh\alpha)x$}
\psplot{1}{2}{x 2 exp -1 add 0.5 exp}
\psplot{1}{2}{x 2 exp -1 add 0.5 exp -1 mul}
\psline{<->}(-1,-0.667)(1.95,1.3)
\rput[r](-1,0){.}
\rput[a](0,-2.25){.}
\end{pspicture}
\end{center}

Observe also that $\sinh\alpha>0$ and $\cosh\alpha>1$.

In the \PMlinkname{hyperbolic angle}{HyperbolicAngle} entry, it is discussed that $\alpha$ is twice the area \PMlinkescapetext{bounded} by the $x$ \PMlinkescapetext{axis}, $H_1$, and the line $y=(\tanh\alpha)x$.  We will use this fact to obtain formulas for $\cosh\alpha$ and $\sinh\alpha$.

In the calculations below, the following integration formula will be used:

\[
\int\sqrt{x^2-1} \, dx=\frac{x}{2}\sqrt{x^2-1}-\frac{1}{2}\ln\left|x+\sqrt{x^2-1}\right|+C
\]

Thus, we have
\begin{align*}
\alpha & =2\left( \int\limits_0^{\cosh\alpha} (\tanh\alpha)x \, dx -\int\limits_1^{\cosh\alpha}\sqrt{x^2-1} \, dx \right) \\
& =\left. (\tanh\alpha)x \right|_0^{\cosh\alpha}-\left. \left( x\sqrt{x^2-1}-\ln\left|x+\sqrt{x^2-1}\right| \right) \right|_1^{\cosh\alpha} \\
& =\tanh\alpha\cosh^2\alpha-\cosh\alpha\sqrt{\cosh^2\alpha-1}+\ln\left(\cosh\alpha+\sqrt{\cosh^2\alpha-1}\right) +\sqrt{1^2-1}-\ln\left(1+\sqrt{1^2-1}\right) \\
& =\cosh\alpha\sinh\alpha-\cosh\alpha\sinh\alpha+\ln(\cosh\alpha+\sinh\alpha) \\
& =\ln(\cosh\alpha+\sinh\alpha).
\end{align*}

Thus, we have
\[
e^{\alpha}=\cosh\alpha+\sinh\alpha.
\]
A \PMlinkescapetext{similar} calculation yields
\[
e^{-\alpha}=\cosh\alpha-\sinh\alpha.
\]
The formulas for the hyperbolic functions are easily \PMlinkescapetext{derived} from the above equations:
\begin{align*}
\cosh\alpha=\frac{e^{\alpha}+e^{-\alpha}}{2} \\
\sinh\alpha=\frac{e^{\alpha}-e^{-\alpha}}{2}
\end{align*}

As mentioned in the \PMlinkname{hyperbolic angle}{HyperbolicAngle} entry, these formulas can be extended to all $\alpha\in\mathbb{R}$ and from there to all $\alpha\in\mathbb{C}$.
%%%%%
%%%%%
\end{document}
